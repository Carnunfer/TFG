%
\begin{isabellebody}%
\setisabellecontext{CancelacionInyectiva}%
%
\isadelimtheory
%
\endisadelimtheory
%
\isatagtheory
%
\endisatagtheory
{\isafoldtheory}%
%
\isadelimtheory
%
\endisadelimtheory
%
\isadelimdocument
%
\endisadelimdocument
%
\isatagdocument
%
\isamarkupsection{Cancelación de funciones inyectivas%
}
\isamarkuptrue%
%
\endisatagdocument
{\isafolddocument}%
%
\isadelimdocument
%
\endisadelimdocument
%
\begin{isamarkuptext}%
El siguiente teorema prueba una caracterización de las funciones
 inyectivas, en otras palabras, las funciones inyectivas son
 monomorfismos en la categoría de conjuntos. Un monomorfismo es un
 homomorfismo inyectivo y la categoría de conjuntos es la categoría
 cuyos objetos son los conjuntos.
  
  \begin{teorema}
    $f$ es una función inyectiva, si y solo si, para todas $g$ y $h$
    tales que \isa{f\ {\isasymcirc}\ g\ {\isacharequal}\ f\ {\isasymcirc}\ h} se tiene que $g = h$. 
  \end{teorema}

Vamos a hacer dos lemas de nuestro teorema, ya que podemos la doble 
implicación en dos implicaciones y demostrar cada una de ellas por
 separado.

\begin {lema}
$f$ es una función inyectiva si para todas $g$ y $h$ tales que $f \circ
 g = f \circ h$ se tiene que $g = h.$
\end {lema}
  \begin{demostracion}
    La demostración la haremos por doble implicación: 
\begin {enumerate}
\item Supongamos que tenemos que $f \circ g = f \circ h$, queremos
 demostrar que $g = h$, usando que f es inyectiva tenemos que: \\
$$(f \circ g)(x) = (f \circ h)(x) \Longrightarrow f(g(x)) = f(h(x)) = 
g(x) = h(x)$$
\item Supongamos ahora que $g = h$, queremos demostrar que  $f \circ g
 = f \circ h$. \\
$$(f \circ g)(x) = f(g(x)) = f(h(x)) = (f \circ h)(x)$$
\end {enumerate}
.
  \end{demostracion}

\begin {lema} 
Si para toda $g$ y $h$ tales que $f \circ g =  f \circ h$ se tiene que $g
= h$ entonces f es inyectiva.
\end {lema} 

\begin {demostracion}
Supongamos que el dominio de nuestra función $f$ es distinto del vacío.
Tenemos que demostrar que $\forall a,b$ tales que $f(a) = f(b),$ esto
 implica que $a = b.$ \\
Sean $a,b$ tales que $f(a) = f(b)$, sean ahora $g(x) = a \forall x$ y
 $h(x) = b \forall x$ entonces 
$$(f \circ g) = (f \circ h) \Longrightarrow  f(g(x)) = f(h(x)) \Longrightarrow f(a) = f(b)$$
Por hipótesis tenemos entonces que $a = b,$ como queríamos demostrar.
\end {demostracion}


  Su especificación es la siguiente, pero al igual que hemos hecho en la demostración
a mano vamos a demostrarlo a través de dos lemas:%
\end{isamarkuptext}\isamarkuptrue%
\isacommand{theorem}\isamarkupfalse%
\ \isanewline
\ \ {\isachardoublequoteopen}inj\ f\ {\isasymlongleftrightarrow}\ {\isacharparenleft}f\ {\isasymcirc}\ g\ {\isacharequal}\ f\ {\isasymcirc}\ h{\isacharparenright}\ {\isacharequal}\ {\isacharparenleft}g\ {\isacharequal}\ h{\isacharparenright}{\isachardoublequoteclose}\isanewline
%
\isadelimproof
\ \ %
\endisadelimproof
%
\isatagproof
\isacommand{oops}\isamarkupfalse%
%
\endisatagproof
{\isafoldproof}%
%
\isadelimproof
%
\endisadelimproof
%
\begin{isamarkuptext}%
Sus lemas son los siguientes:%
\end{isamarkuptext}\isamarkuptrue%
\isacommand{lemma}\isamarkupfalse%
\ \isanewline
{\isachardoublequoteopen}{\isasymforall}g\ h{\isachardot}\ {\isacharparenleft}f\ {\isasymcirc}\ g\ {\isacharequal}\ f\ {\isasymcirc}\ h\ {\isasymlongrightarrow}\ g\ {\isacharequal}\ h{\isacharparenright}\ {\isasymLongrightarrow}\ inj\ f{\isachardoublequoteclose}\isanewline
%
\isadelimproof
\ \ %
\endisadelimproof
%
\isatagproof
\isacommand{oops}\isamarkupfalse%
%
\endisatagproof
{\isafoldproof}%
%
\isadelimproof
\isanewline
%
\endisadelimproof
\isanewline
\isacommand{lemma}\isamarkupfalse%
\ \isanewline
{\isachardoublequoteopen}inj\ f\ {\isasymLongrightarrow}\ {\isacharparenleft}f\ {\isasymcirc}\ g\ {\isacharequal}\ f\ {\isasymcirc}\ h{\isacharparenright}\ {\isacharequal}\ {\isacharparenleft}g\ {\isacharequal}\ h{\isacharparenright}{\isachardoublequoteclose}\isanewline
%
\isadelimproof
\ \ %
\endisadelimproof
%
\isatagproof
\isacommand{oops}\isamarkupfalse%
%
\endisatagproof
{\isafoldproof}%
%
\isadelimproof
%
\endisadelimproof
%
\begin{isamarkuptext}%
En la especificación anterior, \isa{inj\ f} es una 
  abreviatura de \isa{inj\ f} definida en la teoría
  \href{http://bit.ly/2XuPQx5}{Fun.thy}. Además, contiene la definición
  de \isa{inj{\isacharunderscore}on}
  \begin{itemize}
    \item[] \isa{inj{\isacharunderscore}on\ f\ A\ {\isacharequal}\ {\isacharparenleft}{\isasymforall}x{\isasymin}A{\isachardot}\ {\isasymforall}y{\isasymin}A{\isachardot}\ f\ x\ {\isacharequal}\ f\ y\ {\isasymlongrightarrow}\ x\ {\isacharequal}\ y{\isacharparenright}} \hfill (\isa{inj{\isacharunderscore}on{\isacharunderscore}def})
  \end{itemize} 
  Por su parte, \isa{UNIV} es el conjunto universal definido en la 
  teoría \href{http://bit.ly/2XtHCW6}{Set.thy} como una abreviatura de 
  \isa{top} que, a su vez está definido en la teoría 
  \href{http://bit.ly/2Xyj9Pe}{Orderings.thy} mediante la siguiente
  propiedad 
  \begin{itemize}
    \item[] \isa{\mbox{}\inferrule{\mbox{ordering{\isacharunderscore}top\ less{\isacharunderscore}eq\ less\ top}}{\mbox{less{\isacharunderscore}eq\ a\ top}}} 
      \hfill (\isa{ordering{\isacharunderscore}top{\isachardot}extremum})
  \end{itemize} 
  En el caso de la teoría de conjuntos, la relación de orden es la
  inclusión de conjuntos.

  Presentaremos distintas demostraciones de los lemas. La primera
  demostración es applicativa:%
\end{isamarkuptext}\isamarkuptrue%
\isacommand{lemma}\isamarkupfalse%
\ inyectivapli{\isacharcolon}\isanewline
\ \ {\isachardoublequoteopen}inj\ f\ {\isasymLongrightarrow}\ {\isacharparenleft}f\ {\isasymcirc}\ g\ {\isacharequal}\ f\ {\isasymcirc}\ h{\isacharparenright}\ {\isacharequal}\ {\isacharparenleft}g\ {\isacharequal}\ h{\isacharparenright}{\isachardoublequoteclose}\isanewline
%
\isadelimproof
\ \ %
\endisadelimproof
%
\isatagproof
\isacommand{apply}\isamarkupfalse%
\ {\isacharparenleft}simp\ add{\isacharcolon}\ inj{\isacharunderscore}on{\isacharunderscore}def\ fun{\isacharunderscore}eq{\isacharunderscore}iff{\isacharparenright}\ \isanewline
\ \ \isacommand{apply}\isamarkupfalse%
\ auto\isanewline
\ \ \isacommand{done}\isamarkupfalse%
%
\endisatagproof
{\isafoldproof}%
%
\isadelimproof
\ \isanewline
%
\endisadelimproof
\isanewline
\isacommand{lemma}\isamarkupfalse%
\ inyectivapli{\isadigit{2}}{\isacharcolon}\isanewline
{\isachardoublequoteopen}{\isasymforall}g\ h{\isachardot}\ {\isacharparenleft}f\ {\isasymcirc}\ g\ {\isacharequal}\ f\ {\isasymcirc}\ h\ {\isasymlongrightarrow}\ g\ {\isacharequal}\ h{\isacharparenright}\ {\isasymLongrightarrow}\ inj\ f{\isachardoublequoteclose}\isanewline
%
\isadelimproof
\ \ %
\endisadelimproof
%
\isatagproof
\isacommand{apply}\isamarkupfalse%
\ {\isacharparenleft}rule\ injI{\isacharparenright}\isanewline
\ \ \isacommand{by}\isamarkupfalse%
\ {\isacharparenleft}metis\ fun{\isacharunderscore}upd{\isacharunderscore}apply\ fun{\isacharunderscore}upd{\isacharunderscore}comp{\isacharparenright}%
\endisatagproof
{\isafoldproof}%
%
\isadelimproof
%
\endisadelimproof
%
\begin{isamarkuptext}%
En las demostraciones anteriores se han usado los siguientes
 lemas:
  \begin{itemize}
    \item[] \isa{{\isacharparenleft}f\ {\isacharequal}\ g{\isacharparenright}\ {\isacharequal}\ {\isacharparenleft}{\isasymforall}x{\isachardot}\ f\ x\ {\isacharequal}\ g\ x{\isacharparenright}} 
      \hfill (\isa{fun{\isacharunderscore}eq{\isacharunderscore}iff})
  \end{itemize} 
  \begin{itemize}
    \item[] \isa{{\isacharparenleft}f{\isacharparenleft}x\ {\isacharcolon}{\isacharequal}\ y{\isacharparenright}{\isacharparenright}\ z\ {\isacharequal}\ {\isacharparenleft}\textsf{if}\ z\ {\isacharequal}\ x\ \textsf{then}\ y\ \textsf{else}\ f\ z{\isacharparenright}} 
      \hfill (\isa{fun{\isacharunderscore}upd{\isacharunderscore}apply})
  \end{itemize} 
  \begin{itemize}
    \item[] \isa{{\isacharparenleft}f\ {\isacharequal}\ g{\isacharparenright}\ {\isacharequal}\ {\isacharparenleft}{\isasymforall}x{\isachardot}\ f\ x\ {\isacharequal}\ g\ x{\isacharparenright}} 
      \hfill (\isa{fun{\isacharunderscore}upd{\isacharunderscore}comp})
  \end{itemize} 

  La demostración applicativa sin auto es%
\end{isamarkuptext}\isamarkuptrue%
\isacommand{lemma}\isamarkupfalse%
\isanewline
\ \ {\isachardoublequoteopen}inj\ f\ {\isasymLongrightarrow}\ {\isacharparenleft}f\ {\isasymcirc}\ g\ {\isacharequal}\ f\ {\isasymcirc}\ h{\isacharparenright}\ {\isacharequal}\ {\isacharparenleft}g\ {\isacharequal}\ h{\isacharparenright}{\isachardoublequoteclose}\isanewline
%
\isadelimproof
\ \ %
\endisadelimproof
%
\isatagproof
\isacommand{apply}\isamarkupfalse%
\ {\isacharparenleft}unfold\ inj{\isacharunderscore}on{\isacharunderscore}def{\isacharparenright}\ \isanewline
\ \ \isacommand{apply}\isamarkupfalse%
\ {\isacharparenleft}unfold\ fun{\isacharunderscore}eq{\isacharunderscore}iff{\isacharparenright}\ \isanewline
\ \ \isacommand{apply}\isamarkupfalse%
\ {\isacharparenleft}unfold\ o{\isacharunderscore}apply{\isacharparenright}\isanewline
\ \ \isacommand{apply}\isamarkupfalse%
\ {\isacharparenleft}rule\ iffI{\isacharparenright}\isanewline
\ \ \ \isacommand{apply}\isamarkupfalse%
\ simp{\isacharplus}\isanewline
\ \ \isacommand{done}\isamarkupfalse%
%
\endisatagproof
{\isafoldproof}%
%
\isadelimproof
%
\endisadelimproof
%
\begin{isamarkuptext}%
En la demostración anterior se ha introducido los siguientes
  hechos
  \begin{itemize}
    \item \isa{{\isacharparenleft}f\ {\isasymcirc}\ g{\isacharparenright}\ x\ {\isacharequal}\ f\ {\isacharparenleft}g\ x{\isacharparenright}} \hfill (\isa{o{\isacharunderscore}apply})
    \item \isa{{\isasymlbrakk}P\ {\isasymLongrightarrow}\ Q{\isacharsemicolon}\ Q\ {\isasymLongrightarrow}\ P{\isasymrbrakk}\ {\isasymLongrightarrow}\ P\ {\isacharequal}\ Q} \hfill (\isa{iffI})
  \end{itemize} 

  La demostración automática es%
\end{isamarkuptext}\isamarkuptrue%
\isacommand{lemma}\isamarkupfalse%
\isanewline
\ \ \isakeyword{assumes}\ {\isachardoublequoteopen}inj\ f{\isachardoublequoteclose}\isanewline
\ \ \isakeyword{shows}\ {\isachardoublequoteopen}{\isacharparenleft}f\ {\isasymcirc}\ g\ {\isacharequal}\ f\ {\isasymcirc}\ h{\isacharparenright}\ {\isacharequal}\ {\isacharparenleft}g\ {\isacharequal}\ h{\isacharparenright}{\isachardoublequoteclose}\isanewline
%
\isadelimproof
\ \ %
\endisadelimproof
%
\isatagproof
\isacommand{using}\isamarkupfalse%
\ assms\isanewline
\ \ \isacommand{by}\isamarkupfalse%
\ {\isacharparenleft}auto\ simp\ add{\isacharcolon}\ inj{\isacharunderscore}on{\isacharunderscore}def\ fun{\isacharunderscore}eq{\isacharunderscore}iff{\isacharparenright}%
\endisatagproof
{\isafoldproof}%
%
\isadelimproof
%
\endisadelimproof
%
\begin{isamarkuptext}%
La demostración declarativa%
\end{isamarkuptext}\isamarkuptrue%
\isacommand{lemma}\isamarkupfalse%
\isanewline
\ \ \isakeyword{assumes}\ {\isachardoublequoteopen}inj\ f{\isachardoublequoteclose}\isanewline
\ \ \isakeyword{shows}\ {\isachardoublequoteopen}{\isacharparenleft}f\ {\isasymcirc}\ g\ {\isacharequal}\ f\ {\isasymcirc}\ h{\isacharparenright}\ {\isacharequal}\ {\isacharparenleft}g\ {\isacharequal}\ h{\isacharparenright}{\isachardoublequoteclose}\isanewline
%
\isadelimproof
%
\endisadelimproof
%
\isatagproof
\isacommand{proof}\isamarkupfalse%
\ \isanewline
\ \ \isacommand{assume}\isamarkupfalse%
\ {\isachardoublequoteopen}f\ {\isasymcirc}\ g\ {\isacharequal}\ f\ {\isasymcirc}\ h{\isachardoublequoteclose}\isanewline
\ \ \isacommand{show}\isamarkupfalse%
\ {\isachardoublequoteopen}g\ {\isacharequal}\ h{\isachardoublequoteclose}\isanewline
\ \ \isacommand{proof}\isamarkupfalse%
\isanewline
\ \ \ \ \isacommand{fix}\isamarkupfalse%
\ x\isanewline
\ \ \ \ \isacommand{have}\isamarkupfalse%
\ {\isachardoublequoteopen}{\isacharparenleft}f\ {\isasymcirc}\ g{\isacharparenright}{\isacharparenleft}x{\isacharparenright}\ {\isacharequal}\ {\isacharparenleft}f\ {\isasymcirc}\ h{\isacharparenright}{\isacharparenleft}x{\isacharparenright}{\isachardoublequoteclose}\ \isacommand{using}\isamarkupfalse%
\ {\isacharbackquoteopen}f\ {\isasymcirc}\ g\ {\isacharequal}\ f\ {\isasymcirc}\ h{\isacharbackquoteclose}\ \isacommand{by}\isamarkupfalse%
\ simp\isanewline
\ \ \ \ \isacommand{then}\isamarkupfalse%
\ \isacommand{have}\isamarkupfalse%
\ {\isachardoublequoteopen}f{\isacharparenleft}g{\isacharparenleft}x{\isacharparenright}{\isacharparenright}\ {\isacharequal}\ f{\isacharparenleft}h{\isacharparenleft}x{\isacharparenright}{\isacharparenright}{\isachardoublequoteclose}\ \isacommand{by}\isamarkupfalse%
\ simp\isanewline
\ \ \ \ \isacommand{then}\isamarkupfalse%
\ \isacommand{show}\isamarkupfalse%
\ {\isachardoublequoteopen}g{\isacharparenleft}x{\isacharparenright}\ {\isacharequal}\ h{\isacharparenleft}x{\isacharparenright}{\isachardoublequoteclose}\ \isacommand{using}\isamarkupfalse%
\ {\isacharbackquoteopen}inj\ f{\isacharbackquoteclose}\ \isacommand{by}\isamarkupfalse%
\ {\isacharparenleft}simp\ add{\isacharcolon}inj{\isacharunderscore}on{\isacharunderscore}def{\isacharparenright}\isanewline
\ \ \isacommand{qed}\isamarkupfalse%
\isanewline
\isacommand{next}\isamarkupfalse%
\isanewline
\ \ \isacommand{assume}\isamarkupfalse%
\ {\isachardoublequoteopen}g\ {\isacharequal}\ h{\isachardoublequoteclose}\isanewline
\ \ \isacommand{show}\isamarkupfalse%
\ {\isachardoublequoteopen}f\ {\isasymcirc}\ g\ {\isacharequal}\ f\ {\isasymcirc}\ h{\isachardoublequoteclose}\isanewline
\ \ \isacommand{proof}\isamarkupfalse%
\isanewline
\ \ \ \ \isacommand{fix}\isamarkupfalse%
\ x\isanewline
\ \ \ \ \isacommand{have}\isamarkupfalse%
\ {\isachardoublequoteopen}{\isacharparenleft}f\ {\isasymcirc}\ g{\isacharparenright}\ x\ {\isacharequal}\ f{\isacharparenleft}g{\isacharparenleft}x{\isacharparenright}{\isacharparenright}{\isachardoublequoteclose}\ \isacommand{by}\isamarkupfalse%
\ simp\isanewline
\ \ \ \ \isacommand{also}\isamarkupfalse%
\ \isacommand{have}\isamarkupfalse%
\ {\isachardoublequoteopen}{\isasymdots}\ {\isacharequal}\ f{\isacharparenleft}h{\isacharparenleft}x{\isacharparenright}{\isacharparenright}{\isachardoublequoteclose}\ \isacommand{using}\isamarkupfalse%
\ {\isacharbackquoteopen}g\ {\isacharequal}\ h{\isacharbackquoteclose}\ \isacommand{by}\isamarkupfalse%
\ simp\isanewline
\ \ \ \ \isacommand{also}\isamarkupfalse%
\ \isacommand{have}\isamarkupfalse%
\ {\isachardoublequoteopen}{\isasymdots}\ {\isacharequal}\ {\isacharparenleft}f\ {\isasymcirc}\ h{\isacharparenright}\ x{\isachardoublequoteclose}\ \isacommand{by}\isamarkupfalse%
\ simp\isanewline
\ \ \ \ \isacommand{finally}\isamarkupfalse%
\ \isacommand{show}\isamarkupfalse%
\ {\isachardoublequoteopen}{\isacharparenleft}f\ {\isasymcirc}\ g{\isacharparenright}\ x\ {\isacharequal}\ {\isacharparenleft}f\ {\isasymcirc}\ h{\isacharparenright}\ x{\isachardoublequoteclose}\ \isacommand{by}\isamarkupfalse%
\ simp\isanewline
\ \ \isacommand{qed}\isamarkupfalse%
\isanewline
\isacommand{qed}\isamarkupfalse%
%
\endisatagproof
{\isafoldproof}%
%
\isadelimproof
%
\endisadelimproof
%
\begin{isamarkuptext}%
Otra demostración declarativa es%
\end{isamarkuptext}\isamarkuptrue%
\isacommand{lemma}\isamarkupfalse%
\ \isanewline
\ \ \isakeyword{assumes}\ {\isachardoublequoteopen}inj\ f{\isachardoublequoteclose}\isanewline
\ \ \isakeyword{shows}\ {\isachardoublequoteopen}{\isacharparenleft}f\ {\isasymcirc}\ g\ {\isacharequal}\ f\ {\isasymcirc}\ h{\isacharparenright}\ {\isacharequal}\ {\isacharparenleft}g\ {\isacharequal}\ h{\isacharparenright}{\isachardoublequoteclose}\isanewline
%
\isadelimproof
%
\endisadelimproof
%
\isatagproof
\isacommand{proof}\isamarkupfalse%
\ \isanewline
\ \ \isacommand{assume}\isamarkupfalse%
\ {\isachardoublequoteopen}f\ {\isasymcirc}\ g\ {\isacharequal}\ f\ {\isasymcirc}\ h{\isachardoublequoteclose}\ \isanewline
\ \ \isacommand{then}\isamarkupfalse%
\ \isacommand{show}\isamarkupfalse%
\ {\isachardoublequoteopen}g\ {\isacharequal}\ h{\isachardoublequoteclose}\ \isacommand{using}\isamarkupfalse%
\ {\isacharbackquoteopen}inj\ f{\isacharbackquoteclose}\ \isacommand{by}\isamarkupfalse%
\ {\isacharparenleft}simp\ add{\isacharcolon}\ inj{\isacharunderscore}on{\isacharunderscore}def\ fun{\isacharunderscore}eq{\isacharunderscore}iff{\isacharparenright}\ \isanewline
\isacommand{next}\isamarkupfalse%
\isanewline
\ \ \isacommand{assume}\isamarkupfalse%
\ {\isachardoublequoteopen}g\ {\isacharequal}\ h{\isachardoublequoteclose}\ \isanewline
\ \ \isacommand{then}\isamarkupfalse%
\ \isacommand{show}\isamarkupfalse%
\ {\isachardoublequoteopen}f\ {\isasymcirc}\ g\ {\isacharequal}\ f\ {\isasymcirc}\ h{\isachardoublequoteclose}\ \isacommand{by}\isamarkupfalse%
\ simp\isanewline
\isacommand{qed}\isamarkupfalse%
%
\endisatagproof
{\isafoldproof}%
%
\isadelimproof
%
\endisadelimproof
%
\begin{isamarkuptext}%
En consecuencia, la demostración de nuestro teorema:%
\end{isamarkuptext}\isamarkuptrue%
\isacommand{theorem}\isamarkupfalse%
\ \isanewline
{\isachardoublequoteopen}{\isasymforall}g\ h{\isachardot}\ {\isacharparenleft}f\ {\isasymcirc}\ g\ {\isacharequal}\ f\ {\isasymcirc}\ h\ {\isasymlongrightarrow}\ g\ {\isacharequal}\ h{\isacharparenright}\ {\isasymlongleftrightarrow}\ inj\ f{\isachardoublequoteclose}\isanewline
%
\isadelimproof
\ \ %
\endisadelimproof
%
\isatagproof
\isacommand{oops}\isamarkupfalse%
\isanewline
%
\endisatagproof
{\isafoldproof}%
%
\isadelimproof
%
\endisadelimproof
%
\isadelimtheory
%
\endisadelimtheory
%
\isatagtheory
%
\endisatagtheory
{\isafoldtheory}%
%
\isadelimtheory
%
\endisadelimtheory
%
\end{isabellebody}%
\endinput
%:%file=~/ownCloud/alonso/curso-TFG/Carlos/TFG/CancelacionInyectiva.thy%:%
%:%24=8%:%
%:%36=10%:%
%:%37=11%:%
%:%38=12%:%
%:%39=13%:%
%:%40=14%:%
%:%41=15%:%
%:%42=16%:%
%:%43=17%:%
%:%44=18%:%
%:%45=19%:%
%:%46=20%:%
%:%47=21%:%
%:%48=22%:%
%:%49=23%:%
%:%50=24%:%
%:%51=25%:%
%:%52=26%:%
%:%53=27%:%
%:%54=28%:%
%:%55=29%:%
%:%56=30%:%
%:%57=31%:%
%:%58=32%:%
%:%59=33%:%
%:%60=34%:%
%:%61=35%:%
%:%62=36%:%
%:%63=37%:%
%:%64=38%:%
%:%65=39%:%
%:%66=40%:%
%:%67=41%:%
%:%68=42%:%
%:%69=43%:%
%:%70=44%:%
%:%71=45%:%
%:%72=46%:%
%:%73=47%:%
%:%74=48%:%
%:%75=49%:%
%:%76=50%:%
%:%77=51%:%
%:%78=52%:%
%:%79=53%:%
%:%80=54%:%
%:%81=55%:%
%:%82=56%:%
%:%83=57%:%
%:%84=58%:%
%:%85=59%:%
%:%86=60%:%
%:%88=63%:%
%:%89=63%:%
%:%90=64%:%
%:%93=65%:%
%:%97=65%:%
%:%107=68%:%
%:%109=70%:%
%:%110=70%:%
%:%111=71%:%
%:%114=72%:%
%:%118=72%:%
%:%124=72%:%
%:%127=73%:%
%:%128=74%:%
%:%129=74%:%
%:%130=75%:%
%:%133=76%:%
%:%137=76%:%
%:%147=79%:%
%:%148=80%:%
%:%149=81%:%
%:%150=82%:%
%:%151=83%:%
%:%152=84%:%
%:%153=85%:%
%:%154=86%:%
%:%155=87%:%
%:%156=88%:%
%:%157=89%:%
%:%158=90%:%
%:%159=91%:%
%:%160=92%:%
%:%161=93%:%
%:%162=94%:%
%:%163=95%:%
%:%164=96%:%
%:%165=97%:%
%:%166=98%:%
%:%167=99%:%
%:%169=101%:%
%:%170=101%:%
%:%171=102%:%
%:%174=103%:%
%:%178=103%:%
%:%179=103%:%
%:%180=104%:%
%:%181=104%:%
%:%182=105%:%
%:%188=105%:%
%:%191=106%:%
%:%192=107%:%
%:%193=107%:%
%:%194=108%:%
%:%197=109%:%
%:%201=109%:%
%:%202=109%:%
%:%203=110%:%
%:%204=110%:%
%:%213=113%:%
%:%214=114%:%
%:%215=115%:%
%:%216=116%:%
%:%217=117%:%
%:%218=118%:%
%:%219=119%:%
%:%220=120%:%
%:%221=121%:%
%:%222=122%:%
%:%223=123%:%
%:%224=124%:%
%:%225=125%:%
%:%226=126%:%
%:%227=127%:%
%:%228=128%:%
%:%230=130%:%
%:%231=130%:%
%:%232=131%:%
%:%235=132%:%
%:%239=132%:%
%:%240=132%:%
%:%241=133%:%
%:%242=133%:%
%:%243=134%:%
%:%244=134%:%
%:%245=135%:%
%:%246=135%:%
%:%247=136%:%
%:%248=136%:%
%:%249=137%:%
%:%259=139%:%
%:%260=140%:%
%:%261=141%:%
%:%262=142%:%
%:%263=143%:%
%:%264=144%:%
%:%265=145%:%
%:%266=146%:%
%:%268=148%:%
%:%269=148%:%
%:%270=149%:%
%:%271=150%:%
%:%274=151%:%
%:%278=151%:%
%:%279=151%:%
%:%280=152%:%
%:%281=152%:%
%:%290=154%:%
%:%292=156%:%
%:%293=156%:%
%:%294=157%:%
%:%295=158%:%
%:%302=159%:%
%:%303=159%:%
%:%304=160%:%
%:%305=160%:%
%:%306=161%:%
%:%307=161%:%
%:%308=162%:%
%:%309=162%:%
%:%310=163%:%
%:%311=163%:%
%:%312=164%:%
%:%313=164%:%
%:%314=164%:%
%:%315=164%:%
%:%316=165%:%
%:%317=165%:%
%:%318=165%:%
%:%319=165%:%
%:%320=166%:%
%:%321=166%:%
%:%322=166%:%
%:%323=166%:%
%:%324=166%:%
%:%325=167%:%
%:%326=167%:%
%:%327=168%:%
%:%328=168%:%
%:%329=169%:%
%:%330=169%:%
%:%331=170%:%
%:%332=170%:%
%:%333=171%:%
%:%334=171%:%
%:%335=172%:%
%:%336=172%:%
%:%337=173%:%
%:%338=173%:%
%:%339=173%:%
%:%340=174%:%
%:%341=174%:%
%:%342=174%:%
%:%343=174%:%
%:%344=174%:%
%:%345=175%:%
%:%346=175%:%
%:%347=175%:%
%:%348=175%:%
%:%349=176%:%
%:%350=176%:%
%:%351=176%:%
%:%352=176%:%
%:%353=177%:%
%:%354=177%:%
%:%355=178%:%
%:%365=180%:%
%:%367=182%:%
%:%368=182%:%
%:%369=183%:%
%:%370=184%:%
%:%377=185%:%
%:%378=185%:%
%:%379=186%:%
%:%380=186%:%
%:%381=187%:%
%:%382=187%:%
%:%383=187%:%
%:%384=187%:%
%:%385=187%:%
%:%386=188%:%
%:%387=188%:%
%:%388=189%:%
%:%389=189%:%
%:%390=190%:%
%:%391=190%:%
%:%392=190%:%
%:%393=190%:%
%:%394=191%:%
%:%404=193%:%
%:%406=195%:%
%:%407=195%:%
%:%408=196%:%
%:%411=197%:%
%:%415=197%:%
%:%416=197%:%