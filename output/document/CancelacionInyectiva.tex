%
\begin{isabellebody}%
\setisabellecontext{CancelacionInyectiva}%
%
\isadelimtheory
\isanewline
%
\endisadelimtheory
%
\isatagtheory
%
\endisatagtheory
{\isafoldtheory}%
%
\isadelimtheory
%
\endisadelimtheory
%
\isadelimdocument
%
\endisadelimdocument
%
\isatagdocument
%
\isamarkupsection{Demostración en lenguaje natural%
}
\isamarkuptrue%
%
\endisatagdocument
{\isafolddocument}%
%
\isadelimdocument
%
\endisadelimdocument
%
\begin{isamarkuptext}%
\comentario{Estructurar en secciones.}%
\end{isamarkuptext}\isamarkuptrue%
%
\begin{isamarkuptext}%
\comentario{Hacer demostraciones detalladas.}%
\end{isamarkuptext}\isamarkuptrue%
%
\begin{isamarkuptext}%
\comentario{Añadir lemas usados al Soporte.}%
\end{isamarkuptext}\isamarkuptrue%
%
\begin{isamarkuptext}%
El siguiente teorema que se va a probar es una caracterización de
 las funciones inyectivas. Primero se definirá el significado de
 inyectividad de una función y la propiedad de ser cancelativa por la
 izquierda. \\
 Una función $f : B \longrightarrow C$ es inyectiva si 
$$\forall x,y \in \ B : f(x) = f(y) \Longrightarrow x =
 y.$$
Una función $f : B \longrightarrow C$ es cancelativa por la izquierda si 
$$\forall A: (\forall g,h: X \longrightarrow Y) : f \circ g = f \circ h
 \Longrightarrow g = h.$$

Luego el teorema es el siguiente:

Luego el teorema es el siguiente:
  
  \begin{teorema}
  $f$ es una función inyectiva, si y solo si, para todas funciones 
 $g$ y $h$  tales que  $f \circ g = f \circ h$ se tiene que $g = h$. 
  \end{teorema}

Vamos a hacer dos lemas de nuestro teorema, ya que se  descompone la
doble implicación en dos implicaciones y se va a  demostrar cada una de
 ellas por  separado.

\begin{lema}[Condición necesaria]
 Si $f$ es una función inyectiva entonces para todas funciones $g$ y $h$
 tales que  $f \circ g = f \circ h$ se tiene que $g = h.$
\end {lema}
  \begin{demostracion}
Por hipótesis se tiene que $f \circ g = f \circ h$, hay que probar que
$g = h$. Usando que f es inyectiva tenemos que: \\
$$(f \circ g)(x) = (f \circ h)(x) \Longrightarrow f(g(x)) = f(h(x)) = 
g(x) = h(x)$$
  \end{demostracion}

\begin {lema}[Condición suficiente] 
Si para toda $g$ y $h$ tales que $f \circ g =  f \circ h$ se tiene que $g
= h$ entonces f es inyectiva.
\end {lema} 

\begin {demostracion}
Si el dominio de la función $f$ fuese vacío, f  es inyectiva.
Supongamos que el dominio de la función $f$ es distinto del vacío y que 
f verifica la propiedad de ser cancelativa por la izquierda.
Hay que demostrar que $\forall a,b$ tales que $f(a) = f(b),$ esto
 implica que $a = b.$ \\
Sean $a,b$ tales que $f(a) = f(b)$. \\
Definiendo  $g(x) = a  \ \forall x$  y $h(x) = b \  \forall x$ entonces 
$$(f \circ g) = (f \circ h) \Longrightarrow  f(g(x)) = f(h(x))
 \Longrightarrow f(a) = f(b)$$

Por hipótesis, entonces $a = b,$ como se quería demostrar.
\end {demostracion}%
\end{isamarkuptext}\isamarkuptrue%
%
\isadelimdocument
%
\endisadelimdocument
%
\isatagdocument
%
\isamarkupsection{Especificación en Isabelle/Hol%
}
\isamarkuptrue%
%
\endisatagdocument
{\isafolddocument}%
%
\isadelimdocument
%
\endisadelimdocument
%
\begin{isamarkuptext}%
Su especificación es la siguiente, pero al igual que se ha  hecho en
 la demostración a mano se va a demostrar a través de dos lemas:%
\end{isamarkuptext}\isamarkuptrue%
\isacommand{theorem}\isamarkupfalse%
\ caracterizacion{\isacharunderscore}funcion{\isacharunderscore}inyecctiva{\isacharcolon}\isanewline
\ \ {\isachardoublequoteopen}inj\ f\ {\isasymlongleftrightarrow}\ {\isacharparenleft}{\isasymforall}g\ h{\isachardot}\ {\isacharparenleft}f\ {\isasymcirc}\ g\ {\isacharequal}\ f\ {\isasymcirc}\ h{\isacharparenright}\ {\isasymlongrightarrow}\ {\isacharparenleft}g\ {\isacharequal}\ h{\isacharparenright}{\isacharparenright}{\isachardoublequoteclose}\isanewline
%
\isadelimproof
\ \ %
\endisadelimproof
%
\isatagproof
\isacommand{oops}\isamarkupfalse%
%
\endisatagproof
{\isafoldproof}%
%
\isadelimproof
%
\endisadelimproof
%
\begin{isamarkuptext}%
Sus lemas asociados a cada implicación son los siguientes:%
\end{isamarkuptext}\isamarkuptrue%
\isacommand{lemma}\isamarkupfalse%
\ \isanewline
{\isachardoublequoteopen}{\isasymforall}g\ h{\isachardot}\ {\isacharparenleft}f\ {\isasymcirc}\ g\ {\isacharequal}\ f\ {\isasymcirc}\ h\ {\isasymlongrightarrow}\ g\ {\isacharequal}\ h{\isacharparenright}\ {\isasymLongrightarrow}\ inj\ f{\isachardoublequoteclose}\isanewline
%
\isadelimproof
\ \ %
\endisadelimproof
%
\isatagproof
\isacommand{oops}\isamarkupfalse%
%
\endisatagproof
{\isafoldproof}%
%
\isadelimproof
\isanewline
%
\endisadelimproof
\isanewline
\isacommand{lemma}\isamarkupfalse%
\ \isanewline
{\isachardoublequoteopen}inj\ f\ {\isasymLongrightarrow}\ {\isacharparenleft}{\isasymforall}g\ h{\isachardot}{\isacharparenleft}f\ {\isasymcirc}\ g\ {\isacharequal}\ f\ {\isasymcirc}\ h{\isacharparenright}\ {\isasymlongrightarrow}\ {\isacharparenleft}g\ {\isacharequal}\ h{\isacharparenright}{\isacharparenright}{\isachardoublequoteclose}\isanewline
%
\isadelimproof
\ \ %
\endisadelimproof
%
\isatagproof
\isacommand{oops}\isamarkupfalse%
%
\endisatagproof
{\isafoldproof}%
%
\isadelimproof
%
\endisadelimproof
%
\begin{isamarkuptext}%
En la especificación anterior, \isa{inj\ f} es una 
  abreviatura de \isa{inj\ f} definida en la teoría
  \href{http://bit.ly/2XuPQx5}{Fun.thy}. Además, contiene la definición
  de \isa{inj{\isacharunderscore}on}
  \begin{itemize}
    \item[] \isa{inj{\isacharunderscore}on\ f\ A\ {\isacharequal}\ {\isacharparenleft}{\isasymforall}x{\isasymin}A{\isachardot}\ {\isasymforall}y{\isasymin}A{\isachardot}\ f\ x\ {\isacharequal}\ f\ y\ {\isasymlongrightarrow}\ x\ {\isacharequal}\ y{\isacharparenright}} \hfill (\isa{inj{\isacharunderscore}on{\isacharunderscore}def})
  \end{itemize} 
  Por su parte, \isa{UNIV} es el conjunto universal definido en la 
  teoría \href{http://bit.ly/2XtHCW6}{Set.thy} como una abreviatura de 
  \isa{top} que, a su vez está definido en la teoría 
  \href{http://bit.ly/2Xyj9Pe}{Orderings.thy} mediante la siguiente
  propiedad 
  \begin{itemize}
    \item[] \isa{\mbox{}\inferrule{\mbox{ordering{\isacharunderscore}top\ less{\isacharunderscore}eq\ less\ top}}{\mbox{less{\isacharunderscore}eq\ a\ top}}} 
      \hfill (\isa{ordering{\isacharunderscore}top{\isachardot}extremum})
  \end{itemize} 
  En el caso de la teoría de conjuntos, la relación de orden es la
  inclusión de conjuntos.

  Presentaremos distintas demostraciones de los lemas.%
\end{isamarkuptext}\isamarkuptrue%
%
\isadelimdocument
%
\endisadelimdocument
%
\isatagdocument
%
\isamarkupsection{Demostración aplicativa lemas%
}
\isamarkuptrue%
%
\endisatagdocument
{\isafolddocument}%
%
\isadelimdocument
%
\endisadelimdocument
%
\begin{isamarkuptext}%
Las demostraciones aplicativas de los lemas son  :%
\end{isamarkuptext}\isamarkuptrue%
\isacommand{lemma}\isamarkupfalse%
\ condicion{\isacharunderscore}necesaria{\isacharunderscore}aplicativa{\isacharcolon}\isanewline
\ \ {\isachardoublequoteopen}inj\ f\ {\isasymLongrightarrow}\ {\isacharparenleft}{\isasymforall}g\ h{\isachardot}{\isacharparenleft}f\ {\isasymcirc}\ g\ {\isacharequal}\ f\ {\isasymcirc}\ h{\isacharparenright}\ {\isasymlongrightarrow}\ \ {\isacharparenleft}g\ {\isacharequal}\ h{\isacharparenright}{\isacharparenright}{\isachardoublequoteclose}\isanewline
%
\isadelimproof
\ \ %
\endisadelimproof
%
\isatagproof
\isacommand{apply}\isamarkupfalse%
\ {\isacharparenleft}simp\ add{\isacharcolon}\ inj{\isacharunderscore}on{\isacharunderscore}def\ fun{\isacharunderscore}eq{\isacharunderscore}iff{\isacharparenright}\ \isanewline
\ \ \isacommand{done}\isamarkupfalse%
%
\endisatagproof
{\isafoldproof}%
%
\isadelimproof
\ \isanewline
%
\endisadelimproof
\isanewline
\isacommand{lemma}\isamarkupfalse%
\ condicion{\isacharunderscore}suficiente{\isacharunderscore}aplicativa{\isacharcolon}\isanewline
{\isachardoublequoteopen}{\isasymforall}g\ h{\isachardot}\ {\isacharparenleft}f\ {\isasymcirc}\ g\ {\isacharequal}\ f\ {\isasymcirc}\ h\ {\isasymlongrightarrow}\ g\ {\isacharequal}\ h{\isacharparenright}\ {\isasymLongrightarrow}\ inj\ f{\isachardoublequoteclose}\isanewline
%
\isadelimproof
\ \ %
\endisadelimproof
%
\isatagproof
\isacommand{apply}\isamarkupfalse%
\ {\isacharparenleft}rule\ injI{\isacharparenright}\isanewline
\ \ \isacommand{by}\isamarkupfalse%
\ {\isacharparenleft}metis\ fun{\isacharunderscore}upd{\isacharunderscore}apply\ fun{\isacharunderscore}upd{\isacharunderscore}comp{\isacharparenright}%
\endisatagproof
{\isafoldproof}%
%
\isadelimproof
%
\endisadelimproof
%
\begin{isamarkuptext}%
En las demostraciones anteriores se han usado los siguientes
 lemas:
  \begin{itemize}
    \item[] \isa{{\isacharparenleft}f\ {\isacharequal}\ g{\isacharparenright}\ {\isacharequal}\ {\isacharparenleft}{\isasymforall}x{\isachardot}\ f\ x\ {\isacharequal}\ g\ x{\isacharparenright}} 
      \hfill (\isa{fun{\isacharunderscore}eq{\isacharunderscore}iff})
  \end{itemize} 
  \begin{itemize}
    \item[] \isa{{\isacharparenleft}f{\isacharparenleft}x\ {\isacharcolon}{\isacharequal}\ y{\isacharparenright}{\isacharparenright}\ z\ {\isacharequal}\ {\isacharparenleft}\textsf{if}\ z\ {\isacharequal}\ x\ \textsf{then}\ y\ \textsf{else}\ f\ z{\isacharparenright}} 
      \hfill (\isa{fun{\isacharunderscore}upd{\isacharunderscore}apply})
  \end{itemize} 
  \begin{itemize}
    \item[] \isa{{\isacharparenleft}f\ {\isacharequal}\ g{\isacharparenright}\ {\isacharequal}\ {\isacharparenleft}{\isasymforall}x{\isachardot}\ f\ x\ {\isacharequal}\ g\ x{\isacharparenright}} 
      \hfill (\isa{fun{\isacharunderscore}upd{\isacharunderscore}comp})
  \end{itemize}%
\end{isamarkuptext}\isamarkuptrue%
%
\isadelimdocument
%
\endisadelimdocument
%
\isatagdocument
%
\isamarkupsection{Demostración estructurada lemas%
}
\isamarkuptrue%
%
\endisatagdocument
{\isafolddocument}%
%
\isadelimdocument
%
\endisadelimdocument
%
\begin{isamarkuptext}%
Las demostraciones declarativas son las siguientes:%
\end{isamarkuptext}\isamarkuptrue%
\isacommand{lemma}\isamarkupfalse%
\ condicion{\isacharunderscore}necesaria{\isacharunderscore}detallada{\isacharcolon}\isanewline
\ \ \isakeyword{assumes}\ {\isachardoublequoteopen}inj\ f{\isachardoublequoteclose}\isanewline
\ \ \isakeyword{shows}\ {\isachardoublequoteopen}{\isasymforall}g\ h{\isachardot}{\isacharparenleft}f\ {\isasymcirc}\ g\ {\isacharequal}\ f\ {\isasymcirc}\ h{\isacharparenright}\ {\isasymlongrightarrow}\ {\isacharparenleft}g\ {\isacharequal}\ h{\isacharparenright}{\isachardoublequoteclose}\isanewline
%
\isadelimproof
%
\endisadelimproof
%
\isatagproof
\isacommand{proof}\isamarkupfalse%
\isanewline
\ \ \isacommand{fix}\isamarkupfalse%
\ g{\isacharcolon}{\isacharcolon}\ {\isachardoublequoteopen}{\isacharprime}c\ {\isasymRightarrow}\ {\isacharprime}a{\isachardoublequoteclose}\isanewline
\ \ \isacommand{show}\isamarkupfalse%
\ {\isachardoublequoteopen}{\isasymforall}h{\isachardot}{\isacharparenleft}f\ {\isasymcirc}\ g\ {\isacharequal}\ f\ {\isasymcirc}\ h{\isacharparenright}\ {\isasymlongrightarrow}\ {\isacharparenleft}g\ {\isacharequal}\ h{\isacharparenright}{\isachardoublequoteclose}\isanewline
\ \ \isacommand{proof}\isamarkupfalse%
\ {\isacharparenleft}rule\ allI{\isacharparenright}\isanewline
\ \ \ \ \isacommand{fix}\isamarkupfalse%
\ h\isanewline
\ \ \ \ \isacommand{show}\isamarkupfalse%
\ {\isachardoublequoteopen}f\ {\isasymcirc}\ g\ {\isacharequal}\ f\ {\isasymcirc}\ h\ {\isasymlongrightarrow}\ {\isacharparenleft}g\ {\isacharequal}\ h{\isacharparenright}{\isachardoublequoteclose}\isanewline
\ \ \ \ \isacommand{proof}\isamarkupfalse%
\ {\isacharparenleft}rule\ impI{\isacharparenright}\isanewline
\ \ \ \ \ \ \isacommand{assume}\isamarkupfalse%
\ {\isachardoublequoteopen}f\ {\isasymcirc}\ g\ {\isacharequal}\ f\ {\isasymcirc}\ h{\isachardoublequoteclose}\isanewline
\ \ \ \ \ \ \isacommand{show}\isamarkupfalse%
\ {\isachardoublequoteopen}g\ {\isacharequal}\ h{\isachardoublequoteclose}\isanewline
\ \ \ \ \ \ \isacommand{proof}\isamarkupfalse%
\ \isanewline
\ \ \ \ \ \ \ \ \isacommand{fix}\isamarkupfalse%
\ x\isanewline
\ \ \ \ \ \ \ \ \isacommand{have}\isamarkupfalse%
\ \ {\isachardoublequoteopen}{\isacharparenleft}f\ {\isasymcirc}\ g{\isacharparenright}{\isacharparenleft}x{\isacharparenright}\ {\isacharequal}\ {\isacharparenleft}f\ {\isasymcirc}\ h{\isacharparenright}{\isacharparenleft}x{\isacharparenright}{\isachardoublequoteclose}\ \isacommand{using}\isamarkupfalse%
\ {\isacharbackquoteopen}f\ {\isasymcirc}\ g\ {\isacharequal}\ f\ {\isasymcirc}\ h{\isacharbackquoteclose}\ \isacommand{by}\isamarkupfalse%
\ simp\isanewline
\ \ \ \ \ \ \ \ \isacommand{then}\isamarkupfalse%
\ \isacommand{have}\isamarkupfalse%
\ {\isachardoublequoteopen}f{\isacharparenleft}g{\isacharparenleft}x{\isacharparenright}{\isacharparenright}\ {\isacharequal}\ f{\isacharparenleft}h{\isacharparenleft}x{\isacharparenright}{\isacharparenright}{\isachardoublequoteclose}\ \isacommand{by}\isamarkupfalse%
\ simp\isanewline
\ \ \ \ \ \ \ \ \isacommand{thus}\isamarkupfalse%
\ \ {\isachardoublequoteopen}g{\isacharparenleft}x{\isacharparenright}\ {\isacharequal}\ h{\isacharparenleft}x{\isacharparenright}{\isachardoublequoteclose}\ \isacommand{using}\isamarkupfalse%
\ {\isacharbackquoteopen}inj\ f{\isacharbackquoteclose}\ \isacommand{by}\isamarkupfalse%
\ {\isacharparenleft}simp\ add{\isacharcolon}inj{\isacharunderscore}on{\isacharunderscore}def{\isacharparenright}\isanewline
\ \ \ \ \ \ \isacommand{qed}\isamarkupfalse%
\isanewline
\ \ \ \ \isacommand{qed}\isamarkupfalse%
\isanewline
\ \ \isacommand{qed}\isamarkupfalse%
\isanewline
\isacommand{qed}\isamarkupfalse%
\isanewline
%
\endisatagproof
{\isafoldproof}%
%
\isadelimproof
%
\endisadelimproof
\isanewline
\isacommand{lemma}\isamarkupfalse%
\ condicion{\isacharunderscore}suficiente{\isacharunderscore}detallada{\isacharcolon}\isanewline
\ \ \isakeyword{fixes}\ f\ {\isacharcolon}{\isacharcolon}\ {\isachardoublequoteopen}{\isacharprime}b\ {\isasymRightarrow}\ {\isacharprime}c{\isachardoublequoteclose}\ \isanewline
\ \ \isakeyword{assumes}\ {\isachardoublequoteopen}{\isasymforall}{\isacharparenleft}g\ {\isacharcolon}{\isacharcolon}\ {\isacharprime}a\ {\isasymRightarrow}\ {\isacharprime}b{\isacharparenright}\ {\isacharparenleft}h\ {\isacharcolon}{\isacharcolon}\ {\isacharprime}a\ {\isasymRightarrow}\ {\isacharprime}b{\isacharparenright}{\isachardot}\isanewline
\ \ \ \ \ \ \ \ \ {\isacharparenleft}f\ {\isasymcirc}\ g\ {\isacharequal}\ f\ {\isasymcirc}\ h\ {\isasymlongrightarrow}\ g\ {\isacharequal}\ h{\isacharparenright}{\isachardoublequoteclose}\isanewline
\isakeyword{shows}\ {\isachardoublequoteopen}\ inj\ f{\isachardoublequoteclose}\isanewline
%
\isadelimproof
%
\endisadelimproof
%
\isatagproof
\isacommand{proof}\isamarkupfalse%
\ {\isacharparenleft}rule\ injI{\isacharparenright}\isanewline
\ \ \isacommand{fix}\isamarkupfalse%
\ a\ b\ \isanewline
\ \ \isacommand{assume}\isamarkupfalse%
\ {\isadigit{3}}{\isacharcolon}\ {\isachardoublequoteopen}f\ a\ {\isacharequal}\ f\ b\ {\isachardoublequoteclose}\isanewline
\ \ \isacommand{let}\isamarkupfalse%
\ {\isacharquery}g\ {\isacharequal}\ {\isachardoublequoteopen}{\isasymlambda}x\ {\isacharcolon}{\isacharcolon}\ {\isacharprime}a{\isachardot}\ a{\isachardoublequoteclose}\isanewline
\ \ \isacommand{let}\isamarkupfalse%
\ {\isacharquery}h\ {\isacharequal}\ {\isachardoublequoteopen}{\isasymlambda}x\ {\isacharcolon}{\isacharcolon}\ {\isacharprime}a{\isachardot}\ b{\isachardoublequoteclose}\isanewline
\ \ \isacommand{have}\isamarkupfalse%
\ {\isachardoublequoteopen}{\isasymforall}{\isacharparenleft}h\ {\isacharcolon}{\isacharcolon}\ {\isacharprime}a\ {\isasymRightarrow}\ {\isacharprime}b{\isacharparenright}{\isachardot}\ {\isacharparenleft}f\ {\isasymcirc}\ {\isacharquery}g\ {\isacharequal}\ f\ {\isasymcirc}\ h\ {\isasymlongrightarrow}\ {\isacharquery}g\ {\isacharequal}\ h{\isacharparenright}{\isachardoublequoteclose}\isanewline
\ \ \ \ \isacommand{using}\isamarkupfalse%
\ assms\ \isacommand{by}\isamarkupfalse%
\ {\isacharparenleft}rule\ allE{\isacharparenright}\isanewline
\ \ \isacommand{hence}\isamarkupfalse%
\ {\isadigit{1}}{\isacharcolon}\ {\isachardoublequoteopen}\ {\isacharparenleft}f\ {\isasymcirc}\ {\isacharquery}g\ {\isacharequal}\ f\ {\isasymcirc}\ {\isacharquery}h\ {\isasymlongrightarrow}\ {\isacharquery}g\ {\isacharequal}\ {\isacharquery}h{\isacharparenright}{\isachardoublequoteclose}\ \ \isacommand{by}\isamarkupfalse%
\ {\isacharparenleft}rule\ allE{\isacharparenright}\ \isanewline
\ \ \isacommand{have}\isamarkupfalse%
\ {\isadigit{2}}{\isacharcolon}\ {\isachardoublequoteopen}f\ {\isasymcirc}\ {\isacharquery}g\ {\isacharequal}\ f\ {\isasymcirc}\ {\isacharquery}h{\isachardoublequoteclose}\ \isanewline
\ \ \isacommand{proof}\isamarkupfalse%
\ \isanewline
\ \ \ \ \isacommand{fix}\isamarkupfalse%
\ x\isanewline
\ \ \ \ \isacommand{have}\isamarkupfalse%
\ {\isachardoublequoteopen}\ {\isacharparenleft}f\ {\isasymcirc}\ {\isacharparenleft}{\isasymlambda}x\ {\isacharcolon}{\isacharcolon}\ {\isacharprime}a{\isachardot}\ a{\isacharparenright}{\isacharparenright}\ x\ {\isacharequal}\ f{\isacharparenleft}a{\isacharparenright}\ {\isachardoublequoteclose}\ \isacommand{by}\isamarkupfalse%
\ simp\isanewline
\ \ \ \ \isacommand{also}\isamarkupfalse%
\ \isacommand{have}\isamarkupfalse%
\ {\isachardoublequoteopen}{\isachardot}{\isachardot}{\isachardot}\ {\isacharequal}\ f{\isacharparenleft}b{\isacharparenright}{\isachardoublequoteclose}\ \isacommand{using}\isamarkupfalse%
\ {\isadigit{3}}\ \isacommand{by}\isamarkupfalse%
\ simp\isanewline
\ \ \ \ \isacommand{also}\isamarkupfalse%
\ \isacommand{have}\isamarkupfalse%
\ {\isachardoublequoteopen}{\isachardot}{\isachardot}{\isachardot}\ {\isacharequal}\ \ {\isacharparenleft}f\ {\isasymcirc}\ {\isacharparenleft}{\isasymlambda}x\ {\isacharcolon}{\isacharcolon}\ {\isacharprime}a{\isachardot}\ b{\isacharparenright}{\isacharparenright}\ x{\isachardoublequoteclose}\ \isacommand{by}\isamarkupfalse%
\ simp\isanewline
\ \ \ \ \isacommand{finally}\isamarkupfalse%
\ \isacommand{show}\isamarkupfalse%
\ {\isachardoublequoteopen}\ {\isacharparenleft}f\ {\isasymcirc}\ {\isacharparenleft}{\isasymlambda}x\ {\isacharcolon}{\isacharcolon}\ {\isacharprime}a{\isachardot}\ a{\isacharparenright}{\isacharparenright}\ x\ {\isacharequal}\ \ {\isacharparenleft}f\ {\isasymcirc}\ {\isacharparenleft}{\isasymlambda}x\ {\isacharcolon}{\isacharcolon}\ {\isacharprime}a{\isachardot}\ b{\isacharparenright}{\isacharparenright}\ x{\isachardoublequoteclose}\isanewline
\ \ \ \ \ \ \isacommand{by}\isamarkupfalse%
\ simp\isanewline
\ \ \isacommand{qed}\isamarkupfalse%
\isanewline
\ \ \isacommand{have}\isamarkupfalse%
\ {\isachardoublequoteopen}{\isacharquery}g\ {\isacharequal}\ {\isacharquery}h{\isachardoublequoteclose}\ \isacommand{using}\isamarkupfalse%
\ {\isadigit{1}}\ {\isadigit{2}}\ \isacommand{by}\isamarkupfalse%
\ {\isacharparenleft}rule\ mp{\isacharparenright}\isanewline
\ \ \isacommand{then}\isamarkupfalse%
\ \isacommand{show}\isamarkupfalse%
\ {\isachardoublequoteopen}\ a\ {\isacharequal}\ b{\isachardoublequoteclose}\ \isacommand{by}\isamarkupfalse%
\ {\isacharparenleft}rule\ fun{\isacharunderscore}cong{\isacharparenright}\isanewline
\isacommand{qed}\isamarkupfalse%
%
\endisatagproof
{\isafoldproof}%
%
\isadelimproof
%
\endisadelimproof
%
\begin{isamarkuptext}%
En la anterior demostración se ha introducito la regla: 
  \begin{itemize}
    \item[] \isa{\mbox{}\inferrule{\mbox{{\isacharparenleft}f\ {\isacharcolon}{\isacharcolon}\ {\isacharprime}a\ {\isasymRightarrow}\ {\isacharprime}b{\isacharparenright}\ {\isacharequal}\ {\isacharparenleft}g\ {\isacharcolon}{\isacharcolon}\ {\isacharprime}a\ {\isasymRightarrow}\ {\isacharprime}b{\isacharparenright}}}{\mbox{f\ {\isacharparenleft}x\ {\isacharcolon}{\isacharcolon}\ {\isacharprime}a{\isacharparenright}\ {\isacharequal}\ g\ x}}} 
      \hfill (\isa{fun{\isacharunderscore}cong})
  \end{itemize}
Otras demostraciones declarativas usando auto y blast son:%
\end{isamarkuptext}\isamarkuptrue%
\isacommand{lemma}\isamarkupfalse%
\ condicion{\isacharunderscore}necesaria{\isacharunderscore}detallada{\isadigit{1}}{\isacharcolon}\isanewline
\ \ \isakeyword{assumes}\ {\isachardoublequoteopen}inj\ f{\isachardoublequoteclose}\isanewline
\ \ \isakeyword{shows}\ {\isachardoublequoteopen}{\isacharparenleft}f\ {\isasymcirc}\ g\ {\isacharequal}\ f\ {\isasymcirc}\ h{\isacharparenright}\ {\isasymlongrightarrow}{\isacharparenleft}g\ {\isacharequal}\ h{\isacharparenright}{\isachardoublequoteclose}\isanewline
%
\isadelimproof
%
\endisadelimproof
%
\isatagproof
\isacommand{proof}\isamarkupfalse%
\ \isanewline
\ \ \isacommand{assume}\isamarkupfalse%
\ {\isachardoublequoteopen}f\ {\isasymcirc}\ g\ {\isacharequal}\ f\ {\isasymcirc}\ h{\isachardoublequoteclose}\ \isanewline
\ \ \isacommand{then}\isamarkupfalse%
\ \isacommand{show}\isamarkupfalse%
\ {\isachardoublequoteopen}g\ {\isacharequal}\ h{\isachardoublequoteclose}\ \isacommand{using}\isamarkupfalse%
\ {\isacharbackquoteopen}inj\ f{\isacharbackquoteclose}\ \isacommand{by}\isamarkupfalse%
\ {\isacharparenleft}simp\ add{\isacharcolon}\ inj{\isacharunderscore}on{\isacharunderscore}def\ fun{\isacharunderscore}eq{\isacharunderscore}iff{\isacharparenright}\ \isanewline
\isacommand{qed}\isamarkupfalse%
%
\endisatagproof
{\isafoldproof}%
%
\isadelimproof
\isanewline
%
\endisadelimproof
\isanewline
\isacommand{lemma}\isamarkupfalse%
\ condicion{\isacharunderscore}suficiente{\isacharunderscore}detallada{\isadigit{1}}{\isacharcolon}\isanewline
\ \ \isakeyword{fixes}\ f\ {\isacharcolon}{\isacharcolon}\ {\isachardoublequoteopen}{\isacharprime}b\ {\isasymRightarrow}\ {\isacharprime}c{\isachardoublequoteclose}\ \isanewline
\ \ \isakeyword{assumes}\ {\isachardoublequoteopen}{\isasymforall}{\isacharparenleft}g\ {\isacharcolon}{\isacharcolon}\ {\isacharprime}a\ {\isasymRightarrow}\ {\isacharprime}b{\isacharparenright}\ {\isacharparenleft}h\ {\isacharcolon}{\isacharcolon}\ {\isacharprime}a\ {\isasymRightarrow}\ {\isacharprime}b{\isacharparenright}{\isachardot}\isanewline
\ \ \ \ \ \ \ \ \ {\isacharparenleft}f\ {\isasymcirc}\ g\ {\isacharequal}\ f\ {\isasymcirc}\ h\ {\isasymlongrightarrow}\ g\ {\isacharequal}\ h{\isacharparenright}{\isachardoublequoteclose}\isanewline
\ \ \isakeyword{shows}\ {\isachardoublequoteopen}\ inj\ f{\isachardoublequoteclose}\isanewline
%
\isadelimproof
%
\endisadelimproof
%
\isatagproof
\isacommand{proof}\isamarkupfalse%
\ {\isacharparenleft}rule\ injI{\isacharparenright}\isanewline
\ \ \isacommand{fix}\isamarkupfalse%
\ a\ b\ \isanewline
\ \ \isacommand{assume}\isamarkupfalse%
\ {\isadigit{1}}{\isacharcolon}\ {\isachardoublequoteopen}f\ a\ {\isacharequal}\ f\ b\ {\isachardoublequoteclose}\isanewline
\ \ \isacommand{let}\isamarkupfalse%
\ {\isacharquery}g\ {\isacharequal}\ {\isachardoublequoteopen}{\isasymlambda}x\ {\isacharcolon}{\isacharcolon}\ {\isacharprime}a{\isachardot}\ a{\isachardoublequoteclose}\isanewline
\ \ \isacommand{let}\isamarkupfalse%
\ {\isacharquery}h\ {\isacharequal}\ {\isachardoublequoteopen}{\isasymlambda}x\ {\isacharcolon}{\isacharcolon}\ {\isacharprime}a{\isachardot}\ b{\isachardoublequoteclose}\isanewline
\ \ \isacommand{have}\isamarkupfalse%
\ {\isadigit{2}}{\isacharcolon}\ {\isachardoublequoteopen}\ {\isacharparenleft}f\ {\isasymcirc}\ {\isacharquery}g\ {\isacharequal}\ f\ {\isasymcirc}\ {\isacharquery}h\ {\isasymlongrightarrow}\ {\isacharquery}g\ {\isacharequal}\ {\isacharquery}h{\isacharparenright}{\isachardoublequoteclose}\ \ \isacommand{using}\isamarkupfalse%
\ assms\ \isacommand{by}\isamarkupfalse%
\ blast\isanewline
\ \ \isacommand{have}\isamarkupfalse%
\ {\isadigit{3}}{\isacharcolon}\ {\isachardoublequoteopen}f\ {\isasymcirc}\ {\isacharquery}g\ {\isacharequal}\ f\ {\isasymcirc}\ {\isacharquery}h{\isachardoublequoteclose}\ \isanewline
\ \ \isacommand{proof}\isamarkupfalse%
\ \isanewline
\ \ \ \ \isacommand{fix}\isamarkupfalse%
\ x\isanewline
\ \ \ \ \isacommand{have}\isamarkupfalse%
\ {\isachardoublequoteopen}\ {\isacharparenleft}f\ {\isasymcirc}\ {\isacharparenleft}{\isasymlambda}x\ {\isacharcolon}{\isacharcolon}\ {\isacharprime}a{\isachardot}\ a{\isacharparenright}{\isacharparenright}\ x\ {\isacharequal}\ f{\isacharparenleft}a{\isacharparenright}\ {\isachardoublequoteclose}\ \isacommand{by}\isamarkupfalse%
\ simp\isanewline
\ \ \ \ \isacommand{also}\isamarkupfalse%
\ \isacommand{have}\isamarkupfalse%
\ {\isachardoublequoteopen}{\isachardot}{\isachardot}{\isachardot}\ {\isacharequal}\ f{\isacharparenleft}b{\isacharparenright}{\isachardoublequoteclose}\ \isacommand{using}\isamarkupfalse%
\ {\isadigit{1}}\ \isacommand{by}\isamarkupfalse%
\ simp\isanewline
\ \ \ \ \isacommand{also}\isamarkupfalse%
\ \isacommand{have}\isamarkupfalse%
\ {\isachardoublequoteopen}{\isachardot}{\isachardot}{\isachardot}\ {\isacharequal}\ \ {\isacharparenleft}f\ {\isasymcirc}\ {\isacharparenleft}{\isasymlambda}x\ {\isacharcolon}{\isacharcolon}\ {\isacharprime}a{\isachardot}\ b{\isacharparenright}{\isacharparenright}\ x{\isachardoublequoteclose}\ \isacommand{by}\isamarkupfalse%
\ simp\isanewline
\ \ \ \ \isacommand{finally}\isamarkupfalse%
\ \isacommand{show}\isamarkupfalse%
\ {\isachardoublequoteopen}\ {\isacharparenleft}f\ {\isasymcirc}\ {\isacharparenleft}{\isasymlambda}x\ {\isacharcolon}{\isacharcolon}\ {\isacharprime}a{\isachardot}\ a{\isacharparenright}{\isacharparenright}\ x\ {\isacharequal}\ \ {\isacharparenleft}f\ {\isasymcirc}\ {\isacharparenleft}{\isasymlambda}x\ {\isacharcolon}{\isacharcolon}\ {\isacharprime}a{\isachardot}\ b{\isacharparenright}{\isacharparenright}\ x{\isachardoublequoteclose}\isanewline
\ \ \ \ \ \ \isacommand{by}\isamarkupfalse%
\ simp\isanewline
\ \ \isacommand{qed}\isamarkupfalse%
\isanewline
\ \ \isacommand{show}\isamarkupfalse%
\ \ {\isachardoublequoteopen}\ a\ {\isacharequal}\ b{\isachardoublequoteclose}\ \isacommand{using}\isamarkupfalse%
\ {\isadigit{2}}\ {\isadigit{3}}\ \isacommand{by}\isamarkupfalse%
\ meson\isanewline
\isacommand{qed}\isamarkupfalse%
%
\endisatagproof
{\isafoldproof}%
%
\isadelimproof
%
\endisadelimproof
%
\isadelimdocument
%
\endisadelimdocument
%
\isatagdocument
%
\isamarkupsection{Demostración teorema en Isabelle/Hol%
}
\isamarkuptrue%
%
\endisatagdocument
{\isafolddocument}%
%
\isadelimdocument
%
\endisadelimdocument
%
\begin{isamarkuptext}%
En consecuencia, la demostración de nuestro teorema:%
\end{isamarkuptext}\isamarkuptrue%
\isacommand{theorem}\isamarkupfalse%
\ caracterizacion{\isacharunderscore}inyectividad{\isacharcolon}\isanewline
\ \ {\isachardoublequoteopen}inj\ f\ {\isasymlongleftrightarrow}\ {\isacharparenleft}{\isasymforall}g\ h{\isachardot}\ {\isacharparenleft}f\ {\isasymcirc}\ g\ {\isacharequal}\ f\ {\isasymcirc}\ h{\isacharparenright}\ {\isasymlongrightarrow}\ {\isacharparenleft}g\ {\isacharequal}\ h{\isacharparenright}{\isacharparenright}{\isachardoublequoteclose}\isanewline
%
\isadelimproof
\ \ %
\endisadelimproof
%
\isatagproof
\isacommand{using}\isamarkupfalse%
\ condicion{\isacharunderscore}necesaria{\isacharunderscore}detallada\ condicion{\isacharunderscore}suficiente{\isacharunderscore}detallada\isanewline
\ \ \isacommand{by}\isamarkupfalse%
\ auto\isanewline
\isanewline
\isanewline
\isanewline
\isanewline
%
\endisatagproof
{\isafoldproof}%
%
\isadelimproof
%
\endisadelimproof
%
\isadelimtheory
%
\endisadelimtheory
%
\isatagtheory
%
\endisatagtheory
{\isafoldtheory}%
%
\isadelimtheory
%
\endisadelimtheory
%
\end{isabellebody}%
\endinput
%:%file=~/Escritorio/TFG/CancelacionInyectiva.thy%:%
%:%6=1%:%
%:%25=10%:%
%:%37=13%:%
%:%41=15%:%
%:%45=17%:%
%:%49=19%:%
%:%50=20%:%
%:%51=21%:%
%:%52=22%:%
%:%53=23%:%
%:%54=24%:%
%:%55=25%:%
%:%56=26%:%
%:%57=27%:%
%:%58=28%:%
%:%59=29%:%
%:%60=30%:%
%:%61=31%:%
%:%62=32%:%
%:%63=33%:%
%:%64=34%:%
%:%65=35%:%
%:%66=36%:%
%:%67=37%:%
%:%68=38%:%
%:%69=39%:%
%:%70=40%:%
%:%71=41%:%
%:%72=42%:%
%:%73=43%:%
%:%74=44%:%
%:%75=45%:%
%:%76=46%:%
%:%77=47%:%
%:%78=48%:%
%:%79=49%:%
%:%80=50%:%
%:%81=51%:%
%:%82=52%:%
%:%83=53%:%
%:%84=54%:%
%:%85=55%:%
%:%86=56%:%
%:%87=57%:%
%:%88=58%:%
%:%89=59%:%
%:%90=60%:%
%:%91=61%:%
%:%92=62%:%
%:%93=63%:%
%:%94=64%:%
%:%95=65%:%
%:%96=66%:%
%:%97=67%:%
%:%98=68%:%
%:%99=69%:%
%:%100=70%:%
%:%101=71%:%
%:%110=73%:%
%:%122=76%:%
%:%123=77%:%
%:%125=80%:%
%:%126=80%:%
%:%127=81%:%
%:%130=82%:%
%:%134=82%:%
%:%144=86%:%
%:%146=88%:%
%:%147=88%:%
%:%148=89%:%
%:%151=90%:%
%:%155=90%:%
%:%161=90%:%
%:%164=91%:%
%:%165=92%:%
%:%166=92%:%
%:%167=93%:%
%:%170=94%:%
%:%174=94%:%
%:%184=96%:%
%:%185=97%:%
%:%186=98%:%
%:%187=99%:%
%:%188=100%:%
%:%189=101%:%
%:%190=102%:%
%:%191=103%:%
%:%192=104%:%
%:%193=105%:%
%:%194=106%:%
%:%195=107%:%
%:%196=108%:%
%:%197=109%:%
%:%198=110%:%
%:%199=111%:%
%:%200=112%:%
%:%201=113%:%
%:%202=114%:%
%:%203=115%:%
%:%212=117%:%
%:%224=119%:%
%:%226=121%:%
%:%227=121%:%
%:%228=122%:%
%:%231=123%:%
%:%235=123%:%
%:%236=123%:%
%:%237=124%:%
%:%243=124%:%
%:%246=125%:%
%:%247=126%:%
%:%248=126%:%
%:%249=127%:%
%:%252=128%:%
%:%256=128%:%
%:%257=128%:%
%:%258=129%:%
%:%259=129%:%
%:%268=132%:%
%:%269=133%:%
%:%270=134%:%
%:%271=135%:%
%:%272=136%:%
%:%273=137%:%
%:%274=138%:%
%:%275=139%:%
%:%276=140%:%
%:%277=141%:%
%:%278=142%:%
%:%279=143%:%
%:%280=144%:%
%:%281=145%:%
%:%290=147%:%
%:%302=149%:%
%:%304=153%:%
%:%305=153%:%
%:%306=154%:%
%:%307=155%:%
%:%314=156%:%
%:%315=156%:%
%:%316=157%:%
%:%317=157%:%
%:%318=158%:%
%:%319=158%:%
%:%320=159%:%
%:%321=159%:%
%:%322=160%:%
%:%323=160%:%
%:%324=161%:%
%:%325=161%:%
%:%326=162%:%
%:%327=162%:%
%:%328=163%:%
%:%329=163%:%
%:%330=164%:%
%:%331=164%:%
%:%332=165%:%
%:%333=165%:%
%:%334=166%:%
%:%335=166%:%
%:%336=167%:%
%:%337=167%:%
%:%338=167%:%
%:%339=167%:%
%:%340=168%:%
%:%341=168%:%
%:%342=168%:%
%:%343=168%:%
%:%344=169%:%
%:%345=169%:%
%:%346=169%:%
%:%347=169%:%
%:%348=170%:%
%:%349=170%:%
%:%350=171%:%
%:%351=171%:%
%:%352=172%:%
%:%353=172%:%
%:%354=173%:%
%:%355=173%:%
%:%363=176%:%
%:%364=177%:%
%:%365=177%:%
%:%366=178%:%
%:%367=179%:%
%:%368=180%:%
%:%369=181%:%
%:%376=182%:%
%:%377=182%:%
%:%378=183%:%
%:%379=183%:%
%:%380=184%:%
%:%381=184%:%
%:%382=185%:%
%:%383=185%:%
%:%384=186%:%
%:%385=186%:%
%:%386=187%:%
%:%387=187%:%
%:%388=188%:%
%:%389=188%:%
%:%390=188%:%
%:%391=189%:%
%:%392=189%:%
%:%393=189%:%
%:%394=190%:%
%:%395=190%:%
%:%396=191%:%
%:%397=191%:%
%:%398=192%:%
%:%399=192%:%
%:%400=193%:%
%:%401=193%:%
%:%402=193%:%
%:%403=194%:%
%:%404=194%:%
%:%405=194%:%
%:%406=194%:%
%:%407=194%:%
%:%408=195%:%
%:%409=195%:%
%:%410=195%:%
%:%411=195%:%
%:%412=196%:%
%:%413=196%:%
%:%414=196%:%
%:%415=197%:%
%:%416=197%:%
%:%417=198%:%
%:%418=198%:%
%:%419=199%:%
%:%420=199%:%
%:%421=199%:%
%:%422=199%:%
%:%423=200%:%
%:%424=200%:%
%:%425=200%:%
%:%426=200%:%
%:%427=201%:%
%:%437=207%:%
%:%438=208%:%
%:%439=209%:%
%:%440=210%:%
%:%441=211%:%
%:%442=212%:%
%:%444=214%:%
%:%445=214%:%
%:%446=215%:%
%:%447=216%:%
%:%454=217%:%
%:%455=217%:%
%:%456=218%:%
%:%457=218%:%
%:%458=219%:%
%:%459=219%:%
%:%460=219%:%
%:%461=219%:%
%:%462=219%:%
%:%463=220%:%
%:%469=220%:%
%:%472=221%:%
%:%473=222%:%
%:%474=222%:%
%:%475=223%:%
%:%476=224%:%
%:%477=225%:%
%:%478=226%:%
%:%485=227%:%
%:%486=227%:%
%:%487=228%:%
%:%488=228%:%
%:%489=229%:%
%:%490=229%:%
%:%491=230%:%
%:%492=230%:%
%:%493=231%:%
%:%494=231%:%
%:%495=232%:%
%:%496=232%:%
%:%497=232%:%
%:%498=232%:%
%:%499=233%:%
%:%500=233%:%
%:%501=234%:%
%:%502=234%:%
%:%503=235%:%
%:%504=235%:%
%:%505=236%:%
%:%506=236%:%
%:%507=236%:%
%:%508=237%:%
%:%509=237%:%
%:%510=237%:%
%:%511=237%:%
%:%512=237%:%
%:%513=238%:%
%:%514=238%:%
%:%515=238%:%
%:%516=238%:%
%:%517=239%:%
%:%518=239%:%
%:%519=239%:%
%:%520=240%:%
%:%521=240%:%
%:%522=241%:%
%:%523=241%:%
%:%524=242%:%
%:%525=242%:%
%:%526=242%:%
%:%527=242%:%
%:%528=243%:%
%:%543=246%:%
%:%555=247%:%
%:%557=249%:%
%:%558=249%:%
%:%559=250%:%
%:%562=251%:%
%:%566=251%:%
%:%567=251%:%
%:%568=252%:%
%:%569=252%:%
%:%570=253%:%
%:%571=254%:%
%:%572=255%:%
%:%573=256%:%