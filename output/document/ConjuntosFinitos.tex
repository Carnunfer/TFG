%
\begin{isabellebody}%
\setisabellecontext{ConjuntosFinitos}%
%
\isadelimtheory
%
\endisadelimtheory
%
\isatagtheory
%
\endisatagtheory
{\isafoldtheory}%
%
\isadelimtheory
%
\endisadelimtheory
%
\isadelimdocument
%
\endisadelimdocument
%
\isatagdocument
%
\isamarkupsection{Propiedad de los conjuntos finitos de números naturales%
}
\isamarkuptrue%
%
\endisatagdocument
{\isafolddocument}%
%
\isadelimdocument
%
\endisadelimdocument
%
\begin{isamarkuptext}%
El siguiente teorema es una propiedad que verifican todos los conjuntos finitos de números
  naturales  estudiado en el \href{http://bit.ly/2XBW6n2}{tema 10} de la
asignatura de LMF. Su enunciado es el siguiente 

  \begin{teorema} 
    Sea S un conjunto finito de números naturales.  Entonces todos los
 elementos de S son menores o iguales que la suma de los elementos de
 S, es decir, $$\forall m , m \in S \Longrightarrow m \leq \sum S$$ 
\newline
donde $\sum S $ denota la suma de todos los elementos de S.
  \end{teorema} 

\begin{demostracion}
La demostración del teorema la haremos por inducción sobre conjuntos
 finitos naturales.\\
Primero veamos el caso base, es decir, supongamos que $S = \emptyset$:

Tenemos que: $$\forall n, n \in \emptyset \Longrightarrow n \leq \sum
 \emptyset.$$\\
Ya hemos probado el caso base, veamos ahora el paso inductivo:
\newline
Sea S un conjunto finito para el que se cumple la hipótesis, es decir,
 todos los elementos de S son menores o iguales que la suma de todos sus
elementos, sea $a$ un elemento tal que $a \notin S$, ya que si $a \in S$
entonces la demostración es trivial.\\
Hay que probar: $$\forall n , n \in S \cup \{ a \} \Longrightarrow n
 \leq \sum (S \cup \{ a \})$$
Vamos a distinguir dos casos:\\
Caso 1: $n = a$ \\
Si $n = a$ tenemos que $n = a \leq a + \sum S = \sum ( S \cup \{ a
 \})$\\
Caso 2: $n \neq a$\\
Si $n \neq a$ tenemos que $a \notin S$ y que $n \in S \cup \{ a \}$,
 luego esto implica que $n \in S$ y usando la hipótesis de inducción
$$n \in S \Longrightarrow n \leq \sum S \leq \sum S + a = \sum (S \cup
 \{ a \})$$
\end{demostracion}

En la demostración del teorema hemos usado un resultado, que vamos a
 probar en Isabelle después de la especificación del teorema,
 el resultado es $\sum S + a = \sum (S \cup \{ a
 \})$.%
\end{isamarkuptext}\isamarkuptrue%
%
\begin{isamarkuptext}%
Para la especificación del teorema en isabelle, primero debemos
 notar que  \isa{finite\ S} indica que nuestro conjunto $S$ es 
finito  y definir  la función \isa{sumaConj} tal que
 \isa{sumaConj\ n} esla suma de todos los elementos de S.%
\end{isamarkuptext}\isamarkuptrue%
\isacommand{definition}\isamarkupfalse%
\ sumaConj\ {\isacharcolon}{\isacharcolon}\ {\isachardoublequoteopen}nat\ set\ {\isasymRightarrow}\ nat{\isachardoublequoteclose}\ \isakeyword{where}\isanewline
\ \ {\isachardoublequoteopen}sumaConj\ S\ {\isasymequiv}\ {\isasymSum}S{\isachardoublequoteclose}%
\begin{isamarkuptext}%
El enunciado del teorema es el siguiente :%
\end{isamarkuptext}\isamarkuptrue%
\isacommand{lemma}\isamarkupfalse%
\ {\isachardoublequoteopen}finite\ S\ {\isasymLongrightarrow}\ {\isasymforall}x{\isasymin}S{\isachardot}\ x\ {\isasymle}\ sumaConj\ S{\isachardoublequoteclose}\isanewline
%
\isadelimproof
\isanewline
\ \ %
\endisadelimproof
%
\isatagproof
\isacommand{oops}\isamarkupfalse%
%
\endisatagproof
{\isafoldproof}%
%
\isadelimproof
%
\endisadelimproof
%
\begin{isamarkuptext}%
Vamos a demostrar primero el lema enunciado anteriormente%
\end{isamarkuptext}\isamarkuptrue%
\isacommand{lemma}\isamarkupfalse%
\ {\isachardoublequoteopen}\ x\ {\isasymnotin}\ S\ {\isasymand}\ finite\ S\ {\isasymlongrightarrow}\ sumaConj\ S\ {\isacharplus}\ x\ {\isacharequal}\ sumaConj{\isacharparenleft}insert\ x\ S{\isacharparenright}{\isachardoublequoteclose}\isanewline
%
\isadelimproof
\ \ %
\endisadelimproof
%
\isatagproof
\isacommand{by}\isamarkupfalse%
\ {\isacharparenleft}simp\ add{\isacharcolon}\ sumaConj{\isacharunderscore}def{\isacharparenright}%
\endisatagproof
{\isafoldproof}%
%
\isadelimproof
%
\endisadelimproof
%
\begin{isamarkuptext}%
La demostración del lema anterior se ha incluido
 \isa{sumConj{\isacharunderscore}def}, que hace referencia a la definición sumaConj que
 hemos hecho anteriormente. \\
En la demostración se usará la táctica \isa{induct} que hace
  uso del esquema de inducción sobre los conjuntos finitos naturales:
  \begin{itemize}
  \item[] \isa{{\isasymlbrakk}finite\ x{\isacharsemicolon}\ P\ {\isasymemptyset}{\isacharsemicolon}\ {\isasymAnd}A\ a{\isachardot}\ finite\ A\ {\isasymand}\ P\ A\ {\isasymLongrightarrow}\ P\ {\isacharparenleft}{\isacharbraceleft}a{\isacharbraceright}\ {\isasymunion}\ A{\isacharparenright}{\isasymrbrakk}\ {\isasymLongrightarrow}\ P\ x} \hfill (\isa{finite{\isachardot}induct})
  \end{itemize} 

Vamos a ver presentar las diferentes formas de demostración.


La demostración aplicativa es:%
\end{isamarkuptext}\isamarkuptrue%
\isacommand{lemma}\isamarkupfalse%
\ {\isachardoublequoteopen}finite\ S\ {\isasymLongrightarrow}\ {\isasymforall}x{\isasymin}S{\isachardot}\ x\ {\isasymle}\ sumaConj\ S{\isachardoublequoteclose}\isanewline
%
\isadelimproof
\ \ %
\endisadelimproof
%
\isatagproof
\isacommand{apply}\isamarkupfalse%
\ {\isacharparenleft}induct\ rule{\isacharcolon}\ finite{\isacharunderscore}induct{\isacharparenright}\isanewline
\ \ \ \isacommand{apply}\isamarkupfalse%
\ simp\isanewline
\ \ \isacommand{apply}\isamarkupfalse%
\ {\isacharparenleft}simp\ add{\isacharcolon}\ add{\isacharunderscore}increasing\ sumaConj{\isacharunderscore}def{\isacharparenright}\isanewline
\ \ \isacommand{done}\isamarkupfalse%
%
\endisatagproof
{\isafoldproof}%
%
\isadelimproof
%
\endisadelimproof
%
\begin{isamarkuptext}%
En la demostración anterior se ha introducido:
 \begin{itemize}
    \item[] \isa{\mbox{}\inferrule{\mbox{{\isacharparenleft}{\isadigit{0}}\ {\isacharcolon}{\isacharcolon}\ {\isacharprime}a{\isacharparenright}\ {\isasymle}\ a\ {\isasymand}\ b\ {\isasymle}\ c}}{\mbox{b\ {\isasymle}\ a\ {\isacharplus}\ c}}} 
      \hfill (\isa{add{\isacharunderscore}increasing})
  \end{itemize} 
 La demostración automática es:%
\end{isamarkuptext}\isamarkuptrue%
\isacommand{lemma}\isamarkupfalse%
\ {\isachardoublequoteopen}finite\ S\ {\isasymLongrightarrow}\ {\isasymforall}x{\isasymin}S{\isachardot}\ x\ {\isasymle}\ sumaConj\ S{\isachardoublequoteclose}\isanewline
%
\isadelimproof
\ \ %
\endisadelimproof
%
\isatagproof
\isacommand{by}\isamarkupfalse%
\ {\isacharparenleft}induct\ rule{\isacharcolon}\ finite{\isacharunderscore}induct{\isacharparenright}\isanewline
\ \ \ \ \ {\isacharparenleft}auto\ simp\ add{\isacharcolon}\ \ sumaConj{\isacharunderscore}def{\isacharparenright}%
\endisatagproof
{\isafoldproof}%
%
\isadelimproof
%
\endisadelimproof
%
\begin{isamarkuptext}%
La demostración declarativa es:%
\end{isamarkuptext}\isamarkuptrue%
\isacommand{lemma}\isamarkupfalse%
\ sumaConj{\isacharunderscore}acota{\isacharcolon}\ \isanewline
\ \ {\isachardoublequoteopen}finite\ S\ {\isasymLongrightarrow}\ {\isasymforall}x{\isasymin}S{\isachardot}\ x\ {\isasymle}\ sumaConj\ S{\isachardoublequoteclose}\isanewline
%
\isadelimproof
%
\endisadelimproof
%
\isatagproof
\isacommand{proof}\isamarkupfalse%
\ {\isacharparenleft}induct\ rule{\isacharcolon}\ finite{\isacharunderscore}induct{\isacharparenright}\isanewline
\ \ \isacommand{show}\isamarkupfalse%
\ {\isachardoublequoteopen}{\isasymforall}x\ {\isasymin}\ {\isacharbraceleft}{\isacharbraceright}{\isachardot}\ x\ {\isasymle}\ sumaConj\ {\isacharbraceleft}{\isacharbraceright}{\isachardoublequoteclose}\ \isacommand{by}\isamarkupfalse%
\ simp\isanewline
\isacommand{next}\isamarkupfalse%
\isanewline
\ \ \isacommand{fix}\isamarkupfalse%
\ x\ \isakeyword{and}\ F\isanewline
\ \ \isacommand{assume}\isamarkupfalse%
\ fF{\isacharcolon}\ {\isachardoublequoteopen}finite\ F{\isachardoublequoteclose}\ \isanewline
\ \ \ \ \ \isakeyword{and}\ xF{\isacharcolon}\ {\isachardoublequoteopen}x\ {\isasymnotin}\ F{\isachardoublequoteclose}\ \isanewline
\ \ \ \ \ \isakeyword{and}\ HI{\isacharcolon}\ {\isachardoublequoteopen}{\isasymforall}\ x{\isasymin}F{\isachardot}\ x\ {\isasymle}\ sumaConj\ F{\isachardoublequoteclose}\isanewline
\ \ \isacommand{show}\isamarkupfalse%
\ {\isachardoublequoteopen}{\isasymforall}y\ {\isasymin}\ insert\ x\ F{\isachardot}\ y\ {\isasymle}\ sumaConj\ {\isacharparenleft}insert\ x\ F{\isacharparenright}{\isachardoublequoteclose}\isanewline
\ \ \isacommand{proof}\isamarkupfalse%
\ \isanewline
\ \ \ \ \isacommand{fix}\isamarkupfalse%
\ y\ \isanewline
\ \ \ \ \isacommand{assume}\isamarkupfalse%
\ {\isachardoublequoteopen}y\ {\isasymin}\ insert\ x\ F{\isachardoublequoteclose}\isanewline
\ \ \ \ \isacommand{show}\isamarkupfalse%
\ {\isachardoublequoteopen}y\ {\isasymle}\ sumaConj\ {\isacharparenleft}insert\ x\ F{\isacharparenright}{\isachardoublequoteclose}\isanewline
\ \ \ \ \isacommand{proof}\isamarkupfalse%
\ {\isacharparenleft}cases\ {\isachardoublequoteopen}y\ {\isacharequal}\ x{\isachardoublequoteclose}{\isacharparenright}\isanewline
\ \ \ \ \ \ \isacommand{assume}\isamarkupfalse%
\ {\isachardoublequoteopen}y\ {\isacharequal}\ x{\isachardoublequoteclose}\isanewline
\ \ \ \ \ \ \isacommand{then}\isamarkupfalse%
\ \isacommand{have}\isamarkupfalse%
\ {\isachardoublequoteopen}y\ {\isasymle}\ x\ {\isacharplus}\ {\isacharparenleft}sumaConj\ F{\isacharparenright}{\isachardoublequoteclose}\ \isacommand{by}\isamarkupfalse%
\ simp\isanewline
\ \ \ \ \ \ \isacommand{also}\isamarkupfalse%
\ \isacommand{have}\isamarkupfalse%
\ {\isachardoublequoteopen}{\isasymdots}\ {\isacharequal}\ sumaConj\ {\isacharparenleft}insert\ x\ F{\isacharparenright}{\isachardoublequoteclose}\ \ \ \isacommand{by}\isamarkupfalse%
\ {\isacharparenleft}simp\ add{\isacharcolon}\ fF\ sumaConj{\isacharunderscore}def\ xF{\isacharparenright}\ \isanewline
\ \ \ \ \ \ \isacommand{finally}\isamarkupfalse%
\ \isacommand{show}\isamarkupfalse%
\ {\isacharquery}thesis\ \isacommand{{\isachardot}}\isamarkupfalse%
\isanewline
\ \ \ \ \isacommand{next}\isamarkupfalse%
\isanewline
\ \ \ \ \ \ \isacommand{assume}\isamarkupfalse%
\ {\isachardoublequoteopen}y\ {\isasymnoteq}\ x{\isachardoublequoteclose}\isanewline
\ \ \ \ \ \ \isacommand{then}\isamarkupfalse%
\ \isacommand{have}\isamarkupfalse%
\ {\isachardoublequoteopen}y\ {\isasymin}\ F{\isachardoublequoteclose}\ \isacommand{using}\isamarkupfalse%
\ {\isacharbackquoteopen}y\ {\isasymin}\ insert\ x\ F{\isacharbackquoteclose}\ \isacommand{by}\isamarkupfalse%
\ simp\isanewline
\ \ \ \ \ \ \isacommand{then}\isamarkupfalse%
\ \isacommand{have}\isamarkupfalse%
\ {\isachardoublequoteopen}y\ {\isasymle}\ sumaConj\ F{\isachardoublequoteclose}\ \isacommand{using}\isamarkupfalse%
\ HI\ \isacommand{by}\isamarkupfalse%
\ simp\isanewline
\ \ \ \ \ \ \isacommand{also}\isamarkupfalse%
\ \isacommand{have}\isamarkupfalse%
\ {\isachardoublequoteopen}{\isasymdots}\ {\isasymle}\ x\ {\isacharplus}\ {\isacharparenleft}sumaConj\ F{\isacharparenright}{\isachardoublequoteclose}\ \isacommand{by}\isamarkupfalse%
\ simp\isanewline
\ \ \ \ \ \ \isacommand{also}\isamarkupfalse%
\ \isacommand{have}\isamarkupfalse%
\ {\isachardoublequoteopen}{\isasymdots}\ {\isacharequal}\ sumaConj\ {\isacharparenleft}insert\ x\ F{\isacharparenright}{\isachardoublequoteclose}\ \isacommand{using}\isamarkupfalse%
\ fF\ xF\isanewline
\ \ \ \ \ \ \ \ \isacommand{by}\isamarkupfalse%
\ {\isacharparenleft}simp\ add{\isacharcolon}\ sumaConj{\isacharunderscore}def{\isacharparenright}\isanewline
\ \ \ \ \ \ \isacommand{finally}\isamarkupfalse%
\ \isacommand{show}\isamarkupfalse%
\ {\isacharquery}thesis\ \isacommand{{\isachardot}}\isamarkupfalse%
\isanewline
\ \ \ \ \isacommand{qed}\isamarkupfalse%
\isanewline
\ \ \isacommand{qed}\isamarkupfalse%
\isanewline
\isacommand{qed}\isamarkupfalse%
%
\endisatagproof
{\isafoldproof}%
%
\isadelimproof
%
\endisadelimproof
%
\isadelimtheory
%
\endisadelimtheory
%
\isatagtheory
%
\endisatagtheory
{\isafoldtheory}%
%
\isadelimtheory
%
\endisadelimtheory
%
\end{isabellebody}%
\endinput
%:%file=~/Escritorio/TFG/ConjuntosFinitos.thy%:%
%:%24=7%:%
%:%36=9%:%
%:%37=10%:%
%:%38=11%:%
%:%39=12%:%
%:%40=13%:%
%:%41=14%:%
%:%42=15%:%
%:%43=16%:%
%:%44=17%:%
%:%45=18%:%
%:%46=19%:%
%:%47=20%:%
%:%48=21%:%
%:%49=22%:%
%:%50=23%:%
%:%51=24%:%
%:%52=25%:%
%:%53=26%:%
%:%54=27%:%
%:%55=28%:%
%:%56=29%:%
%:%57=30%:%
%:%58=31%:%
%:%59=32%:%
%:%60=33%:%
%:%61=34%:%
%:%62=35%:%
%:%63=36%:%
%:%64=37%:%
%:%65=38%:%
%:%66=39%:%
%:%67=40%:%
%:%68=41%:%
%:%69=42%:%
%:%70=43%:%
%:%71=44%:%
%:%72=45%:%
%:%73=46%:%
%:%74=47%:%
%:%75=48%:%
%:%76=49%:%
%:%77=50%:%
%:%81=54%:%
%:%82=55%:%
%:%83=56%:%
%:%84=57%:%
%:%86=60%:%
%:%87=60%:%
%:%88=61%:%
%:%90=63%:%
%:%92=66%:%
%:%93=66%:%
%:%96=67%:%
%:%97=68%:%
%:%101=68%:%
%:%111=70%:%
%:%113=71%:%
%:%114=71%:%
%:%117=72%:%
%:%121=72%:%
%:%122=72%:%
%:%131=75%:%
%:%132=76%:%
%:%133=77%:%
%:%134=78%:%
%:%135=79%:%
%:%136=80%:%
%:%137=81%:%
%:%138=82%:%
%:%139=83%:%
%:%140=84%:%
%:%141=85%:%
%:%142=86%:%
%:%143=87%:%
%:%145=89%:%
%:%146=89%:%
%:%149=90%:%
%:%153=90%:%
%:%154=90%:%
%:%155=91%:%
%:%156=91%:%
%:%157=92%:%
%:%158=92%:%
%:%159=93%:%
%:%169=95%:%
%:%170=96%:%
%:%171=97%:%
%:%172=98%:%
%:%173=99%:%
%:%174=100%:%
%:%176=102%:%
%:%177=102%:%
%:%180=103%:%
%:%184=103%:%
%:%185=103%:%
%:%186=104%:%
%:%195=106%:%
%:%197=108%:%
%:%198=108%:%
%:%199=109%:%
%:%206=110%:%
%:%207=110%:%
%:%208=111%:%
%:%209=111%:%
%:%210=111%:%
%:%211=112%:%
%:%212=112%:%
%:%213=113%:%
%:%214=113%:%
%:%215=114%:%
%:%216=114%:%
%:%217=115%:%
%:%218=116%:%
%:%219=117%:%
%:%220=117%:%
%:%221=118%:%
%:%222=118%:%
%:%223=119%:%
%:%224=119%:%
%:%225=120%:%
%:%226=120%:%
%:%227=121%:%
%:%228=121%:%
%:%229=122%:%
%:%230=122%:%
%:%231=123%:%
%:%232=123%:%
%:%233=124%:%
%:%234=124%:%
%:%235=124%:%
%:%236=124%:%
%:%237=125%:%
%:%238=125%:%
%:%239=125%:%
%:%240=125%:%
%:%241=126%:%
%:%242=126%:%
%:%243=126%:%
%:%244=126%:%
%:%245=127%:%
%:%246=127%:%
%:%247=128%:%
%:%248=128%:%
%:%249=129%:%
%:%250=129%:%
%:%251=129%:%
%:%252=129%:%
%:%253=129%:%
%:%254=130%:%
%:%255=130%:%
%:%256=130%:%
%:%257=130%:%
%:%258=130%:%
%:%259=131%:%
%:%260=131%:%
%:%261=131%:%
%:%262=131%:%
%:%263=132%:%
%:%264=132%:%
%:%265=132%:%
%:%266=132%:%
%:%267=133%:%
%:%268=133%:%
%:%269=134%:%
%:%270=134%:%
%:%271=134%:%
%:%272=134%:%
%:%273=135%:%
%:%274=135%:%
%:%275=136%:%
%:%276=136%:%
%:%277=137%:%