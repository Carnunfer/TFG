%
\begin{isabellebody}%
\setisabellecontext{ConjuntosFinitos}%
%
\isadelimtheory
\isanewline
%
\endisadelimtheory
%
\isatagtheory
%
\endisatagtheory
{\isafoldtheory}%
%
\isadelimtheory
%
\endisadelimtheory
%
\begin{isamarkuptext}%
\comentario{Estructurar en secciones.}%
\end{isamarkuptext}\isamarkuptrue%
%
\begin{isamarkuptext}%
\comentario{Hacer demostraciones detalladas.}%
\end{isamarkuptext}\isamarkuptrue%
%
\begin{isamarkuptext}%
\comentario{Añadir lemas usados al Soporte.}%
\end{isamarkuptext}\isamarkuptrue%
%
\begin{isamarkuptext}%
El siguiente teorema es una propiedad que verifican todos los conjuntos finitos de números
  naturales  estudiado en el \href{http://bit.ly/2XBW6n2}{tema 10} de la
asignatura de LMF. Su enunciado es el siguiente 

  \begin{teorema} 
    Sea S un conjunto finito de números naturales.  Entonces todos los
 elementos de S son menores o iguales que la suma de los elementos de
 S, es decir, $$\forall m , m \in S \Longrightarrow m \leq \sum S$$ 
\newline
donde $\sum S $ denota la suma de todos los elementos de S.
  \end{teorema} 

\begin{demostracion}
La demostración del teorema la haremos por inducción sobre conjuntos
 finitos naturales.\\
Primero veamos el caso base, es decir, supongamos que $S = \emptyset$:

Tenemos que: $$\forall n, n \in \emptyset \Longrightarrow n \leq \sum
 \emptyset.$$\\
Ya hemos probado el caso base, veamos ahora el paso inductivo:
\newline
Sea S un conjunto finito para el que se cumple la hipótesis, es decir,
 todos los elementos de S son menores o iguales que la suma de todos sus
elementos, sea $a$ un elemento tal que $a \notin S$, ya que si $a \in S$
entonces la demostración es trivial.\\
Hay que probar: $$\forall n , n \in S \cup \{ a \} \Longrightarrow n
 \leq \sum (S \cup \{ a \})$$
Vamos a distinguir dos casos:\\
Caso 1: $n = a$ \\
Si $n = a$ tenemos que $n = a \leq a + \sum S = \sum ( S \cup \{ a
 \})$\\
Caso 2: $n \neq a$\\
Si $n \neq a$ tenemos que $a \notin S$ y que $n \in S \cup \{ a \}$,
 luego esto implica que $n \in S$ y usando la hipótesis de inducción
$$n \in S \Longrightarrow n \leq \sum S \leq \sum S + a = \sum (S \cup
 \{ a \})$$
\end{demostracion}

En la demostración del teorema hemos usado un resultado, que vamos a
 probar en Isabelle después de la especificación del teorema,
 el resultado es $\sum S + a = \sum (S \cup \{ a
 \})$.%
\end{isamarkuptext}\isamarkuptrue%
%
\begin{isamarkuptext}%
Para la especificación del teorema en isabelle, primero debemos
 notar que  \isa{finite\ S} indica que nuestro conjunto $S$ es 
finito  y definir  la función \isa{sumaConj} tal que
 \isa{sumaConj\ n} esla suma de todos los elementos de S.%
\end{isamarkuptext}\isamarkuptrue%
\isacommand{definition}\isamarkupfalse%
\ sumaConj\ {\isacharcolon}{\isacharcolon}\ {\isachardoublequoteopen}nat\ set\ {\isasymRightarrow}\ nat{\isachardoublequoteclose}\ \isakeyword{where}\isanewline
\ \ {\isachardoublequoteopen}sumaConj\ S\ {\isasymequiv}\ {\isasymSum}S{\isachardoublequoteclose}%
\begin{isamarkuptext}%
El enunciado del teorema es el siguiente :%
\end{isamarkuptext}\isamarkuptrue%
\isacommand{lemma}\isamarkupfalse%
\ {\isachardoublequoteopen}finite\ S\ {\isasymLongrightarrow}\ {\isasymforall}x{\isasymin}S{\isachardot}\ x\ {\isasymle}\ sumaConj\ S{\isachardoublequoteclose}\isanewline
%
\isadelimproof
\isanewline
\ \ %
\endisadelimproof
%
\isatagproof
\isacommand{oops}\isamarkupfalse%
%
\endisatagproof
{\isafoldproof}%
%
\isadelimproof
%
\endisadelimproof
%
\begin{isamarkuptext}%
Vamos a demostrar primero el lema enunciado anteriormente%
\end{isamarkuptext}\isamarkuptrue%
\isacommand{lemma}\isamarkupfalse%
\ {\isachardoublequoteopen}\ x\ {\isasymnotin}\ S\ {\isasymand}\ finite\ S\ {\isasymlongrightarrow}\ sumaConj\ S\ {\isacharplus}\ x\ {\isacharequal}\ sumaConj{\isacharparenleft}insert\ x\ S{\isacharparenright}{\isachardoublequoteclose}\isanewline
%
\isadelimproof
\ \ %
\endisadelimproof
%
\isatagproof
\isacommand{by}\isamarkupfalse%
\ {\isacharparenleft}simp\ add{\isacharcolon}\ sumaConj{\isacharunderscore}def{\isacharparenright}%
\endisatagproof
{\isafoldproof}%
%
\isadelimproof
%
\endisadelimproof
%
\begin{isamarkuptext}%
La demostración del lema anterior se ha incluido
 \isa{sumConj{\isacharunderscore}def}, que hace referencia a la definición sumaConj que
 hemos hecho anteriormente. \\
En la demostración se usará la táctica \isa{induct} que hace
  uso del esquema de inducción sobre los conjuntos finitos naturales:
  \begin{itemize}
  \item[] \isa{{\isasymlbrakk}finite\ x{\isacharsemicolon}\ P\ {\isasymemptyset}{\isacharsemicolon}\ {\isasymAnd}A\ a{\isachardot}\ finite\ A\ {\isasymand}\ P\ A\ {\isasymLongrightarrow}\ P\ {\isacharparenleft}{\isacharbraceleft}a{\isacharbraceright}\ {\isasymunion}\ A{\isacharparenright}{\isasymrbrakk}\ {\isasymLongrightarrow}\ P\ x} \hfill (\isa{finite{\isachardot}induct})
  \end{itemize} 

Vamos a ver presentar las diferentes formas de demostración.


La demostración aplicativa es:%
\end{isamarkuptext}\isamarkuptrue%
\isacommand{lemma}\isamarkupfalse%
\ {\isachardoublequoteopen}finite\ S\ {\isasymLongrightarrow}\ {\isasymforall}x{\isasymin}S{\isachardot}\ x\ {\isasymle}\ sumaConj\ S{\isachardoublequoteclose}\isanewline
%
\isadelimproof
\ \ %
\endisadelimproof
%
\isatagproof
\isacommand{apply}\isamarkupfalse%
\ {\isacharparenleft}induct\ rule{\isacharcolon}\ finite{\isacharunderscore}induct{\isacharparenright}\isanewline
\ \ \ \isacommand{apply}\isamarkupfalse%
\ simp\isanewline
\ \ \isacommand{apply}\isamarkupfalse%
\ {\isacharparenleft}simp\ add{\isacharcolon}\ add{\isacharunderscore}increasing\ sumaConj{\isacharunderscore}def{\isacharparenright}\isanewline
\ \ \isacommand{done}\isamarkupfalse%
%
\endisatagproof
{\isafoldproof}%
%
\isadelimproof
%
\endisadelimproof
%
\begin{isamarkuptext}%
En la demostración anterior se ha introducido:
 \begin{itemize}
    \item[] \isa{\mbox{}\inferrule{\mbox{{\isacharparenleft}{\isadigit{0}}\ {\isacharcolon}{\isacharcolon}\ {\isacharprime}a{\isacharparenright}\ {\isasymle}\ a\ {\isasymand}\ b\ {\isasymle}\ c}}{\mbox{b\ {\isasymle}\ a\ {\isacharplus}\ c}}} 
      \hfill (\isa{add{\isacharunderscore}increasing})
  \end{itemize} 
 La demostración automática es:%
\end{isamarkuptext}\isamarkuptrue%
\isacommand{lemma}\isamarkupfalse%
\ {\isachardoublequoteopen}finite\ S\ {\isasymLongrightarrow}\ {\isasymforall}x{\isasymin}S{\isachardot}\ x\ {\isasymle}\ sumaConj\ S{\isachardoublequoteclose}\isanewline
%
\isadelimproof
\ \ %
\endisadelimproof
%
\isatagproof
\isacommand{by}\isamarkupfalse%
\ {\isacharparenleft}induct\ rule{\isacharcolon}\ finite{\isacharunderscore}induct{\isacharparenright}\isanewline
\ \ \ \ \ {\isacharparenleft}auto\ simp\ add{\isacharcolon}\ \ sumaConj{\isacharunderscore}def{\isacharparenright}%
\endisatagproof
{\isafoldproof}%
%
\isadelimproof
%
\endisadelimproof
%
\begin{isamarkuptext}%
La demostración declarativa es:%
\end{isamarkuptext}\isamarkuptrue%
\isacommand{lemma}\isamarkupfalse%
\ sumaConj{\isacharunderscore}acota{\isacharcolon}\ \isanewline
\ \ {\isachardoublequoteopen}finite\ S\ {\isasymLongrightarrow}\ {\isasymforall}x{\isasymin}S{\isachardot}\ x\ {\isasymle}\ sumaConj\ S{\isachardoublequoteclose}\isanewline
%
\isadelimproof
%
\endisadelimproof
%
\isatagproof
\isacommand{proof}\isamarkupfalse%
\ {\isacharparenleft}induct\ rule{\isacharcolon}\ finite{\isacharunderscore}induct{\isacharparenright}\isanewline
\ \ \isacommand{show}\isamarkupfalse%
\ {\isachardoublequoteopen}{\isasymforall}x\ {\isasymin}\ {\isacharbraceleft}{\isacharbraceright}{\isachardot}\ x\ {\isasymle}\ sumaConj\ {\isacharbraceleft}{\isacharbraceright}{\isachardoublequoteclose}\ \isacommand{by}\isamarkupfalse%
\ simp\isanewline
\isacommand{next}\isamarkupfalse%
\isanewline
\ \ \isacommand{fix}\isamarkupfalse%
\ x\ \isakeyword{and}\ F\isanewline
\ \ \isacommand{assume}\isamarkupfalse%
\ fF{\isacharcolon}\ {\isachardoublequoteopen}finite\ F{\isachardoublequoteclose}\ \isanewline
\ \ \ \ \ \isakeyword{and}\ xF{\isacharcolon}\ {\isachardoublequoteopen}x\ {\isasymnotin}\ F{\isachardoublequoteclose}\ \isanewline
\ \ \ \ \ \isakeyword{and}\ HI{\isacharcolon}\ {\isachardoublequoteopen}{\isasymforall}\ x{\isasymin}F{\isachardot}\ x\ {\isasymle}\ sumaConj\ F{\isachardoublequoteclose}\isanewline
\ \ \isacommand{show}\isamarkupfalse%
\ {\isachardoublequoteopen}{\isasymforall}y\ {\isasymin}\ insert\ x\ F{\isachardot}\ y\ {\isasymle}\ sumaConj\ {\isacharparenleft}insert\ x\ F{\isacharparenright}{\isachardoublequoteclose}\isanewline
\ \ \isacommand{proof}\isamarkupfalse%
\ \isanewline
\ \ \ \ \isacommand{fix}\isamarkupfalse%
\ y\ \isanewline
\ \ \ \ \isacommand{assume}\isamarkupfalse%
\ {\isachardoublequoteopen}y\ {\isasymin}\ insert\ x\ F{\isachardoublequoteclose}\isanewline
\ \ \ \ \isacommand{show}\isamarkupfalse%
\ {\isachardoublequoteopen}y\ {\isasymle}\ sumaConj\ {\isacharparenleft}insert\ x\ F{\isacharparenright}{\isachardoublequoteclose}\isanewline
\ \ \ \ \isacommand{proof}\isamarkupfalse%
\ {\isacharparenleft}cases\ {\isachardoublequoteopen}y\ {\isacharequal}\ x{\isachardoublequoteclose}{\isacharparenright}\isanewline
\ \ \ \ \ \ \isacommand{assume}\isamarkupfalse%
\ {\isachardoublequoteopen}y\ {\isacharequal}\ x{\isachardoublequoteclose}\isanewline
\ \ \ \ \ \ \isacommand{then}\isamarkupfalse%
\ \isacommand{have}\isamarkupfalse%
\ {\isachardoublequoteopen}y\ {\isasymle}\ x\ {\isacharplus}\ {\isacharparenleft}sumaConj\ F{\isacharparenright}{\isachardoublequoteclose}\ \isacommand{by}\isamarkupfalse%
\ simp\isanewline
\ \ \ \ \ \ \isacommand{also}\isamarkupfalse%
\ \isacommand{have}\isamarkupfalse%
\ {\isachardoublequoteopen}{\isasymdots}\ {\isacharequal}\ sumaConj\ {\isacharparenleft}insert\ x\ F{\isacharparenright}{\isachardoublequoteclose}\ \ \ \isacommand{by}\isamarkupfalse%
\ {\isacharparenleft}simp\ add{\isacharcolon}\ fF\ sumaConj{\isacharunderscore}def\ xF{\isacharparenright}\ \isanewline
\ \ \ \ \ \ \isacommand{finally}\isamarkupfalse%
\ \isacommand{show}\isamarkupfalse%
\ {\isacharquery}thesis\ \isacommand{{\isachardot}}\isamarkupfalse%
\isanewline
\ \ \ \ \isacommand{next}\isamarkupfalse%
\isanewline
\ \ \ \ \ \ \isacommand{assume}\isamarkupfalse%
\ {\isachardoublequoteopen}y\ {\isasymnoteq}\ x{\isachardoublequoteclose}\isanewline
\ \ \ \ \ \ \isacommand{then}\isamarkupfalse%
\ \isacommand{have}\isamarkupfalse%
\ {\isachardoublequoteopen}y\ {\isasymin}\ F{\isachardoublequoteclose}\ \isacommand{using}\isamarkupfalse%
\ {\isacharbackquoteopen}y\ {\isasymin}\ insert\ x\ F{\isacharbackquoteclose}\ \isacommand{by}\isamarkupfalse%
\ simp\isanewline
\ \ \ \ \ \ \isacommand{then}\isamarkupfalse%
\ \isacommand{have}\isamarkupfalse%
\ {\isachardoublequoteopen}y\ {\isasymle}\ sumaConj\ F{\isachardoublequoteclose}\ \isacommand{using}\isamarkupfalse%
\ HI\ \isacommand{by}\isamarkupfalse%
\ simp\isanewline
\ \ \ \ \ \ \isacommand{also}\isamarkupfalse%
\ \isacommand{have}\isamarkupfalse%
\ {\isachardoublequoteopen}{\isasymdots}\ {\isasymle}\ x\ {\isacharplus}\ {\isacharparenleft}sumaConj\ F{\isacharparenright}{\isachardoublequoteclose}\ \isacommand{by}\isamarkupfalse%
\ simp\isanewline
\ \ \ \ \ \ \isacommand{also}\isamarkupfalse%
\ \isacommand{have}\isamarkupfalse%
\ {\isachardoublequoteopen}{\isasymdots}\ {\isacharequal}\ sumaConj\ {\isacharparenleft}insert\ x\ F{\isacharparenright}{\isachardoublequoteclose}\ \isacommand{using}\isamarkupfalse%
\ fF\ xF\isanewline
\ \ \ \ \ \ \ \ \isacommand{by}\isamarkupfalse%
\ {\isacharparenleft}simp\ add{\isacharcolon}\ sumaConj{\isacharunderscore}def{\isacharparenright}\isanewline
\ \ \ \ \ \ \isacommand{finally}\isamarkupfalse%
\ \isacommand{show}\isamarkupfalse%
\ {\isacharquery}thesis\ \isacommand{{\isachardot}}\isamarkupfalse%
\isanewline
\ \ \ \ \isacommand{qed}\isamarkupfalse%
\isanewline
\ \ \isacommand{qed}\isamarkupfalse%
\isanewline
\isacommand{qed}\isamarkupfalse%
%
\endisatagproof
{\isafoldproof}%
%
\isadelimproof
%
\endisadelimproof
%
\isadelimtheory
%
\endisadelimtheory
%
\isatagtheory
%
\endisatagtheory
{\isafoldtheory}%
%
\isadelimtheory
%
\endisadelimtheory
%
\end{isabellebody}%
\endinput
%:%file=~/ownCloud/alonso/curso-TFG/Carlos/TFG_de_Carlos/ConjuntosFinitos.thy%:%
%:%6=1%:%
%:%20=9%:%
%:%24=11%:%
%:%28=13%:%
%:%32=15%:%
%:%33=16%:%
%:%34=17%:%
%:%35=18%:%
%:%36=19%:%
%:%37=20%:%
%:%38=21%:%
%:%39=22%:%
%:%40=23%:%
%:%41=24%:%
%:%42=25%:%
%:%43=26%:%
%:%44=27%:%
%:%45=28%:%
%:%46=29%:%
%:%47=30%:%
%:%48=31%:%
%:%49=32%:%
%:%50=33%:%
%:%51=34%:%
%:%52=35%:%
%:%53=36%:%
%:%54=37%:%
%:%55=38%:%
%:%56=39%:%
%:%57=40%:%
%:%58=41%:%
%:%59=42%:%
%:%60=43%:%
%:%61=44%:%
%:%62=45%:%
%:%63=46%:%
%:%64=47%:%
%:%65=48%:%
%:%66=49%:%
%:%67=50%:%
%:%68=51%:%
%:%69=52%:%
%:%70=53%:%
%:%71=54%:%
%:%72=55%:%
%:%73=56%:%
%:%77=60%:%
%:%78=61%:%
%:%79=62%:%
%:%80=63%:%
%:%82=66%:%
%:%83=66%:%
%:%84=67%:%
%:%86=69%:%
%:%88=72%:%
%:%89=72%:%
%:%92=73%:%
%:%93=74%:%
%:%97=74%:%
%:%107=76%:%
%:%109=77%:%
%:%110=77%:%
%:%113=78%:%
%:%117=78%:%
%:%118=78%:%
%:%127=81%:%
%:%128=82%:%
%:%129=83%:%
%:%130=84%:%
%:%131=85%:%
%:%132=86%:%
%:%133=87%:%
%:%134=88%:%
%:%135=89%:%
%:%136=90%:%
%:%137=91%:%
%:%138=92%:%
%:%139=93%:%
%:%141=95%:%
%:%142=95%:%
%:%145=96%:%
%:%149=96%:%
%:%150=96%:%
%:%151=97%:%
%:%152=97%:%
%:%153=98%:%
%:%154=98%:%
%:%155=99%:%
%:%165=101%:%
%:%166=102%:%
%:%167=103%:%
%:%168=104%:%
%:%169=105%:%
%:%170=106%:%
%:%172=108%:%
%:%173=108%:%
%:%176=109%:%
%:%180=109%:%
%:%181=109%:%
%:%182=110%:%
%:%191=112%:%
%:%193=114%:%
%:%194=114%:%
%:%195=115%:%
%:%202=116%:%
%:%203=116%:%
%:%204=117%:%
%:%205=117%:%
%:%206=117%:%
%:%207=118%:%
%:%208=118%:%
%:%209=119%:%
%:%210=119%:%
%:%211=120%:%
%:%212=120%:%
%:%213=121%:%
%:%214=122%:%
%:%215=123%:%
%:%216=123%:%
%:%217=124%:%
%:%218=124%:%
%:%219=125%:%
%:%220=125%:%
%:%221=126%:%
%:%222=126%:%
%:%223=127%:%
%:%224=127%:%
%:%225=128%:%
%:%226=128%:%
%:%227=129%:%
%:%228=129%:%
%:%229=130%:%
%:%230=130%:%
%:%231=130%:%
%:%232=130%:%
%:%233=131%:%
%:%234=131%:%
%:%235=131%:%
%:%236=131%:%
%:%237=132%:%
%:%238=132%:%
%:%239=132%:%
%:%240=132%:%
%:%241=133%:%
%:%242=133%:%
%:%243=134%:%
%:%244=134%:%
%:%245=135%:%
%:%246=135%:%
%:%247=135%:%
%:%248=135%:%
%:%249=135%:%
%:%250=136%:%
%:%251=136%:%
%:%252=136%:%
%:%253=136%:%
%:%254=136%:%
%:%255=137%:%
%:%256=137%:%
%:%257=137%:%
%:%258=137%:%
%:%259=138%:%
%:%260=138%:%
%:%261=138%:%
%:%262=138%:%
%:%263=139%:%
%:%264=139%:%
%:%265=140%:%
%:%266=140%:%
%:%267=140%:%
%:%268=140%:%
%:%269=141%:%
%:%270=141%:%
%:%271=142%:%
%:%272=142%:%
%:%273=143%:%