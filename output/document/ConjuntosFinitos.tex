%
\begin{isabellebody}%
\setisabellecontext{ConjuntosFinitos}%
%
\isadelimtheory
\isanewline
%
\endisadelimtheory
%
\isatagtheory
%
\endisatagtheory
{\isafoldtheory}%
%
\isadelimtheory
%
\endisadelimtheory
%
\begin{isamarkuptext}%
\comentario{Estructurar en secciones.}%
\end{isamarkuptext}\isamarkuptrue%
%
\begin{isamarkuptext}%
\comentario{Hacer demostraciones detalladas.}%
\end{isamarkuptext}\isamarkuptrue%
%
\begin{isamarkuptext}%
\comentario{Añadir lemas usados al Soporte.}%
\end{isamarkuptext}\isamarkuptrue%
%
\isadelimdocument
%
\endisadelimdocument
%
\isatagdocument
%
\isamarkupsection{Demostración en lenguaje natural%
}
\isamarkuptrue%
%
\endisatagdocument
{\isafolddocument}%
%
\isadelimdocument
%
\endisadelimdocument
%
\begin{isamarkuptext}%
El siguiente teorema es una propiedad que verifican todos los 
conjuntos finitos de números naturales  estudiado en el 
\href{http://bit.ly/2XBW6n2}{tema 10} de la
asignatura de LMF de tercer curso del grado en Matemáticas. Su enunciado
 es el siguiente 

  \begin{teorema} 
    Sea S un conjunto finito de números naturales.  Entonces todos los
 elementos de S son menores o iguales que la suma de los elementos de
 S, es decir,

 $$\forall m \in S \Longrightarrow m \leq \sum S$$ 

donde $\sum S $ denota la suma de todos los elementos de S.
  \end{teorema} 

\begin{demostracion}
La demostración del teorema la haremos por inducción sobre conjuntos
 finitos.

  
 (Base de la inducción) El caso $S = \emptyset$ es trivial.

 (Paso de la inducción) Supongamos que se verifica el teorema para un
 conjunto finito de números naturales, que se denotará por $S.$ 
 
Sea $a \in \Bbb{N}$ tal que $a \notin S,$ Ya que si $a \in S$ se tendría
probado el teorema. Luego hay que probar que: 

$$\forall n \in S \cup \{a\} \Longrightarrow n \leq \sum (S \cup
 \{a\})$$

Distingamos dos casos ahora:

Caso 1: $n = a$.

Si $n = a$, se tiene que:

$$n = a \leq a + \sum S = \sum (S \cup \{a\}).$$

Caso 2: $n \neq a.$

Si $n \neq a,$ tenemos que $n \in S,$ luego usando la hipótesis de
 inducción:

$$n \leq \sum S \leq \sum S + a = \sum (S \cup \{a\}).$$
\end{demostracion}

En la demostración del teorema hemos usado un resultado, que vamos a
 probar en Isabelle después de la especificación del teorema,
 el resultado es $\sum S + a = \sum (S \cup \{ a\})$.%
\end{isamarkuptext}\isamarkuptrue%
%
\isadelimdocument
%
\endisadelimdocument
%
\isatagdocument
%
\isamarkupsection{Especificación en Isabelle/HOL%
}
\isamarkuptrue%
%
\endisatagdocument
{\isafolddocument}%
%
\isadelimdocument
%
\endisadelimdocument
%
\begin{isamarkuptext}%
Para la especificación del teorema en Isabelle, primero debemos
 notar que  \isa{finite\ S} indica que un conjunto $S$ es 
finito  y definir  la función \isa{sumaConj} tal que
 \isa{sumaConj\ n} es la suma de todos los elementos de S.%
\end{isamarkuptext}\isamarkuptrue%
\isacommand{definition}\isamarkupfalse%
\ sumaConj\ {\isacharcolon}{\isacharcolon}\ {\isachardoublequoteopen}nat\ set\ {\isasymRightarrow}\ nat{\isachardoublequoteclose}\ \isakeyword{where}\isanewline
\ \ {\isachardoublequoteopen}sumaConj\ S\ {\isasymequiv}\ {\isasymSum}S{\isachardoublequoteclose}%
\begin{isamarkuptext}%
El enunciado del teorema es el siguiente :%
\end{isamarkuptext}\isamarkuptrue%
\isacommand{lemma}\isamarkupfalse%
\ {\isachardoublequoteopen}finite\ S\ {\isasymLongrightarrow}\ {\isasymforall}x\ {\isasymin}\ S{\isachardot}\ x\ {\isasymle}\ sumaConj\ S{\isachardoublequoteclose}\isanewline
%
\isadelimproof
\isanewline
\ \ %
\endisadelimproof
%
\isatagproof
\isacommand{oops}\isamarkupfalse%
%
\endisatagproof
{\isafoldproof}%
%
\isadelimproof
%
\endisadelimproof
%
\begin{isamarkuptext}%
Vamos a demostrar primero el lema enunciado anteriormente%
\end{isamarkuptext}\isamarkuptrue%
\isacommand{lemma}\isamarkupfalse%
\ aux{\isacharunderscore}propiedad{\isacharunderscore}conjuntos{\isacharunderscore}finitos{\isacharcolon}\isanewline
\ {\isachardoublequoteopen}\ x\ {\isasymnotin}\ S\ {\isasymand}\ finite\ S\ {\isasymlongrightarrow}\ sumaConj\ S\ {\isacharplus}\ x\ {\isacharequal}\ sumaConj{\isacharparenleft}insert\ x\ S{\isacharparenright}{\isachardoublequoteclose}\isanewline
%
\isadelimproof
\ \ %
\endisadelimproof
%
\isatagproof
\isacommand{by}\isamarkupfalse%
\ {\isacharparenleft}simp\ add{\isacharcolon}\ sumaConj{\isacharunderscore}def{\isacharparenright}%
\endisatagproof
{\isafoldproof}%
%
\isadelimproof
%
\endisadelimproof
%
\begin{isamarkuptext}%
La demostración del lema anterior se ha incluido
 \isa{sumConj{\isacharunderscore}def}, que hace referencia a la definición sumaConj que
 hemos hecho anteriormente.


En la demostración se usará la táctica \isa{induct} que hace
  uso del esquema de inducción sobre los conjuntos finitos:
  \begin{itemize}
  \item[] \isa{{\isasymlbrakk}finite\ x{\isacharsemicolon}\ P\ {\isasymemptyset}{\isacharsemicolon}\ {\isasymAnd}A\ a{\isachardot}\ finite\ A\ {\isasymand}\ P\ A\ {\isasymLongrightarrow}\ P\ {\isacharparenleft}{\isacharbraceleft}a{\isacharbraceright}\ {\isasymunion}\ A{\isacharparenright}{\isasymrbrakk}\ {\isasymLongrightarrow}\ P\ x}
 \hfill (\isa{finite{\isachardot}induct})
  \end{itemize} 

Vamos a presentar diferentes formas de demostración:%
\end{isamarkuptext}\isamarkuptrue%
%
\isadelimdocument
%
\endisadelimdocument
%
\isatagdocument
%
\isamarkupsection{Demostración aplicativa%
}
\isamarkuptrue%
%
\endisatagdocument
{\isafolddocument}%
%
\isadelimdocument
%
\endisadelimdocument
%
\begin{isamarkuptext}%
La demostración aplicativa del teorema es:%
\end{isamarkuptext}\isamarkuptrue%
\isacommand{lemma}\isamarkupfalse%
\ {\isachardoublequoteopen}finite\ S\ {\isasymLongrightarrow}\ {\isasymforall}x{\isasymin}S{\isachardot}\ x\ {\isasymle}\ sumaConj\ S{\isachardoublequoteclose}\isanewline
%
\isadelimproof
\ \ %
\endisadelimproof
%
\isatagproof
\isacommand{apply}\isamarkupfalse%
\ {\isacharparenleft}induct\ rule{\isacharcolon}\ finite{\isacharunderscore}induct{\isacharparenright}\isanewline
\ \ \ \isacommand{apply}\isamarkupfalse%
\ simp\isanewline
\ \ \isacommand{apply}\isamarkupfalse%
\ {\isacharparenleft}simp\ add{\isacharcolon}\ add{\isacharunderscore}increasing\ sumaConj{\isacharunderscore}def{\isacharparenright}\isanewline
\ \ \isacommand{done}\isamarkupfalse%
%
\endisatagproof
{\isafoldproof}%
%
\isadelimproof
%
\endisadelimproof
%
\begin{isamarkuptext}%
En la demostración anterior se ha introducido:
 \begin{itemize}
    \item[] \isa{\mbox{}\inferrule{\mbox{{\isacharparenleft}{\isadigit{0}}\ {\isacharcolon}{\isacharcolon}\ {\isacharprime}a{\isacharparenright}\ {\isasymle}\ a\ {\isasymand}\ b\ {\isasymle}\ c}}{\mbox{b\ {\isasymle}\ a\ {\isacharplus}\ c}}} 
      \hfill (\isa{add{\isacharunderscore}increasing})
  \end{itemize}%
\end{isamarkuptext}\isamarkuptrue%
%
\isadelimdocument
%
\endisadelimdocument
%
\isatagdocument
%
\isamarkupsection{Demostración automática%
}
\isamarkuptrue%
%
\endisatagdocument
{\isafolddocument}%
%
\isadelimdocument
%
\endisadelimdocument
%
\begin{isamarkuptext}%
La demostración automática es:%
\end{isamarkuptext}\isamarkuptrue%
\isacommand{lemma}\isamarkupfalse%
\ {\isachardoublequoteopen}finite\ S\ {\isasymLongrightarrow}\ {\isasymforall}x{\isasymin}S{\isachardot}\ x\ {\isasymle}\ sumaConj\ S{\isachardoublequoteclose}\isanewline
%
\isadelimproof
\ \ %
\endisadelimproof
%
\isatagproof
\isacommand{by}\isamarkupfalse%
\ {\isacharparenleft}induct\ rule{\isacharcolon}\ finite{\isacharunderscore}induct{\isacharparenright}\isanewline
\ \ \ \ \ {\isacharparenleft}auto\ simp\ add{\isacharcolon}\ \ sumaConj{\isacharunderscore}def{\isacharparenright}%
\endisatagproof
{\isafoldproof}%
%
\isadelimproof
%
\endisadelimproof
%
\isadelimdocument
%
\endisadelimdocument
%
\isatagdocument
%
\isamarkupsection{Demostración detallada%
}
\isamarkuptrue%
%
\endisatagdocument
{\isafolddocument}%
%
\isadelimdocument
%
\endisadelimdocument
%
\begin{isamarkuptext}%
La demostración declarativa es:%
\end{isamarkuptext}\isamarkuptrue%
\isacommand{lemma}\isamarkupfalse%
\ sumaConj{\isacharunderscore}acota{\isacharcolon}\ \isanewline
\ \ {\isachardoublequoteopen}finite\ S\ {\isasymLongrightarrow}\ {\isasymforall}x{\isasymin}S{\isachardot}\ x\ {\isasymle}\ sumaConj\ S{\isachardoublequoteclose}\isanewline
%
\isadelimproof
%
\endisadelimproof
%
\isatagproof
\isacommand{proof}\isamarkupfalse%
\ {\isacharparenleft}induct\ rule{\isacharcolon}\ finite{\isacharunderscore}induct{\isacharparenright}\isanewline
\ \ \isacommand{show}\isamarkupfalse%
\ {\isachardoublequoteopen}{\isasymforall}x\ {\isasymin}\ {\isacharbraceleft}{\isacharbraceright}{\isachardot}\ x\ {\isasymle}\ sumaConj\ {\isacharbraceleft}{\isacharbraceright}{\isachardoublequoteclose}\ \isacommand{by}\isamarkupfalse%
\ simp\isanewline
\isacommand{next}\isamarkupfalse%
\isanewline
\ \ \isacommand{fix}\isamarkupfalse%
\ x\ \isakeyword{and}\ F\isanewline
\ \ \isacommand{assume}\isamarkupfalse%
\ fF{\isacharcolon}\ {\isachardoublequoteopen}finite\ F{\isachardoublequoteclose}\ \isanewline
\ \ \ \ \ \isakeyword{and}\ xF{\isacharcolon}\ {\isachardoublequoteopen}x\ {\isasymnotin}\ F{\isachardoublequoteclose}\ \isanewline
\ \ \ \ \ \isakeyword{and}\ HI{\isacharcolon}\ {\isachardoublequoteopen}{\isasymforall}\ x{\isasymin}F{\isachardot}\ x\ {\isasymle}\ sumaConj\ F{\isachardoublequoteclose}\isanewline
\ \ \isacommand{show}\isamarkupfalse%
\ {\isachardoublequoteopen}{\isasymforall}y\ {\isasymin}\ insert\ x\ F{\isachardot}\ y\ {\isasymle}\ sumaConj\ {\isacharparenleft}insert\ x\ F{\isacharparenright}{\isachardoublequoteclose}\isanewline
\ \ \isacommand{proof}\isamarkupfalse%
\ \isanewline
\ \ \ \ \isacommand{fix}\isamarkupfalse%
\ y\ \isanewline
\ \ \ \ \isacommand{assume}\isamarkupfalse%
\ {\isachardoublequoteopen}y\ {\isasymin}\ insert\ x\ F{\isachardoublequoteclose}\isanewline
\ \ \ \ \isacommand{show}\isamarkupfalse%
\ {\isachardoublequoteopen}y\ {\isasymle}\ sumaConj\ {\isacharparenleft}insert\ x\ F{\isacharparenright}{\isachardoublequoteclose}\isanewline
\ \ \ \ \isacommand{proof}\isamarkupfalse%
\ {\isacharparenleft}cases\ {\isachardoublequoteopen}y\ {\isacharequal}\ x{\isachardoublequoteclose}{\isacharparenright}\isanewline
\ \ \ \ \ \ \isacommand{assume}\isamarkupfalse%
\ {\isachardoublequoteopen}y\ {\isacharequal}\ x{\isachardoublequoteclose}\isanewline
\ \ \ \ \ \ \isacommand{then}\isamarkupfalse%
\ \isacommand{have}\isamarkupfalse%
\ {\isachardoublequoteopen}y\ {\isasymle}\ x\ {\isacharplus}\ {\isacharparenleft}sumaConj\ F{\isacharparenright}{\isachardoublequoteclose}\ \isacommand{by}\isamarkupfalse%
\ simp\isanewline
\ \ \ \ \ \ \isacommand{also}\isamarkupfalse%
\ \isacommand{have}\isamarkupfalse%
\ {\isachardoublequoteopen}{\isasymdots}\ {\isacharequal}\ sumaConj\ {\isacharparenleft}insert\ x\ F{\isacharparenright}{\isachardoublequoteclose}\ \isanewline
\ \ \ \ \ \ \ \ \isacommand{by}\isamarkupfalse%
\ {\isacharparenleft}simp\ add{\isacharcolon}\ fF\ sumaConj{\isacharunderscore}def\ xF{\isacharparenright}\ \isanewline
\ \ \ \ \ \ \isacommand{finally}\isamarkupfalse%
\ \isacommand{show}\isamarkupfalse%
\ {\isacharquery}thesis\ \isacommand{{\isachardot}}\isamarkupfalse%
\isanewline
\ \ \ \ \isacommand{next}\isamarkupfalse%
\isanewline
\ \ \ \ \ \ \isacommand{assume}\isamarkupfalse%
\ {\isachardoublequoteopen}y\ {\isasymnoteq}\ x{\isachardoublequoteclose}\isanewline
\ \ \ \ \ \ \isacommand{then}\isamarkupfalse%
\ \isacommand{have}\isamarkupfalse%
\ {\isachardoublequoteopen}y\ {\isasymin}\ F{\isachardoublequoteclose}\ \isacommand{using}\isamarkupfalse%
\ {\isacharbackquoteopen}y\ {\isasymin}\ insert\ x\ F{\isacharbackquoteclose}\ \isacommand{by}\isamarkupfalse%
\ simp\isanewline
\ \ \ \ \ \ \isacommand{then}\isamarkupfalse%
\ \isacommand{have}\isamarkupfalse%
\ {\isachardoublequoteopen}y\ {\isasymle}\ sumaConj\ F{\isachardoublequoteclose}\ \isacommand{using}\isamarkupfalse%
\ HI\ \isacommand{by}\isamarkupfalse%
\ simp\isanewline
\ \ \ \ \ \ \isacommand{also}\isamarkupfalse%
\ \isacommand{have}\isamarkupfalse%
\ {\isachardoublequoteopen}{\isasymdots}\ {\isasymle}\ x\ {\isacharplus}\ {\isacharparenleft}sumaConj\ F{\isacharparenright}{\isachardoublequoteclose}\ \isacommand{by}\isamarkupfalse%
\ simp\isanewline
\ \ \ \ \ \ \isacommand{also}\isamarkupfalse%
\ \isacommand{have}\isamarkupfalse%
\ {\isachardoublequoteopen}{\isasymdots}\ {\isacharequal}\ sumaConj\ {\isacharparenleft}insert\ x\ F{\isacharparenright}{\isachardoublequoteclose}\ \isacommand{using}\isamarkupfalse%
\ fF\ xF\isanewline
\ \ \ \ \ \ \ \ \isacommand{by}\isamarkupfalse%
\ {\isacharparenleft}simp\ add{\isacharcolon}\ sumaConj{\isacharunderscore}def{\isacharparenright}\isanewline
\ \ \ \ \ \ \isacommand{finally}\isamarkupfalse%
\ \isacommand{show}\isamarkupfalse%
\ {\isacharquery}thesis\ \isacommand{{\isachardot}}\isamarkupfalse%
\isanewline
\ \ \ \ \isacommand{qed}\isamarkupfalse%
\isanewline
\ \ \isacommand{qed}\isamarkupfalse%
\isanewline
\isacommand{qed}\isamarkupfalse%
%
\endisatagproof
{\isafoldproof}%
%
\isadelimproof
%
\endisadelimproof
%
\begin{isamarkuptext}%
En esta última demostración hemos usado el método de prueba por
 casos,el método blast y también el simp("simplificador") añadiéndole 
\isa{sumaConj{\isacharunderscore}def}.%
\end{isamarkuptext}\isamarkuptrue%
%
\isadelimtheory
%
\endisadelimtheory
%
\isatagtheory
%
\endisatagtheory
{\isafoldtheory}%
%
\isadelimtheory
%
\endisadelimtheory
%
\end{isabellebody}%
\endinput
%:%file=~/Escritorio/TFG/ConjuntosFinitos.thy%:%
%:%6=1%:%
%:%20=9%:%
%:%24=11%:%
%:%28=13%:%
%:%37=15%:%
%:%49=18%:%
%:%50=19%:%
%:%51=20%:%
%:%52=21%:%
%:%53=22%:%
%:%54=23%:%
%:%55=24%:%
%:%56=25%:%
%:%57=26%:%
%:%58=27%:%
%:%59=28%:%
%:%60=29%:%
%:%61=30%:%
%:%62=31%:%
%:%63=32%:%
%:%64=33%:%
%:%65=34%:%
%:%66=35%:%
%:%67=36%:%
%:%68=37%:%
%:%69=38%:%
%:%70=39%:%
%:%71=40%:%
%:%72=41%:%
%:%73=42%:%
%:%74=43%:%
%:%75=44%:%
%:%76=45%:%
%:%77=46%:%
%:%78=47%:%
%:%79=48%:%
%:%80=49%:%
%:%81=50%:%
%:%82=51%:%
%:%83=52%:%
%:%84=53%:%
%:%85=54%:%
%:%86=55%:%
%:%87=56%:%
%:%88=57%:%
%:%89=58%:%
%:%90=59%:%
%:%91=60%:%
%:%92=61%:%
%:%93=62%:%
%:%94=63%:%
%:%95=64%:%
%:%96=65%:%
%:%97=66%:%
%:%98=67%:%
%:%99=68%:%
%:%108=71%:%
%:%120=73%:%
%:%121=74%:%
%:%122=75%:%
%:%123=76%:%
%:%125=79%:%
%:%126=79%:%
%:%127=80%:%
%:%129=82%:%
%:%131=85%:%
%:%132=85%:%
%:%135=86%:%
%:%136=87%:%
%:%140=87%:%
%:%150=89%:%
%:%152=91%:%
%:%153=91%:%
%:%154=92%:%
%:%157=93%:%
%:%161=93%:%
%:%162=93%:%
%:%171=96%:%
%:%172=97%:%
%:%173=98%:%
%:%174=99%:%
%:%175=100%:%
%:%176=101%:%
%:%177=102%:%
%:%178=103%:%
%:%179=104%:%
%:%180=105%:%
%:%181=106%:%
%:%182=107%:%
%:%183=108%:%
%:%192=111%:%
%:%204=113%:%
%:%206=115%:%
%:%207=115%:%
%:%210=116%:%
%:%214=116%:%
%:%215=116%:%
%:%216=117%:%
%:%217=117%:%
%:%218=118%:%
%:%219=118%:%
%:%220=119%:%
%:%230=121%:%
%:%231=122%:%
%:%232=123%:%
%:%233=124%:%
%:%234=125%:%
%:%243=128%:%
%:%255=130%:%
%:%257=132%:%
%:%258=132%:%
%:%261=133%:%
%:%265=133%:%
%:%266=133%:%
%:%267=134%:%
%:%281=136%:%
%:%293=138%:%
%:%295=140%:%
%:%296=140%:%
%:%297=141%:%
%:%304=142%:%
%:%305=142%:%
%:%306=143%:%
%:%307=143%:%
%:%308=143%:%
%:%309=144%:%
%:%310=144%:%
%:%311=145%:%
%:%312=145%:%
%:%313=146%:%
%:%314=146%:%
%:%315=147%:%
%:%316=148%:%
%:%317=149%:%
%:%318=149%:%
%:%319=150%:%
%:%320=150%:%
%:%321=151%:%
%:%322=151%:%
%:%323=152%:%
%:%324=152%:%
%:%325=153%:%
%:%326=153%:%
%:%327=154%:%
%:%328=154%:%
%:%329=155%:%
%:%330=155%:%
%:%331=156%:%
%:%332=156%:%
%:%333=156%:%
%:%334=156%:%
%:%335=157%:%
%:%336=157%:%
%:%337=157%:%
%:%338=158%:%
%:%339=158%:%
%:%340=159%:%
%:%341=159%:%
%:%342=159%:%
%:%343=159%:%
%:%344=160%:%
%:%345=160%:%
%:%346=161%:%
%:%347=161%:%
%:%348=162%:%
%:%349=162%:%
%:%350=162%:%
%:%351=162%:%
%:%352=162%:%
%:%353=163%:%
%:%354=163%:%
%:%355=163%:%
%:%356=163%:%
%:%357=163%:%
%:%358=164%:%
%:%359=164%:%
%:%360=164%:%
%:%361=164%:%
%:%362=165%:%
%:%363=165%:%
%:%364=165%:%
%:%365=165%:%
%:%366=166%:%
%:%367=166%:%
%:%368=167%:%
%:%369=167%:%
%:%370=167%:%
%:%371=167%:%
%:%372=168%:%
%:%373=168%:%
%:%374=169%:%
%:%375=169%:%
%:%376=170%:%
%:%386=173%:%
%:%387=174%:%
%:%388=175%:%