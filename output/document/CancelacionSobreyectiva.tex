%
\begin{isabellebody}%
\setisabellecontext{CancelacionSobreyectiva}%
%
\isadelimtheory
\isanewline
%
\endisadelimtheory
%
\isatagtheory
%
\endisatagtheory
{\isafoldtheory}%
%
\isadelimtheory
%
\endisadelimtheory
%
\begin{isamarkuptext}%
\comentario{Estructurar en secciones.}%
\end{isamarkuptext}\isamarkuptrue%
%
\begin{isamarkuptext}%
\comentario{Hacer demostraciones detalladas.}%
\end{isamarkuptext}\isamarkuptrue%
%
\begin{isamarkuptext}%
\comentario{Añadir lemas usados al Soporte.}%
\end{isamarkuptext}\isamarkuptrue%
%
\begin{isamarkuptext}%
El siguiente teorema prueba una caracterización de las funciones
 sobreyectivas, en otras palabras, las funciones sobreyectivas son
 epimorfismos en la categoría de conjuntos. Donde un epimorfismo es un
 homomorfismo sobreyectivo y la categoría de conjuntos es la categoría
 donde los objetos son conjuntos.


\begin {teorema}
  f es sobreyectiva si y solo si  para todas funciones g y h tal que 
$g \circ f  = h \circ f$ se tiene que g = h.
\end {teorema}
 
El teorema lo podemos dividir en dos lemas, ya que el teorema se
 demuestra por una doble implicación, luego vamos a dividir el teorema
 en las dos implicaciones.

\begin {lema}
  f es sobreyectiva entonces  para todas funciones g y h tal que 
$g \circ f = h \circ f$ se tiene que $g = h$.
\end {lema}
\begin {demostracion}
\begin {itemize}
\item Supongamos que tenemos que $g \circ  f = h \circ f$, queremos
 probar que $g = h.$ Usando la definición de sobreyectividad
 $(\forall y \in Y,  \exists x | y = f(x))$ y nuestra hipótesis,
 tenemos que: $$g(y) = g(f(x)) = (g \circ f) (x) = (h \circ f) (x) =
 h(f(x)) = h(y).$$
\item Supongamos que $g = h$, hay que probar que
 $g \circ f = h \circ f.$ Usando nuestra hipótesis, tenemos que:
$$ (g \circ f)(x) = g(f(x)) = h(f(x)) = (h \circ f) (x).$$
\end {itemize}
(*<*).(*>*)
\end {demostracion}

\begin {lema}
 Si  para todas funciones g y h tal que $g \circ f  = h \circ f$ se 
tiene que g = h entonces f es sobreyectiva.
\end {lema}

\begin {demostracion}
Para la demostración del ejercicios, primero debemos señalar los
 dominios y codominios de las funciones que vamos a usar.
 $f : C \longrightarrow A,$ $g,h: A \longrightarrow B.$ También debemos
 notar que nuestro conjunto  $B$ tiene que tener almenos dos elementos
 diferentes, supongamos que $B = \{a,b\}.$ \\
La prueba la vamos a realizar por reducción al absurdo. Luego supongamos
que nuestra función $f$ no es sobreyectiva, es decir, $\exists y_{1} \in
 A \ \isa{tal\ que} \  \nexists x \in C \ : f(x) = y.$ \\
Definamos ahora las funciones $g,h:$
$$g(y) = a \  \forall y \in A$$
$$h(y) = a  \ \isa{si} \  y \neq y_{1} \ h(y) =  b \ 
 \isa{si} \  y =  y_{1}$$

Entonces sabemos que $g(y) \neq h(y)  \forall y \in A.$ Sin embargo,
 por hipótesis tenemos que si $g \circ f = h \circ f$, lo cual es
 cierto, se tiene que $h = g.$ Por lo que hemos llegado a una
 contradicción, entonces $f$ es sobreyectiva.
\end {demostracion}


Su especificación es la siguiente, que la dividiremos en dos al igual que 
en la demostración a mano:%
\end{isamarkuptext}\isamarkuptrue%
\isacommand{theorem}\isamarkupfalse%
\isanewline
\ {\isachardoublequoteopen}surj\ f\ {\isasymlongleftrightarrow}\ {\isacharparenleft}{\isasymforall}g\ h{\isachardot}{\isacharparenleft}g\ {\isasymcirc}\ f\ {\isacharequal}\ h\ {\isasymcirc}\ f{\isacharparenright}\ {\isasymlongrightarrow}\ {\isacharparenleft}g\ {\isacharequal}\ h{\isacharparenright}{\isacharparenright}{\isachardoublequoteclose}\isanewline
%
\isadelimproof
\ \ %
\endisadelimproof
%
\isatagproof
\isacommand{oops}\isamarkupfalse%
%
\endisatagproof
{\isafoldproof}%
%
\isadelimproof
\isanewline
%
\endisadelimproof
\isanewline
\isacommand{lemma}\isamarkupfalse%
\ \isanewline
{\isachardoublequoteopen}surj\ f\ {\isasymLongrightarrow}\ \ {\isacharparenleft}{\isasymforall}g\ h{\isachardot}\ {\isacharparenleft}g\ {\isasymcirc}\ f\ {\isacharequal}\ h\ {\isasymcirc}\ f{\isacharparenright}\ {\isasymlongrightarrow}\ {\isacharparenleft}g\ {\isacharequal}\ h{\isacharparenright}{\isacharparenright}{\isachardoublequoteclose}\isanewline
%
\isadelimproof
\ \ %
\endisadelimproof
%
\isatagproof
\isacommand{oops}\isamarkupfalse%
%
\endisatagproof
{\isafoldproof}%
%
\isadelimproof
\isanewline
%
\endisadelimproof
\isanewline
\isacommand{lemma}\isamarkupfalse%
\ \isanewline
{\isachardoublequoteopen}{\isasymforall}g\ h{\isachardot}\ {\isacharparenleft}g\ {\isasymcirc}\ f\ {\isacharequal}\ h\ {\isasymcirc}\ f\ {\isasymlongrightarrow}\ g\ {\isacharequal}\ h{\isacharparenright}\ {\isasymlongrightarrow}\ surj\ f{\isachardoublequoteclose}\isanewline
%
\isadelimproof
\ \ %
\endisadelimproof
%
\isatagproof
\isacommand{oops}\isamarkupfalse%
%
\endisatagproof
{\isafoldproof}%
%
\isadelimproof
%
\endisadelimproof
%
\begin{isamarkuptext}%
En la especificación anterior, \isa{surj\ f} es una abreviatura de 
  \isa{range\ f\ {\isacharequal}\ UNIV}, donde \isa{range\ f} es el rango o imagen
de la función f.
 Por otra parte, \isa{UNIV} es el conjunto universal definido en la 
  teoría \href{http://bit.ly/2XtHCW6}{Set.thy} como una abreviatura de 
  \isa{top} que, a su vez está definido en la teoría 
  \href{http://bit.ly/2Xyj9Pe}{Orderings.thy} mediante la siguiente
  propiedad 
  \begin{itemize}
    \item[] \isa{\mbox{}\inferrule{\mbox{ordering{\isacharunderscore}top\ less{\isacharunderscore}eq\ less\ top}}{\mbox{less{\isacharunderscore}eq\ a\ top}}} 
      \hfill (\isa{ordering{\isacharunderscore}top{\isachardot}extremum})
  \end{itemize} 
Además queda añadir que la teoría donde se encuentra definido
 \isa{surj\ f} es en \href{http://bit.ly/2XuPQx5}{Fun.thy}. Esta
 teoría contiene la definicion \isa{surj{\isacharunderscore}def}.
 \begin{itemize}
    \item[] \isa{surj\ f\ {\isacharequal}\ {\isacharparenleft}{\isasymforall}y{\isachardot}\ {\isasymexists}x{\isachardot}\ y\ {\isacharequal}\ f\ x{\isacharparenright}}
 \hfill (\isa{inj{\isacharunderscore}on{\isacharunderscore}def})
  \end{itemize} 

Presentaremos distintas demostraciones de los lemas. Las primeras son
 las detalladas:%
\end{isamarkuptext}\isamarkuptrue%
\isanewline
\isacommand{lemma}\isamarkupfalse%
\ sobreyectivadetallada{\isacharcolon}\isanewline
\ \ \isakeyword{assumes}\ {\isachardoublequoteopen}surj\ f{\isachardoublequoteclose}\ \isanewline
\ \ \isakeyword{shows}\ {\isachardoublequoteopen}{\isasymforall}g\ h{\isachardot}\ {\isacharparenleft}\ g\ {\isasymcirc}\ f\ {\isacharequal}\ h\ {\isasymcirc}\ f\ {\isacharparenright}\ {\isasymlongrightarrow}\ {\isacharparenleft}g\ {\isacharequal}\ h{\isacharparenright}{\isachardoublequoteclose}\isanewline
%
\isadelimproof
%
\endisadelimproof
%
\isatagproof
\isacommand{proof}\isamarkupfalse%
\ {\isacharparenleft}rule\ allI{\isacharparenright}\isanewline
\ \ \isacommand{fix}\isamarkupfalse%
\ g\ {\isacharcolon}{\isacharcolon}\ {\isachardoublequoteopen}{\isacharprime}a\ {\isasymRightarrow}{\isacharprime}c{\isachardoublequoteclose}\ \isanewline
\ \ \isacommand{show}\isamarkupfalse%
\ {\isachardoublequoteopen}{\isasymforall}h{\isachardot}\ {\isacharparenleft}g\ {\isasymcirc}\ f\ {\isacharequal}\ h\ {\isasymcirc}\ f{\isacharparenright}\ {\isasymlongrightarrow}\ {\isacharparenleft}g\ {\isacharequal}\ h{\isacharparenright}{\isachardoublequoteclose}\isanewline
\ \ \isacommand{proof}\isamarkupfalse%
\ {\isacharparenleft}rule\ allI{\isacharparenright}\isanewline
\ \ \ \ \isacommand{fix}\isamarkupfalse%
\ h\isanewline
\ \ \ \ \isacommand{show}\isamarkupfalse%
\ {\isachardoublequoteopen}{\isacharparenleft}g\ {\isasymcirc}\ f\ {\isacharequal}\ h\ {\isasymcirc}\ f{\isacharparenright}\ {\isasymlongrightarrow}\ {\isacharparenleft}g\ {\isacharequal}\ h{\isacharparenright}{\isachardoublequoteclose}\ \isanewline
\ \ \ \ \isacommand{proof}\isamarkupfalse%
\ {\isacharparenleft}rule\ impI{\isacharparenright}\isanewline
\ \ \ \ \ \ \isacommand{assume}\isamarkupfalse%
\ {\isadigit{1}}{\isacharcolon}\ {\isachardoublequoteopen}g\ {\isasymcirc}\ f\ {\isacharequal}\ h\ {\isasymcirc}\ f{\isachardoublequoteclose}\isanewline
\ \ \ \ \ \ \isacommand{show}\isamarkupfalse%
\ {\isachardoublequoteopen}g\ {\isacharequal}\ h{\isachardoublequoteclose}\isanewline
\ \ \ \ \ \ \isacommand{proof}\isamarkupfalse%
\ \ \isanewline
\ \ \ \ \ \ \ \ \isacommand{fix}\isamarkupfalse%
\ x\isanewline
\ \ \ \ \ \ \ \ \isacommand{have}\isamarkupfalse%
\ {\isachardoublequoteopen}\ {\isasymexists}y\ {\isachardot}\ x\ {\isacharequal}\ f{\isacharparenleft}y{\isacharparenright}{\isachardoublequoteclose}\ \isacommand{using}\isamarkupfalse%
\ assms\ \isacommand{by}\isamarkupfalse%
\ {\isacharparenleft}simp\ add{\isacharcolon}surj{\isacharunderscore}def{\isacharparenright}\isanewline
\ \ \ \ \ \ \ \ \isacommand{then}\isamarkupfalse%
\ \isacommand{obtain}\isamarkupfalse%
\ \ {\isachardoublequoteopen}y{\isachardoublequoteclose}\ \isakeyword{where}\ {\isadigit{2}}{\isacharcolon}{\isachardoublequoteopen}x\ {\isacharequal}\ f{\isacharparenleft}y{\isacharparenright}{\isachardoublequoteclose}\ \isacommand{by}\isamarkupfalse%
\ {\isacharparenleft}rule\ exE{\isacharparenright}\isanewline
\ \ \ \ \ \ \ \ \isacommand{then}\isamarkupfalse%
\ \isacommand{have}\isamarkupfalse%
\ \ {\isachardoublequoteopen}g{\isacharparenleft}x{\isacharparenright}\ {\isacharequal}\ g{\isacharparenleft}f{\isacharparenleft}y{\isacharparenright}{\isacharparenright}{\isachardoublequoteclose}\ \isacommand{by}\isamarkupfalse%
\ simp\isanewline
\ \ \ \ \ \ \ \ \isacommand{also}\isamarkupfalse%
\ \isacommand{have}\isamarkupfalse%
\ \ {\isachardoublequoteopen}{\isachardot}{\isachardot}{\isachardot}\ {\isacharequal}\ {\isacharparenleft}g\ {\isasymcirc}\ f{\isacharparenright}\ {\isacharparenleft}y{\isacharparenright}\ \ {\isachardoublequoteclose}\ \isacommand{by}\isamarkupfalse%
\ simp\isanewline
\ \ \ \ \ \ \ \ \isacommand{also}\isamarkupfalse%
\ \isacommand{have}\isamarkupfalse%
\ \ {\isachardoublequoteopen}{\isachardot}{\isachardot}{\isachardot}\ {\isacharequal}\ {\isacharparenleft}h\ {\isasymcirc}\ f{\isacharparenright}\ {\isacharparenleft}y{\isacharparenright}{\isachardoublequoteclose}\ \isacommand{using}\isamarkupfalse%
\ {\isadigit{1}}\ \isacommand{by}\isamarkupfalse%
\ simp\isanewline
\ \ \ \ \ \ \ \ \isacommand{also}\isamarkupfalse%
\ \isacommand{have}\isamarkupfalse%
\ \ {\isachardoublequoteopen}{\isachardot}{\isachardot}{\isachardot}\ {\isacharequal}\ h{\isacharparenleft}f{\isacharparenleft}y{\isacharparenright}{\isacharparenright}{\isachardoublequoteclose}\ \isacommand{by}\isamarkupfalse%
\ simp\isanewline
\ \ \ \ \ \ \ \ \isacommand{also}\isamarkupfalse%
\ \isacommand{have}\isamarkupfalse%
\ \ {\isachardoublequoteopen}{\isachardot}{\isachardot}{\isachardot}\ {\isacharequal}\ h{\isacharparenleft}x{\isacharparenright}{\isachardoublequoteclose}\ \isacommand{using}\isamarkupfalse%
\ {\isadigit{2}}\ \ \ \isacommand{by}\isamarkupfalse%
\ {\isacharparenleft}simp\ add{\isacharcolon}\ {\isacartoucheopen}x\ {\isacharequal}\ f\ y{\isacartoucheclose}{\isacharparenright}\isanewline
\ \ \ \ \ \ \ \ \isacommand{finally}\isamarkupfalse%
\ \isacommand{show}\isamarkupfalse%
\ \ {\isachardoublequoteopen}\ g{\isacharparenleft}x{\isacharparenright}\ {\isacharequal}\ h{\isacharparenleft}x{\isacharparenright}\ {\isachardoublequoteclose}\ \isacommand{by}\isamarkupfalse%
\ simp\isanewline
\ \ \ \ \ \ \isacommand{qed}\isamarkupfalse%
\isanewline
\ \ \ \ \isacommand{qed}\isamarkupfalse%
\isanewline
\ \ \isacommand{qed}\isamarkupfalse%
\isanewline
\isacommand{qed}\isamarkupfalse%
%
\endisatagproof
{\isafoldproof}%
%
\isadelimproof
\isanewline
%
\endisadelimproof
\isanewline
\isanewline
\isacommand{lemma}\isamarkupfalse%
\ sobreyectivadetallada{\isadigit{2}}{\isacharcolon}\isanewline
\ \ \isakeyword{fixes}\ f\ {\isacharcolon}{\isacharcolon}\ {\isachardoublequoteopen}{\isacharprime}c\ {\isasymRightarrow}\ {\isacharprime}a{\isachardoublequoteclose}\ \isanewline
\ \ \isakeyword{assumes}\ {\isachardoublequoteopen}{\isasymforall}{\isacharparenleft}g\ {\isacharcolon}{\isacharcolon}\ {\isacharprime}a\ {\isasymRightarrow}\ {\isacharprime}b{\isacharparenright}\ {\isacharparenleft}h\ {\isacharcolon}{\isacharcolon}\ {\isacharprime}a\ {\isasymRightarrow}\ {\isacharprime}b{\isacharparenright}{\isachardot}\ {\isacharparenleft}\ g\ {\isasymcirc}\ f\ {\isacharequal}\ h\ {\isasymcirc}\ f\ {\isacharparenright}\ {\isasymlongrightarrow}\ {\isacharparenleft}g\ {\isacharequal}\ h{\isacharparenright}{\isachardoublequoteclose}\isanewline
\ \ \isakeyword{shows}\ {\isachardoublequoteopen}surj\ f{\isachardoublequoteclose}\isanewline
%
\isadelimproof
%
\endisadelimproof
%
\isatagproof
\isacommand{proof}\isamarkupfalse%
\ {\isacharparenleft}rule\ surjI{\isacharparenright}\isanewline
\ \ \isacommand{assume}\isamarkupfalse%
\ {\isadigit{1}}{\isacharcolon}{\isachardoublequoteopen}\ {\isasymnot}\ surj\ f{\isachardoublequoteclose}\isanewline
\ \ \isacommand{have}\isamarkupfalse%
\ {\isachardoublequoteopen}\ {\isasymnot}{\isacharparenleft}{\isasymforall}y{\isachardot}\ {\isasymexists}x{\isachardot}\ y\ {\isacharequal}\ f\ x{\isacharparenright}{\isachardoublequoteclose}\ \isacommand{using}\isamarkupfalse%
\ {\isadigit{1}}\ \isacommand{by}\isamarkupfalse%
\ {\isacharparenleft}simp\ add{\isacharcolon}\ surj{\isacharunderscore}def{\isacharparenright}\isanewline
\ \ \isacommand{then}\isamarkupfalse%
\ \isacommand{have}\isamarkupfalse%
\ {\isachardoublequoteopen}{\isasymexists}y{\isachardot}\ {\isasymnexists}x{\isachardot}\ y\ {\isacharequal}\ f\ x{\isachardoublequoteclose}\ \isacommand{by}\isamarkupfalse%
\ simp\isanewline
\ \ \isacommand{then}\isamarkupfalse%
\ \isacommand{obtain}\isamarkupfalse%
\ y{\isadigit{1}}\ \isakeyword{where}\ {\isachardoublequoteopen}{\isasymnexists}x{\isachardot}\ y{\isadigit{1}}\ {\isacharequal}\ f\ x{\isachardoublequoteclose}\ \isacommand{by}\isamarkupfalse%
\ {\isacharparenleft}rule\ exE{\isacharparenright}\isanewline
\ \ \isacommand{then}\isamarkupfalse%
\ \isacommand{have}\isamarkupfalse%
\ {\isachardoublequoteopen}{\isasymforall}x{\isachardot}\ y{\isadigit{1}}\ {\isasymnoteq}\ f\ x{\isachardoublequoteclose}\ \ \isacommand{by}\isamarkupfalse%
\ simp\isanewline
\ \ \isacommand{let}\isamarkupfalse%
\ {\isacharquery}g\ {\isacharequal}\ {\isachardoublequoteopen}{\isasymlambda}x\ {\isacharcolon}{\isacharcolon}\ {\isacharprime}a{\isachardot}\ a\ {\isacharcolon}{\isacharcolon}\ {\isacharprime}b{\isachardoublequoteclose}\ \isanewline
\ \ \isacommand{let}\isamarkupfalse%
\ {\isacharquery}h\ {\isacharequal}{\isachardoublequoteopen}\ fun{\isacharunderscore}upd\ {\isacharquery}g\ y{\isadigit{1}}\ {\isacharparenleft}b\ {\isacharcolon}{\isacharcolon}\ {\isacharprime}b{\isacharparenright}{\isachardoublequoteclose}\isanewline
\ \ \isacommand{have}\isamarkupfalse%
\ {\isadigit{2}}{\isacharcolon}{\isachardoublequoteopen}{\isacharquery}g\ {\isasymcirc}\ f\ {\isacharequal}\ {\isacharquery}h\ {\isasymcirc}\ f\ {\isasymlongrightarrow}\ {\isacharquery}g\ {\isacharequal}\ {\isacharquery}h{\isachardoublequoteclose}\ \isacommand{using}\isamarkupfalse%
\ assms\ \isacommand{by}\isamarkupfalse%
\ blast\isanewline
\ \ \isacommand{have}\isamarkupfalse%
\ {\isadigit{3}}{\isacharcolon}{\isachardoublequoteopen}{\isacharquery}g\ {\isasymcirc}\ f\ {\isacharequal}\ {\isacharquery}h\ {\isasymcirc}\ f{\isachardoublequoteclose}\ \isanewline
\ \ \ \ \isacommand{by}\isamarkupfalse%
\ {\isacharparenleft}metis\ {\isacharparenleft}mono{\isacharunderscore}tags{\isacharcomma}\ lifting{\isacharparenright}\ fun{\isacharunderscore}upd{\isacharunderscore}def\ {\isacartoucheopen}{\isasymnexists}x\ {\isacharcolon}{\isacharcolon}\ {\isacharprime}c{\isachardot}\ {\isacharparenleft}y{\isadigit{1}}\ {\isacharcolon}{\isacharcolon}\ {\isacharprime}a{\isacharparenright}\ {\isacharequal}\isanewline
\ {\isacharparenleft}f\ {\isacharcolon}{\isacharcolon}\ {\isacharprime}c\ {\isasymRightarrow}\ {\isacharprime}a{\isacharparenright}\ x{\isacartoucheclose}\ f{\isacharunderscore}inv{\isacharunderscore}into{\isacharunderscore}f\ fun{\isachardot}map{\isacharunderscore}cong{\isadigit{0}}{\isacharparenright}\isanewline
\ \ \isacommand{have}\isamarkupfalse%
\ {\isachardoublequoteopen}{\isacharquery}g\ {\isacharequal}\ {\isacharquery}h{\isachardoublequoteclose}\ \isacommand{using}\isamarkupfalse%
\ {\isadigit{2}}\ {\isadigit{3}}\ \isacommand{by}\isamarkupfalse%
\ {\isacharparenleft}rule\ mp{\isacharparenright}\isanewline
\ \ \isacommand{have}\isamarkupfalse%
\ {\isachardoublequoteopen}{\isacharquery}g\ {\isasymnoteq}\ {\isacharquery}h{\isachardoublequoteclose}\ \isanewline
\ \ \isacommand{proof}\isamarkupfalse%
\ \isanewline
\ \ \ \ \isacommand{assume}\isamarkupfalse%
\ {\isadigit{4}}{\isacharcolon}\ {\isachardoublequoteopen}{\isacharquery}g\ {\isacharequal}\ {\isacharquery}h{\isachardoublequoteclose}\isanewline
\ \ \ \ \isacommand{show}\isamarkupfalse%
\ False\isanewline
\ \ \ \ \isacommand{proof}\isamarkupfalse%
\ {\isacharminus}\isanewline
\ \ \ \ \ \ \isacommand{have}\isamarkupfalse%
\ {\isachardoublequoteopen}{\isacharquery}g\ {\isacharequal}\ fun{\isacharunderscore}upd\ {\isacharquery}g\ y{\isadigit{1}}\ {\isacharparenleft}b\ {\isacharcolon}{\isacharcolon}\ {\isacharprime}b{\isacharparenright}{\isachardoublequoteclose}\ \isacommand{using}\isamarkupfalse%
\ {\isadigit{4}}\ \isacommand{by}\isamarkupfalse%
\ simp\isanewline
\ \ \ \ \ \ \isacommand{also}\isamarkupfalse%
\ \isacommand{have}\isamarkupfalse%
\ {\isachardoublequoteopen}{\isachardot}{\isachardot}{\isachardot}\ {\isacharequal}\ \ {\isacharparenleft}{\isasymlambda}x{\isachardot}\ if\ x\ {\isacharequal}\ y{\isadigit{1}}\ then\ b\ else\ {\isacharquery}g\ x{\isacharparenright}{\isachardoublequoteclose}\ \ \isacommand{by}\isamarkupfalse%
\ {\isacharparenleft}simp\ add{\isacharcolon}\isanewline
fun{\isacharunderscore}upd{\isacharunderscore}def{\isacharparenright}\isanewline
\ \ \ \ \ \ \isacommand{finally}\isamarkupfalse%
\ \isacommand{have}\isamarkupfalse%
\ {\isadigit{5}}{\isacharcolon}\ {\isachardoublequoteopen}{\isacharquery}g\ {\isacharequal}\ {\isacharparenleft}{\isasymlambda}x{\isachardot}\ if\ x\ {\isacharequal}\ y{\isadigit{1}}\ then\ b\ else\ {\isacharquery}g\ x{\isacharparenright}{\isachardoublequoteclose}\ \isacommand{by}\isamarkupfalse%
\ simp\isanewline
\ \ \ \ \ \ \isacommand{show}\isamarkupfalse%
\ False\isanewline
\ \ \ \ \ \ \isacommand{proof}\isamarkupfalse%
\ {\isacharparenleft}cases{\isacharparenright}\isanewline
\ \ \ \ \ \ \ \ \isacommand{oops}\isamarkupfalse%
%
\endisatagproof
{\isafoldproof}%
%
\isadelimproof
%
\endisadelimproof
%
\begin{isamarkuptext}%
En la demostración hemos introducido: 
 \begin{itemize}
    \item[] \isa{\mbox{}\inferrule{\mbox{{\isasymexists}x\ {\isacharcolon}{\isacharcolon}\ {\isacharprime}a{\isachardot}\ {\isacharparenleft}P\ {\isacharcolon}{\isacharcolon}\ {\isacharprime}a\ {\isasymRightarrow}\ bool{\isacharparenright}\ x}\\\ \mbox{{\isasymAnd}x\ {\isacharcolon}{\isacharcolon}\ {\isacharprime}a{\isachardot}\ \mbox{}\inferrule{\mbox{P\ x}}{\mbox{Q\ {\isacharcolon}{\isacharcolon}\ bool}}}}{\mbox{Q}}} 
      \hfill (\isa{rule\ exE}) 
  \end{itemize} 
 \begin{itemize}
    \item[] \isa{{\isasymlbrakk}P\ {\isacharcolon}{\isacharcolon}\ bool\ {\isasymLongrightarrow}\ Q\ {\isacharcolon}{\isacharcolon}\ bool{\isacharsemicolon}\ Q\ {\isasymLongrightarrow}\ P{\isasymrbrakk}\ {\isasymLongrightarrow}\ P\ {\isacharequal}\ Q} 
      \hfill (\isa{iffI})
  \end{itemize} 

La demostración aplicativa es:%
\end{isamarkuptext}\isamarkuptrue%
\isacommand{lemma}\isamarkupfalse%
\ {\isachardoublequoteopen}surj\ f\ {\isasymLongrightarrow}\ {\isacharparenleft}{\isacharparenleft}g\ {\isasymcirc}\ f{\isacharparenright}\ {\isacharequal}\ {\isacharparenleft}h\ {\isasymcirc}\ f{\isacharparenright}\ {\isacharparenright}\ {\isasymlongrightarrow}\ {\isacharparenleft}g\ {\isacharequal}\ h{\isacharparenright}{\isachardoublequoteclose}\isanewline
%
\isadelimproof
\ \ %
\endisadelimproof
%
\isatagproof
\isacommand{apply}\isamarkupfalse%
\ {\isacharparenleft}simp\ add{\isacharcolon}\ surj{\isacharunderscore}def\ fun{\isacharunderscore}eq{\isacharunderscore}iff{\isacharparenright}\isanewline
\ \ \isacommand{apply}\isamarkupfalse%
\ metis\isanewline
\ \ \isacommand{done}\isamarkupfalse%
%
\endisatagproof
{\isafoldproof}%
%
\isadelimproof
\isanewline
%
\endisadelimproof
\isanewline
\isacommand{lemma}\isamarkupfalse%
\ {\isachardoublequoteopen}surj\ f\ {\isasymLongrightarrow}\ {\isacharparenleft}{\isacharparenleft}g\ {\isasymcirc}\ f{\isacharparenright}\ {\isacharequal}\ {\isacharparenleft}h\ {\isasymcirc}\ f{\isacharparenright}\ {\isacharparenright}\ {\isasymlongrightarrow}{\isacharparenleft}g\ {\isacharequal}\ h{\isacharparenright}{\isachardoublequoteclose}\isanewline
%
\isadelimproof
\ \ %
\endisadelimproof
%
\isatagproof
\isacommand{apply}\isamarkupfalse%
\ {\isacharparenleft}simp\ add{\isacharcolon}\ surj{\isacharunderscore}def\ fun{\isacharunderscore}eq{\isacharunderscore}iff\ {\isacharparenright}\ \isanewline
\ \ \isacommand{by}\isamarkupfalse%
\ metis%
\endisatagproof
{\isafoldproof}%
%
\isadelimproof
%
\endisadelimproof
%
\begin{isamarkuptext}%
En esta demostración hemos introducido:
 \begin{itemize}
    \item[] \isa{{\isacharparenleft}{\isacharparenleft}f\ {\isacharcolon}{\isacharcolon}\ {\isacharprime}a\ {\isasymRightarrow}\ {\isacharprime}b{\isacharparenright}\ {\isacharequal}\ {\isacharparenleft}g\ {\isacharcolon}{\isacharcolon}\ {\isacharprime}a\ {\isasymRightarrow}\ {\isacharprime}b{\isacharparenright}{\isacharparenright}\ {\isacharequal}\ {\isacharparenleft}{\isasymforall}x\ {\isacharcolon}{\isacharcolon}\ {\isacharprime}a{\isachardot}\ f\ x\ {\isacharequal}\ g\ x{\isacharparenright}} 
      \hfill (\isa{fun{\isacharunderscore}eq{\isacharunderscore}iff})
  \end{itemize}%
\end{isamarkuptext}\isamarkuptrue%
%
\isadelimtheory
%
\endisadelimtheory
%
\isatagtheory
%
\endisatagtheory
{\isafoldtheory}%
%
\isadelimtheory
%
\endisadelimtheory
%
\end{isabellebody}%
\endinput
%:%file=~/ownCloud/alonso/curso-TFG/Carlos/TFG_de_Carlos/CancelacionSobreyectiva.thy%:%
%:%6=1%:%
%:%20=9%:%
%:%24=11%:%
%:%28=13%:%
%:%32=16%:%
%:%33=17%:%
%:%34=18%:%
%:%35=19%:%
%:%36=20%:%
%:%37=21%:%
%:%38=22%:%
%:%39=23%:%
%:%40=24%:%
%:%41=25%:%
%:%42=26%:%
%:%43=27%:%
%:%44=28%:%
%:%45=29%:%
%:%46=30%:%
%:%47=31%:%
%:%48=32%:%
%:%49=33%:%
%:%50=34%:%
%:%51=35%:%
%:%52=36%:%
%:%53=37%:%
%:%54=38%:%
%:%55=39%:%
%:%56=40%:%
%:%57=41%:%
%:%58=42%:%
%:%59=43%:%
%:%60=44%:%
%:%61=45%:%
%:%62=46%:%
%:%63=47%:%
%:%64=48%:%
%:%65=49%:%
%:%66=50%:%
%:%67=51%:%
%:%68=52%:%
%:%69=53%:%
%:%70=54%:%
%:%71=55%:%
%:%72=56%:%
%:%73=57%:%
%:%74=58%:%
%:%75=59%:%
%:%76=60%:%
%:%77=61%:%
%:%78=62%:%
%:%79=63%:%
%:%80=64%:%
%:%81=65%:%
%:%82=66%:%
%:%83=67%:%
%:%84=68%:%
%:%85=69%:%
%:%86=70%:%
%:%87=71%:%
%:%88=72%:%
%:%89=73%:%
%:%90=74%:%
%:%91=75%:%
%:%92=76%:%
%:%93=77%:%
%:%95=80%:%
%:%96=80%:%
%:%97=81%:%
%:%100=82%:%
%:%104=82%:%
%:%110=82%:%
%:%113=83%:%
%:%114=84%:%
%:%115=84%:%
%:%116=85%:%
%:%119=86%:%
%:%123=86%:%
%:%129=86%:%
%:%132=87%:%
%:%133=88%:%
%:%134=88%:%
%:%135=89%:%
%:%138=90%:%
%:%142=90%:%
%:%152=94%:%
%:%153=95%:%
%:%154=96%:%
%:%155=97%:%
%:%156=98%:%
%:%157=99%:%
%:%158=100%:%
%:%159=101%:%
%:%160=102%:%
%:%161=103%:%
%:%162=104%:%
%:%163=105%:%
%:%164=106%:%
%:%165=107%:%
%:%166=108%:%
%:%167=109%:%
%:%168=110%:%
%:%169=111%:%
%:%170=112%:%
%:%171=113%:%
%:%172=114%:%
%:%173=115%:%
%:%175=119%:%
%:%176=120%:%
%:%177=120%:%
%:%178=121%:%
%:%179=122%:%
%:%186=123%:%
%:%187=123%:%
%:%188=124%:%
%:%189=124%:%
%:%190=125%:%
%:%191=125%:%
%:%192=126%:%
%:%193=126%:%
%:%194=127%:%
%:%195=127%:%
%:%196=128%:%
%:%197=128%:%
%:%198=129%:%
%:%199=129%:%
%:%200=130%:%
%:%201=130%:%
%:%202=131%:%
%:%203=131%:%
%:%204=132%:%
%:%205=132%:%
%:%206=133%:%
%:%207=133%:%
%:%208=134%:%
%:%209=134%:%
%:%210=134%:%
%:%211=134%:%
%:%212=135%:%
%:%213=135%:%
%:%214=135%:%
%:%215=135%:%
%:%216=136%:%
%:%217=136%:%
%:%218=136%:%
%:%219=136%:%
%:%220=137%:%
%:%221=137%:%
%:%222=137%:%
%:%223=137%:%
%:%224=138%:%
%:%225=138%:%
%:%226=138%:%
%:%227=138%:%
%:%228=138%:%
%:%229=139%:%
%:%230=139%:%
%:%231=139%:%
%:%232=139%:%
%:%233=140%:%
%:%234=140%:%
%:%235=140%:%
%:%236=140%:%
%:%237=140%:%
%:%238=141%:%
%:%239=141%:%
%:%240=141%:%
%:%241=141%:%
%:%242=142%:%
%:%243=142%:%
%:%244=143%:%
%:%245=143%:%
%:%246=144%:%
%:%247=144%:%
%:%248=145%:%
%:%254=145%:%
%:%257=146%:%
%:%258=147%:%
%:%259=148%:%
%:%260=148%:%
%:%261=149%:%
%:%262=150%:%
%:%263=151%:%
%:%270=152%:%
%:%271=152%:%
%:%272=153%:%
%:%273=153%:%
%:%274=154%:%
%:%275=154%:%
%:%276=154%:%
%:%277=154%:%
%:%278=155%:%
%:%279=155%:%
%:%280=155%:%
%:%281=155%:%
%:%282=156%:%
%:%283=156%:%
%:%284=156%:%
%:%285=156%:%
%:%286=157%:%
%:%287=157%:%
%:%288=157%:%
%:%289=157%:%
%:%290=158%:%
%:%291=158%:%
%:%292=159%:%
%:%293=159%:%
%:%294=160%:%
%:%295=160%:%
%:%296=160%:%
%:%297=160%:%
%:%298=161%:%
%:%299=161%:%
%:%300=162%:%
%:%301=162%:%
%:%302=163%:%
%:%303=164%:%
%:%304=164%:%
%:%305=164%:%
%:%306=164%:%
%:%307=165%:%
%:%308=165%:%
%:%309=166%:%
%:%310=166%:%
%:%311=167%:%
%:%312=167%:%
%:%313=168%:%
%:%314=168%:%
%:%315=169%:%
%:%316=169%:%
%:%317=170%:%
%:%318=170%:%
%:%319=170%:%
%:%320=170%:%
%:%321=171%:%
%:%322=171%:%
%:%323=171%:%
%:%324=171%:%
%:%325=172%:%
%:%326=173%:%
%:%327=173%:%
%:%328=173%:%
%:%329=173%:%
%:%330=174%:%
%:%331=174%:%
%:%332=175%:%
%:%333=175%:%
%:%334=176%:%
%:%344=180%:%
%:%345=181%:%
%:%346=182%:%
%:%347=183%:%
%:%348=184%:%
%:%349=185%:%
%:%350=186%:%
%:%351=187%:%
%:%352=188%:%
%:%353=189%:%
%:%354=190%:%
%:%356=192%:%
%:%357=192%:%
%:%360=193%:%
%:%364=193%:%
%:%365=193%:%
%:%366=194%:%
%:%367=194%:%
%:%368=195%:%
%:%374=195%:%
%:%377=196%:%
%:%378=197%:%
%:%379=197%:%
%:%382=198%:%
%:%386=198%:%
%:%387=198%:%
%:%388=199%:%
%:%389=199%:%
%:%398=202%:%
%:%399=203%:%
%:%400=204%:%
%:%401=205%:%
%:%402=206%:%