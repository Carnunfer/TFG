%
\begin{isabellebody}%
\setisabellecontext{CancelacionSobreyectiva}%
%
\isadelimtheory
\isanewline
%
\endisadelimtheory
%
\isatagtheory
%
\endisatagtheory
{\isafoldtheory}%
%
\isadelimtheory
%
\endisadelimtheory
%
\begin{isamarkuptext}%
\comentario{Estructurar en secciones.}%
\end{isamarkuptext}\isamarkuptrue%
%
\begin{isamarkuptext}%
\comentario{Hacer demostraciones detalladas.}%
\end{isamarkuptext}\isamarkuptrue%
%
\begin{isamarkuptext}%
\comentario{Añadir lemas usados al Soporte.}%
\end{isamarkuptext}\isamarkuptrue%
%
\isadelimdocument
%
\endisadelimdocument
%
\isatagdocument
%
\isamarkupsection{Demostración en Lenguaje natural%
}
\isamarkuptrue%
%
\endisatagdocument
{\isafolddocument}%
%
\isadelimdocument
%
\endisadelimdocument
%
\begin{isamarkuptext}%
El siguiente teorema prueba una caracterización de las funciones
 sobreyectivas. Primero se definirá el significado de la sobreyectividad
de una función y de la propiedad de ser cancelativa por la derecha. \\
Una función $f : A \longrightarrow B$ es sobreyectiva si 
$$\forall y \in B : \exists x \in A : f(x) = y$$
Una función $f : A \longrightarrow B$ tiene la propiedad de ser
 canletiva por la izquierda si: 
$$\forall C : (\forall g,h: B \longrightarrow C) : g \circ f = h \circ f
\Longrightarrow g = h$$

Luego el teorema es el siguiente: 

\begin {teorema}
  f es sobreyectiva si y solo si  para todas funciones g y h tal que 
$g \circ f  = h \circ f$ se tiene que g = h.
\end {teorema}
 
El teorema se puede dividir en dos lemas, ya que  se
 demuestra por una doble implicación.

\begin {lema}[Condición necesaria]
 Si $f$ es sobreyectiva entonces  para todas funciones g y h tal que 
$g \circ f = h \circ f$ se tiene que $g = h$.
\end {lema}

\begin {demostracion}
Supongamos que tenemos que $g \circ  f = h \circ f$, queremos
 probar que $g = h.$ Usando la definición de sobreyectividad
 $(\forall y \in Y,  \exists x \| y = f(x))$ y nuestra hipótesis,
 tenemos que: $$g(y) = g(f(x)) = (g \circ f) (x) = (h \circ f) (x) =
 h(f(x)) = h(y).$$
\end {demostracion}

\begin {lema}[Condición necesaria] 
 Si  para todas funciones g y h tal que $g \circ f  = h \circ f$ se 
tiene que g = h entonces f es sobreyectiva.
\end {lema}

\begin {demostracion}
Para la demostración del lema, primero se debe señalar los
 dominios y codominios de las funciones que se van a usar.
 $f : C \longrightarrow A,$ $g,h: A \longrightarrow B.$ También se debe
 notar que el conjunto  $B$ tiene que tener almenos dos elementos
 diferentes,luego supongamos que $B = \{a,b\}.$ \\
La prueba se va a realizar por reducción al absurdo. Luego supongamos
que nuestra función $f$ no es sobreyectiva, es decir, $\exists y_{1} \in
 A \ \isa{tal\ que} \  \nexists x \in C \ : f(x) = y.$ \\
Definamos ahora las funciones $g,h:$
$$g(y) = a \  \forall y \in A$$


$$h(y)= \left\{ \begin{array}{lcc}
             a &   si  & y \neq y_1 \\
             b &  si & y = y_1
             \end{array}
   \right.$$


Entonces $g(y) \neq h(y).$ Sin embargo,
 por hipótesis se tiene  que si $g \circ f = h \circ f$, lo cual es
 cierto, entonces $h = g.$ Por lo que hemos llegado a una
 contradicción, por lo tanto, $f$ es sobreyectiva.
\end {demostracion}%
\end{isamarkuptext}\isamarkuptrue%
%
\isadelimdocument
%
\endisadelimdocument
%
\isatagdocument
%
\isamarkupsection{Especificación en Isabelle/Hol%
}
\isamarkuptrue%
%
\endisatagdocument
{\isafolddocument}%
%
\isadelimdocument
%
\endisadelimdocument
%
\begin{isamarkuptext}%
Su especificación es la siguiente, que se dividira en dos al igual que 
en la demostración a mano:%
\end{isamarkuptext}\isamarkuptrue%
\isacommand{theorem}\isamarkupfalse%
\ caracterizacion{\isacharunderscore}funciones{\isacharunderscore}sobreyectivas{\isacharcolon}\isanewline
\ {\isachardoublequoteopen}surj\ f\ {\isasymlongleftrightarrow}\ {\isacharparenleft}{\isasymforall}g\ h{\isachardot}{\isacharparenleft}g\ {\isasymcirc}\ f\ {\isacharequal}\ h\ {\isasymcirc}\ f{\isacharparenright}\ {\isasymlongrightarrow}\ {\isacharparenleft}g\ {\isacharequal}\ h{\isacharparenright}{\isacharparenright}{\isachardoublequoteclose}\isanewline
%
\isadelimproof
\ \ %
\endisadelimproof
%
\isatagproof
\isacommand{oops}\isamarkupfalse%
%
\endisatagproof
{\isafoldproof}%
%
\isadelimproof
\isanewline
%
\endisadelimproof
\isanewline
\isacommand{lemma}\isamarkupfalse%
\ condicion{\isacharunderscore}suficiente{\isacharcolon}\isanewline
{\isachardoublequoteopen}surj\ f\ {\isasymLongrightarrow}\ \ {\isacharparenleft}{\isasymforall}g\ h{\isachardot}\ {\isacharparenleft}g\ {\isasymcirc}\ f\ {\isacharequal}\ h\ {\isasymcirc}\ f{\isacharparenright}\ {\isasymlongrightarrow}\ {\isacharparenleft}g\ {\isacharequal}\ h{\isacharparenright}{\isacharparenright}{\isachardoublequoteclose}\isanewline
%
\isadelimproof
\ \ %
\endisadelimproof
%
\isatagproof
\isacommand{oops}\isamarkupfalse%
%
\endisatagproof
{\isafoldproof}%
%
\isadelimproof
\isanewline
%
\endisadelimproof
\isanewline
\isacommand{lemma}\isamarkupfalse%
\ condicion{\isacharunderscore}necesaria{\isacharcolon}\isanewline
{\isachardoublequoteopen}{\isasymforall}g\ h{\isachardot}\ {\isacharparenleft}g\ {\isasymcirc}\ f\ {\isacharequal}\ h\ {\isasymcirc}\ f\ {\isasymlongrightarrow}\ g\ {\isacharequal}\ h{\isacharparenright}\ {\isasymlongrightarrow}\ surj\ f{\isachardoublequoteclose}\isanewline
%
\isadelimproof
\ \ %
\endisadelimproof
%
\isatagproof
\isacommand{oops}\isamarkupfalse%
%
\endisatagproof
{\isafoldproof}%
%
\isadelimproof
%
\endisadelimproof
%
\begin{isamarkuptext}%
En la especificación anterior, \isa{surj\ f} es una abreviatura de 
  \isa{range\ f\ {\isacharequal}\ UNIV}, donde \isa{range\ f} es el rango o imagen
de la función f y \isa{UNIV} es el conjunto universal definido en la 
  teoría \href{http://bit.ly/2XtHCW6}{Set.thy} como una abreviatura de 
  \isa{top} que, a su vez está definido en la teoría 
  \href{http://bit.ly/2Xyj9Pe}{Orderings.thy} mediante la siguiente
  propiedad 
  \begin{itemize}
    \item[] \isa{\mbox{}\inferrule{\mbox{ordering{\isacharunderscore}top\ less{\isacharunderscore}eq\ less\ top}}{\mbox{less{\isacharunderscore}eq\ a\ top}}} 
      \hfill (\isa{ordering{\isacharunderscore}top{\isachardot}extremum})
  \end{itemize} 
Además queda añadir que la teoría donde se encuentra definido
 \isa{surj\ f} es en \href{http://bit.ly/2XuPQx5}{Fun.thy}. Esta
 teoría contiene la definicion \isa{surj{\isacharunderscore}def}.
 \begin{itemize}
    \item[] \isa{surj\ f\ {\isacharequal}\ {\isacharparenleft}{\isasymforall}y{\isachardot}\ {\isasymexists}x{\isachardot}\ y\ {\isacharequal}\ f\ x{\isacharparenright}}
 \hfill (\isa{surj{\isacharunderscore}{\isacharunderscore}def})
  \end{itemize}%
\end{isamarkuptext}\isamarkuptrue%
%
\isadelimdocument
%
\endisadelimdocument
%
\isatagdocument
%
\isamarkupsection{Demostración estructurada%
}
\isamarkuptrue%
%
\endisatagdocument
{\isafolddocument}%
%
\isadelimdocument
%
\endisadelimdocument
%
\begin{isamarkuptext}%
Presentaremos distintas demostraciones de los lemas. Las primeras son
 las detalladas:%
\end{isamarkuptext}\isamarkuptrue%
\isacommand{lemma}\isamarkupfalse%
\ condicion{\isacharunderscore}suficiente{\isacharunderscore}detallada{\isacharcolon}\isanewline
\ \ \isakeyword{assumes}\ {\isachardoublequoteopen}surj\ f{\isachardoublequoteclose}\ \isanewline
\ \ \isakeyword{shows}\ {\isachardoublequoteopen}{\isasymforall}g\ h{\isachardot}\ {\isacharparenleft}\ g\ {\isasymcirc}\ f\ {\isacharequal}\ h\ {\isasymcirc}\ f\ {\isacharparenright}\ {\isasymlongrightarrow}\ {\isacharparenleft}g\ {\isacharequal}\ h{\isacharparenright}{\isachardoublequoteclose}\isanewline
%
\isadelimproof
%
\endisadelimproof
%
\isatagproof
\isacommand{proof}\isamarkupfalse%
\ {\isacharparenleft}rule\ allI{\isacharparenright}\isanewline
\ \ \isacommand{fix}\isamarkupfalse%
\ g\ {\isacharcolon}{\isacharcolon}\ {\isachardoublequoteopen}{\isacharprime}a\ {\isasymRightarrow}{\isacharprime}c{\isachardoublequoteclose}\ \isanewline
\ \ \isacommand{show}\isamarkupfalse%
\ {\isachardoublequoteopen}{\isasymforall}h{\isachardot}\ {\isacharparenleft}g\ {\isasymcirc}\ f\ {\isacharequal}\ h\ {\isasymcirc}\ f{\isacharparenright}\ {\isasymlongrightarrow}\ {\isacharparenleft}g\ {\isacharequal}\ h{\isacharparenright}{\isachardoublequoteclose}\isanewline
\ \ \isacommand{proof}\isamarkupfalse%
\ {\isacharparenleft}rule\ allI{\isacharparenright}\isanewline
\ \ \ \ \isacommand{fix}\isamarkupfalse%
\ h\isanewline
\ \ \ \ \isacommand{show}\isamarkupfalse%
\ {\isachardoublequoteopen}{\isacharparenleft}g\ {\isasymcirc}\ f\ {\isacharequal}\ h\ {\isasymcirc}\ f{\isacharparenright}\ {\isasymlongrightarrow}\ {\isacharparenleft}g\ {\isacharequal}\ h{\isacharparenright}{\isachardoublequoteclose}\ \isanewline
\ \ \ \ \isacommand{proof}\isamarkupfalse%
\ {\isacharparenleft}rule\ impI{\isacharparenright}\isanewline
\ \ \ \ \ \ \isacommand{assume}\isamarkupfalse%
\ {\isadigit{1}}{\isacharcolon}\ {\isachardoublequoteopen}g\ {\isasymcirc}\ f\ {\isacharequal}\ h\ {\isasymcirc}\ f{\isachardoublequoteclose}\isanewline
\ \ \ \ \ \ \isacommand{show}\isamarkupfalse%
\ {\isachardoublequoteopen}g\ {\isacharequal}\ h{\isachardoublequoteclose}\isanewline
\ \ \ \ \ \ \isacommand{proof}\isamarkupfalse%
\ \ \isanewline
\ \ \ \ \ \ \ \ \isacommand{fix}\isamarkupfalse%
\ x\isanewline
\ \ \ \ \ \ \ \ \isacommand{have}\isamarkupfalse%
\ {\isachardoublequoteopen}\ {\isasymexists}y\ {\isachardot}\ x\ {\isacharequal}\ f{\isacharparenleft}y{\isacharparenright}{\isachardoublequoteclose}\ \isacommand{using}\isamarkupfalse%
\ assms\ \isacommand{by}\isamarkupfalse%
\ {\isacharparenleft}simp\ add{\isacharcolon}surj{\isacharunderscore}def{\isacharparenright}\isanewline
\ \ \ \ \ \ \ \ \isacommand{then}\isamarkupfalse%
\ \isacommand{obtain}\isamarkupfalse%
\ \ {\isachardoublequoteopen}y{\isachardoublequoteclose}\ \isakeyword{where}\ {\isadigit{2}}{\isacharcolon}{\isachardoublequoteopen}x\ {\isacharequal}\ f{\isacharparenleft}y{\isacharparenright}{\isachardoublequoteclose}\ \isacommand{by}\isamarkupfalse%
\ {\isacharparenleft}rule\ exE{\isacharparenright}\isanewline
\ \ \ \ \ \ \ \ \isacommand{then}\isamarkupfalse%
\ \isacommand{have}\isamarkupfalse%
\ \ {\isachardoublequoteopen}g{\isacharparenleft}x{\isacharparenright}\ {\isacharequal}\ g{\isacharparenleft}f{\isacharparenleft}y{\isacharparenright}{\isacharparenright}{\isachardoublequoteclose}\ \isacommand{by}\isamarkupfalse%
\ simp\isanewline
\ \ \ \ \ \ \ \ \isacommand{also}\isamarkupfalse%
\ \isacommand{have}\isamarkupfalse%
\ \ {\isachardoublequoteopen}{\isachardot}{\isachardot}{\isachardot}\ {\isacharequal}\ {\isacharparenleft}g\ {\isasymcirc}\ f{\isacharparenright}\ {\isacharparenleft}y{\isacharparenright}\ \ {\isachardoublequoteclose}\ \isacommand{by}\isamarkupfalse%
\ simp\isanewline
\ \ \ \ \ \ \ \ \isacommand{also}\isamarkupfalse%
\ \isacommand{have}\isamarkupfalse%
\ \ {\isachardoublequoteopen}{\isachardot}{\isachardot}{\isachardot}\ {\isacharequal}\ {\isacharparenleft}h\ {\isasymcirc}\ f{\isacharparenright}\ {\isacharparenleft}y{\isacharparenright}{\isachardoublequoteclose}\ \isacommand{using}\isamarkupfalse%
\ {\isadigit{1}}\ \isacommand{by}\isamarkupfalse%
\ simp\isanewline
\ \ \ \ \ \ \ \ \isacommand{also}\isamarkupfalse%
\ \isacommand{have}\isamarkupfalse%
\ \ {\isachardoublequoteopen}{\isachardot}{\isachardot}{\isachardot}\ {\isacharequal}\ h{\isacharparenleft}f{\isacharparenleft}y{\isacharparenright}{\isacharparenright}{\isachardoublequoteclose}\ \isacommand{by}\isamarkupfalse%
\ simp\isanewline
\ \ \ \ \ \ \ \ \isacommand{also}\isamarkupfalse%
\ \isacommand{have}\isamarkupfalse%
\ \ {\isachardoublequoteopen}{\isachardot}{\isachardot}{\isachardot}\ {\isacharequal}\ h{\isacharparenleft}x{\isacharparenright}{\isachardoublequoteclose}\ \isacommand{using}\isamarkupfalse%
\ {\isadigit{2}}\ \ \ \isacommand{by}\isamarkupfalse%
\ {\isacharparenleft}simp\ add{\isacharcolon}\ {\isacartoucheopen}x\ {\isacharequal}\ f\ y{\isacartoucheclose}{\isacharparenright}\isanewline
\ \ \ \ \ \ \ \ \isacommand{finally}\isamarkupfalse%
\ \isacommand{show}\isamarkupfalse%
\ \ {\isachardoublequoteopen}\ g{\isacharparenleft}x{\isacharparenright}\ {\isacharequal}\ h{\isacharparenleft}x{\isacharparenright}\ {\isachardoublequoteclose}\ \isacommand{by}\isamarkupfalse%
\ simp\isanewline
\ \ \ \ \ \ \isacommand{qed}\isamarkupfalse%
\isanewline
\ \ \ \ \isacommand{qed}\isamarkupfalse%
\isanewline
\ \ \isacommand{qed}\isamarkupfalse%
\isanewline
\isacommand{qed}\isamarkupfalse%
%
\endisatagproof
{\isafoldproof}%
%
\isadelimproof
%
\endisadelimproof
%
\begin{isamarkuptext}%
En la siguiente demostración nos hará falta la introducción de
 los pequeños lemas que demostraremos a continuación:%
\end{isamarkuptext}\isamarkuptrue%
\isacommand{lemma}\isamarkupfalse%
\ auxiliar{\isacharunderscore}{\isadigit{1}}{\isacharcolon}\isanewline
\ \ \isakeyword{assumes}\ {\isachardoublequoteopen}{\isasymnot}{\isacharparenleft}{\isasymforall}x{\isachardot}\ P{\isacharparenleft}x{\isacharparenright}{\isacharparenright}{\isachardoublequoteclose}\isanewline
\ \ \isakeyword{shows}\ \ \ {\isachardoublequoteopen}{\isasymexists}x{\isachardot}\ {\isasymnot}P{\isacharparenleft}x{\isacharparenright}{\isachardoublequoteclose}\isanewline
%
\isadelimproof
%
\endisadelimproof
%
\isatagproof
\isacommand{using}\isamarkupfalse%
\ assms\isanewline
\ \ \isacommand{by}\isamarkupfalse%
\ auto%
\endisatagproof
{\isafoldproof}%
%
\isadelimproof
\isanewline
%
\endisadelimproof
\isanewline
\isacommand{lemma}\isamarkupfalse%
\ auxiliar{\isacharunderscore}{\isadigit{2}}{\isacharcolon}\isanewline
\ \ \isakeyword{assumes}\ {\isachardoublequoteopen}{\isasymnot}{\isacharparenleft}{\isasymexists}x{\isachardot}\ P{\isacharparenleft}x{\isacharparenright}{\isacharparenright}{\isachardoublequoteclose}\isanewline
\ \ \isakeyword{shows}\ \ \ {\isachardoublequoteopen}{\isasymforall}x{\isachardot}\ {\isasymnot}P{\isacharparenleft}x{\isacharparenright}{\isachardoublequoteclose}\isanewline
%
\isadelimproof
%
\endisadelimproof
%
\isatagproof
\isacommand{using}\isamarkupfalse%
\ assms\isanewline
\isacommand{by}\isamarkupfalse%
\ auto%
\endisatagproof
{\isafoldproof}%
%
\isadelimproof
\isanewline
%
\endisadelimproof
\isanewline
\isacommand{lemma}\isamarkupfalse%
\ condicion{\isacharunderscore}necesaria{\isacharunderscore}detallada{\isacharcolon}\isanewline
\ \ \isakeyword{assumes}\ {\isachardoublequoteopen}{\isasymforall}{\isacharparenleft}g\ {\isacharcolon}{\isacharcolon}\ {\isacharprime}b\ {\isasymRightarrow}\ {\isacharprime}c{\isacharparenright}\ h\ {\isachardot}{\isacharparenleft}g\ {\isasymcirc}\ {\isacharparenleft}f\ {\isacharcolon}{\isacharcolon}\ {\isacharprime}a\ {\isasymRightarrow}\ {\isacharprime}b{\isacharparenright}\ {\isacharequal}\ h\ {\isasymcirc}\ f{\isacharparenright}\ {\isasymlongrightarrow}\ {\isacharparenleft}g\ {\isacharequal}\ h{\isacharparenright}{\isachardoublequoteclose}\isanewline
\ \ \ \ \ \ \ \ \ \ \ {\isachardoublequoteopen}{\isasymexists}\ {\isacharparenleft}x{\isadigit{0}}{\isacharcolon}{\isacharcolon}{\isacharprime}c{\isacharparenright}\ {\isacharparenleft}x{\isadigit{1}}{\isacharcolon}{\isacharcolon}{\isacharprime}c{\isacharparenright}{\isachardot}\ x{\isadigit{0}}\ {\isasymnoteq}\ x{\isadigit{1}}{\isachardoublequoteclose}\isanewline
\ \ \ \ \ \ \ \ \ \isakeyword{shows}\ {\isachardoublequoteopen}\ {\isasymforall}\ {\isacharparenleft}y{\isacharcolon}{\isacharcolon}{\isacharprime}b{\isacharparenright}{\isachardot}\ {\isacharparenleft}{\isasymexists}\ {\isacharparenleft}x{\isacharcolon}{\isacharcolon}\ {\isacharprime}a{\isacharparenright}{\isachardot}\ f\ x\ {\isacharequal}\ y{\isacharparenright}{\isachardoublequoteclose}\isanewline
%
\isadelimproof
%
\endisadelimproof
%
\isatagproof
\isacommand{proof}\isamarkupfalse%
\ {\isacharparenleft}rule\ ccontr{\isacharparenright}\isanewline
\ \ \isacommand{assume}\isamarkupfalse%
\ {\isachardoublequoteopen}{\isasymnot}\ {\isacharparenleft}{\isasymforall}y\ {\isacharcolon}{\isacharcolon}\ {\isacharprime}b{\isachardot}\ {\isasymexists}x\ {\isacharcolon}{\isacharcolon}\ {\isacharprime}a{\isachardot}\ f\ x\ {\isacharequal}\ y{\isacharparenright}{\isachardoublequoteclose}\isanewline
\ \ \isacommand{hence}\isamarkupfalse%
\ {\isachardoublequoteopen}{\isasymexists}y\ {\isacharcolon}{\isacharcolon}\ {\isacharprime}b\ {\isachardot}\ {\isasymnot}\ {\isacharparenleft}{\isasymexists}x\ {\isacharcolon}{\isacharcolon}\ {\isacharprime}a{\isachardot}\ f\ x\ {\isacharequal}\ y{\isacharparenright}{\isachardoublequoteclose}\ \isacommand{by}\isamarkupfalse%
\ {\isacharparenleft}rule\ auxiliar{\isacharunderscore}{\isadigit{1}}{\isacharparenright}\isanewline
\ \ \isacommand{then}\isamarkupfalse%
\ \isacommand{obtain}\isamarkupfalse%
\ y{\isadigit{0}}\ \isakeyword{where}\ \ {\isachardoublequoteopen}{\isasymnot}\ {\isacharparenleft}{\isasymexists}x\ {\isacharcolon}{\isacharcolon}\ {\isacharprime}a{\isachardot}\ f\ x\ {\isacharequal}\ y{\isadigit{0}}{\isacharparenright}{\isachardoublequoteclose}\ \isacommand{by}\isamarkupfalse%
\ {\isacharparenleft}rule\ exE{\isacharparenright}\isanewline
\ \ \isacommand{hence}\isamarkupfalse%
\ {\isachardoublequoteopen}{\isasymforall}x\ {\isacharcolon}{\isacharcolon}\ {\isacharprime}a{\isachardot}\ {\isacharparenleft}{\isasymnot}\ {\isacharparenleft}f\ x\ {\isacharequal}\ y{\isadigit{0}}{\isacharparenright}{\isacharparenright}{\isachardoublequoteclose}\ \isacommand{by}\isamarkupfalse%
\ {\isacharparenleft}rule\ auxiliar{\isacharunderscore}{\isadigit{2}}{\isacharparenright}\isanewline
\ \ \isacommand{obtain}\isamarkupfalse%
\ a{\isadigit{0}}\ \isakeyword{where}\ {\isachardoublequoteopen}\ {\isasymexists}{\isacharparenleft}x{\isadigit{1}}{\isacharcolon}{\isacharcolon}{\isacharprime}c{\isacharparenright}{\isachardot}\ a{\isadigit{0}}\ {\isasymnoteq}\ x{\isadigit{1}}{\isachardoublequoteclose}\ \isacommand{using}\isamarkupfalse%
\ assms{\isacharparenleft}{\isadigit{2}}{\isacharparenright}\ \isacommand{by}\isamarkupfalse%
\ {\isacharparenleft}rule\ exE{\isacharparenright}\isanewline
\ \ \isacommand{then}\isamarkupfalse%
\ \isacommand{obtain}\isamarkupfalse%
\ a{\isadigit{1}}\ \isakeyword{where}\ {\isachardoublequoteopen}a{\isadigit{0}}\ {\isasymnoteq}\ a{\isadigit{1}}{\isachardoublequoteclose}\ \isacommand{by}\isamarkupfalse%
\ {\isacharparenleft}rule\ exE{\isacharparenright}\isanewline
\ \ \isacommand{let}\isamarkupfalse%
\ {\isacharquery}g\ {\isacharequal}\ {\isachardoublequoteopen}{\isacharparenleft}{\isasymlambda}x{\isachardot}\ a{\isadigit{0}}{\isacharparenright}\ \ {\isacharcolon}{\isacharcolon}\ {\isacharprime}b\ {\isasymRightarrow}\ {\isacharprime}c{\isachardoublequoteclose}\isanewline
\ \ \isacommand{let}\isamarkupfalse%
\ {\isacharquery}h\ {\isacharequal}\ {\isachardoublequoteopen}{\isacharquery}g{\isacharparenleft}y{\isadigit{0}}{\isacharcolon}{\isacharequal}a{\isadigit{1}}{\isacharparenright}{\isachardoublequoteclose}\isanewline
\ \ \isacommand{have}\isamarkupfalse%
\ {\isachardoublequoteopen}{\isasymforall}h\ {\isachardot}{\isacharparenleft}{\isacharquery}g\ {\isasymcirc}\ {\isacharparenleft}f\ {\isacharcolon}{\isacharcolon}\ {\isacharprime}a\ {\isasymRightarrow}\ {\isacharprime}b{\isacharparenright}\ {\isacharequal}\ h\ {\isasymcirc}\ f{\isacharparenright}\ {\isasymlongrightarrow}\ {\isacharparenleft}{\isacharquery}g\ {\isacharequal}\ h{\isacharparenright}{\isachardoublequoteclose}\isanewline
\ \ \ \ \isacommand{using}\isamarkupfalse%
\ assms{\isacharparenleft}{\isadigit{1}}{\isacharparenright}\ \isacommand{by}\isamarkupfalse%
\ {\isacharparenleft}rule\ allE{\isacharparenright}\isanewline
\ \ \isacommand{hence}\isamarkupfalse%
\ {\isadigit{1}}{\isacharcolon}{\isachardoublequoteopen}{\isacharparenleft}{\isacharquery}g\ {\isasymcirc}\ {\isacharparenleft}f\ {\isacharcolon}{\isacharcolon}\ {\isacharprime}a\ {\isasymRightarrow}\ {\isacharprime}b{\isacharparenright}\ {\isacharequal}\ {\isacharquery}h\ {\isasymcirc}\ f{\isacharparenright}\ {\isasymlongrightarrow}\ {\isacharparenleft}{\isacharquery}g\ {\isacharequal}\ {\isacharquery}h{\isacharparenright}{\isachardoublequoteclose}\ \isacommand{by}\isamarkupfalse%
\ {\isacharparenleft}rule\ allE{\isacharparenright}\isanewline
\ \ \isacommand{have}\isamarkupfalse%
\ {\isadigit{2}}{\isacharcolon}\ {\isachardoublequoteopen}{\isacharparenleft}{\isacharquery}g\ {\isasymcirc}\ {\isacharparenleft}f\ {\isacharcolon}{\isacharcolon}\ {\isacharprime}a\ {\isasymRightarrow}\ {\isacharprime}b{\isacharparenright}\ {\isacharequal}\ {\isacharquery}h\ {\isasymcirc}\ f{\isacharparenright}{\isachardoublequoteclose}\isanewline
\ \ \ \ \isacommand{using}\isamarkupfalse%
\ {\isacharbrackleft}{\isacharbrackleft}simp{\isacharunderscore}trace{\isacharbrackright}{\isacharbrackright}\isanewline
\ \ \ \ \isacommand{using}\isamarkupfalse%
\ {\isacartoucheopen}{\isasymnexists}x\ {\isacharcolon}{\isacharcolon}\ {\isacharprime}a{\isachardot}\ {\isacharparenleft}f\ {\isacharcolon}{\isacharcolon}\ {\isacharprime}a\ {\isasymRightarrow}\ {\isacharprime}b{\isacharparenright}\ x\ {\isacharequal}\ {\isacharparenleft}y{\isadigit{0}}\ {\isacharcolon}{\isacharcolon}\ {\isacharprime}b{\isacharparenright}{\isacartoucheclose}\ \isacommand{by}\isamarkupfalse%
\ auto\isanewline
\ \ \isacommand{have}\isamarkupfalse%
\ {\isachardoublequoteopen}{\isacharparenleft}{\isacharquery}g\ {\isacharequal}\ {\isacharquery}h{\isacharparenright}{\isachardoublequoteclose}\ \isacommand{using}\isamarkupfalse%
\ {\isadigit{1}}\ {\isadigit{2}}\ \isacommand{by}\isamarkupfalse%
\ {\isacharparenleft}rule\ mp{\isacharparenright}\isanewline
\ \ \isacommand{hence}\isamarkupfalse%
\ {\isachardoublequoteopen}a{\isadigit{0}}\ {\isacharequal}\ a{\isadigit{1}}{\isachardoublequoteclose}\ \isacommand{by}\isamarkupfalse%
\ {\isacharparenleft}metis\ fun{\isacharunderscore}upd{\isacharunderscore}idem{\isacharunderscore}iff{\isacharparenright}\isanewline
\ \ \isacommand{with}\isamarkupfalse%
\ {\isacharbackquoteopen}a{\isadigit{0}}\ {\isasymnoteq}\ a{\isadigit{1}}{\isacharbackquoteclose}\ \isacommand{show}\isamarkupfalse%
\ False\ \isacommand{by}\isamarkupfalse%
\ {\isacharparenleft}rule\ notE{\isacharparenright}\isanewline
\isacommand{qed}\isamarkupfalse%
%
\endisatagproof
{\isafoldproof}%
%
\isadelimproof
\isanewline
%
\endisadelimproof
\isanewline
\isacommand{lemma}\isamarkupfalse%
\ condicion{\isacharunderscore}necesaria{\isacharunderscore}detallada{\isacharunderscore}{\isadigit{2}}{\isacharcolon}\isanewline
\ \ \isakeyword{assumes}\ {\isachardoublequoteopen}\ {\isasymforall}\ {\isacharparenleft}y{\isacharcolon}{\isacharcolon}{\isacharprime}b{\isacharparenright}{\isachardot}\ {\isacharparenleft}{\isasymexists}\ {\isacharparenleft}x{\isacharcolon}{\isacharcolon}\ {\isacharprime}a{\isacharparenright}{\isachardot}\ f\ x\ {\isacharequal}\ y{\isacharparenright}{\isachardoublequoteclose}\isanewline
\ \ \isakeyword{shows}\ {\isachardoublequoteopen}surj\ f{\isachardoublequoteclose}\isanewline
%
\isadelimproof
\ \ %
\endisadelimproof
%
\isatagproof
\isacommand{by}\isamarkupfalse%
\ {\isacharparenleft}metis\ assms\ surj{\isacharunderscore}def{\isacharparenright}%
\endisatagproof
{\isafoldproof}%
%
\isadelimproof
%
\endisadelimproof
%
\begin{isamarkuptext}%
En la demostración hemos introducido: 
 \begin{itemize}
    \item[] \isa{\mbox{}\inferrule{\mbox{{\isasymexists}x{\isachardot}\ P\ x}\\\ \mbox{{\isasymAnd}x{\isachardot}\ \mbox{}\inferrule{\mbox{P\ x}}{\mbox{Q}}}}{\mbox{Q}}} 
      \hfill (\isa{rule\ exE}) 
  \end{itemize} 
 \begin{itemize}
    \item[] \isa{{\isasymlbrakk}P\ {\isasymLongrightarrow}\ Q{\isacharsemicolon}\ Q\ {\isasymLongrightarrow}\ P{\isasymrbrakk}\ {\isasymLongrightarrow}\ P\ {\isacharequal}\ Q} 
      \hfill (\isa{iffI})
  \end{itemize}%
\end{isamarkuptext}\isamarkuptrue%
%
\isadelimdocument
%
\endisadelimdocument
%
\isatagdocument
%
\isamarkupsection{Demostración aplicativa%
}
\isamarkuptrue%
%
\endisatagdocument
{\isafolddocument}%
%
\isadelimdocument
%
\endisadelimdocument
%
\begin{isamarkuptext}%
Las demostraciones aplicativas son:%
\end{isamarkuptext}\isamarkuptrue%
\isacommand{lemma}\isamarkupfalse%
\ demostracion{\isacharunderscore}suficiente{\isacharunderscore}aplicativa{\isacharcolon}\isanewline
{\isachardoublequoteopen}surj\ f\ {\isasymLongrightarrow}\ {\isacharparenleft}{\isacharparenleft}g\ {\isasymcirc}\ f{\isacharparenright}\ {\isacharequal}\ {\isacharparenleft}h\ {\isasymcirc}\ f{\isacharparenright}\ {\isacharparenright}\ {\isasymlongrightarrow}\ {\isacharparenleft}g\ {\isacharequal}\ h{\isacharparenright}{\isachardoublequoteclose}\isanewline
%
\isadelimproof
\ \ %
\endisadelimproof
%
\isatagproof
\isacommand{apply}\isamarkupfalse%
\ {\isacharparenleft}simp\ add{\isacharcolon}\ surj{\isacharunderscore}def\ fun{\isacharunderscore}eq{\isacharunderscore}iff{\isacharparenright}\isanewline
\ \ \isacommand{apply}\isamarkupfalse%
\ metis\isanewline
\ \ \isacommand{done}\isamarkupfalse%
%
\endisatagproof
{\isafoldproof}%
%
\isadelimproof
\isanewline
%
\endisadelimproof
\isanewline
\isacommand{lemma}\isamarkupfalse%
\ demostracion{\isacharunderscore}necesaria{\isacharunderscore}aplicativa{\isacharcolon}\isanewline
\ \ {\isachardoublequoteopen}{\isasymlbrakk}{\isasymforall}{\isacharparenleft}g\ {\isacharcolon}{\isacharcolon}\ {\isacharprime}b\ {\isasymRightarrow}\ {\isacharprime}c{\isacharparenright}\ h\ {\isachardot}{\isacharparenleft}g\ {\isasymcirc}\ {\isacharparenleft}f\ {\isacharcolon}{\isacharcolon}\ {\isacharprime}a\ {\isasymRightarrow}\ {\isacharprime}b{\isacharparenright}\ {\isacharequal}\ h\ {\isasymcirc}\ f{\isacharparenright}\ {\isasymlongrightarrow}\ {\isacharparenleft}g\ {\isacharequal}\ h{\isacharparenright}{\isacharsemicolon}\isanewline
\ \ \ \ \ \ \ \ \ \ {\isasymexists}\ {\isacharparenleft}x{\isadigit{0}}{\isacharcolon}{\isacharcolon}{\isacharprime}c{\isacharparenright}\ {\isacharparenleft}x{\isadigit{1}}{\isacharcolon}{\isacharcolon}{\isacharprime}c{\isacharparenright}{\isachardot}\ x{\isadigit{0}}\ {\isasymnoteq}\ x{\isadigit{1}}{\isasymrbrakk}{\isasymLongrightarrow}\ surj\ f{\isachardoublequoteclose}\isanewline
%
\isadelimproof
\ \ %
\endisadelimproof
%
\isatagproof
\isacommand{apply}\isamarkupfalse%
\ {\isacharparenleft}rule\ surjI{\isacharparenright}\isanewline
\ \ \isacommand{apply}\isamarkupfalse%
\ {\isacharparenleft}drule\ condicion{\isacharunderscore}necesaria{\isacharunderscore}detallada{\isacharparenright}\isanewline
\ \ \ \isacommand{apply}\isamarkupfalse%
\ simp\isanewline
\ \ \isacommand{apply}\isamarkupfalse%
\ {\isacharparenleft}erule\ allE{\isacharparenright}\isanewline
\ \ \isacommand{apply}\isamarkupfalse%
\ {\isacharparenleft}erule\ exE{\isacharparenright}{\isacharplus}\isanewline
\ \ \isacommand{apply}\isamarkupfalse%
\ {\isacharparenleft}erule\ Hilbert{\isacharunderscore}Choice{\isachardot}someI{\isacharparenright}\isanewline
\ \ \isacommand{done}\isamarkupfalse%
%
\endisatagproof
{\isafoldproof}%
%
\isadelimproof
%
\endisadelimproof
%
\begin{isamarkuptext}%
En estas hemos introducido:
 \begin{itemize}
    \item[] \isa{{\isacharparenleft}f\ {\isacharequal}\ g{\isacharparenright}\ {\isacharequal}\ {\isacharparenleft}{\isasymforall}x{\isachardot}\ f\ x\ {\isacharequal}\ g\ x{\isacharparenright}} 
      \hfill (\isa{fun{\isacharunderscore}eq{\isacharunderscore}iff})
     \item[] \isa{\mbox{}\inferrule{\mbox{P\ x}}{\mbox{P\ {\isacharparenleft}Eps\ P{\isacharparenright}}}} 
      \hfill (\isa{Hilbert{\isacharunderscore}Choice{\isachardot}someI})
\end{itemize}%
\end{isamarkuptext}\isamarkuptrue%
%
\isadelimdocument
%
\endisadelimdocument
%
\isatagdocument
%
\isamarkupsection{Demostración teorema%
}
\isamarkuptrue%
%
\endisatagdocument
{\isafolddocument}%
%
\isadelimdocument
%
\endisadelimdocument
%
\begin{isamarkuptext}%
En consecuencia, la demostración del teorema es%
\end{isamarkuptext}\isamarkuptrue%
\isacommand{theorem}\isamarkupfalse%
\ caracterizacion{\isacharunderscore}funciones{\isacharunderscore}sobreyectivas{\isacharcolon}\isanewline
\ {\isachardoublequoteopen}surj\ f\ {\isasymlongleftrightarrow}\ \ {\isacharparenleft}{\isasymforall}g\ h{\isachardot}{\isacharparenleft}g\ {\isasymcirc}\ f\ {\isacharequal}\ h\ {\isasymcirc}\ f{\isacharparenright}\ {\isasymlongrightarrow}\ {\isacharparenleft}g\ {\isacharequal}\ h{\isacharparenright}{\isacharparenright}{\isachardoublequoteclose}\isanewline
%
\isadelimproof
\ \ %
\endisadelimproof
%
\isatagproof
\isacommand{oops}\isamarkupfalse%
\isanewline
\ \ \isanewline
%
\endisatagproof
{\isafoldproof}%
%
\isadelimproof
%
\endisadelimproof
%
\isadelimtheory
%
\endisadelimtheory
%
\isatagtheory
%
\endisatagtheory
{\isafoldtheory}%
%
\isadelimtheory
%
\endisadelimtheory
%
\end{isabellebody}%
\endinput
%:%file=~/Escritorio/TFG/CancelacionSobreyectiva.thy%:%
%:%6=1%:%
%:%20=9%:%
%:%24=11%:%
%:%28=13%:%
%:%37=15%:%
%:%49=17%:%
%:%50=18%:%
%:%51=19%:%
%:%52=20%:%
%:%53=21%:%
%:%54=22%:%
%:%55=23%:%
%:%56=24%:%
%:%57=25%:%
%:%58=26%:%
%:%59=27%:%
%:%60=28%:%
%:%61=29%:%
%:%62=30%:%
%:%63=31%:%
%:%64=32%:%
%:%65=33%:%
%:%66=34%:%
%:%67=35%:%
%:%68=36%:%
%:%69=37%:%
%:%70=38%:%
%:%71=39%:%
%:%72=40%:%
%:%73=41%:%
%:%74=42%:%
%:%75=43%:%
%:%76=44%:%
%:%77=45%:%
%:%78=46%:%
%:%79=47%:%
%:%80=48%:%
%:%81=49%:%
%:%82=50%:%
%:%83=51%:%
%:%84=52%:%
%:%85=53%:%
%:%86=54%:%
%:%87=55%:%
%:%88=56%:%
%:%89=57%:%
%:%90=58%:%
%:%91=59%:%
%:%92=60%:%
%:%93=61%:%
%:%94=62%:%
%:%95=63%:%
%:%96=64%:%
%:%97=65%:%
%:%98=66%:%
%:%99=67%:%
%:%100=68%:%
%:%101=69%:%
%:%102=70%:%
%:%103=71%:%
%:%104=72%:%
%:%105=73%:%
%:%106=74%:%
%:%107=75%:%
%:%108=76%:%
%:%109=77%:%
%:%110=78%:%
%:%111=79%:%
%:%120=81%:%
%:%132=84%:%
%:%133=85%:%
%:%135=87%:%
%:%136=87%:%
%:%137=88%:%
%:%140=89%:%
%:%144=89%:%
%:%150=89%:%
%:%153=90%:%
%:%154=91%:%
%:%155=91%:%
%:%156=92%:%
%:%159=93%:%
%:%163=93%:%
%:%169=93%:%
%:%172=94%:%
%:%173=95%:%
%:%174=95%:%
%:%175=96%:%
%:%178=97%:%
%:%182=97%:%
%:%192=101%:%
%:%193=102%:%
%:%194=103%:%
%:%195=104%:%
%:%196=105%:%
%:%197=106%:%
%:%198=107%:%
%:%199=108%:%
%:%200=109%:%
%:%201=110%:%
%:%202=111%:%
%:%203=112%:%
%:%204=113%:%
%:%205=114%:%
%:%206=115%:%
%:%207=116%:%
%:%208=117%:%
%:%209=118%:%
%:%218=122%:%
%:%230=125%:%
%:%231=126%:%
%:%233=130%:%
%:%234=130%:%
%:%235=131%:%
%:%236=132%:%
%:%243=133%:%
%:%244=133%:%
%:%245=134%:%
%:%246=134%:%
%:%247=135%:%
%:%248=135%:%
%:%249=136%:%
%:%250=136%:%
%:%251=137%:%
%:%252=137%:%
%:%253=138%:%
%:%254=138%:%
%:%255=139%:%
%:%256=139%:%
%:%257=140%:%
%:%258=140%:%
%:%259=141%:%
%:%260=141%:%
%:%261=142%:%
%:%262=142%:%
%:%263=143%:%
%:%264=143%:%
%:%265=144%:%
%:%266=144%:%
%:%267=144%:%
%:%268=144%:%
%:%269=145%:%
%:%270=145%:%
%:%271=145%:%
%:%272=145%:%
%:%273=146%:%
%:%274=146%:%
%:%275=146%:%
%:%276=146%:%
%:%277=147%:%
%:%278=147%:%
%:%279=147%:%
%:%280=147%:%
%:%281=148%:%
%:%282=148%:%
%:%283=148%:%
%:%284=148%:%
%:%285=148%:%
%:%286=149%:%
%:%287=149%:%
%:%288=149%:%
%:%289=149%:%
%:%290=150%:%
%:%291=150%:%
%:%292=150%:%
%:%293=150%:%
%:%294=150%:%
%:%295=151%:%
%:%296=151%:%
%:%297=151%:%
%:%298=151%:%
%:%299=152%:%
%:%300=152%:%
%:%301=153%:%
%:%302=153%:%
%:%303=154%:%
%:%304=154%:%
%:%305=155%:%
%:%315=158%:%
%:%316=159%:%
%:%318=162%:%
%:%319=162%:%
%:%320=163%:%
%:%321=164%:%
%:%328=165%:%
%:%329=165%:%
%:%330=166%:%
%:%331=166%:%
%:%336=166%:%
%:%339=167%:%
%:%340=168%:%
%:%341=168%:%
%:%342=169%:%
%:%343=170%:%
%:%350=171%:%
%:%351=171%:%
%:%352=172%:%
%:%353=172%:%
%:%358=172%:%
%:%361=173%:%
%:%362=174%:%
%:%363=174%:%
%:%364=175%:%
%:%365=176%:%
%:%366=177%:%
%:%373=178%:%
%:%374=178%:%
%:%375=179%:%
%:%376=179%:%
%:%377=180%:%
%:%378=180%:%
%:%379=180%:%
%:%380=181%:%
%:%381=181%:%
%:%382=181%:%
%:%383=181%:%
%:%384=182%:%
%:%385=182%:%
%:%386=182%:%
%:%387=183%:%
%:%388=183%:%
%:%389=183%:%
%:%390=183%:%
%:%391=184%:%
%:%392=184%:%
%:%393=184%:%
%:%394=184%:%
%:%395=185%:%
%:%396=185%:%
%:%397=186%:%
%:%398=186%:%
%:%399=187%:%
%:%400=187%:%
%:%401=188%:%
%:%402=188%:%
%:%403=188%:%
%:%404=189%:%
%:%405=189%:%
%:%406=189%:%
%:%407=190%:%
%:%408=190%:%
%:%409=191%:%
%:%410=191%:%
%:%411=192%:%
%:%412=192%:%
%:%413=192%:%
%:%414=193%:%
%:%415=193%:%
%:%416=193%:%
%:%417=193%:%
%:%418=194%:%
%:%419=194%:%
%:%420=194%:%
%:%421=195%:%
%:%422=195%:%
%:%423=195%:%
%:%424=195%:%
%:%425=196%:%
%:%431=196%:%
%:%434=197%:%
%:%435=198%:%
%:%436=198%:%
%:%437=199%:%
%:%438=200%:%
%:%441=201%:%
%:%445=201%:%
%:%446=201%:%
%:%455=203%:%
%:%456=204%:%
%:%457=205%:%
%:%458=206%:%
%:%459=207%:%
%:%460=208%:%
%:%461=209%:%
%:%462=210%:%
%:%463=211%:%
%:%472=214%:%
%:%484=216%:%
%:%486=218%:%
%:%487=218%:%
%:%488=219%:%
%:%491=220%:%
%:%495=220%:%
%:%496=220%:%
%:%497=221%:%
%:%498=221%:%
%:%499=222%:%
%:%505=222%:%
%:%508=223%:%
%:%509=224%:%
%:%510=224%:%
%:%511=225%:%
%:%512=226%:%
%:%515=227%:%
%:%519=227%:%
%:%520=227%:%
%:%521=228%:%
%:%522=228%:%
%:%523=229%:%
%:%524=229%:%
%:%525=230%:%
%:%526=230%:%
%:%527=231%:%
%:%528=231%:%
%:%529=232%:%
%:%530=232%:%
%:%531=233%:%
%:%541=236%:%
%:%542=237%:%
%:%543=238%:%
%:%544=239%:%
%:%545=240%:%
%:%546=241%:%
%:%547=242%:%
%:%556=244%:%
%:%568=245%:%
%:%570=247%:%
%:%571=247%:%
%:%572=248%:%
%:%575=249%:%
%:%579=249%:%
%:%580=249%:%
%:%581=250%:%