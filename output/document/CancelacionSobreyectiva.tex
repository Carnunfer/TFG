%
\begin{isabellebody}%
\setisabellecontext{CancelacionSobreyectiva}%
%
\isadelimtheory
%
\endisadelimtheory
%
\isatagtheory
%
\endisatagtheory
{\isafoldtheory}%
%
\isadelimtheory
%
\endisadelimtheory
%
\isadelimdocument
%
\endisadelimdocument
%
\isatagdocument
%
\isamarkupsection{Cancelación de las funciones sobreyectivas%
}
\isamarkuptrue%
%
\endisatagdocument
{\isafolddocument}%
%
\isadelimdocument
%
\endisadelimdocument
%
\begin{isamarkuptext}%
El siguiente teorema prueba una caracterización de las funciones
 sobreyectivas, en otras palabras, las funciones sobreyectivas son
 epimorfismos en la categoría de conjuntos. Donde un epimorfismo es un
 homomorfismo sobreyectivo y la categoría de conjuntos es la categoría
 donde los objetos son conjuntos.


\begin {teorema}
  f es sobreyectiva si y solo si  para todas funciones g y h tal que 
$g \circ f  = h \circ f$ se tiene que g = h.
\end {teorema}
 
El teorema lo podemos dividir en dos lemas, ya que el teorema se
 demuestra por una doble implicación, luego vamos a dividir el teorema
 en las dos implicaciones.

\begin {lema}
  f es sobreyectiva entonces  para todas funciones g y h tal que 
$g \circ f = h \circ f$ se tiene que $g = h$.
\end {lema}
\begin {demostracion}
\begin {itemize}
\item Supongamos que tenemos que $g \circ  f = h \circ f$, queremos
 probar que $g = h.$ Usando la definición de sobreyectividad
 $(\forall y \in Y,  \exists x | y = f(x))$ y nuestra hipótesis,
 tenemos que: $$g(y) = g(f(x)) = (g \circ f) (x) = (h \circ f) (x) =
 h(f(x)) = h(y).$$
\item Supongamos que $g = h$, hay que probar que
 $g \circ f = h \circ f.$ Usando nuestra hipótesis, tenemos que:
$$ (g \circ f)(x) = g(f(x)) = h(f(x)) = (h \circ f) (x).$$
\end {itemize}
(*<*).(*>*)
\end {demostracion}

\begin {lema}
 Si  para todas funciones g y h tal que $g \circ f  = h \circ f$ se 
tiene que g = h entonces f es sobreyectiva.
\end {lema}

\begin {demostracion}
Para la demostración del ejercicios, primero debemos señalar los
 dominios y codominios de las funciones que vamos a usar.
 $f : C \longrightarrow A,$ $g,h: A \longrightarrow B.$ También debemos
 notar que nuestro conjunto  $B$ tiene que tener almenos dos elementos
 diferentes, supongamos que $B = \{a,b\}.$ \\
La prueba la vamos a realizar por reducción al absurdo. Luego supongamos
que nuestra función $f$ no es sobreyectiva, es decir, $\exists y_{1} \in
 A \ \isa{tal\ que} \  \nexists x \in C \ : f(x) = y.$ \\
Definamos ahora las funciones $g,h:$
$$g(y) = a \  \forall y \in A$$
$$h(y) = a  \ \isa{si} \  y \neq y_{1} \ h(y) =  b \ 
 \isa{si} \  y =  y_{1}$$

Entonces sabemos que $g(y) \neq h(y)  \forall y \in A.$ Sin embargo,
 por hipótesis tenemos que si $g \circ f = h \circ f$, lo cual es
 cierto, se tiene que $h = g.$ Por lo que hemos llegado a una
 contradicción, entonces $f$ es sobreyectiva.
\end {demostracion}


Su especificación es la siguiente, que la dividiremos en dos al igual que 
en la demostración a mano:%
\end{isamarkuptext}\isamarkuptrue%
\isacommand{theorem}\isamarkupfalse%
\isanewline
\ {\isachardoublequoteopen}surj\ f\ {\isasymlongleftrightarrow}\ {\isacharparenleft}{\isasymforall}g\ h{\isachardot}{\isacharparenleft}g\ {\isasymcirc}\ f\ {\isacharequal}\ h\ {\isasymcirc}\ f{\isacharparenright}\ {\isasymlongrightarrow}\ {\isacharparenleft}g\ {\isacharequal}\ h{\isacharparenright}{\isacharparenright}{\isachardoublequoteclose}\isanewline
%
\isadelimproof
\ \ %
\endisadelimproof
%
\isatagproof
\isacommand{oops}\isamarkupfalse%
%
\endisatagproof
{\isafoldproof}%
%
\isadelimproof
\isanewline
%
\endisadelimproof
\isanewline
\isacommand{lemma}\isamarkupfalse%
\ \isanewline
{\isachardoublequoteopen}surj\ f\ {\isasymLongrightarrow}\ \ {\isacharparenleft}{\isasymforall}g\ h{\isachardot}\ {\isacharparenleft}g\ {\isasymcirc}\ f\ {\isacharequal}\ h\ {\isasymcirc}\ f{\isacharparenright}\ {\isasymlongrightarrow}\ {\isacharparenleft}g\ {\isacharequal}\ h{\isacharparenright}{\isacharparenright}{\isachardoublequoteclose}\isanewline
%
\isadelimproof
\ \ %
\endisadelimproof
%
\isatagproof
\isacommand{oops}\isamarkupfalse%
%
\endisatagproof
{\isafoldproof}%
%
\isadelimproof
\isanewline
%
\endisadelimproof
\isanewline
\isacommand{lemma}\isamarkupfalse%
\ \isanewline
{\isachardoublequoteopen}{\isasymforall}g\ h{\isachardot}\ {\isacharparenleft}g\ {\isasymcirc}\ f\ {\isacharequal}\ h\ {\isasymcirc}\ f\ {\isasymlongrightarrow}\ g\ {\isacharequal}\ h{\isacharparenright}\ {\isasymlongrightarrow}\ surj\ f{\isachardoublequoteclose}\isanewline
%
\isadelimproof
\ \ %
\endisadelimproof
%
\isatagproof
\isacommand{oops}\isamarkupfalse%
%
\endisatagproof
{\isafoldproof}%
%
\isadelimproof
%
\endisadelimproof
%
\begin{isamarkuptext}%
En la especificación anterior, \isa{surj\ f} es una abreviatura de 
  \isa{range\ f\ {\isacharequal}\ UNIV}, donde \isa{range\ f} es el rango o imagen
de la función f.
 Por otra parte, \isa{UNIV} es el conjunto universal definido en la 
  teoría \href{http://bit.ly/2XtHCW6}{Set.thy} como una abreviatura de 
  \isa{top} que, a su vez está definido en la teoría 
  \href{http://bit.ly/2Xyj9Pe}{Orderings.thy} mediante la siguiente
  propiedad 
  \begin{itemize}
    \item[] \isa{\mbox{}\inferrule{\mbox{ordering{\isacharunderscore}top\ less{\isacharunderscore}eq\ less\ top}}{\mbox{less{\isacharunderscore}eq\ a\ top}}} 
      \hfill (\isa{ordering{\isacharunderscore}top{\isachardot}extremum})
  \end{itemize} 
Además queda añadir que la teoría donde se encuentra definido
 \isa{surj\ f} es en \href{http://bit.ly/2XuPQx5}{Fun.thy}. Esta
 teoría contiene la definicion \isa{surj{\isacharunderscore}def}.
 \begin{itemize}
    \item[] \isa{surj\ f\ {\isacharequal}\ {\isacharparenleft}{\isasymforall}y{\isachardot}\ {\isasymexists}x{\isachardot}\ y\ {\isacharequal}\ f\ x{\isacharparenright}}
 \hfill (\isa{inj{\isacharunderscore}on{\isacharunderscore}def})
  \end{itemize} 

Presentaremos distintas demostraciones de los lemas. Las primeras son
 las detalladas:%
\end{isamarkuptext}\isamarkuptrue%
\isanewline
\isacommand{lemma}\isamarkupfalse%
\ sobreyectivadetallada{\isacharcolon}\isanewline
\ \ \isakeyword{assumes}\ {\isachardoublequoteopen}surj\ f{\isachardoublequoteclose}\ \isanewline
\ \ \isakeyword{shows}\ {\isachardoublequoteopen}{\isasymforall}g\ h{\isachardot}\ {\isacharparenleft}\ g\ {\isasymcirc}\ f\ {\isacharequal}\ h\ {\isasymcirc}\ f\ {\isacharparenright}\ {\isasymlongrightarrow}\ {\isacharparenleft}g\ {\isacharequal}\ h{\isacharparenright}{\isachardoublequoteclose}\isanewline
%
\isadelimproof
%
\endisadelimproof
%
\isatagproof
\isacommand{proof}\isamarkupfalse%
\ {\isacharparenleft}rule\ allI{\isacharparenright}\isanewline
\ \ \isacommand{fix}\isamarkupfalse%
\ g\ {\isacharcolon}{\isacharcolon}\ {\isachardoublequoteopen}{\isacharprime}a\ {\isasymRightarrow}{\isacharprime}c{\isachardoublequoteclose}\ \isanewline
\ \ \isacommand{show}\isamarkupfalse%
\ {\isachardoublequoteopen}{\isasymforall}h{\isachardot}\ {\isacharparenleft}g\ {\isasymcirc}\ f\ {\isacharequal}\ h\ {\isasymcirc}\ f{\isacharparenright}\ {\isasymlongrightarrow}\ {\isacharparenleft}g\ {\isacharequal}\ h{\isacharparenright}{\isachardoublequoteclose}\isanewline
\ \ \isacommand{proof}\isamarkupfalse%
\ {\isacharparenleft}rule\ allI{\isacharparenright}\isanewline
\ \ \ \ \isacommand{fix}\isamarkupfalse%
\ h\isanewline
\ \ \ \ \isacommand{show}\isamarkupfalse%
\ {\isachardoublequoteopen}{\isacharparenleft}g\ {\isasymcirc}\ f\ {\isacharequal}\ h\ {\isasymcirc}\ f{\isacharparenright}\ {\isasymlongrightarrow}\ {\isacharparenleft}g\ {\isacharequal}\ h{\isacharparenright}{\isachardoublequoteclose}\ \isanewline
\ \ \ \ \isacommand{proof}\isamarkupfalse%
\ {\isacharparenleft}rule\ impI{\isacharparenright}\isanewline
\ \ \ \ \ \ \isacommand{assume}\isamarkupfalse%
\ {\isadigit{1}}{\isacharcolon}\ {\isachardoublequoteopen}g\ {\isasymcirc}\ f\ {\isacharequal}\ h\ {\isasymcirc}\ f{\isachardoublequoteclose}\isanewline
\ \ \ \ \ \ \isacommand{show}\isamarkupfalse%
\ {\isachardoublequoteopen}g\ {\isacharequal}\ h{\isachardoublequoteclose}\isanewline
\ \ \ \ \ \ \isacommand{proof}\isamarkupfalse%
\ \ \isanewline
\ \ \ \ \ \ \ \ \isacommand{fix}\isamarkupfalse%
\ x\isanewline
\ \ \ \ \ \ \ \ \isacommand{have}\isamarkupfalse%
\ {\isachardoublequoteopen}\ {\isasymexists}y\ {\isachardot}\ x\ {\isacharequal}\ f{\isacharparenleft}y{\isacharparenright}{\isachardoublequoteclose}\ \isacommand{using}\isamarkupfalse%
\ assms\ \isacommand{by}\isamarkupfalse%
\ {\isacharparenleft}simp\ add{\isacharcolon}surj{\isacharunderscore}def{\isacharparenright}\isanewline
\ \ \ \ \ \ \ \ \isacommand{then}\isamarkupfalse%
\ \isacommand{obtain}\isamarkupfalse%
\ \ {\isachardoublequoteopen}y{\isachardoublequoteclose}\ \isakeyword{where}\ {\isadigit{2}}{\isacharcolon}{\isachardoublequoteopen}x\ {\isacharequal}\ f{\isacharparenleft}y{\isacharparenright}{\isachardoublequoteclose}\ \isacommand{by}\isamarkupfalse%
\ {\isacharparenleft}rule\ exE{\isacharparenright}\isanewline
\ \ \ \ \ \ \ \ \isacommand{then}\isamarkupfalse%
\ \isacommand{have}\isamarkupfalse%
\ \ {\isachardoublequoteopen}g{\isacharparenleft}x{\isacharparenright}\ {\isacharequal}\ g{\isacharparenleft}f{\isacharparenleft}y{\isacharparenright}{\isacharparenright}{\isachardoublequoteclose}\ \isacommand{by}\isamarkupfalse%
\ simp\isanewline
\ \ \ \ \ \ \ \ \isacommand{also}\isamarkupfalse%
\ \isacommand{have}\isamarkupfalse%
\ \ {\isachardoublequoteopen}{\isachardot}{\isachardot}{\isachardot}\ {\isacharequal}\ {\isacharparenleft}g\ {\isasymcirc}\ f{\isacharparenright}\ {\isacharparenleft}y{\isacharparenright}\ \ {\isachardoublequoteclose}\ \isacommand{by}\isamarkupfalse%
\ simp\isanewline
\ \ \ \ \ \ \ \ \isacommand{also}\isamarkupfalse%
\ \isacommand{have}\isamarkupfalse%
\ \ {\isachardoublequoteopen}{\isachardot}{\isachardot}{\isachardot}\ {\isacharequal}\ {\isacharparenleft}h\ {\isasymcirc}\ f{\isacharparenright}\ {\isacharparenleft}y{\isacharparenright}{\isachardoublequoteclose}\ \isacommand{using}\isamarkupfalse%
\ {\isadigit{1}}\ \isacommand{by}\isamarkupfalse%
\ simp\isanewline
\ \ \ \ \ \ \ \ \isacommand{also}\isamarkupfalse%
\ \isacommand{have}\isamarkupfalse%
\ \ {\isachardoublequoteopen}{\isachardot}{\isachardot}{\isachardot}\ {\isacharequal}\ h{\isacharparenleft}f{\isacharparenleft}y{\isacharparenright}{\isacharparenright}{\isachardoublequoteclose}\ \isacommand{by}\isamarkupfalse%
\ simp\isanewline
\ \ \ \ \ \ \ \ \isacommand{also}\isamarkupfalse%
\ \isacommand{have}\isamarkupfalse%
\ \ {\isachardoublequoteopen}{\isachardot}{\isachardot}{\isachardot}\ {\isacharequal}\ h{\isacharparenleft}x{\isacharparenright}{\isachardoublequoteclose}\ \isacommand{using}\isamarkupfalse%
\ {\isadigit{2}}\ \ \ \isacommand{by}\isamarkupfalse%
\ {\isacharparenleft}simp\ add{\isacharcolon}\ {\isacartoucheopen}x\ {\isacharequal}\ f\ y{\isacartoucheclose}{\isacharparenright}\isanewline
\ \ \ \ \ \ \ \ \isacommand{finally}\isamarkupfalse%
\ \isacommand{show}\isamarkupfalse%
\ \ {\isachardoublequoteopen}\ g{\isacharparenleft}x{\isacharparenright}\ {\isacharequal}\ h{\isacharparenleft}x{\isacharparenright}\ {\isachardoublequoteclose}\ \isacommand{by}\isamarkupfalse%
\ simp\isanewline
\ \ \ \ \ \ \isacommand{qed}\isamarkupfalse%
\isanewline
\ \ \ \ \isacommand{qed}\isamarkupfalse%
\isanewline
\ \ \isacommand{qed}\isamarkupfalse%
\isanewline
\isacommand{qed}\isamarkupfalse%
%
\endisatagproof
{\isafoldproof}%
%
\isadelimproof
\isanewline
%
\endisadelimproof
\isanewline
\isanewline
\isacommand{lemma}\isamarkupfalse%
\ sobreyectivadetallada{\isadigit{2}}{\isacharcolon}\isanewline
\ \ \isakeyword{fixes}\ f\ {\isacharcolon}{\isacharcolon}\ {\isachardoublequoteopen}{\isacharprime}c\ {\isasymRightarrow}\ {\isacharprime}a{\isachardoublequoteclose}\ \isanewline
\ \ \isakeyword{assumes}\ {\isachardoublequoteopen}{\isasymforall}{\isacharparenleft}g\ {\isacharcolon}{\isacharcolon}\ {\isacharprime}a\ {\isasymRightarrow}\ {\isacharprime}b{\isacharparenright}\ {\isacharparenleft}h\ {\isacharcolon}{\isacharcolon}\ {\isacharprime}a\ {\isasymRightarrow}\ {\isacharprime}b{\isacharparenright}{\isachardot}\ {\isacharparenleft}\ g\ {\isasymcirc}\ f\ {\isacharequal}\ h\ {\isasymcirc}\ f\ {\isacharparenright}\ {\isasymlongrightarrow}\ {\isacharparenleft}g\ {\isacharequal}\ h{\isacharparenright}{\isachardoublequoteclose}\isanewline
\ \ \isakeyword{shows}\ {\isachardoublequoteopen}surj\ f{\isachardoublequoteclose}\isanewline
%
\isadelimproof
%
\endisadelimproof
%
\isatagproof
\isacommand{proof}\isamarkupfalse%
\ {\isacharparenleft}rule\ surjI{\isacharparenright}\isanewline
\ \ \isacommand{assume}\isamarkupfalse%
\ {\isadigit{1}}{\isacharcolon}{\isachardoublequoteopen}\ {\isasymnot}\ surj\ f{\isachardoublequoteclose}\isanewline
\ \ \isacommand{have}\isamarkupfalse%
\ {\isachardoublequoteopen}\ {\isasymnot}{\isacharparenleft}{\isasymforall}y{\isachardot}\ {\isasymexists}x{\isachardot}\ y\ {\isacharequal}\ f\ x{\isacharparenright}{\isachardoublequoteclose}\ \isacommand{using}\isamarkupfalse%
\ {\isadigit{1}}\ \isacommand{by}\isamarkupfalse%
\ {\isacharparenleft}simp\ add{\isacharcolon}\ surj{\isacharunderscore}def{\isacharparenright}\isanewline
\ \ \isacommand{then}\isamarkupfalse%
\ \isacommand{have}\isamarkupfalse%
\ {\isachardoublequoteopen}{\isasymexists}y{\isachardot}\ {\isasymnexists}x{\isachardot}\ y\ {\isacharequal}\ f\ x{\isachardoublequoteclose}\ \isacommand{by}\isamarkupfalse%
\ simp\isanewline
\ \ \isacommand{then}\isamarkupfalse%
\ \isacommand{obtain}\isamarkupfalse%
\ y{\isadigit{1}}\ \isakeyword{where}\ {\isachardoublequoteopen}{\isasymnexists}x{\isachardot}\ y{\isadigit{1}}\ {\isacharequal}\ f\ x{\isachardoublequoteclose}\ \isacommand{by}\isamarkupfalse%
\ {\isacharparenleft}rule\ exE{\isacharparenright}\isanewline
\ \ \isacommand{then}\isamarkupfalse%
\ \isacommand{have}\isamarkupfalse%
\ {\isachardoublequoteopen}{\isasymforall}x{\isachardot}\ y{\isadigit{1}}\ {\isasymnoteq}\ f\ x{\isachardoublequoteclose}\ \ \isacommand{by}\isamarkupfalse%
\ simp\isanewline
\ \ \isacommand{let}\isamarkupfalse%
\ {\isacharquery}g\ {\isacharequal}\ {\isachardoublequoteopen}{\isasymlambda}x\ {\isacharcolon}{\isacharcolon}\ {\isacharprime}a{\isachardot}\ a\ {\isacharcolon}{\isacharcolon}\ {\isacharprime}b{\isachardoublequoteclose}\ \isanewline
\ \ \isacommand{let}\isamarkupfalse%
\ {\isacharquery}h\ {\isacharequal}{\isachardoublequoteopen}\ fun{\isacharunderscore}upd\ {\isacharquery}g\ y{\isadigit{1}}\ {\isacharparenleft}b\ {\isacharcolon}{\isacharcolon}\ {\isacharprime}b{\isacharparenright}{\isachardoublequoteclose}\isanewline
\ \ \isacommand{have}\isamarkupfalse%
\ {\isadigit{2}}{\isacharcolon}{\isachardoublequoteopen}{\isacharquery}g\ {\isasymcirc}\ f\ {\isacharequal}\ {\isacharquery}h\ {\isasymcirc}\ f\ {\isasymlongrightarrow}\ {\isacharquery}g\ {\isacharequal}\ {\isacharquery}h{\isachardoublequoteclose}\ \isacommand{using}\isamarkupfalse%
\ assms\ \isacommand{by}\isamarkupfalse%
\ blast\isanewline
\ \ \isacommand{have}\isamarkupfalse%
\ {\isadigit{3}}{\isacharcolon}{\isachardoublequoteopen}{\isacharquery}g\ {\isasymcirc}\ f\ {\isacharequal}\ {\isacharquery}h\ {\isasymcirc}\ f{\isachardoublequoteclose}\ \isanewline
\ \ \ \ \isacommand{by}\isamarkupfalse%
\ {\isacharparenleft}metis\ {\isacharparenleft}mono{\isacharunderscore}tags{\isacharcomma}\ lifting{\isacharparenright}\ fun{\isacharunderscore}upd{\isacharunderscore}def\ {\isacartoucheopen}{\isasymnexists}x\ {\isacharcolon}{\isacharcolon}\ {\isacharprime}c{\isachardot}\ {\isacharparenleft}y{\isadigit{1}}\ {\isacharcolon}{\isacharcolon}\ {\isacharprime}a{\isacharparenright}\ {\isacharequal}\isanewline
\ {\isacharparenleft}f\ {\isacharcolon}{\isacharcolon}\ {\isacharprime}c\ {\isasymRightarrow}\ {\isacharprime}a{\isacharparenright}\ x{\isacartoucheclose}\ f{\isacharunderscore}inv{\isacharunderscore}into{\isacharunderscore}f\ fun{\isachardot}map{\isacharunderscore}cong{\isadigit{0}}{\isacharparenright}\isanewline
\ \ \isacommand{have}\isamarkupfalse%
\ {\isachardoublequoteopen}{\isacharquery}g\ {\isacharequal}\ {\isacharquery}h{\isachardoublequoteclose}\ \isacommand{using}\isamarkupfalse%
\ {\isadigit{2}}\ {\isadigit{3}}\ \isacommand{by}\isamarkupfalse%
\ {\isacharparenleft}rule\ mp{\isacharparenright}\isanewline
\ \ \isacommand{have}\isamarkupfalse%
\ {\isachardoublequoteopen}{\isacharquery}g\ {\isasymnoteq}\ {\isacharquery}h{\isachardoublequoteclose}\ \isanewline
\ \ \isacommand{proof}\isamarkupfalse%
\ \isanewline
\ \ \ \ \isacommand{assume}\isamarkupfalse%
\ {\isadigit{4}}{\isacharcolon}\ {\isachardoublequoteopen}{\isacharquery}g\ {\isacharequal}\ {\isacharquery}h{\isachardoublequoteclose}\isanewline
\ \ \ \ \isacommand{show}\isamarkupfalse%
\ False\isanewline
\ \ \ \ \isacommand{proof}\isamarkupfalse%
\ {\isacharminus}\isanewline
\ \ \ \ \ \ \isacommand{have}\isamarkupfalse%
\ {\isachardoublequoteopen}{\isacharquery}g\ {\isacharequal}\ fun{\isacharunderscore}upd\ {\isacharquery}g\ y{\isadigit{1}}\ {\isacharparenleft}b\ {\isacharcolon}{\isacharcolon}\ {\isacharprime}b{\isacharparenright}{\isachardoublequoteclose}\ \isacommand{using}\isamarkupfalse%
\ {\isadigit{4}}\ \isacommand{by}\isamarkupfalse%
\ simp\isanewline
\ \ \ \ \ \ \isacommand{also}\isamarkupfalse%
\ \isacommand{have}\isamarkupfalse%
\ {\isachardoublequoteopen}{\isachardot}{\isachardot}{\isachardot}\ {\isacharequal}\ \ {\isacharparenleft}{\isasymlambda}x{\isachardot}\ if\ x\ {\isacharequal}\ y{\isadigit{1}}\ then\ b\ else\ {\isacharquery}g\ x{\isacharparenright}{\isachardoublequoteclose}\ \ \isacommand{by}\isamarkupfalse%
\ {\isacharparenleft}simp\ add{\isacharcolon}\isanewline
fun{\isacharunderscore}upd{\isacharunderscore}def{\isacharparenright}\isanewline
\ \ \ \ \ \ \isacommand{finally}\isamarkupfalse%
\ \isacommand{have}\isamarkupfalse%
\ {\isadigit{5}}{\isacharcolon}\ {\isachardoublequoteopen}{\isacharquery}g\ {\isacharequal}\ {\isacharparenleft}{\isasymlambda}x{\isachardot}\ if\ x\ {\isacharequal}\ y{\isadigit{1}}\ then\ b\ else\ {\isacharquery}g\ x{\isacharparenright}{\isachardoublequoteclose}\ \isacommand{by}\isamarkupfalse%
\ simp\isanewline
\ \ \ \ \ \ \isacommand{show}\isamarkupfalse%
\ False\isanewline
\ \ \ \ \ \ \isacommand{proof}\isamarkupfalse%
\ {\isacharparenleft}cases{\isacharparenright}\isanewline
\ \ \ \ \ \ \ \ \isacommand{oops}\isamarkupfalse%
%
\endisatagproof
{\isafoldproof}%
%
\isadelimproof
%
\endisadelimproof
%
\begin{isamarkuptext}%
En la demostración hemos introducido: 
 \begin{itemize}
    \item[] \isa{\mbox{}\inferrule{\mbox{{\isasymexists}x\ {\isacharcolon}{\isacharcolon}\ {\isacharprime}a{\isachardot}\ {\isacharparenleft}P\ {\isacharcolon}{\isacharcolon}\ {\isacharprime}a\ {\isasymRightarrow}\ bool{\isacharparenright}\ x}\\\ \mbox{{\isasymAnd}x\ {\isacharcolon}{\isacharcolon}\ {\isacharprime}a{\isachardot}\ \mbox{}\inferrule{\mbox{P\ x}}{\mbox{Q\ {\isacharcolon}{\isacharcolon}\ bool}}}}{\mbox{Q}}} 
      \hfill (\isa{rule\ exE}) 
  \end{itemize} 
 \begin{itemize}
    \item[] \isa{{\isasymlbrakk}P\ {\isacharcolon}{\isacharcolon}\ bool\ {\isasymLongrightarrow}\ Q\ {\isacharcolon}{\isacharcolon}\ bool{\isacharsemicolon}\ Q\ {\isasymLongrightarrow}\ P{\isasymrbrakk}\ {\isasymLongrightarrow}\ P\ {\isacharequal}\ Q} 
      \hfill (\isa{iffI})
  \end{itemize} 

La demostración aplicativa es:%
\end{isamarkuptext}\isamarkuptrue%
\isacommand{lemma}\isamarkupfalse%
\ {\isachardoublequoteopen}surj\ f\ {\isasymLongrightarrow}\ {\isacharparenleft}{\isacharparenleft}g\ {\isasymcirc}\ f{\isacharparenright}\ {\isacharequal}\ {\isacharparenleft}h\ {\isasymcirc}\ f{\isacharparenright}\ {\isacharparenright}\ {\isasymlongrightarrow}\ {\isacharparenleft}g\ {\isacharequal}\ h{\isacharparenright}{\isachardoublequoteclose}\isanewline
%
\isadelimproof
\ \ %
\endisadelimproof
%
\isatagproof
\isacommand{apply}\isamarkupfalse%
\ {\isacharparenleft}simp\ add{\isacharcolon}\ surj{\isacharunderscore}def\ fun{\isacharunderscore}eq{\isacharunderscore}iff{\isacharparenright}\isanewline
\ \ \isacommand{apply}\isamarkupfalse%
\ metis\isanewline
\ \ \isacommand{done}\isamarkupfalse%
%
\endisatagproof
{\isafoldproof}%
%
\isadelimproof
\isanewline
%
\endisadelimproof
\isanewline
\isacommand{lemma}\isamarkupfalse%
\ {\isachardoublequoteopen}surj\ f\ {\isasymLongrightarrow}\ {\isacharparenleft}{\isacharparenleft}g\ {\isasymcirc}\ f{\isacharparenright}\ {\isacharequal}\ {\isacharparenleft}h\ {\isasymcirc}\ f{\isacharparenright}\ {\isacharparenright}\ {\isasymlongrightarrow}{\isacharparenleft}g\ {\isacharequal}\ h{\isacharparenright}{\isachardoublequoteclose}\isanewline
%
\isadelimproof
\ \ %
\endisadelimproof
%
\isatagproof
\isacommand{apply}\isamarkupfalse%
\ {\isacharparenleft}simp\ add{\isacharcolon}\ surj{\isacharunderscore}def\ fun{\isacharunderscore}eq{\isacharunderscore}iff\ {\isacharparenright}\ \isanewline
\ \ \isacommand{by}\isamarkupfalse%
\ metis%
\endisatagproof
{\isafoldproof}%
%
\isadelimproof
%
\endisadelimproof
%
\begin{isamarkuptext}%
En esta demostración hemos introducido:
 \begin{itemize}
    \item[] \isa{{\isacharparenleft}{\isacharparenleft}f\ {\isacharcolon}{\isacharcolon}\ {\isacharprime}a\ {\isasymRightarrow}\ {\isacharprime}b{\isacharparenright}\ {\isacharequal}\ {\isacharparenleft}g\ {\isacharcolon}{\isacharcolon}\ {\isacharprime}a\ {\isasymRightarrow}\ {\isacharprime}b{\isacharparenright}{\isacharparenright}\ {\isacharequal}\ {\isacharparenleft}{\isasymforall}x\ {\isacharcolon}{\isacharcolon}\ {\isacharprime}a{\isachardot}\ f\ x\ {\isacharequal}\ g\ x{\isacharparenright}} 
      \hfill (\isa{fun{\isacharunderscore}eq{\isacharunderscore}iff})
  \end{itemize}%
\end{isamarkuptext}\isamarkuptrue%
%
\isadelimtheory
%
\endisadelimtheory
%
\isatagtheory
%
\endisatagtheory
{\isafoldtheory}%
%
\isadelimtheory
%
\endisadelimtheory
%
\end{isabellebody}%
\endinput
%:%file=~/Escritorio/TFG/CancelacionSobreyectiva.thy%:%
%:%24=7%:%
%:%36=10%:%
%:%37=11%:%
%:%38=12%:%
%:%39=13%:%
%:%40=14%:%
%:%41=15%:%
%:%42=16%:%
%:%43=17%:%
%:%44=18%:%
%:%45=19%:%
%:%46=20%:%
%:%47=21%:%
%:%48=22%:%
%:%49=23%:%
%:%50=24%:%
%:%51=25%:%
%:%52=26%:%
%:%53=27%:%
%:%54=28%:%
%:%55=29%:%
%:%56=30%:%
%:%57=31%:%
%:%58=32%:%
%:%59=33%:%
%:%60=34%:%
%:%61=35%:%
%:%62=36%:%
%:%63=37%:%
%:%64=38%:%
%:%65=39%:%
%:%66=40%:%
%:%67=41%:%
%:%68=42%:%
%:%69=43%:%
%:%70=44%:%
%:%71=45%:%
%:%72=46%:%
%:%73=47%:%
%:%74=48%:%
%:%75=49%:%
%:%76=50%:%
%:%77=51%:%
%:%78=52%:%
%:%79=53%:%
%:%80=54%:%
%:%81=55%:%
%:%82=56%:%
%:%83=57%:%
%:%84=58%:%
%:%85=59%:%
%:%86=60%:%
%:%87=61%:%
%:%88=62%:%
%:%89=63%:%
%:%90=64%:%
%:%91=65%:%
%:%92=66%:%
%:%93=67%:%
%:%94=68%:%
%:%95=69%:%
%:%96=70%:%
%:%97=71%:%
%:%99=74%:%
%:%100=74%:%
%:%101=75%:%
%:%104=76%:%
%:%108=76%:%
%:%114=76%:%
%:%117=77%:%
%:%118=78%:%
%:%119=78%:%
%:%120=79%:%
%:%123=80%:%
%:%127=80%:%
%:%133=80%:%
%:%136=81%:%
%:%137=82%:%
%:%138=82%:%
%:%139=83%:%
%:%142=84%:%
%:%146=84%:%
%:%156=88%:%
%:%157=89%:%
%:%158=90%:%
%:%159=91%:%
%:%160=92%:%
%:%161=93%:%
%:%162=94%:%
%:%163=95%:%
%:%164=96%:%
%:%165=97%:%
%:%166=98%:%
%:%167=99%:%
%:%168=100%:%
%:%169=101%:%
%:%170=102%:%
%:%171=103%:%
%:%172=104%:%
%:%173=105%:%
%:%174=106%:%
%:%175=107%:%
%:%176=108%:%
%:%177=109%:%
%:%179=113%:%
%:%180=114%:%
%:%181=114%:%
%:%182=115%:%
%:%183=116%:%
%:%190=117%:%
%:%191=117%:%
%:%192=118%:%
%:%193=118%:%
%:%194=119%:%
%:%195=119%:%
%:%196=120%:%
%:%197=120%:%
%:%198=121%:%
%:%199=121%:%
%:%200=122%:%
%:%201=122%:%
%:%202=123%:%
%:%203=123%:%
%:%204=124%:%
%:%205=124%:%
%:%206=125%:%
%:%207=125%:%
%:%208=126%:%
%:%209=126%:%
%:%210=127%:%
%:%211=127%:%
%:%212=128%:%
%:%213=128%:%
%:%214=128%:%
%:%215=128%:%
%:%216=129%:%
%:%217=129%:%
%:%218=129%:%
%:%219=129%:%
%:%220=130%:%
%:%221=130%:%
%:%222=130%:%
%:%223=130%:%
%:%224=131%:%
%:%225=131%:%
%:%226=131%:%
%:%227=131%:%
%:%228=132%:%
%:%229=132%:%
%:%230=132%:%
%:%231=132%:%
%:%232=132%:%
%:%233=133%:%
%:%234=133%:%
%:%235=133%:%
%:%236=133%:%
%:%237=134%:%
%:%238=134%:%
%:%239=134%:%
%:%240=134%:%
%:%241=134%:%
%:%242=135%:%
%:%243=135%:%
%:%244=135%:%
%:%245=135%:%
%:%246=136%:%
%:%247=136%:%
%:%248=137%:%
%:%249=137%:%
%:%250=138%:%
%:%251=138%:%
%:%252=139%:%
%:%258=139%:%
%:%261=140%:%
%:%262=141%:%
%:%263=142%:%
%:%264=142%:%
%:%265=143%:%
%:%266=144%:%
%:%267=145%:%
%:%274=146%:%
%:%275=146%:%
%:%276=147%:%
%:%277=147%:%
%:%278=148%:%
%:%279=148%:%
%:%280=148%:%
%:%281=148%:%
%:%282=149%:%
%:%283=149%:%
%:%284=149%:%
%:%285=149%:%
%:%286=150%:%
%:%287=150%:%
%:%288=150%:%
%:%289=150%:%
%:%290=151%:%
%:%291=151%:%
%:%292=151%:%
%:%293=151%:%
%:%294=152%:%
%:%295=152%:%
%:%296=153%:%
%:%297=153%:%
%:%298=154%:%
%:%299=154%:%
%:%300=154%:%
%:%301=154%:%
%:%302=155%:%
%:%303=155%:%
%:%304=156%:%
%:%305=156%:%
%:%306=157%:%
%:%307=158%:%
%:%308=158%:%
%:%309=158%:%
%:%310=158%:%
%:%311=159%:%
%:%312=159%:%
%:%313=160%:%
%:%314=160%:%
%:%315=161%:%
%:%316=161%:%
%:%317=162%:%
%:%318=162%:%
%:%319=163%:%
%:%320=163%:%
%:%321=164%:%
%:%322=164%:%
%:%323=164%:%
%:%324=164%:%
%:%325=165%:%
%:%326=165%:%
%:%327=165%:%
%:%328=165%:%
%:%329=166%:%
%:%330=167%:%
%:%331=167%:%
%:%332=167%:%
%:%333=167%:%
%:%334=168%:%
%:%335=168%:%
%:%336=169%:%
%:%337=169%:%
%:%338=170%:%
%:%348=174%:%
%:%349=175%:%
%:%350=176%:%
%:%351=177%:%
%:%352=178%:%
%:%353=179%:%
%:%354=180%:%
%:%355=181%:%
%:%356=182%:%
%:%357=183%:%
%:%358=184%:%
%:%360=186%:%
%:%361=186%:%
%:%364=187%:%
%:%368=187%:%
%:%369=187%:%
%:%370=188%:%
%:%371=188%:%
%:%372=189%:%
%:%378=189%:%
%:%381=190%:%
%:%382=191%:%
%:%383=191%:%
%:%386=192%:%
%:%390=192%:%
%:%391=192%:%
%:%392=193%:%
%:%393=193%:%
%:%402=196%:%
%:%403=197%:%
%:%404=198%:%
%:%405=199%:%
%:%406=200%:%