%
\begin{isabellebody}%
\setisabellecontext{TeoremaCantor}%
%
\isadelimtheory
\isanewline
%
\endisadelimtheory
%
\isatagtheory
%
\endisatagtheory
{\isafoldtheory}%
%
\isadelimtheory
%
\endisadelimtheory
%
\begin{isamarkuptext}%
\comentario{Estructurar en secciones.}%
\end{isamarkuptext}\isamarkuptrue%
%
\begin{isamarkuptext}%
\comentario{Hacer demostraciones detalladas.}%
\end{isamarkuptext}\isamarkuptrue%
%
\begin{isamarkuptext}%
\comentario{Añadir lemas usados al Soporte.}%
\end{isamarkuptext}\isamarkuptrue%
%
\isadelimdocument
%
\endisadelimdocument
%
\isatagdocument
%
\isamarkupsection{Demostración en lenguaje natural%
}
\isamarkuptrue%
%
\endisatagdocument
{\isafolddocument}%
%
\isadelimdocument
%
\endisadelimdocument
%
\begin{isamarkuptext}%
El siguiente, denominado  teorema de Cantor por el matemático
 Georg Cantor, es un resultado importante de la teoría
 de conjuntos. 

El matemático Georg Ferdinand Ludwig Philipp Cantor fue un matemático y
lógico nacido en Rusia en el siglo XIX. Fue inventor junto con Dedekind
 y Frege de la teoría de conjuntos, que es la base de las matemáticas
 modernas.


Para la comprensión del teorema vamos a definir una serie de conceptos:

\begin {itemize}

\item Conjunto de potencia $A$  $(\mathcal{P}(A))$: conjunto formado por
todos los subconjuntos de $A$.

\item Cardinal del conjunto $A$ (Denotado $\# A$): número de elementos 
del propio  conjunto.

\end {itemize}
El enunciado original del teorema es el siguiente : 


\begin {teorema}
El cardinal del conjunto potencia de cualquier conjunto A es
 estrictamente mayor que el cardinal de A, o lo que es lo mismo,
$\# \mathcal{P}(A) > \# A.$


\end {teorema}
Pero el enunciado del teorema lo podemos reformular como: 
\begin{teorema}
Dado un conjunto $A$, $\nexists  f : A \longrightarrow \mathcal{P}(A)$ 
que sea sobreyectiva.

\end{teorema}

El teorema se puede reescribir de la anterior forma, ya que si
 se supone que $\exists f$ tal que $f: A \longrightarrow \mathcal{P}(A)$
es sobreyectiva, entonces tenemos que $f(A) = \mathcal{P}(A)$ y por lo
 tanto, $\# f(A) \geq \# \mathcal{P}(A)$, de lo que se deduce esta
 reformulación. Recíprocamente, es trivial ver que esta reformulación
 implica la primera del teorema. 


\begin{demostracion}

La prueba se va a realizar por reducción al absurdo.

Supongamos que $\exists f : A \longrightarrow \mathcal{P}(A)$
 sobreyectiva, es decir, $\forall B \in \mathcal{P}(A) ,  \exists x \in
 A$ tal que $B = f(x)$.

En particular, tomemos el conjunto
 $$B = \{ x \in A : x \notin f(x) \}$$

 y  supongamos que $\exists a \in A : B = f(a)$, ya que $B$ es un
 subconjunto de A. Luego podemos distinguir dos casos :

$1.$ Si $a \in B.$

 Entonces por definición del conjunto $B$ se tiene que
$a \notin B$, luego se llega a una contradicción. 

$2.$ Si $a \notin B.$
 Entonces por definición de B se tiene que $a \in 
B$, luego se ha llegado a otra contradicción. 

En los dos casos se ha llegado a una contradicción,por lo que no existe
 $a$ y $f$ no es sobreyectiva.
\end{demostracion}%
\end{isamarkuptext}\isamarkuptrue%
%
\isadelimdocument
%
\endisadelimdocument
%
\isatagdocument
%
\isamarkupsection{Especificación en Isabelle/HOL%
}
\isamarkuptrue%
%
\endisatagdocument
{\isafolddocument}%
%
\isadelimdocument
%
\endisadelimdocument
%
\begin{isamarkuptext}%
Para la especificación del teorema en Isabelle, primero debemos notar
 que $$f :: \, 'a \Rightarrow \,'a \: set$$ significa que es una función 
de tipos, donde $'a$ significa un tipo y para poder denotar
el conjunto potencia tenemos que poner $'a \ set$ que significa que es
 de un tipo formado por conjuntos del tipo $'a$.


El enunciado del teorema es el siguiente :%
\end{isamarkuptext}\isamarkuptrue%
\isacommand{theorem}\isamarkupfalse%
\ Cantor{\isacharcolon}\ {\isachardoublequoteopen}{\isasymnexists}f\ {\isacharcolon}{\isacharcolon}\ {\isacharprime}a\ {\isasymRightarrow}\ {\isacharprime}a\ set{\isachardot}\ {\isasymforall}A{\isachardot}\ {\isasymexists}x{\isachardot}\ A\ {\isacharequal}\ f\ x{\isachardoublequoteclose}\isanewline
\isanewline
%
\isadelimproof
\isanewline
\ \ %
\endisadelimproof
%
\isatagproof
\isacommand{oops}\isamarkupfalse%
%
\endisatagproof
{\isafoldproof}%
%
\isadelimproof
%
\endisadelimproof
%
\begin{isamarkuptext}%
La demostración la haremos por la regla la introducción a la
negación, la cual es una simplificación de la regla de 
reducción al absurdo, cuyo esquema mostramos a continuación:   
 \begin{itemize}
  \item[] \isa{{\isacharparenleft}P\ {\isasymLongrightarrow}\ False{\isacharparenright}\ {\isasymLongrightarrow}\ {\isasymnot}\ P} \hfill (\isa{notI})
  \end{itemize}



A continuación presentaremos diferentes formas de demostración del
 teorema.%
\end{isamarkuptext}\isamarkuptrue%
%
\isadelimdocument
%
\endisadelimdocument
%
\isatagdocument
%
\isamarkupsection{Demostración detallada%
}
\isamarkuptrue%
%
\endisatagdocument
{\isafolddocument}%
%
\isadelimdocument
%
\endisadelimdocument
%
\begin{isamarkuptext}%
La primera es la demostración detallada del teorema:%
\end{isamarkuptext}\isamarkuptrue%
\isacommand{theorem}\isamarkupfalse%
\ CantorDetallada{\isacharcolon}\ {\isachardoublequoteopen}{\isasymnexists}f\ {\isacharcolon}{\isacharcolon}\ {\isacharprime}a\ {\isasymRightarrow}\ {\isacharprime}a\ set{\isachardot}\ {\isasymforall}B{\isachardot}\ {\isasymexists}x{\isachardot}\ B\ {\isacharequal}\ f\ x{\isachardoublequoteclose}\isanewline
%
\isadelimproof
%
\endisadelimproof
%
\isatagproof
\isacommand{proof}\isamarkupfalse%
\ {\isacharparenleft}rule\ notI{\isacharparenright}\isanewline
\ \ \isacommand{assume}\isamarkupfalse%
\ {\isachardoublequoteopen}{\isasymexists}f\ {\isacharcolon}{\isacharcolon}\ {\isacharprime}a\ {\isasymRightarrow}\ {\isacharprime}a\ set{\isachardot}\ {\isasymforall}A{\isachardot}\ {\isasymexists}x{\isachardot}\ A\ {\isacharequal}\ f\ x{\isachardoublequoteclose}\isanewline
\ \ \isacommand{then}\isamarkupfalse%
\ \isacommand{obtain}\isamarkupfalse%
\ f\ {\isacharcolon}{\isacharcolon}\ {\isachardoublequoteopen}{\isacharprime}a\ {\isasymRightarrow}\ {\isacharprime}a\ set{\isachardoublequoteclose}\ \isakeyword{where}\ {\isacharasterisk}{\isacharcolon}\ {\isachardoublequoteopen}{\isasymforall}A{\isachardot}\ {\isasymexists}x{\isachardot}\ A\ {\isacharequal}\ f\ x{\isachardoublequoteclose}\ \isacommand{by}\isamarkupfalse%
\ {\isacharparenleft}rule\isanewline
\ \ \ \ \ \ \ \ exE{\isacharparenright}\isanewline
\ \ \isacommand{let}\isamarkupfalse%
\ {\isacharquery}B\ {\isacharequal}\ {\isachardoublequoteopen}{\isacharbraceleft}x{\isachardot}\ x\ {\isasymnotin}\ f\ x{\isacharbraceright}{\isachardoublequoteclose}\isanewline
\ \ \isacommand{from}\isamarkupfalse%
\ {\isacharasterisk}\ \isacommand{obtain}\isamarkupfalse%
\ {\isachardoublequoteopen}\ {\isasymexists}x{\isachardot}\ {\isacharquery}B\ {\isacharequal}\ f\ x\ {\isachardoublequoteclose}\ \isacommand{by}\isamarkupfalse%
\ {\isacharparenleft}rule\ allE{\isacharparenright}\isanewline
\ \ \isacommand{then}\isamarkupfalse%
\ \ \isacommand{obtain}\isamarkupfalse%
\ a\ \isakeyword{where}\ {\isadigit{1}}{\isacharcolon}{\isachardoublequoteopen}{\isacharquery}B\ {\isacharequal}\ f\ a{\isachardoublequoteclose}\ \isacommand{by}\isamarkupfalse%
\ {\isacharparenleft}rule\ exE{\isacharparenright}\isanewline
\ \ \isacommand{show}\isamarkupfalse%
\ False\isanewline
\ \ \isacommand{proof}\isamarkupfalse%
\ {\isacharparenleft}cases{\isacharparenright}\isanewline
\ \ \ \ \isacommand{assume}\isamarkupfalse%
\ {\isachardoublequoteopen}a\ {\isasymin}\ {\isacharquery}B{\isachardoublequoteclose}\ \ \isanewline
\ \ \ \ \isacommand{then}\isamarkupfalse%
\ \isacommand{show}\isamarkupfalse%
\ False\ \ \isacommand{using}\isamarkupfalse%
\ {\isadigit{1}}\ \isacommand{by}\isamarkupfalse%
\ blast\isanewline
\ \ \isacommand{next}\isamarkupfalse%
\ \isanewline
\ \ \ \ \isacommand{assume}\isamarkupfalse%
\ {\isachardoublequoteopen}a\ {\isasymnotin}\ {\isacharquery}B{\isachardoublequoteclose}\isanewline
\ \ \ \ \isacommand{thus}\isamarkupfalse%
\ False\ \isacommand{using}\isamarkupfalse%
\ {\isadigit{1}}\ \isacommand{by}\isamarkupfalse%
\ blast\isanewline
\ \ \isacommand{qed}\isamarkupfalse%
\isanewline
\isacommand{qed}\isamarkupfalse%
%
\endisatagproof
{\isafoldproof}%
%
\isadelimproof
%
\endisadelimproof
%
\isadelimdocument
%
\endisadelimdocument
%
\isatagdocument
%
\isamarkupsection{Demostración aplicativa%
}
\isamarkuptrue%
%
\endisatagdocument
{\isafolddocument}%
%
\isadelimdocument
%
\endisadelimdocument
%
\begin{isamarkuptext}%
La demostración aplicativa del teorema es:%
\end{isamarkuptext}\isamarkuptrue%
\isacommand{theorem}\isamarkupfalse%
\ CantorAplicativa\ {\isacharcolon}\isanewline
\ {\isachardoublequoteopen}{\isasymnexists}f\ {\isacharcolon}{\isacharcolon}\ {\isacharprime}a\ {\isasymRightarrow}\ {\isacharprime}a\ set{\isachardot}\ {\isasymforall}A{\isachardot}\ {\isasymexists}x{\isachardot}\ A\ {\isacharequal}\ f\ x{\isachardoublequoteclose}\isanewline
%
\isadelimproof
\ \ %
\endisadelimproof
%
\isatagproof
\isacommand{apply}\isamarkupfalse%
\ {\isacharparenleft}rule\ notI{\isacharparenright}\isanewline
\ \ \isacommand{apply}\isamarkupfalse%
\ {\isacharparenleft}erule\ exE{\isacharparenright}\isanewline
\ \ \isacommand{apply}\isamarkupfalse%
\ {\isacharparenleft}erule{\isacharunderscore}tac\ x\ {\isacharequal}\ {\isachardoublequoteopen}{\isacharbraceleft}x{\isachardot}\ x\ {\isasymnotin}\ f\ x{\isacharbraceright}{\isachardoublequoteclose}\ \isakeyword{in}\ allE{\isacharparenright}\isanewline
\ \ \isacommand{apply}\isamarkupfalse%
\ {\isacharparenleft}erule\ exE{\isacharparenright}\isanewline
\ \ \isacommand{apply}\isamarkupfalse%
\ \ blast\ \isanewline
\ \ \isacommand{done}\isamarkupfalse%
%
\endisatagproof
{\isafoldproof}%
%
\isadelimproof
%
\endisadelimproof
%
\isadelimdocument
%
\endisadelimdocument
%
\isatagdocument
%
\isamarkupsection{Demostración autmática%
}
\isamarkuptrue%
%
\endisatagdocument
{\isafolddocument}%
%
\isadelimdocument
%
\endisadelimdocument
%
\begin{isamarkuptext}%
La demostración automática del teorema es:%
\end{isamarkuptext}\isamarkuptrue%
\isacommand{theorem}\isamarkupfalse%
\ CantorAutomatic{\isacharcolon}\ {\isachardoublequoteopen}{\isasymnexists}f\ {\isacharcolon}{\isacharcolon}\ {\isacharprime}a\ {\isasymRightarrow}\ {\isacharprime}a\ set{\isachardot}\ {\isasymforall}B{\isachardot}\ {\isasymexists}x{\isachardot}\ B\ {\isacharequal}\ f\ x{\isachardoublequoteclose}\isanewline
%
\isadelimproof
\ \ %
\endisadelimproof
%
\isatagproof
\isacommand{by}\isamarkupfalse%
\ best%
\endisatagproof
{\isafoldproof}%
%
\isadelimproof
%
\endisadelimproof
%
\begin{isamarkuptext}%
En la demostración de isabelle hemos utilizado el método de prueba
rule con las siguientes reglas, tanto en la aplicativa como en la
 detallada:
 \begin{itemize}
  \item[] \isa{\mbox{}\inferrule{\mbox{\mbox{}\inferrule{\mbox{P}}{\mbox{False}}}}{\mbox{{\isasymnot}\ P}}} \hfill (\isa{notI})
  \end{itemize}
 \begin{itemize}
  \item[] \isa{\mbox{}\inferrule{\mbox{{\isasymexists}x{\isachardot}\ P\ x}\\\ \mbox{{\isasymAnd}x{\isachardot}\ \mbox{}\inferrule{\mbox{P\ x}}{\mbox{Q}}}}{\mbox{Q}}} \hfill (\isa{exE})
  \end{itemize}
 \begin{itemize}
  \item[] \isa{\mbox{}\inferrule{\mbox{{\isasymforall}x{\isachardot}\ P\ x}\\\ \mbox{\mbox{}\inferrule{\mbox{P\ x}}{\mbox{R}}}}{\mbox{R}}} \hfill (\isa{allE})
  \end{itemize}
También hacemos uso de blast, que es un conjunto de reglas lógicas y 
 la demostración automática la hacemos por medio de "best".%
\end{isamarkuptext}\isamarkuptrue%
%
\isadelimtheory
%
\endisadelimtheory
%
\isatagtheory
%
\endisatagtheory
{\isafoldtheory}%
%
\isadelimtheory
%
\endisadelimtheory
%
\end{isabellebody}%
\endinput
%:%file=~/Escritorio/TFG/TeoremaCantor.thy%:%
%:%6=1%:%
%:%20=9%:%
%:%24=11%:%
%:%28=13%:%
%:%37=15%:%
%:%49=17%:%
%:%50=18%:%
%:%51=19%:%
%:%52=20%:%
%:%53=21%:%
%:%54=22%:%
%:%55=23%:%
%:%56=24%:%
%:%57=25%:%
%:%58=26%:%
%:%59=27%:%
%:%60=28%:%
%:%61=29%:%
%:%62=30%:%
%:%63=31%:%
%:%64=32%:%
%:%65=33%:%
%:%66=34%:%
%:%67=35%:%
%:%68=36%:%
%:%69=37%:%
%:%70=38%:%
%:%71=39%:%
%:%72=40%:%
%:%73=41%:%
%:%74=42%:%
%:%75=43%:%
%:%76=44%:%
%:%77=45%:%
%:%78=46%:%
%:%79=47%:%
%:%80=48%:%
%:%81=49%:%
%:%82=50%:%
%:%83=51%:%
%:%84=52%:%
%:%85=53%:%
%:%86=54%:%
%:%87=55%:%
%:%88=56%:%
%:%89=57%:%
%:%90=58%:%
%:%91=59%:%
%:%92=60%:%
%:%93=61%:%
%:%94=62%:%
%:%95=63%:%
%:%96=64%:%
%:%97=65%:%
%:%98=66%:%
%:%99=67%:%
%:%100=68%:%
%:%101=69%:%
%:%102=70%:%
%:%103=71%:%
%:%104=72%:%
%:%105=73%:%
%:%106=74%:%
%:%107=75%:%
%:%108=76%:%
%:%109=77%:%
%:%110=78%:%
%:%111=79%:%
%:%112=80%:%
%:%113=81%:%
%:%114=82%:%
%:%115=83%:%
%:%116=84%:%
%:%117=85%:%
%:%118=86%:%
%:%119=87%:%
%:%120=88%:%
%:%129=92%:%
%:%141=95%:%
%:%142=96%:%
%:%143=97%:%
%:%144=98%:%
%:%145=99%:%
%:%146=100%:%
%:%147=101%:%
%:%148=102%:%
%:%150=104%:%
%:%151=104%:%
%:%152=105%:%
%:%155=106%:%
%:%156=107%:%
%:%160=107%:%
%:%170=109%:%
%:%171=110%:%
%:%172=111%:%
%:%173=112%:%
%:%174=113%:%
%:%175=114%:%
%:%176=115%:%
%:%177=116%:%
%:%178=117%:%
%:%179=118%:%
%:%180=119%:%
%:%189=121%:%
%:%201=124%:%
%:%203=127%:%
%:%204=127%:%
%:%211=128%:%
%:%212=128%:%
%:%213=129%:%
%:%214=129%:%
%:%215=130%:%
%:%216=130%:%
%:%217=130%:%
%:%218=130%:%
%:%219=131%:%
%:%220=132%:%
%:%221=132%:%
%:%222=133%:%
%:%223=133%:%
%:%224=133%:%
%:%225=133%:%
%:%226=134%:%
%:%227=134%:%
%:%228=134%:%
%:%229=134%:%
%:%230=135%:%
%:%231=135%:%
%:%232=136%:%
%:%233=136%:%
%:%234=137%:%
%:%235=137%:%
%:%236=138%:%
%:%237=138%:%
%:%238=138%:%
%:%239=138%:%
%:%240=138%:%
%:%241=139%:%
%:%242=139%:%
%:%243=140%:%
%:%244=140%:%
%:%245=141%:%
%:%246=141%:%
%:%247=141%:%
%:%248=141%:%
%:%249=142%:%
%:%250=142%:%
%:%251=143%:%
%:%266=145%:%
%:%278=148%:%
%:%280=151%:%
%:%281=151%:%
%:%282=152%:%
%:%285=153%:%
%:%289=153%:%
%:%290=153%:%
%:%291=154%:%
%:%292=154%:%
%:%293=155%:%
%:%294=155%:%
%:%295=156%:%
%:%296=156%:%
%:%297=157%:%
%:%298=157%:%
%:%299=158%:%
%:%314=160%:%
%:%326=162%:%
%:%328=165%:%
%:%329=165%:%
%:%332=166%:%
%:%336=166%:%
%:%337=166%:%
%:%346=168%:%
%:%347=169%:%
%:%348=170%:%
%:%349=171%:%
%:%350=172%:%
%:%351=173%:%
%:%352=174%:%
%:%353=175%:%
%:%354=176%:%
%:%355=177%:%
%:%356=178%:%
%:%357=179%:%
%:%358=180%:%
%:%359=181%:%