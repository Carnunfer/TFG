%
\begin{isabellebody}%
\setisabellecontext{TeoremaCantor}%
%
\isadelimtheory
\isanewline
%
\endisadelimtheory
%
\isatagtheory
%
\endisatagtheory
{\isafoldtheory}%
%
\isadelimtheory
%
\endisadelimtheory
%
\begin{isamarkuptext}%
\comentario{Estructurar en secciones.}%
\end{isamarkuptext}\isamarkuptrue%
%
\begin{isamarkuptext}%
\comentario{Hacer demostraciones detalladas.}%
\end{isamarkuptext}\isamarkuptrue%
%
\begin{isamarkuptext}%
\comentario{Añadir lemas usados al Soporte.}%
\end{isamarkuptext}\isamarkuptrue%
%
\begin{isamarkuptext}%
El siguiente, denominado  teorema de Cantor por el matemático
 Georg Cantor, es un resultado importante de la teoría
 de conjuntos. 

El matemático Georg Ferdinand Ludwig Philipp Cantor fue un matemático y
lógico nacido en Rusia en el siglo XIX. Fue inventor junto con Dedekind
 y Frege de la teoría de conjuntos, que es la base de las matemáticas
 modernas.



Para la comprensión del teorema vamos a definir una serie de conceptos:

\begin {itemize}

\item Conjunto de potencia $A$  $(\mathcal{P}(A))$: conjunto formado por
todos los subconjuntos de $A$.
\item Cardinal del conjunto $A$ (Denotado $\# A$): número de elementos del propio
 conjunto.

\end {itemize}
El enunciado original del teorema es el siguiente : 


\begin {teorema}
El cardinal del conjunto potencia de cualquier conjunto A es
 estrictamente mayor que el cardinal de A, o lo que es lo mismo,
$\# \mathcal{P}(A) > \# A.$


\end {teorema}
Pero el enunciado del teorema lo podemos reformular como: 
\begin{teorema}
Dado un conjunto A, $\nexists  f : A \longrightarrow \mathcal{P}(A)$ que
sea sobreyectiva.

\end{teorema}

El teorema lo hemos podido reescribir de la anterior forma, ya que si
 suponemos que $\exists f$ tal que $f: A \longrightarrow \mathcal{P}(A)$
es sobreyectiva, entonces tenemos que $f(A) = \mathcal{P}(A)$ y por lo
 tanto, $\# f(A) \geq \# \mathcal{P}(A)$, de lo que se deduce esta
 reformulación. Reciprocamente, es trivial ver que esta reformulación
 implica la primera.
 con el teorema. \\
El teorema de Cantor es trivial para conjuntos finitos, ya que el
 conjunto potencia, de conjuntos finitos de n elementos tiene
 $2^n$ elementos.

Por ello,  vamos a realizar la prueba para conjuntos infinitos. 


\begin{demostracion}
 
Vamos a realizar la prueba por reducción al absurdo.\\
Supongamos que $\exists f : A \longrightarrow \mathcal{P}(A)$ sobreyectiva, es
 decir, $\forall C \in \rho(A) ,  \exists x \in A$ tal que $C = f(x)$.
En particular, tomemos el conjunto $$B = \{ x \in A : x \notin f(x) \}$$
 y  supongamos que $\exists a \in A : B = f(a)$, ya que $B$ es un
 subconjunto de A, luego podemos distinguir dos casos $:$ \\
$1.$ Si $a \in B$, entonces por definición del conjunto $B$ tenemos que
$a \notin B$, luego llegamos a una contradicción. \\
$2.$ Si $a \notin B$, entonces por definición de B tenemos que $a \in 
B$, luego hemos llegado a otra contradicción. 

En las dos hipótesis hemos llegado a una contradicción,
por lo que no existe $a$ y $f$ no es sobreyectiva.
\end{demostracion}


Para la especificación del teorema en Isabelle, primero debemos notar
 que $$f :: \, 'a \Rightarrow \,'a \: set$$
 significa que es una función 
de tipos, donde $'a$ significa un tipo y para poder denotar
el conjunto potencia tenemos que poner $'a \ set$ que significa que es
 de un tipo formado por conjuntos del tipo $'a$.




El enunciado del teorema es el siguiente :%
\end{isamarkuptext}\isamarkuptrue%
\isacommand{theorem}\isamarkupfalse%
\ Cantor{\isacharcolon}\ {\isachardoublequoteopen}{\isasymnexists}f\ {\isacharcolon}{\isacharcolon}\ {\isacharprime}a\ {\isasymRightarrow}\ {\isacharprime}a\ set{\isachardot}\ {\isasymforall}A{\isachardot}\ {\isasymexists}x{\isachardot}\ A\ {\isacharequal}\ f\ x{\isachardoublequoteclose}\isanewline
\isanewline
%
\isadelimproof
\isanewline
\ \ %
\endisadelimproof
%
\isatagproof
\isacommand{oops}\isamarkupfalse%
%
\endisatagproof
{\isafoldproof}%
%
\isadelimproof
%
\endisadelimproof
%
\begin{isamarkuptext}%
La demostración la haremos por la regla la introducción a la
negación, la cual es una simplificación de la regla de 
reducción al absurdo, cuyo esquema mostramos a continuación:   
 \begin{itemize}
  \item[] \isa{{\isacharparenleft}P\ {\isasymLongrightarrow}\ False{\isacharparenright}\ {\isasymLongrightarrow}\ {\isasymnot}\ P} \hfill (\isa{notI})
  \end{itemize}


Esta es la demostración detallada del teorema:%
\end{isamarkuptext}\isamarkuptrue%
\isacommand{theorem}\isamarkupfalse%
\ CantorDetallada{\isacharcolon}\ {\isachardoublequoteopen}{\isasymnexists}f\ {\isacharcolon}{\isacharcolon}\ {\isacharprime}a\ {\isasymRightarrow}\ {\isacharprime}a\ set{\isachardot}\ {\isasymforall}B{\isachardot}\ {\isasymexists}x{\isachardot}\ B\ {\isacharequal}\ f\ x{\isachardoublequoteclose}\isanewline
%
\isadelimproof
%
\endisadelimproof
%
\isatagproof
\isacommand{proof}\isamarkupfalse%
\ {\isacharparenleft}rule\ notI{\isacharparenright}\isanewline
\ \ \isacommand{assume}\isamarkupfalse%
\ {\isachardoublequoteopen}{\isasymexists}f\ {\isacharcolon}{\isacharcolon}\ {\isacharprime}a\ {\isasymRightarrow}\ {\isacharprime}a\ set{\isachardot}\ {\isasymforall}A{\isachardot}\ {\isasymexists}x{\isachardot}\ A\ {\isacharequal}\ f\ x{\isachardoublequoteclose}\isanewline
\ \ \isacommand{then}\isamarkupfalse%
\ \isacommand{obtain}\isamarkupfalse%
\ f\ {\isacharcolon}{\isacharcolon}\ {\isachardoublequoteopen}{\isacharprime}a\ {\isasymRightarrow}\ {\isacharprime}a\ set{\isachardoublequoteclose}\ \isakeyword{where}\ {\isacharasterisk}{\isacharcolon}\ {\isachardoublequoteopen}{\isasymforall}A{\isachardot}\ {\isasymexists}x{\isachardot}\ A\ {\isacharequal}\ f\ x{\isachardoublequoteclose}\ \isacommand{by}\isamarkupfalse%
\ {\isacharparenleft}rule\isanewline
\ \ \ \ \ \ \ \ exE{\isacharparenright}\isanewline
\ \ \isacommand{let}\isamarkupfalse%
\ {\isacharquery}B\ {\isacharequal}\ {\isachardoublequoteopen}{\isacharbraceleft}x{\isachardot}\ x\ {\isasymnotin}\ f\ x{\isacharbraceright}{\isachardoublequoteclose}\isanewline
\ \ \isacommand{from}\isamarkupfalse%
\ {\isacharasterisk}\ \isacommand{obtain}\isamarkupfalse%
\ {\isachardoublequoteopen}\ {\isasymexists}x{\isachardot}\ {\isacharquery}B\ {\isacharequal}\ f\ x\ {\isachardoublequoteclose}\ \isacommand{by}\isamarkupfalse%
\ {\isacharparenleft}rule\ allE{\isacharparenright}\isanewline
\ \ \isacommand{then}\isamarkupfalse%
\ \ \isacommand{obtain}\isamarkupfalse%
\ a\ \isakeyword{where}\ {\isadigit{1}}{\isacharcolon}{\isachardoublequoteopen}{\isacharquery}B\ {\isacharequal}\ f\ a{\isachardoublequoteclose}\ \isacommand{by}\isamarkupfalse%
\ {\isacharparenleft}rule\ exE{\isacharparenright}\isanewline
\ \ \isacommand{show}\isamarkupfalse%
\ False\isanewline
\ \ \isacommand{proof}\isamarkupfalse%
\ {\isacharparenleft}cases{\isacharparenright}\isanewline
\ \ \ \ \isacommand{assume}\isamarkupfalse%
\ {\isachardoublequoteopen}a\ {\isasymin}\ {\isacharquery}B{\isachardoublequoteclose}\ \ \isanewline
\ \ \ \ \isacommand{then}\isamarkupfalse%
\ \isacommand{show}\isamarkupfalse%
\ False\ \ \isacommand{using}\isamarkupfalse%
\ {\isadigit{1}}\ \isacommand{by}\isamarkupfalse%
\ blast\isanewline
\ \ \isacommand{next}\isamarkupfalse%
\ \isanewline
\ \ \ \ \isacommand{assume}\isamarkupfalse%
\ {\isachardoublequoteopen}a\ {\isasymnotin}\ {\isacharquery}B{\isachardoublequoteclose}\isanewline
\ \ \ \ \isacommand{thus}\isamarkupfalse%
\ False\ \isacommand{using}\isamarkupfalse%
\ {\isadigit{1}}\ \isacommand{by}\isamarkupfalse%
\ blast\isanewline
\ \ \isacommand{qed}\isamarkupfalse%
\isanewline
\isacommand{qed}\isamarkupfalse%
%
\endisatagproof
{\isafoldproof}%
%
\isadelimproof
%
\endisadelimproof
%
\begin{isamarkuptext}%
Esta es la demostración aplicativa del teorema:%
\end{isamarkuptext}\isamarkuptrue%
\isacommand{theorem}\isamarkupfalse%
\ CantorAplicativa\ {\isacharcolon}\isanewline
\ {\isachardoublequoteopen}{\isasymnexists}f\ {\isacharcolon}{\isacharcolon}\ {\isacharprime}a\ {\isasymRightarrow}\ {\isacharprime}a\ set{\isachardot}\ {\isasymforall}A{\isachardot}\ {\isasymexists}x{\isachardot}\ A\ {\isacharequal}\ f\ x{\isachardoublequoteclose}\isanewline
%
\isadelimproof
\ \ %
\endisadelimproof
%
\isatagproof
\isacommand{apply}\isamarkupfalse%
\ {\isacharparenleft}rule\ notI{\isacharparenright}\isanewline
\ \ \isacommand{apply}\isamarkupfalse%
\ {\isacharparenleft}erule\ exE{\isacharparenright}\isanewline
\ \ \isacommand{apply}\isamarkupfalse%
\ {\isacharparenleft}erule{\isacharunderscore}tac\ x\ {\isacharequal}\ {\isachardoublequoteopen}{\isacharbraceleft}x{\isachardot}\ x\ {\isasymnotin}\ f\ x{\isacharbraceright}{\isachardoublequoteclose}\ \isakeyword{in}\ allE{\isacharparenright}\isanewline
\ \ \isacommand{apply}\isamarkupfalse%
\ {\isacharparenleft}erule\ exE{\isacharparenright}\isanewline
\ \ \isacommand{apply}\isamarkupfalse%
\ \ blast\ \isanewline
\ \ \isacommand{done}\isamarkupfalse%
%
\endisatagproof
{\isafoldproof}%
%
\isadelimproof
%
\endisadelimproof
%
\begin{isamarkuptext}%
Esta es la demostración automática del teorema:%
\end{isamarkuptext}\isamarkuptrue%
\isacommand{theorem}\isamarkupfalse%
\ CantorAutomatic{\isacharcolon}\ {\isachardoublequoteopen}{\isasymnexists}f\ {\isacharcolon}{\isacharcolon}\ {\isacharprime}a\ {\isasymRightarrow}\ {\isacharprime}a\ set{\isachardot}\ {\isasymforall}B{\isachardot}\ {\isasymexists}x{\isachardot}\ B\ {\isacharequal}\ f\ x{\isachardoublequoteclose}\isanewline
%
\isadelimproof
\ \ %
\endisadelimproof
%
\isatagproof
\isacommand{by}\isamarkupfalse%
\ best%
\endisatagproof
{\isafoldproof}%
%
\isadelimproof
%
\endisadelimproof
%
\begin{isamarkuptext}%
En la demostración de isabelle hemos utilizado el método de prueba
rule con las siguientes reglas, tanto en la aplicativa como en la
 detallada:
 \begin{itemize}
  \item[] \isa{\mbox{}\inferrule{\mbox{\mbox{}\inferrule{\mbox{P}}{\mbox{False}}}}{\mbox{{\isasymnot}\ P}}} \hfill (\isa{notI})
  \end{itemize}
 \begin{itemize}
  \item[] \isa{\mbox{}\inferrule{\mbox{{\isasymexists}x{\isachardot}\ P\ x}\\\ \mbox{{\isasymAnd}x{\isachardot}\ \mbox{}\inferrule{\mbox{P\ x}}{\mbox{Q}}}}{\mbox{Q}}} \hfill (\isa{exE})
  \end{itemize}
 \begin{itemize}
  \item[] \isa{\mbox{}\inferrule{\mbox{{\isasymforall}x{\isachardot}\ P\ x}\\\ \mbox{\mbox{}\inferrule{\mbox{P\ x}}{\mbox{R}}}}{\mbox{R}}} \hfill (\isa{allE})
  \end{itemize}
También hacemos uso de blast, que es un conjunto de reglas lógicas y 
 la demostración automática la hacemos por medio de "best".%
\end{isamarkuptext}\isamarkuptrue%
%
\isadelimtheory
%
\endisadelimtheory
%
\isatagtheory
%
\endisatagtheory
{\isafoldtheory}%
%
\isadelimtheory
%
\endisadelimtheory
%
\end{isabellebody}%
\endinput
%:%file=~/ownCloud/alonso/curso-TFG/Carlos/TFG_de_Carlos/TeoremaCantor.thy%:%
%:%6=1%:%
%:%20=9%:%
%:%24=11%:%
%:%28=13%:%
%:%32=15%:%
%:%33=16%:%
%:%34=17%:%
%:%35=18%:%
%:%36=19%:%
%:%37=20%:%
%:%38=21%:%
%:%39=22%:%
%:%40=23%:%
%:%41=24%:%
%:%42=25%:%
%:%43=26%:%
%:%44=27%:%
%:%45=28%:%
%:%46=29%:%
%:%47=30%:%
%:%48=31%:%
%:%49=32%:%
%:%50=33%:%
%:%51=34%:%
%:%52=35%:%
%:%53=36%:%
%:%54=37%:%
%:%55=38%:%
%:%56=39%:%
%:%57=40%:%
%:%58=41%:%
%:%59=42%:%
%:%60=43%:%
%:%61=44%:%
%:%62=45%:%
%:%63=46%:%
%:%64=47%:%
%:%65=48%:%
%:%66=49%:%
%:%67=50%:%
%:%68=51%:%
%:%69=52%:%
%:%70=53%:%
%:%71=54%:%
%:%72=55%:%
%:%73=56%:%
%:%74=57%:%
%:%75=58%:%
%:%76=59%:%
%:%77=60%:%
%:%78=61%:%
%:%79=62%:%
%:%80=63%:%
%:%81=64%:%
%:%82=65%:%
%:%83=66%:%
%:%84=67%:%
%:%85=68%:%
%:%86=69%:%
%:%87=70%:%
%:%88=71%:%
%:%89=72%:%
%:%90=73%:%
%:%91=74%:%
%:%92=75%:%
%:%93=76%:%
%:%94=77%:%
%:%95=78%:%
%:%96=79%:%
%:%97=80%:%
%:%98=81%:%
%:%99=82%:%
%:%100=83%:%
%:%101=84%:%
%:%102=85%:%
%:%103=86%:%
%:%104=87%:%
%:%105=88%:%
%:%106=89%:%
%:%107=90%:%
%:%108=91%:%
%:%109=92%:%
%:%110=93%:%
%:%111=94%:%
%:%112=95%:%
%:%114=97%:%
%:%115=97%:%
%:%116=98%:%
%:%119=99%:%
%:%120=100%:%
%:%124=100%:%
%:%134=102%:%
%:%135=103%:%
%:%136=104%:%
%:%137=105%:%
%:%138=106%:%
%:%139=107%:%
%:%140=108%:%
%:%141=109%:%
%:%142=110%:%
%:%144=112%:%
%:%145=112%:%
%:%152=113%:%
%:%153=113%:%
%:%154=114%:%
%:%155=114%:%
%:%156=115%:%
%:%157=115%:%
%:%158=115%:%
%:%159=115%:%
%:%160=116%:%
%:%161=117%:%
%:%162=117%:%
%:%163=118%:%
%:%164=118%:%
%:%165=118%:%
%:%166=118%:%
%:%167=119%:%
%:%168=119%:%
%:%169=119%:%
%:%170=119%:%
%:%171=120%:%
%:%172=120%:%
%:%173=121%:%
%:%174=121%:%
%:%175=122%:%
%:%176=122%:%
%:%177=123%:%
%:%178=123%:%
%:%179=123%:%
%:%180=123%:%
%:%181=123%:%
%:%182=124%:%
%:%183=124%:%
%:%184=125%:%
%:%185=125%:%
%:%186=126%:%
%:%187=126%:%
%:%188=126%:%
%:%189=126%:%
%:%190=127%:%
%:%191=127%:%
%:%192=128%:%
%:%202=130%:%
%:%204=133%:%
%:%205=133%:%
%:%206=134%:%
%:%209=135%:%
%:%213=135%:%
%:%214=135%:%
%:%215=136%:%
%:%216=136%:%
%:%217=137%:%
%:%218=137%:%
%:%219=138%:%
%:%220=138%:%
%:%221=139%:%
%:%222=139%:%
%:%223=140%:%
%:%233=142%:%
%:%235=143%:%
%:%236=143%:%
%:%239=144%:%
%:%243=144%:%
%:%244=144%:%
%:%253=146%:%
%:%254=147%:%
%:%255=148%:%
%:%256=149%:%
%:%257=150%:%
%:%258=151%:%
%:%259=152%:%
%:%260=153%:%
%:%261=154%:%
%:%262=155%:%
%:%263=156%:%
%:%264=157%:%
%:%265=158%:%
%:%266=159%:%