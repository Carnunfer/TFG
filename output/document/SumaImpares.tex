%
\begin{isabellebody}%
\setisabellecontext{SumaImpares}%
%
\isadelimtheory
%
\endisadelimtheory
%
\isatagtheory
%
\endisatagtheory
{\isafoldtheory}%
%
\isadelimtheory
%
\endisadelimtheory
%
\isadelimdocument
%
\endisadelimdocument
%
\isatagdocument
%
\isamarkupsection{Suma de los primeros números impares%
}
\isamarkuptrue%
%
\endisatagdocument
{\isafolddocument}%
%
\isadelimdocument
%
\endisadelimdocument
%
\begin{isamarkuptext}%
El primer teorema es una propiedad de los números naturales.

  \begin{teorema}
    La suma de los $n$ primeros números impares es $n^2$.
  \end{teorema}

  \begin{demostracion}
    La demostración la haremos en inducción sobre $n$.
    
    (Base de la inducción) El caso $n = 0$ es trivial.
    
    (Paso de la inducción) Supongamos que la propiedad se verifica para
    $n$ y veamos que también se verifica para $n+1$. 
 
    Tenemos que demostrar que $\sum_{j=1}^{n+1} k_j = (n+1)^2$ donde
    $k_j$ el j--ésimo impar; es decir, $k_j = 2j - 1$.

    $$\begin{array}{l}
      \sum_{j = 1}^{n+1} k_j \\
      = k_{n+1} + \sum^{n}_{j=1} k_j \\ 
      = k_{n+1} + n^2 \\
      = 2(n+1) - 1 + n^2 \\
      = n^2 + 2n + 1 \\ 
      = (n+1)^2
      \end{array}$$ 
  \end{demostracion}

  Para especificar el teorema en Isabelle, se comienza definiendo 
  la función \isa{suma{\isacharunderscore}impares} tal que \isa{suma{\isacharunderscore}impares\ n} es
  la suma de los $n$ primeros números impares%
\end{isamarkuptext}\isamarkuptrue%
\isacommand{fun}\isamarkupfalse%
\ suma{\isacharunderscore}impares\ {\isacharcolon}{\isacharcolon}\ {\isachardoublequoteopen}nat\ {\isasymRightarrow}\ nat{\isachardoublequoteclose}\ \isakeyword{where}\isanewline
\ \ {\isachardoublequoteopen}suma{\isacharunderscore}impares\ {\isadigit{0}}\ {\isacharequal}\ {\isadigit{0}}{\isachardoublequoteclose}\ \isanewline
{\isacharbar}\ {\isachardoublequoteopen}suma{\isacharunderscore}impares\ {\isacharparenleft}Suc\ n{\isacharparenright}\ {\isacharequal}\ {\isacharparenleft}{\isadigit{2}}{\isacharasterisk}{\isacharparenleft}Suc\ n{\isacharparenright}\ {\isacharminus}\ {\isadigit{1}}{\isacharparenright}\ {\isacharplus}\ suma{\isacharunderscore}impares\ n{\isachardoublequoteclose}%
\begin{isamarkuptext}%
El enunciado del teorema es el siguiente:%
\end{isamarkuptext}\isamarkuptrue%
\isacommand{lemma}\isamarkupfalse%
\ {\isachardoublequoteopen}suma{\isacharunderscore}impares\ n\ {\isacharequal}\ n\ {\isacharasterisk}\ n{\isachardoublequoteclose}\isanewline
%
\isadelimproof
%
\endisadelimproof
%
\isatagproof
\isacommand{oops}\isamarkupfalse%
%
\endisatagproof
{\isafoldproof}%
%
\isadelimproof
%
\endisadelimproof
%
\begin{isamarkuptext}%
En la demostración se usará la táctica \isa{induct} que hace
  uso del esquema de inducción sobre los naturales:
  \begin{itemize}
  \item[] \isa{\mbox{}\inferrule{\mbox{P\ {\isadigit{0}}}\\\ \mbox{{\isasymAnd}nat{\isachardot}\ \mbox{}\inferrule{\mbox{P\ nat}}{\mbox{P\ {\isacharparenleft}Suc\ nat{\isacharparenright}}}}}{\mbox{P\ nat}}} 
          \hfill (\isa{nat{\isachardot}induct})
  \end{itemize}

  Vamos a presentar distintas demostraciones del teorema. La 
  primera es la demostración aplicativa detallada.%
\end{isamarkuptext}\isamarkuptrue%
\isacommand{lemma}\isamarkupfalse%
\ {\isachardoublequoteopen}suma{\isacharunderscore}impares\ n\ {\isacharequal}\ n\ {\isacharasterisk}\ n{\isachardoublequoteclose}\isanewline
%
\isadelimproof
\ \ %
\endisadelimproof
%
\isatagproof
\isacommand{apply}\isamarkupfalse%
\ {\isacharparenleft}induct\ n{\isacharparenright}\ \isanewline
\ \ \ \isacommand{apply}\isamarkupfalse%
\ {\isacharparenleft}simp\ only{\isacharcolon}\ suma{\isacharunderscore}impares{\isachardot}simps{\isacharparenleft}{\isadigit{1}}{\isacharparenright}{\isacharparenright}\isanewline
\ \ \isacommand{apply}\isamarkupfalse%
\ {\isacharparenleft}simp\ only{\isacharcolon}\ suma{\isacharunderscore}impares{\isachardot}simps{\isacharparenleft}{\isadigit{2}}{\isacharparenright}{\isacharparenright}\isanewline
\ \ \isacommand{apply}\isamarkupfalse%
\ {\isacharparenleft}simp\ only{\isacharcolon}\ mult{\isacharunderscore}Suc\ mult{\isacharunderscore}Suc{\isacharunderscore}right{\isacharparenright}\isanewline
\ \ \isacommand{done}\isamarkupfalse%
%
\endisatagproof
{\isafoldproof}%
%
\isadelimproof
%
\endisadelimproof
%
\begin{isamarkuptext}%
En la demostración anterior hemos usado dentro del método 
\isa{simp} únicamente la definición de \isa{suma{\isacharunderscore}impares},
 para ello lo  hemos indicado con \isa{simp\ only}. A parte hemos
 usado lo siguiente :
  \begin{itemize}
  \item[] \isa{Suc\ m\ {\isacharasterisk}\ n\ {\isacharequal}\ n\ {\isacharplus}\ m\ {\isacharasterisk}\ n} 
          \hfill (\isa{mult{\isacharunderscore}Suc})
  \end{itemize}
  \begin{itemize}
  \item[] \isa{m\ {\isacharasterisk}\ Suc\ n\ {\isacharequal}\ m\ {\isacharplus}\ m\ {\isacharasterisk}\ n} 
          \hfill (\isa{mult{\isacharunderscore}Suc{\isacharunderscore}right})
  \end{itemize}%
\end{isamarkuptext}\isamarkuptrue%
%
\begin{isamarkuptext}%
Se puede eliminar los detalles de la demostración anterior.%
\end{isamarkuptext}\isamarkuptrue%
\isacommand{lemma}\isamarkupfalse%
\ {\isachardoublequoteopen}suma{\isacharunderscore}impares\ n\ {\isacharequal}\ n\ {\isacharasterisk}\ n{\isachardoublequoteclose}\isanewline
%
\isadelimproof
\ \ %
\endisadelimproof
%
\isatagproof
\isacommand{apply}\isamarkupfalse%
\ {\isacharparenleft}induct\ n{\isacharparenright}\ \isanewline
\ \ \ \isacommand{apply}\isamarkupfalse%
\ simp{\isacharunderscore}all\isanewline
\ \ \isacommand{done}\isamarkupfalse%
%
\endisatagproof
{\isafoldproof}%
%
\isadelimproof
%
\endisadelimproof
%
\begin{isamarkuptext}%
La correspondiente demostración automática es%
\end{isamarkuptext}\isamarkuptrue%
\isacommand{lemma}\isamarkupfalse%
\ {\isachardoublequoteopen}suma{\isacharunderscore}impares\ n\ {\isacharequal}\ n\ {\isacharasterisk}\ n{\isachardoublequoteclose}\isanewline
%
\isadelimproof
\ \ %
\endisadelimproof
%
\isatagproof
\isacommand{by}\isamarkupfalse%
\ {\isacharparenleft}induct\ n{\isacharparenright}\ simp{\isacharunderscore}all%
\endisatagproof
{\isafoldproof}%
%
\isadelimproof
%
\endisadelimproof
%
\begin{isamarkuptext}%
La demostración estructurada y detallada del lema anterior es%
\end{isamarkuptext}\isamarkuptrue%
\isacommand{lemma}\isamarkupfalse%
\ {\isachardoublequoteopen}suma{\isacharunderscore}impares\ n\ {\isacharequal}\ n\ {\isacharasterisk}\ n{\isachardoublequoteclose}\isanewline
%
\isadelimproof
%
\endisadelimproof
%
\isatagproof
\isacommand{proof}\isamarkupfalse%
\ {\isacharparenleft}induct\ n{\isacharparenright}\isanewline
\ \ \isacommand{have}\isamarkupfalse%
\ {\isachardoublequoteopen}suma{\isacharunderscore}impares\ {\isadigit{0}}\ {\isacharequal}\ {\isadigit{0}}{\isachardoublequoteclose}\ \isanewline
\ \ \ \ \isacommand{by}\isamarkupfalse%
\ {\isacharparenleft}simp\ only{\isacharcolon}\ suma{\isacharunderscore}impares{\isachardot}simps{\isacharparenleft}{\isadigit{1}}{\isacharparenright}{\isacharparenright}\isanewline
\ \ \isacommand{also}\isamarkupfalse%
\ \isacommand{have}\isamarkupfalse%
\ {\isachardoublequoteopen}{\isasymdots}\ {\isacharequal}\ {\isadigit{0}}\ {\isacharasterisk}\ {\isadigit{0}}{\isachardoublequoteclose}\isanewline
\ \ \ \ \isacommand{by}\isamarkupfalse%
\ {\isacharparenleft}simp\ only{\isacharcolon}\ mult{\isacharunderscore}{\isadigit{0}}{\isacharparenright}\isanewline
\ \ \isacommand{finally}\isamarkupfalse%
\ \isacommand{show}\isamarkupfalse%
\ {\isachardoublequoteopen}suma{\isacharunderscore}impares\ {\isadigit{0}}\ {\isacharequal}\ {\isadigit{0}}\ {\isacharasterisk}\ {\isadigit{0}}{\isachardoublequoteclose}\isanewline
\ \ \ \ \isacommand{by}\isamarkupfalse%
\ simp\isanewline
\isacommand{next}\isamarkupfalse%
\isanewline
\ \ \isacommand{fix}\isamarkupfalse%
\ n\ \isanewline
\ \ \isacommand{assume}\isamarkupfalse%
\ HI{\isacharcolon}\ {\isachardoublequoteopen}suma{\isacharunderscore}impares\ n\ {\isacharequal}\ n\ {\isacharasterisk}\ n{\isachardoublequoteclose}\isanewline
\ \ \isacommand{have}\isamarkupfalse%
\ {\isachardoublequoteopen}suma{\isacharunderscore}impares\ {\isacharparenleft}Suc\ n{\isacharparenright}\ {\isacharequal}\ {\isacharparenleft}{\isadigit{2}}\ {\isacharasterisk}\ {\isacharparenleft}Suc\ n{\isacharparenright}\ {\isacharminus}\ {\isadigit{1}}{\isacharparenright}\ {\isacharplus}\ suma{\isacharunderscore}impares\ n{\isachardoublequoteclose}\ \isanewline
\ \ \ \ \isacommand{by}\isamarkupfalse%
\ {\isacharparenleft}simp\ only{\isacharcolon}\ suma{\isacharunderscore}impares{\isachardot}simps{\isacharparenleft}{\isadigit{2}}{\isacharparenright}{\isacharparenright}\isanewline
\ \ \isacommand{also}\isamarkupfalse%
\ \isacommand{have}\isamarkupfalse%
\ {\isachardoublequoteopen}{\isasymdots}\ {\isacharequal}\ {\isacharparenleft}{\isadigit{2}}\ {\isacharasterisk}\ {\isacharparenleft}Suc\ n{\isacharparenright}\ {\isacharminus}\ {\isadigit{1}}{\isacharparenright}\ {\isacharplus}\ n\ {\isacharasterisk}\ n{\isachardoublequoteclose}\ \isanewline
\ \ \ \ \isacommand{by}\isamarkupfalse%
\ {\isacharparenleft}simp\ only{\isacharcolon}\ HI{\isacharparenright}\isanewline
\ \ \isacommand{also}\isamarkupfalse%
\ \isacommand{have}\isamarkupfalse%
\ {\isachardoublequoteopen}{\isasymdots}\ {\isacharequal}\ n\ {\isacharasterisk}\ n\ {\isacharplus}\ {\isadigit{2}}\ {\isacharasterisk}\ n\ {\isacharplus}\ {\isadigit{1}}{\isachardoublequoteclose}\ \isanewline
\ \ \ \ \isacommand{by}\isamarkupfalse%
\ {\isacharparenleft}simp\ only{\isacharcolon}\ mult{\isacharunderscore}Suc{\isacharunderscore}right{\isacharparenright}\isanewline
\ \ \isacommand{also}\isamarkupfalse%
\ \isacommand{have}\isamarkupfalse%
\ {\isachardoublequoteopen}{\isasymdots}\ {\isacharequal}\ {\isacharparenleft}Suc\ n{\isacharparenright}\ {\isacharasterisk}\ {\isacharparenleft}Suc\ n{\isacharparenright}{\isachardoublequoteclose}\isanewline
\ \ \ \ \isacommand{by}\isamarkupfalse%
\ {\isacharparenleft}simp\ only{\isacharcolon}\ mult{\isacharunderscore}Suc\ mult{\isacharunderscore}Suc{\isacharunderscore}right{\isacharparenright}\isanewline
\ \ \isacommand{finally}\isamarkupfalse%
\ \isacommand{show}\isamarkupfalse%
\ {\isachardoublequoteopen}suma{\isacharunderscore}impares\ {\isacharparenleft}Suc\ n{\isacharparenright}\ {\isacharequal}\ {\isacharparenleft}Suc\ n{\isacharparenright}\ {\isacharasterisk}\ {\isacharparenleft}Suc\ n{\isacharparenright}{\isachardoublequoteclose}\ \isanewline
\ \ \ \ \isacommand{by}\isamarkupfalse%
\ simp\isanewline
\isacommand{qed}\isamarkupfalse%
%
\endisatagproof
{\isafoldproof}%
%
\isadelimproof
%
\endisadelimproof
%
\begin{isamarkuptext}%
En la demostración anterior se pueden ocultar detalles.%
\end{isamarkuptext}\isamarkuptrue%
\isacommand{lemma}\isamarkupfalse%
\ {\isachardoublequoteopen}suma{\isacharunderscore}impares\ n\ {\isacharequal}\ n\ {\isacharasterisk}\ n{\isachardoublequoteclose}\isanewline
%
\isadelimproof
%
\endisadelimproof
%
\isatagproof
\isacommand{proof}\isamarkupfalse%
\ {\isacharparenleft}induct\ n{\isacharparenright}\isanewline
\ \ \isacommand{show}\isamarkupfalse%
\ {\isachardoublequoteopen}suma{\isacharunderscore}impares\ {\isadigit{0}}\ {\isacharequal}\ {\isadigit{0}}\ {\isacharasterisk}\ {\isadigit{0}}{\isachardoublequoteclose}\ \isacommand{by}\isamarkupfalse%
\ simp\isanewline
\isacommand{next}\isamarkupfalse%
\isanewline
\ \ \isacommand{fix}\isamarkupfalse%
\ n\ \isanewline
\ \ \isacommand{assume}\isamarkupfalse%
\ HI{\isacharcolon}\ {\isachardoublequoteopen}suma{\isacharunderscore}impares\ n\ {\isacharequal}\ n\ {\isacharasterisk}\ n{\isachardoublequoteclose}\isanewline
\ \ \isacommand{have}\isamarkupfalse%
\ {\isachardoublequoteopen}suma{\isacharunderscore}impares\ {\isacharparenleft}Suc\ n{\isacharparenright}\ {\isacharequal}\ {\isacharparenleft}{\isadigit{2}}\ {\isacharasterisk}\ {\isacharparenleft}Suc\ n{\isacharparenright}\ {\isacharminus}\ {\isadigit{1}}{\isacharparenright}\ {\isacharplus}\ suma{\isacharunderscore}impares\ n{\isachardoublequoteclose}\ \isanewline
\ \ \ \ \isacommand{by}\isamarkupfalse%
\ simp\isanewline
\ \ \isacommand{also}\isamarkupfalse%
\ \isacommand{have}\isamarkupfalse%
\ {\isachardoublequoteopen}{\isasymdots}\ {\isacharequal}\ {\isacharparenleft}{\isadigit{2}}\ {\isacharasterisk}\ {\isacharparenleft}Suc\ n{\isacharparenright}\ {\isacharminus}\ {\isadigit{1}}{\isacharparenright}\ {\isacharplus}\ n\ {\isacharasterisk}\ n{\isachardoublequoteclose}\ \isanewline
\ \ \ \ \isacommand{using}\isamarkupfalse%
\ HI\ \isacommand{by}\isamarkupfalse%
\ simp\isanewline
\ \ \isacommand{also}\isamarkupfalse%
\ \isacommand{have}\isamarkupfalse%
\ {\isachardoublequoteopen}{\isasymdots}\ {\isacharequal}\ {\isacharparenleft}Suc\ n{\isacharparenright}\ {\isacharasterisk}\ {\isacharparenleft}Suc\ n{\isacharparenright}{\isachardoublequoteclose}\ \isanewline
\ \ \ \ \isacommand{by}\isamarkupfalse%
\ simp\isanewline
\ \ \isacommand{finally}\isamarkupfalse%
\ \isacommand{show}\isamarkupfalse%
\ {\isachardoublequoteopen}suma{\isacharunderscore}impares\ {\isacharparenleft}Suc\ n{\isacharparenright}\ {\isacharequal}\ {\isacharparenleft}Suc\ n{\isacharparenright}\ {\isacharasterisk}\ {\isacharparenleft}Suc\ n{\isacharparenright}{\isachardoublequoteclose}\ \isanewline
\ \ \ \ \isacommand{by}\isamarkupfalse%
\ simp\isanewline
\isacommand{qed}\isamarkupfalse%
%
\endisatagproof
{\isafoldproof}%
%
\isadelimproof
%
\endisadelimproof
%
\begin{isamarkuptext}%
La demostración anterior se puede simplificar usando patrones.%
\end{isamarkuptext}\isamarkuptrue%
\isacommand{lemma}\isamarkupfalse%
\ {\isachardoublequoteopen}suma{\isacharunderscore}impares\ n\ {\isacharequal}\ n\ {\isacharasterisk}\ n{\isachardoublequoteclose}\ {\isacharparenleft}\isakeyword{is}\ {\isachardoublequoteopen}{\isacharquery}P\ n\ {\isacharequal}\ {\isacharquery}Q\ n{\isachardoublequoteclose}{\isacharparenright}\isanewline
%
\isadelimproof
%
\endisadelimproof
%
\isatagproof
\isacommand{proof}\isamarkupfalse%
\ {\isacharparenleft}induct\ n{\isacharparenright}\isanewline
\ \ \isacommand{show}\isamarkupfalse%
\ {\isachardoublequoteopen}{\isacharquery}P\ {\isadigit{0}}\ {\isacharequal}\ {\isacharquery}Q\ {\isadigit{0}}{\isachardoublequoteclose}\ \isacommand{by}\isamarkupfalse%
\ simp\isanewline
\isacommand{next}\isamarkupfalse%
\isanewline
\ \ \isacommand{fix}\isamarkupfalse%
\ n\ \isanewline
\ \ \isacommand{assume}\isamarkupfalse%
\ HI{\isacharcolon}\ {\isachardoublequoteopen}{\isacharquery}P\ n\ {\isacharequal}\ {\isacharquery}Q\ n{\isachardoublequoteclose}\isanewline
\ \ \isacommand{have}\isamarkupfalse%
\ {\isachardoublequoteopen}{\isacharquery}P\ {\isacharparenleft}Suc\ n{\isacharparenright}\ {\isacharequal}\ {\isacharparenleft}{\isadigit{2}}\ {\isacharasterisk}\ {\isacharparenleft}Suc\ n{\isacharparenright}\ {\isacharminus}\ {\isadigit{1}}{\isacharparenright}\ {\isacharplus}\ suma{\isacharunderscore}impares\ n{\isachardoublequoteclose}\ \isanewline
\ \ \ \ \isacommand{by}\isamarkupfalse%
\ simp\isanewline
\ \ \isacommand{also}\isamarkupfalse%
\ \isacommand{have}\isamarkupfalse%
\ {\isachardoublequoteopen}{\isasymdots}\ {\isacharequal}\ {\isacharparenleft}{\isadigit{2}}\ {\isacharasterisk}\ {\isacharparenleft}Suc\ n{\isacharparenright}\ {\isacharminus}\ {\isadigit{1}}{\isacharparenright}\ {\isacharplus}\ n\ {\isacharasterisk}\ n{\isachardoublequoteclose}\ \isacommand{using}\isamarkupfalse%
\ HI\ \isacommand{by}\isamarkupfalse%
\ simp\isanewline
\ \ \isacommand{also}\isamarkupfalse%
\ \isacommand{have}\isamarkupfalse%
\ {\isachardoublequoteopen}{\isasymdots}\ {\isacharequal}\ {\isacharquery}Q\ {\isacharparenleft}Suc\ n{\isacharparenright}{\isachardoublequoteclose}\ \isacommand{by}\isamarkupfalse%
\ simp\isanewline
\ \ \isacommand{finally}\isamarkupfalse%
\ \isacommand{show}\isamarkupfalse%
\ {\isachardoublequoteopen}{\isacharquery}P\ {\isacharparenleft}Suc\ n{\isacharparenright}\ {\isacharequal}\ {\isacharquery}Q\ {\isacharparenleft}Suc\ n{\isacharparenright}{\isachardoublequoteclose}\ \isacommand{by}\isamarkupfalse%
\ simp\isanewline
\isacommand{qed}\isamarkupfalse%
%
\endisatagproof
{\isafoldproof}%
%
\isadelimproof
%
\endisadelimproof
%
\begin{isamarkuptext}%
La demostración usando otro patrón es%
\end{isamarkuptext}\isamarkuptrue%
\isacommand{lemma}\isamarkupfalse%
\ {\isachardoublequoteopen}suma{\isacharunderscore}impares\ n\ {\isacharequal}\ n\ {\isacharasterisk}\ n{\isachardoublequoteclose}\ {\isacharparenleft}\isakeyword{is}\ {\isachardoublequoteopen}{\isacharquery}P\ n{\isachardoublequoteclose}{\isacharparenright}\isanewline
%
\isadelimproof
%
\endisadelimproof
%
\isatagproof
\isacommand{proof}\isamarkupfalse%
\ {\isacharparenleft}induct\ n{\isacharparenright}\isanewline
\ \ \isacommand{show}\isamarkupfalse%
\ {\isachardoublequoteopen}{\isacharquery}P\ {\isadigit{0}}{\isachardoublequoteclose}\ \isacommand{by}\isamarkupfalse%
\ simp\isanewline
\isacommand{next}\isamarkupfalse%
\isanewline
\ \ \isacommand{fix}\isamarkupfalse%
\ n\ \isanewline
\ \ \isacommand{assume}\isamarkupfalse%
\ {\isachardoublequoteopen}{\isacharquery}P\ n{\isachardoublequoteclose}\isanewline
\ \ \isacommand{then}\isamarkupfalse%
\ \isacommand{show}\isamarkupfalse%
\ {\isachardoublequoteopen}{\isacharquery}P\ {\isacharparenleft}Suc\ n{\isacharparenright}{\isachardoublequoteclose}\ \isacommand{by}\isamarkupfalse%
\ simp\isanewline
\isacommand{qed}\isamarkupfalse%
\isanewline
%
\endisatagproof
{\isafoldproof}%
%
\isadelimproof
%
\endisadelimproof
%
\isadelimtheory
%
\endisadelimtheory
%
\isatagtheory
%
\endisatagtheory
{\isafoldtheory}%
%
\isadelimtheory
%
\endisadelimtheory
%
\end{isabellebody}%
\endinput
%:%file=~/Escritorio/TFG/SumaImpares.thy%:%
%:%24=8%:%
%:%36=10%:%
%:%37=11%:%
%:%38=12%:%
%:%39=13%:%
%:%40=14%:%
%:%41=15%:%
%:%42=16%:%
%:%43=17%:%
%:%44=18%:%
%:%45=19%:%
%:%46=20%:%
%:%47=21%:%
%:%48=22%:%
%:%49=23%:%
%:%50=24%:%
%:%51=25%:%
%:%52=26%:%
%:%53=27%:%
%:%54=28%:%
%:%55=29%:%
%:%56=30%:%
%:%57=31%:%
%:%58=32%:%
%:%59=33%:%
%:%60=34%:%
%:%61=35%:%
%:%62=36%:%
%:%63=37%:%
%:%64=38%:%
%:%65=39%:%
%:%67=41%:%
%:%68=41%:%
%:%69=42%:%
%:%70=43%:%
%:%72=45%:%
%:%74=47%:%
%:%75=47%:%
%:%82=48%:%
%:%92=50%:%
%:%93=51%:%
%:%94=52%:%
%:%95=53%:%
%:%96=54%:%
%:%97=55%:%
%:%98=56%:%
%:%99=57%:%
%:%100=58%:%
%:%102=60%:%
%:%103=60%:%
%:%106=61%:%
%:%110=61%:%
%:%111=61%:%
%:%112=62%:%
%:%113=62%:%
%:%114=63%:%
%:%115=63%:%
%:%116=64%:%
%:%117=64%:%
%:%118=65%:%
%:%128=67%:%
%:%129=68%:%
%:%130=69%:%
%:%131=70%:%
%:%132=71%:%
%:%133=72%:%
%:%134=73%:%
%:%135=74%:%
%:%136=75%:%
%:%137=76%:%
%:%138=77%:%
%:%139=78%:%
%:%143=81%:%
%:%145=83%:%
%:%146=83%:%
%:%149=84%:%
%:%153=84%:%
%:%154=84%:%
%:%155=85%:%
%:%156=85%:%
%:%157=86%:%
%:%167=88%:%
%:%169=90%:%
%:%170=90%:%
%:%173=91%:%
%:%177=91%:%
%:%178=91%:%
%:%187=93%:%
%:%189=95%:%
%:%190=95%:%
%:%197=96%:%
%:%198=96%:%
%:%199=97%:%
%:%200=97%:%
%:%201=98%:%
%:%202=98%:%
%:%203=99%:%
%:%204=99%:%
%:%205=99%:%
%:%206=100%:%
%:%207=100%:%
%:%208=101%:%
%:%209=101%:%
%:%210=101%:%
%:%211=102%:%
%:%212=102%:%
%:%213=103%:%
%:%214=103%:%
%:%215=104%:%
%:%216=104%:%
%:%217=105%:%
%:%218=105%:%
%:%219=106%:%
%:%220=106%:%
%:%221=107%:%
%:%222=107%:%
%:%223=108%:%
%:%224=108%:%
%:%225=108%:%
%:%226=109%:%
%:%227=109%:%
%:%228=110%:%
%:%229=110%:%
%:%230=110%:%
%:%231=111%:%
%:%232=111%:%
%:%233=112%:%
%:%234=112%:%
%:%235=112%:%
%:%236=113%:%
%:%237=113%:%
%:%238=114%:%
%:%239=114%:%
%:%240=114%:%
%:%241=115%:%
%:%242=115%:%
%:%243=116%:%
%:%253=118%:%
%:%255=120%:%
%:%256=120%:%
%:%263=121%:%
%:%264=121%:%
%:%265=122%:%
%:%266=122%:%
%:%267=122%:%
%:%268=123%:%
%:%269=123%:%
%:%270=124%:%
%:%271=124%:%
%:%272=125%:%
%:%273=125%:%
%:%274=126%:%
%:%275=126%:%
%:%276=127%:%
%:%277=127%:%
%:%278=128%:%
%:%279=128%:%
%:%280=128%:%
%:%281=129%:%
%:%282=129%:%
%:%283=129%:%
%:%284=130%:%
%:%285=130%:%
%:%286=130%:%
%:%287=131%:%
%:%288=131%:%
%:%289=132%:%
%:%290=132%:%
%:%291=132%:%
%:%292=133%:%
%:%293=133%:%
%:%294=134%:%
%:%304=136%:%
%:%306=138%:%
%:%307=138%:%
%:%314=139%:%
%:%315=139%:%
%:%316=140%:%
%:%317=140%:%
%:%318=140%:%
%:%319=141%:%
%:%320=141%:%
%:%321=142%:%
%:%322=142%:%
%:%323=143%:%
%:%324=143%:%
%:%325=144%:%
%:%326=144%:%
%:%327=145%:%
%:%328=145%:%
%:%329=146%:%
%:%330=146%:%
%:%331=146%:%
%:%332=146%:%
%:%333=146%:%
%:%334=147%:%
%:%335=147%:%
%:%336=147%:%
%:%337=147%:%
%:%338=148%:%
%:%339=148%:%
%:%340=148%:%
%:%341=148%:%
%:%342=149%:%
%:%352=151%:%
%:%354=153%:%
%:%355=153%:%
%:%362=154%:%
%:%363=154%:%
%:%364=155%:%
%:%365=155%:%
%:%366=155%:%
%:%367=156%:%
%:%368=156%:%
%:%369=157%:%
%:%370=157%:%
%:%371=158%:%
%:%372=158%:%
%:%373=159%:%
%:%374=159%:%
%:%375=159%:%
%:%376=159%:%
%:%377=160%:%
%:%378=160%:%