%
\begin{isabellebody}%
\setisabellecontext{SumaImpares}%
%
\isadelimtheory
\isanewline
%
\endisadelimtheory
%
\isatagtheory
%
\endisatagtheory
{\isafoldtheory}%
%
\isadelimtheory
%
\endisadelimtheory
%
\isadelimdocument
%
\endisadelimdocument
%
\isatagdocument
%
\isamarkupsection{Demostración en lenguaje natural%
}
\isamarkuptrue%
%
\endisatagdocument
{\isafolddocument}%
%
\isadelimdocument
%
\endisadelimdocument
%
\begin{isamarkuptext}%
El primer teorema es una propiedad de los números naturales.

  \begin{teorema}
    La suma de los $n$ primeros números impares es $n^2$.
  \end{teorema}

  \begin{demostracion}
    La demostración la haremos en inducción sobre $n$.
    
    (Base de la inducción) El caso $n = 0$ es trivial.
    
    (Paso de la inducción) Supongamos que la propiedad se verifica para
    $n$ y veamos que también se verifica para $n+1$. 
 
    Tenemos que demostrar que $\sum_{j=1}^{n+1} k_j = (n+1)^2$ donde
    $k_j$ el j--ésimo impar; es decir, $k_j = 2j - 1$.

    $$\begin{array}{l}
      \sum_{j = 1}^{n+1} k_j \\
      = k_{n+1} + \sum^{n}_{j=1} k_j \\ 
      = k_{n+1} + n^2 \\
      = 2(n+1) - 1 + n^2 \\
      = n^2 + 2n + 1 \\ 
      = (n+1)^2
      \end{array}$$ 
  \end{demostracion}%
\end{isamarkuptext}\isamarkuptrue%
%
\isadelimdocument
%
\endisadelimdocument
%
\isatagdocument
%
\isamarkupsection{Especificación en Isabelle/HOL%
}
\isamarkuptrue%
%
\endisatagdocument
{\isafolddocument}%
%
\isadelimdocument
%
\endisadelimdocument
%
\begin{isamarkuptext}%
Para especificar el teorema en Isabelle, se comienza definiendo 
  la función \isa{suma{\isacharunderscore}impares} tal que \isa{suma{\isacharunderscore}impares\ n} es
  la suma de los $n$ primeros números impares%
\end{isamarkuptext}\isamarkuptrue%
\isacommand{fun}\isamarkupfalse%
\ suma{\isacharunderscore}impares\ {\isacharcolon}{\isacharcolon}\ {\isachardoublequoteopen}nat\ {\isasymRightarrow}\ nat{\isachardoublequoteclose}\ \isakeyword{where}\isanewline
\ \ {\isachardoublequoteopen}suma{\isacharunderscore}impares\ {\isadigit{0}}\ {\isacharequal}\ {\isadigit{0}}{\isachardoublequoteclose}\ \isanewline
{\isacharbar}\ {\isachardoublequoteopen}suma{\isacharunderscore}impares\ {\isacharparenleft}Suc\ n{\isacharparenright}\ {\isacharequal}\ {\isacharparenleft}{\isadigit{2}}{\isacharasterisk}{\isacharparenleft}Suc\ n{\isacharparenright}\ {\isacharminus}\ {\isadigit{1}}{\isacharparenright}\ {\isacharplus}\ suma{\isacharunderscore}impares\ n{\isachardoublequoteclose}%
\begin{isamarkuptext}%
El enunciado del teorema es el siguiente:%
\end{isamarkuptext}\isamarkuptrue%
\isacommand{lemma}\isamarkupfalse%
\ {\isachardoublequoteopen}suma{\isacharunderscore}impares\ n\ {\isacharequal}\ n\ {\isacharasterisk}\ n{\isachardoublequoteclose}\isanewline
%
\isadelimproof
%
\endisadelimproof
%
\isatagproof
\isacommand{oops}\isamarkupfalse%
%
\endisatagproof
{\isafoldproof}%
%
\isadelimproof
%
\endisadelimproof
%
\isadelimdocument
%
\endisadelimdocument
%
\isatagdocument
%
\isamarkupsection{Demostración aplicativa%
}
\isamarkuptrue%
%
\endisatagdocument
{\isafolddocument}%
%
\isadelimdocument
%
\endisadelimdocument
%
\begin{isamarkuptext}%
En la demostración se usará la táctica \isa{induct} que hace
  uso del esquema de inducción sobre los naturales:
  \begin{itemize}
  \item[] \isa{\mbox{}\inferrule{\mbox{P\ {\isadigit{0}}}\\\ \mbox{{\isasymAnd}nat{\isachardot}\ \mbox{}\inferrule{\mbox{P\ nat}}{\mbox{P\ {\isacharparenleft}Suc\ nat{\isacharparenright}}}}}{\mbox{P\ nat}}} 
          \hfill (\isa{nat{\isachardot}induct})
  \end{itemize}

  Vamos a presentar distintas demostraciones del teorema. La 
  primera es la demostración aplicativa detallada.%
\end{isamarkuptext}\isamarkuptrue%
\isacommand{lemma}\isamarkupfalse%
\ {\isachardoublequoteopen}suma{\isacharunderscore}impares\ n\ {\isacharequal}\ n\ {\isacharasterisk}\ n{\isachardoublequoteclose}\isanewline
%
\isadelimproof
\ \ %
\endisadelimproof
%
\isatagproof
\isacommand{apply}\isamarkupfalse%
\ {\isacharparenleft}induct\ n{\isacharparenright}\ \isanewline
\ \ \ \isacommand{apply}\isamarkupfalse%
\ {\isacharparenleft}simp\ only{\isacharcolon}\ suma{\isacharunderscore}impares{\isachardot}simps{\isacharparenleft}{\isadigit{1}}{\isacharparenright}{\isacharparenright}\isanewline
\ \ \isacommand{apply}\isamarkupfalse%
\ {\isacharparenleft}simp\ only{\isacharcolon}\ suma{\isacharunderscore}impares{\isachardot}simps{\isacharparenleft}{\isadigit{2}}{\isacharparenright}{\isacharparenright}\isanewline
\ \ \isacommand{apply}\isamarkupfalse%
\ {\isacharparenleft}simp\ only{\isacharcolon}\ mult{\isacharunderscore}Suc\ mult{\isacharunderscore}Suc{\isacharunderscore}right{\isacharparenright}\isanewline
\ \ \isacommand{done}\isamarkupfalse%
%
\endisatagproof
{\isafoldproof}%
%
\isadelimproof
%
\endisadelimproof
%
\begin{isamarkuptext}%
En la demostración anterior hemos usado dentro del método 
\isa{simp} únicamente la definición de \isa{suma{\isacharunderscore}impares},
 para ello lo  hemos indicado con \isa{simp\ only}. A parte hemos
 usado lo siguiente :
  \begin{itemize}
  \item[] \isa{Suc\ m\ {\isacharasterisk}\ n\ {\isacharequal}\ n\ {\isacharplus}\ m\ {\isacharasterisk}\ n} 
          \hfill (\isa{mult{\isacharunderscore}Suc})
  \end{itemize}
  \begin{itemize}
  \item[] \isa{m\ {\isacharasterisk}\ Suc\ n\ {\isacharequal}\ m\ {\isacharplus}\ m\ {\isacharasterisk}\ n} 
          \hfill (\isa{mult{\isacharunderscore}Suc{\isacharunderscore}right})
  \end{itemize}%
\end{isamarkuptext}\isamarkuptrue%
%
\begin{isamarkuptext}%
Se puede eliminar los detalles de la demostración anterior.%
\end{isamarkuptext}\isamarkuptrue%
\isacommand{lemma}\isamarkupfalse%
\ {\isachardoublequoteopen}suma{\isacharunderscore}impares\ n\ {\isacharequal}\ n\ {\isacharasterisk}\ n{\isachardoublequoteclose}\isanewline
%
\isadelimproof
\ \ %
\endisadelimproof
%
\isatagproof
\isacommand{apply}\isamarkupfalse%
\ {\isacharparenleft}induct\ n{\isacharparenright}\ \isanewline
\ \ \ \isacommand{apply}\isamarkupfalse%
\ simp{\isacharunderscore}all\isanewline
\ \ \isacommand{done}\isamarkupfalse%
%
\endisatagproof
{\isafoldproof}%
%
\isadelimproof
%
\endisadelimproof
%
\isadelimdocument
%
\endisadelimdocument
%
\isatagdocument
%
\isamarkupsection{Demostración automática%
}
\isamarkuptrue%
%
\endisatagdocument
{\isafolddocument}%
%
\isadelimdocument
%
\endisadelimdocument
%
\begin{isamarkuptext}%
La correspondiente demostración automática es%
\end{isamarkuptext}\isamarkuptrue%
\isacommand{lemma}\isamarkupfalse%
\ {\isachardoublequoteopen}suma{\isacharunderscore}impares\ n\ {\isacharequal}\ n\ {\isacharasterisk}\ n{\isachardoublequoteclose}\isanewline
%
\isadelimproof
\ \ %
\endisadelimproof
%
\isatagproof
\isacommand{by}\isamarkupfalse%
\ {\isacharparenleft}induct\ n{\isacharparenright}\ simp{\isacharunderscore}all%
\endisatagproof
{\isafoldproof}%
%
\isadelimproof
%
\endisadelimproof
%
\isadelimdocument
%
\endisadelimdocument
%
\isatagdocument
%
\isamarkupsection{Demostración estructurada%
}
\isamarkuptrue%
%
\endisatagdocument
{\isafolddocument}%
%
\isadelimdocument
%
\endisadelimdocument
%
\begin{isamarkuptext}%
La demostración estructurada y detallada del lema anterior es%
\end{isamarkuptext}\isamarkuptrue%
\isacommand{lemma}\isamarkupfalse%
\ {\isachardoublequoteopen}suma{\isacharunderscore}impares\ n\ {\isacharequal}\ n\ {\isacharasterisk}\ n{\isachardoublequoteclose}\isanewline
%
\isadelimproof
%
\endisadelimproof
%
\isatagproof
\isacommand{proof}\isamarkupfalse%
\ {\isacharparenleft}induct\ n{\isacharparenright}\isanewline
\ \ \isacommand{have}\isamarkupfalse%
\ {\isachardoublequoteopen}suma{\isacharunderscore}impares\ {\isadigit{0}}\ {\isacharequal}\ {\isadigit{0}}{\isachardoublequoteclose}\ \isanewline
\ \ \ \ \isacommand{by}\isamarkupfalse%
\ {\isacharparenleft}simp\ only{\isacharcolon}\ suma{\isacharunderscore}impares{\isachardot}simps{\isacharparenleft}{\isadigit{1}}{\isacharparenright}{\isacharparenright}\isanewline
\ \ \isacommand{also}\isamarkupfalse%
\ \isacommand{have}\isamarkupfalse%
\ {\isachardoublequoteopen}{\isasymdots}\ {\isacharequal}\ {\isadigit{0}}\ {\isacharasterisk}\ {\isadigit{0}}{\isachardoublequoteclose}\isanewline
\ \ \ \ \isacommand{by}\isamarkupfalse%
\ {\isacharparenleft}simp\ only{\isacharcolon}\ \ mult{\isacharunderscore}{\isadigit{0}}{\isacharparenright}\isanewline
\ \ \isacommand{finally}\isamarkupfalse%
\ \isacommand{show}\isamarkupfalse%
\ {\isachardoublequoteopen}suma{\isacharunderscore}impares\ {\isadigit{0}}\ {\isacharequal}\ {\isadigit{0}}\ {\isacharasterisk}\ {\isadigit{0}}{\isachardoublequoteclose}\isanewline
\ \ \ \ \isacommand{by}\isamarkupfalse%
\ simp\isanewline
\isacommand{next}\isamarkupfalse%
\isanewline
\ \ \isacommand{fix}\isamarkupfalse%
\ n\ \isanewline
\ \ \isacommand{assume}\isamarkupfalse%
\ HI{\isacharcolon}\ {\isachardoublequoteopen}suma{\isacharunderscore}impares\ n\ {\isacharequal}\ n\ {\isacharasterisk}\ n{\isachardoublequoteclose}\isanewline
\ \ \isacommand{have}\isamarkupfalse%
\ {\isachardoublequoteopen}suma{\isacharunderscore}impares\ {\isacharparenleft}Suc\ n{\isacharparenright}\ {\isacharequal}\ {\isacharparenleft}{\isadigit{2}}\ {\isacharasterisk}\ {\isacharparenleft}Suc\ n{\isacharparenright}\ {\isacharminus}\ {\isadigit{1}}{\isacharparenright}\ {\isacharplus}\ suma{\isacharunderscore}impares\ n{\isachardoublequoteclose}\ \isanewline
\ \ \ \ \isacommand{by}\isamarkupfalse%
\ {\isacharparenleft}simp\ only{\isacharcolon}\ suma{\isacharunderscore}impares{\isachardot}simps{\isacharparenleft}{\isadigit{2}}{\isacharparenright}{\isacharparenright}\isanewline
\ \ \isacommand{also}\isamarkupfalse%
\ \isacommand{have}\isamarkupfalse%
\ {\isachardoublequoteopen}{\isasymdots}\ {\isacharequal}\ {\isacharparenleft}{\isadigit{2}}\ {\isacharasterisk}\ {\isacharparenleft}Suc\ n{\isacharparenright}\ {\isacharminus}\ {\isadigit{1}}{\isacharparenright}\ {\isacharplus}\ n\ {\isacharasterisk}\ n{\isachardoublequoteclose}\ \isanewline
\ \ \ \ \isacommand{by}\isamarkupfalse%
\ {\isacharparenleft}simp\ only{\isacharcolon}\ HI{\isacharparenright}\isanewline
\ \ \isacommand{also}\isamarkupfalse%
\ \isacommand{have}\isamarkupfalse%
\ {\isachardoublequoteopen}{\isasymdots}\ {\isacharequal}\ n\ {\isacharasterisk}\ n\ {\isacharplus}\ {\isadigit{2}}\ {\isacharasterisk}\ n\ {\isacharplus}\ {\isadigit{1}}{\isachardoublequoteclose}\ \isanewline
\ \ \ \ \isacommand{by}\isamarkupfalse%
\ {\isacharparenleft}simp\ only{\isacharcolon}\ mult{\isacharunderscore}Suc{\isacharunderscore}right{\isacharparenright}\isanewline
\ \ \isacommand{also}\isamarkupfalse%
\ \isacommand{have}\isamarkupfalse%
\ {\isachardoublequoteopen}{\isasymdots}\ {\isacharequal}\ {\isacharparenleft}Suc\ n{\isacharparenright}\ {\isacharasterisk}\ {\isacharparenleft}Suc\ n{\isacharparenright}{\isachardoublequoteclose}\isanewline
\ \ \ \ \isacommand{by}\isamarkupfalse%
\ {\isacharparenleft}simp\ only{\isacharcolon}\ mult{\isacharunderscore}Suc\ mult{\isacharunderscore}Suc{\isacharunderscore}right{\isacharparenright}\isanewline
\ \ \isacommand{finally}\isamarkupfalse%
\ \isacommand{show}\isamarkupfalse%
\ {\isachardoublequoteopen}suma{\isacharunderscore}impares\ {\isacharparenleft}Suc\ n{\isacharparenright}\ {\isacharequal}\ {\isacharparenleft}Suc\ n{\isacharparenright}\ {\isacharasterisk}\ {\isacharparenleft}Suc\ n{\isacharparenright}{\isachardoublequoteclose}\ \isanewline
\ \ \ \ \isacommand{by}\isamarkupfalse%
\ simp\isanewline
\isacommand{qed}\isamarkupfalse%
%
\endisatagproof
{\isafoldproof}%
%
\isadelimproof
%
\endisadelimproof
%
\begin{isamarkuptext}%
En la demostración anterior se pueden ocultar detalles.%
\end{isamarkuptext}\isamarkuptrue%
\isacommand{lemma}\isamarkupfalse%
\ {\isachardoublequoteopen}suma{\isacharunderscore}impares\ n\ {\isacharequal}\ n\ {\isacharasterisk}\ n{\isachardoublequoteclose}\isanewline
%
\isadelimproof
%
\endisadelimproof
%
\isatagproof
\isacommand{proof}\isamarkupfalse%
\ {\isacharparenleft}induct\ n{\isacharparenright}\isanewline
\ \ \isacommand{show}\isamarkupfalse%
\ {\isachardoublequoteopen}suma{\isacharunderscore}impares\ {\isadigit{0}}\ {\isacharequal}\ {\isadigit{0}}\ {\isacharasterisk}\ {\isadigit{0}}{\isachardoublequoteclose}\ \isacommand{by}\isamarkupfalse%
\ simp\isanewline
\isacommand{next}\isamarkupfalse%
\isanewline
\ \ \isacommand{fix}\isamarkupfalse%
\ n\ \isanewline
\ \ \isacommand{assume}\isamarkupfalse%
\ HI{\isacharcolon}\ {\isachardoublequoteopen}suma{\isacharunderscore}impares\ n\ {\isacharequal}\ n\ {\isacharasterisk}\ n{\isachardoublequoteclose}\isanewline
\ \ \isacommand{have}\isamarkupfalse%
\ {\isachardoublequoteopen}suma{\isacharunderscore}impares\ {\isacharparenleft}Suc\ n{\isacharparenright}\ {\isacharequal}\ {\isacharparenleft}{\isadigit{2}}\ {\isacharasterisk}\ {\isacharparenleft}Suc\ n{\isacharparenright}\ {\isacharminus}\ {\isadigit{1}}{\isacharparenright}\ {\isacharplus}\ suma{\isacharunderscore}impares\ n{\isachardoublequoteclose}\ \isanewline
\ \ \ \ \isacommand{by}\isamarkupfalse%
\ simp\isanewline
\ \ \isacommand{also}\isamarkupfalse%
\ \isacommand{have}\isamarkupfalse%
\ {\isachardoublequoteopen}{\isasymdots}\ {\isacharequal}\ {\isacharparenleft}{\isadigit{2}}\ {\isacharasterisk}\ {\isacharparenleft}Suc\ n{\isacharparenright}\ {\isacharminus}\ {\isadigit{1}}{\isacharparenright}\ {\isacharplus}\ n\ {\isacharasterisk}\ n{\isachardoublequoteclose}\ \isanewline
\ \ \ \ \isacommand{using}\isamarkupfalse%
\ HI\ \isacommand{by}\isamarkupfalse%
\ simp\isanewline
\ \ \isacommand{also}\isamarkupfalse%
\ \isacommand{have}\isamarkupfalse%
\ {\isachardoublequoteopen}{\isasymdots}\ {\isacharequal}\ {\isacharparenleft}Suc\ n{\isacharparenright}\ {\isacharasterisk}\ {\isacharparenleft}Suc\ n{\isacharparenright}{\isachardoublequoteclose}\ \isanewline
\ \ \ \ \isacommand{by}\isamarkupfalse%
\ simp\isanewline
\ \ \isacommand{finally}\isamarkupfalse%
\ \isacommand{show}\isamarkupfalse%
\ {\isachardoublequoteopen}suma{\isacharunderscore}impares\ {\isacharparenleft}Suc\ n{\isacharparenright}\ {\isacharequal}\ {\isacharparenleft}Suc\ n{\isacharparenright}\ {\isacharasterisk}\ {\isacharparenleft}Suc\ n{\isacharparenright}{\isachardoublequoteclose}\ \isanewline
\ \ \ \ \isacommand{by}\isamarkupfalse%
\ simp\isanewline
\isacommand{qed}\isamarkupfalse%
%
\endisatagproof
{\isafoldproof}%
%
\isadelimproof
%
\endisadelimproof
%
\isadelimdocument
%
\endisadelimdocument
%
\isatagdocument
%
\isamarkupsection{Demostración con patrones%
}
\isamarkuptrue%
%
\endisatagdocument
{\isafolddocument}%
%
\isadelimdocument
%
\endisadelimdocument
%
\begin{isamarkuptext}%
La demostración anterior se puede simplificar usando patrones.%
\end{isamarkuptext}\isamarkuptrue%
\isacommand{lemma}\isamarkupfalse%
\ {\isachardoublequoteopen}suma{\isacharunderscore}impares\ n\ {\isacharequal}\ n\ {\isacharasterisk}\ n{\isachardoublequoteclose}\ {\isacharparenleft}\isakeyword{is}\ {\isachardoublequoteopen}{\isacharquery}P\ n\ {\isacharequal}\ {\isacharquery}Q\ n{\isachardoublequoteclose}{\isacharparenright}\isanewline
%
\isadelimproof
%
\endisadelimproof
%
\isatagproof
\isacommand{proof}\isamarkupfalse%
\ {\isacharparenleft}induct\ n{\isacharparenright}\isanewline
\ \ \isacommand{show}\isamarkupfalse%
\ {\isachardoublequoteopen}{\isacharquery}P\ {\isadigit{0}}\ {\isacharequal}\ {\isacharquery}Q\ {\isadigit{0}}{\isachardoublequoteclose}\ \isacommand{by}\isamarkupfalse%
\ simp\isanewline
\isacommand{next}\isamarkupfalse%
\isanewline
\ \ \isacommand{fix}\isamarkupfalse%
\ n\ \isanewline
\ \ \isacommand{assume}\isamarkupfalse%
\ HI{\isacharcolon}\ {\isachardoublequoteopen}{\isacharquery}P\ n\ {\isacharequal}\ {\isacharquery}Q\ n{\isachardoublequoteclose}\isanewline
\ \ \isacommand{have}\isamarkupfalse%
\ {\isachardoublequoteopen}{\isacharquery}P\ {\isacharparenleft}Suc\ n{\isacharparenright}\ {\isacharequal}\ {\isacharparenleft}{\isadigit{2}}\ {\isacharasterisk}\ {\isacharparenleft}Suc\ n{\isacharparenright}\ {\isacharminus}\ {\isadigit{1}}{\isacharparenright}\ {\isacharplus}\ suma{\isacharunderscore}impares\ n{\isachardoublequoteclose}\ \isanewline
\ \ \ \ \isacommand{by}\isamarkupfalse%
\ simp\isanewline
\ \ \isacommand{also}\isamarkupfalse%
\ \isacommand{have}\isamarkupfalse%
\ {\isachardoublequoteopen}{\isasymdots}\ {\isacharequal}\ {\isacharparenleft}{\isadigit{2}}\ {\isacharasterisk}\ {\isacharparenleft}Suc\ n{\isacharparenright}\ {\isacharminus}\ {\isadigit{1}}{\isacharparenright}\ {\isacharplus}\ n\ {\isacharasterisk}\ n{\isachardoublequoteclose}\ \isacommand{using}\isamarkupfalse%
\ HI\ \isacommand{by}\isamarkupfalse%
\ simp\isanewline
\ \ \isacommand{also}\isamarkupfalse%
\ \isacommand{have}\isamarkupfalse%
\ {\isachardoublequoteopen}{\isasymdots}\ {\isacharequal}\ {\isacharquery}Q\ {\isacharparenleft}Suc\ n{\isacharparenright}{\isachardoublequoteclose}\ \isacommand{by}\isamarkupfalse%
\ simp\isanewline
\ \ \isacommand{finally}\isamarkupfalse%
\ \isacommand{show}\isamarkupfalse%
\ {\isachardoublequoteopen}{\isacharquery}P\ {\isacharparenleft}Suc\ n{\isacharparenright}\ {\isacharequal}\ {\isacharquery}Q\ {\isacharparenleft}Suc\ n{\isacharparenright}{\isachardoublequoteclose}\ \isacommand{by}\isamarkupfalse%
\ simp\isanewline
\isacommand{qed}\isamarkupfalse%
%
\endisatagproof
{\isafoldproof}%
%
\isadelimproof
%
\endisadelimproof
%
\begin{isamarkuptext}%
La demostración usando otro patrón es%
\end{isamarkuptext}\isamarkuptrue%
\isacommand{lemma}\isamarkupfalse%
\ {\isachardoublequoteopen}suma{\isacharunderscore}impares\ n\ {\isacharequal}\ n\ {\isacharasterisk}\ n{\isachardoublequoteclose}\ {\isacharparenleft}\isakeyword{is}\ {\isachardoublequoteopen}{\isacharquery}P\ n{\isachardoublequoteclose}{\isacharparenright}\isanewline
%
\isadelimproof
%
\endisadelimproof
%
\isatagproof
\isacommand{proof}\isamarkupfalse%
\ {\isacharparenleft}induct\ n{\isacharparenright}\isanewline
\ \ \isacommand{show}\isamarkupfalse%
\ {\isachardoublequoteopen}{\isacharquery}P\ {\isadigit{0}}{\isachardoublequoteclose}\ \isacommand{by}\isamarkupfalse%
\ simp\isanewline
\isacommand{next}\isamarkupfalse%
\isanewline
\ \ \isacommand{fix}\isamarkupfalse%
\ n\ \isanewline
\ \ \isacommand{assume}\isamarkupfalse%
\ {\isachardoublequoteopen}{\isacharquery}P\ n{\isachardoublequoteclose}\isanewline
\ \ \isacommand{then}\isamarkupfalse%
\ \isacommand{show}\isamarkupfalse%
\ {\isachardoublequoteopen}{\isacharquery}P\ {\isacharparenleft}Suc\ n{\isacharparenright}{\isachardoublequoteclose}\ \isacommand{by}\isamarkupfalse%
\ simp\isanewline
\isacommand{qed}\isamarkupfalse%
\isanewline
%
\endisatagproof
{\isafoldproof}%
%
\isadelimproof
%
\endisadelimproof
%
\isadelimtheory
%
\endisadelimtheory
%
\isatagtheory
%
\endisatagtheory
{\isafoldtheory}%
%
\isadelimtheory
%
\endisadelimtheory
%
\end{isabellebody}%
\endinput
%:%file=~/Escritorio/TFG/SumaImpares.thy%:%
%:%6=1%:%
%:%25=9%:%
%:%37=11%:%
%:%38=12%:%
%:%39=13%:%
%:%40=14%:%
%:%41=15%:%
%:%42=16%:%
%:%43=17%:%
%:%44=18%:%
%:%45=19%:%
%:%46=20%:%
%:%47=21%:%
%:%48=22%:%
%:%49=23%:%
%:%50=24%:%
%:%51=25%:%
%:%52=26%:%
%:%53=27%:%
%:%54=28%:%
%:%55=29%:%
%:%56=30%:%
%:%57=31%:%
%:%58=32%:%
%:%59=33%:%
%:%60=34%:%
%:%61=35%:%
%:%62=36%:%
%:%71=39%:%
%:%83=41%:%
%:%84=42%:%
%:%85=43%:%
%:%87=45%:%
%:%88=45%:%
%:%89=46%:%
%:%90=47%:%
%:%92=49%:%
%:%94=51%:%
%:%95=51%:%
%:%102=52%:%
%:%117=54%:%
%:%129=56%:%
%:%130=57%:%
%:%131=58%:%
%:%132=59%:%
%:%133=60%:%
%:%134=61%:%
%:%135=62%:%
%:%136=63%:%
%:%137=64%:%
%:%139=66%:%
%:%140=66%:%
%:%143=67%:%
%:%147=67%:%
%:%148=67%:%
%:%149=68%:%
%:%150=68%:%
%:%151=69%:%
%:%152=69%:%
%:%153=70%:%
%:%154=70%:%
%:%155=71%:%
%:%165=73%:%
%:%166=74%:%
%:%167=75%:%
%:%168=76%:%
%:%169=77%:%
%:%170=78%:%
%:%171=79%:%
%:%172=80%:%
%:%173=81%:%
%:%174=82%:%
%:%175=83%:%
%:%176=84%:%
%:%180=87%:%
%:%182=89%:%
%:%183=89%:%
%:%186=90%:%
%:%190=90%:%
%:%191=90%:%
%:%192=91%:%
%:%193=91%:%
%:%194=92%:%
%:%209=94%:%
%:%221=96%:%
%:%223=98%:%
%:%224=98%:%
%:%227=99%:%
%:%231=99%:%
%:%232=99%:%
%:%246=101%:%
%:%258=103%:%
%:%260=105%:%
%:%261=105%:%
%:%268=106%:%
%:%269=106%:%
%:%270=107%:%
%:%271=107%:%
%:%272=108%:%
%:%273=108%:%
%:%274=109%:%
%:%275=109%:%
%:%276=109%:%
%:%277=110%:%
%:%278=110%:%
%:%279=111%:%
%:%280=111%:%
%:%281=111%:%
%:%282=112%:%
%:%283=112%:%
%:%284=113%:%
%:%285=113%:%
%:%286=114%:%
%:%287=114%:%
%:%288=115%:%
%:%289=115%:%
%:%290=116%:%
%:%291=116%:%
%:%292=117%:%
%:%293=117%:%
%:%294=118%:%
%:%295=118%:%
%:%296=118%:%
%:%297=119%:%
%:%298=119%:%
%:%299=120%:%
%:%300=120%:%
%:%301=120%:%
%:%302=121%:%
%:%303=121%:%
%:%304=122%:%
%:%305=122%:%
%:%306=122%:%
%:%307=123%:%
%:%308=123%:%
%:%309=124%:%
%:%310=124%:%
%:%311=124%:%
%:%312=125%:%
%:%313=125%:%
%:%314=126%:%
%:%324=128%:%
%:%326=130%:%
%:%327=130%:%
%:%334=131%:%
%:%335=131%:%
%:%336=132%:%
%:%337=132%:%
%:%338=132%:%
%:%339=133%:%
%:%340=133%:%
%:%341=134%:%
%:%342=134%:%
%:%343=135%:%
%:%344=135%:%
%:%345=136%:%
%:%346=136%:%
%:%347=137%:%
%:%348=137%:%
%:%349=138%:%
%:%350=138%:%
%:%351=138%:%
%:%352=139%:%
%:%353=139%:%
%:%354=139%:%
%:%355=140%:%
%:%356=140%:%
%:%357=140%:%
%:%358=141%:%
%:%359=141%:%
%:%360=142%:%
%:%361=142%:%
%:%362=142%:%
%:%363=143%:%
%:%364=143%:%
%:%365=144%:%
%:%380=146%:%
%:%392=148%:%
%:%394=150%:%
%:%395=150%:%
%:%402=151%:%
%:%403=151%:%
%:%404=152%:%
%:%405=152%:%
%:%406=152%:%
%:%407=153%:%
%:%408=153%:%
%:%409=154%:%
%:%410=154%:%
%:%411=155%:%
%:%412=155%:%
%:%413=156%:%
%:%414=156%:%
%:%415=157%:%
%:%416=157%:%
%:%417=158%:%
%:%418=158%:%
%:%419=158%:%
%:%420=158%:%
%:%421=158%:%
%:%422=159%:%
%:%423=159%:%
%:%424=159%:%
%:%425=159%:%
%:%426=160%:%
%:%427=160%:%
%:%428=160%:%
%:%429=160%:%
%:%430=161%:%
%:%440=163%:%
%:%442=165%:%
%:%443=165%:%
%:%450=166%:%
%:%451=166%:%
%:%452=167%:%
%:%453=167%:%
%:%454=167%:%
%:%455=168%:%
%:%456=168%:%
%:%457=169%:%
%:%458=169%:%
%:%459=170%:%
%:%460=170%:%
%:%461=171%:%
%:%462=171%:%
%:%463=171%:%
%:%464=171%:%
%:%465=172%:%
%:%466=172%:%