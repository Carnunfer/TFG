%
\begin{isabellebody}%
\setisabellecontext{EjerciciosACGA}%
%
\isadelimtheory
%
\endisadelimtheory
%
\isatagtheory
%
\endisatagtheory
{\isafoldtheory}%
%
\isadelimtheory
%
\endisadelimtheory
%
\isadelimdocument
%
\endisadelimdocument
%
\isatagdocument
%
\isamarkupsection{Ejercicio para entregar de ACGA%
}
\isamarkuptrue%
%
\endisatagdocument
{\isafolddocument}%
%
\isadelimdocument
%
\endisadelimdocument
%
\begin{isamarkuptext}%
\begin {ejercicio} 
Sean $A$,$B$ dos anillos 
\begin {enumerate}
\item Probar que si $I \subseteq A$ y $J \subseteq B$ son ideales,
 entonces $I \times J \subseteq A \times B$ es un ideal.
\item Probar que todo ideal de $A \times B$ es de esta forma.
\item Si $A$ y $B$ son noetherianos, probar que $A \times B$ también lo
 es.
\end{enumerate}

\end {ejercicio} 

\begin {solucion}

\begin{enumerate}

\item Para que $I \times J \subseteq A \times B$ sea un ideal de $A \times B$
 tiene que verificar que sea un subgrupo aditivo y que $\forall  x \in
 I \times J$ y $\forall y \in A \times B$ tenemos que $xy \in I
 \times J$.
\item Sea $K \subseteq A \times B$ un ideal, definamos: 

$$I = \{ a \in A | (a,b) \in K para algun  b \in B \}$$

$$J = \{ b \in B | (a,b) \in K para algun a \in A \}$$

Es fácil ver $I$,$J$ son ideales de $A$ y $B$ respectivamente.

Si $a , c \in I \Longrightarrow \exists b , d \in B$ tal que 
 $(a,b),(c,d) \in K \Longrightarrow (a + c, b + d) \in K$, luego
 verifica que es un subgrupo aditivo. Ahora tenemos que $\forall r \in
A,  (r,0) \times (a,b) = (ra , 0) \in K$ por ser $K$ un ideal, luego esto
significa que $ra \in I$ y ,por lo tanto, I es un ideal de A. \\
Para demostrar que $J$ es un ideal hacemos el mismo razonamiento. \\
Veamos ahora que $K = I \times J$, por doble contención. \\
Para la contención hacía la derecha, sea $(a,b) \in K$ esto por
 definición de $I$,$J$, significa que $(a,b) \in I \times J$. \\
Para la contención hacía la izquierda, sea $(a,b) \in I \times J$.
 Consideremos $a \in I$ y $b \in J$ luego  esto implica que $\exists c
 \in B$ tal que $(a,c) \in K$ y $\exists d \in A$ tal que $(d,b) \in
 K$. Como K es un ideal de $A \times B$, $(1,0) \times (a,c) = (a,0) \in
K$, con el mismo razonamiento llegamos a que $(0,b) \in K$. Entonces 
$(a,0) + (0,b) = (a,b) \in K$. 

\item Consideremos una cadena de ideales creciente de $A \times B$. 
$$I_{1} \subseteq I_{2} \subseteq \ldots$$

\end {enumerate}

.
\end{solucion}%
\end{isamarkuptext}\isamarkuptrue%
%
\isadelimtheory
%
\endisadelimtheory
%
\isatagtheory
%
\endisatagtheory
{\isafoldtheory}%
%
\isadelimtheory
%
\endisadelimtheory
%
\end{isabellebody}%
\endinput
%:%file=/home/carlos/Escritorio/ACGA ejercicios propuestos/EjerciciosACGA.thy%:%
%:%24=7%:%
%:%36=9%:%
%:%37=10%:%
%:%38=11%:%
%:%39=12%:%
%:%40=13%:%
%:%41=14%:%
%:%42=15%:%
%:%43=16%:%
%:%44=17%:%
%:%45=18%:%
%:%46=19%:%
%:%47=20%:%
%:%48=21%:%
%:%49=22%:%
%:%50=23%:%
%:%51=24%:%
%:%52=25%:%
%:%53=26%:%
%:%54=27%:%
%:%55=28%:%
%:%56=29%:%
%:%57=30%:%
%:%58=31%:%
%:%59=32%:%
%:%60=33%:%
%:%61=34%:%
%:%62=35%:%
%:%63=36%:%
%:%64=37%:%
%:%65=38%:%
%:%66=39%:%
%:%67=40%:%
%:%68=41%:%
%:%69=42%:%
%:%70=43%:%
%:%71=44%:%
%:%72=45%:%
%:%73=46%:%
%:%74=47%:%
%:%75=48%:%
%:%76=49%:%
%:%77=50%:%
%:%78=51%:%
%:%79=52%:%
%:%80=53%:%
%:%81=54%:%
%:%82=55%:%
%:%83=56%:%
%:%84=57%:%
%:%85=58%:%
%:%86=59%:%