%
\begin{isabellebody}%
\setisabellecontext{CancelacionSobreyectiva}%
%
\isadelimtheory
%
\endisadelimtheory
%
\isatagtheory
%
\endisatagtheory
{\isafoldtheory}%
%
\isadelimtheory
%
\endisadelimtheory
%
\isadelimdocument
%
\endisadelimdocument
%
\isatagdocument
%
\isamarkupsection{Cancelación de las funciones sobreyectivas%
}
\isamarkuptrue%
%
\endisatagdocument
{\isafolddocument}%
%
\isadelimdocument
%
\endisadelimdocument
%
\begin{isamarkuptext}%
El siguiente teorema prueba una propiedad de las funciones
 sobreyectivas. El enunciado es el siguiente: 
\begin {teorema}
Las funciones sobreyectivas son cancelativas por la derecha. Es decir,
 si f es sobreyectiva entonces para todas funciones g y h tal que g o f
 = h o f se tiene que g = h.
\end {teorema}
 
\begin {demostracion}
\begin {itemize}
\item Supongamos que tenemos que $g \circ  f = h \circ f$, queremos probar que $g =
 h.$ Usando la definición de sobreyectividad $(\forall y \in Y,
 \exists x | y = f(x))$ y nuestra hipótesis, tenemos que:
$$g(y) = g(f(x)) = (g o f) (x) = (h o f) (x) = h(f(x)) = h(y)$$
\item Supongamos que $g = h$, hay que probar que $g o f = h o f.$ Usando
nuestra hipótesis, tenemos que:
$$ (g o f)(x) = g(f(x)) = h(f(x)) = (h o f) (x).$$
\end {itemize}
.
\end {demostracion}

Su especificación es la siguiente:%
\end{isamarkuptext}\isamarkuptrue%
\isacommand{lemma}\isamarkupfalse%
\ {\isachardoublequoteopen}surj\ f\ {\isasymLongrightarrow}\ {\isacharparenleft}\ g\ {\isasymo}\ f\ {\isacharequal}\ h\ {\isasymo}\ f\ {\isacharparenright}\ {\isacharequal}\ {\isacharparenleft}g\ {\isacharequal}\ h{\isacharparenright}{\isachardoublequoteclose}\isanewline
%
\isadelimproof
\ \ %
\endisadelimproof
%
\isatagproof
\isacommand{oops}\isamarkupfalse%
%
\endisatagproof
{\isafoldproof}%
%
\isadelimproof
%
\endisadelimproof
%
\begin{isamarkuptext}%
En la especificación anterior, \isa{surj\ f} es una abreviatura de 
  \isa{range\ f\ {\isacharequal}\ UNIV}, donde \isa{range\ f} es el rango o imagen
de la función f.
 Por otra parte, \isa{UNIV} es el conjunto universal definido en la 
  teoría \href{http://bit.ly/2XtHCW6}{Set.thy} como una abreviatura de 
  \isa{top} que, a su vez está definido en la teoría 
  \href{http://bit.ly/2Xyj9Pe}{Orderings.thy} mediante la siguiente
  propiedad 
  \begin{itemize}
    \item[] \isa{\mbox{}\inferrule{\mbox{ordering{\isacharunderscore}top\ less{\isacharunderscore}eq\ less\ top}}{\mbox{less{\isacharunderscore}eq\ a\ top}}} 
      \hfill (\isa{ordering{\isacharunderscore}top{\isachardot}extremum})
  \end{itemize} 
Además queda añadir que la teoría donde se encuentra definido \isa{surj\ f}
 es en \href{http://bit.ly/2XuPQx5}{Fun.thy}. Esta teoría contiene la
 definicion \isa{surj{\isacharunderscore}def}.
 \begin{itemize}
    \item[] \isa{surj\ f\ {\isacharequal}\ {\isacharparenleft}{\isasymforall}y{\isachardot}\ {\isasymexists}x{\isachardot}\ y\ {\isacharequal}\ f\ x{\isacharparenright}} \hfill (\isa{inj{\isacharunderscore}on{\isacharunderscore}def})
  \end{itemize} 

Presentaremos distintas demostraciones del teorema. La primera es la
 detallada:%
\end{isamarkuptext}\isamarkuptrue%
\isacommand{lemma}\isamarkupfalse%
\ \isanewline
\ \ \isakeyword{assumes}\ {\isachardoublequoteopen}surj\ f{\isachardoublequoteclose}\ \isanewline
\ \ \isakeyword{shows}\ {\isachardoublequoteopen}{\isacharparenleft}\ g\ {\isasymcirc}\ f\ {\isacharequal}\ h\ {\isasymcirc}\ f\ {\isacharparenright}\ {\isacharequal}\ {\isacharparenleft}g\ {\isacharequal}\ h{\isacharparenright}{\isachardoublequoteclose}\isanewline
%
\isadelimproof
%
\endisadelimproof
%
\isatagproof
\isacommand{proof}\isamarkupfalse%
\ {\isacharparenleft}rule\ iffI{\isacharparenright}\isanewline
\ \ \isacommand{assume}\isamarkupfalse%
\ {\isadigit{1}}{\isacharcolon}\ {\isachardoublequoteopen}\ g\ {\isasymcirc}\ f\ {\isacharequal}\ h\ {\isasymcirc}\ f\ {\isachardoublequoteclose}\isanewline
\ \ \isacommand{show}\isamarkupfalse%
\ {\isachardoublequoteopen}g\ {\isacharequal}\ h{\isachardoublequoteclose}\ \isanewline
\ \ \isacommand{proof}\isamarkupfalse%
\ \isanewline
\ \ \ \ \isacommand{fix}\isamarkupfalse%
\ x\isanewline
\isanewline
\ \ \ \ \isacommand{have}\isamarkupfalse%
\ {\isachardoublequoteopen}\ {\isasymexists}y\ {\isachardot}\ x\ {\isacharequal}\ f{\isacharparenleft}y{\isacharparenright}{\isachardoublequoteclose}\ \isacommand{using}\isamarkupfalse%
\ assms\ \isacommand{by}\isamarkupfalse%
\ {\isacharparenleft}simp\ add{\isacharcolon}surj{\isacharunderscore}def{\isacharparenright}\isanewline
\ \ \ \ \isacommand{then}\isamarkupfalse%
\ \isacommand{obtain}\isamarkupfalse%
\ {\isachardoublequoteopen}y{\isachardoublequoteclose}\ \isakeyword{where}\ {\isadigit{2}}{\isacharcolon}{\isachardoublequoteopen}x\ {\isacharequal}\ f{\isacharparenleft}y{\isacharparenright}{\isachardoublequoteclose}\ \isacommand{by}\isamarkupfalse%
\ {\isacharparenleft}rule\ exE{\isacharparenright}\isanewline
\ \ \ \ \isacommand{then}\isamarkupfalse%
\ \isacommand{have}\isamarkupfalse%
\ {\isachardoublequoteopen}g{\isacharparenleft}x{\isacharparenright}\ {\isacharequal}\ g{\isacharparenleft}f{\isacharparenleft}y{\isacharparenright}{\isacharparenright}{\isachardoublequoteclose}\ \isacommand{by}\isamarkupfalse%
\ simp\isanewline
\ \ \ \ \isacommand{also}\isamarkupfalse%
\ \isacommand{have}\isamarkupfalse%
\ {\isachardoublequoteopen}{\isachardot}{\isachardot}{\isachardot}\ {\isacharequal}\ {\isacharparenleft}g\ {\isasymcirc}\ f{\isacharparenright}\ {\isacharparenleft}y{\isacharparenright}\ \ {\isachardoublequoteclose}\ \isacommand{by}\isamarkupfalse%
\ simp\isanewline
\ \ \ \ \isacommand{also}\isamarkupfalse%
\ \isacommand{have}\isamarkupfalse%
\ {\isachardoublequoteopen}{\isachardot}{\isachardot}{\isachardot}\ {\isacharequal}\ {\isacharparenleft}h\ {\isasymcirc}\ f{\isacharparenright}\ {\isacharparenleft}y{\isacharparenright}{\isachardoublequoteclose}\ \isacommand{using}\isamarkupfalse%
\ {\isadigit{1}}\ \isacommand{by}\isamarkupfalse%
\ simp\isanewline
\ \ \ \ \isacommand{also}\isamarkupfalse%
\ \isacommand{have}\isamarkupfalse%
\ {\isachardoublequoteopen}{\isachardot}{\isachardot}{\isachardot}\ {\isacharequal}\ h{\isacharparenleft}f{\isacharparenleft}y{\isacharparenright}{\isacharparenright}{\isachardoublequoteclose}\ \isacommand{by}\isamarkupfalse%
\ simp\isanewline
\ \ \ \ \isacommand{also}\isamarkupfalse%
\ \isacommand{have}\isamarkupfalse%
\ {\isachardoublequoteopen}{\isachardot}{\isachardot}{\isachardot}\ {\isacharequal}\ h{\isacharparenleft}x{\isacharparenright}{\isachardoublequoteclose}\ \isacommand{using}\isamarkupfalse%
\ {\isadigit{2}}\ \ \ \isacommand{by}\isamarkupfalse%
\ {\isacharparenleft}simp\ add{\isacharcolon}\ {\isacartoucheopen}x\ {\isacharequal}\ f\ y{\isacartoucheclose}{\isacharparenright}\isanewline
\ \ \ \ \isacommand{finally}\isamarkupfalse%
\ \isacommand{show}\isamarkupfalse%
\ \ {\isachardoublequoteopen}\ g{\isacharparenleft}x{\isacharparenright}\ {\isacharequal}\ h{\isacharparenleft}x{\isacharparenright}\ {\isachardoublequoteclose}\ \isacommand{by}\isamarkupfalse%
\ simp\isanewline
\ \ \isacommand{qed}\isamarkupfalse%
\isanewline
\isacommand{next}\isamarkupfalse%
\isanewline
\ \ \isacommand{assume}\isamarkupfalse%
\ {\isachardoublequoteopen}g\ {\isacharequal}\ h{\isachardoublequoteclose}\ \isanewline
\ \ \isacommand{show}\isamarkupfalse%
\ {\isachardoublequoteopen}g\ {\isasymcirc}\ f\ {\isacharequal}\ h\ {\isasymcirc}\ f{\isachardoublequoteclose}\isanewline
\ \ \isacommand{proof}\isamarkupfalse%
\isanewline
\ \ \ \ \isacommand{fix}\isamarkupfalse%
\ x\isanewline
\ \ \ \ \isacommand{have}\isamarkupfalse%
\ {\isachardoublequoteopen}{\isacharparenleft}g\ {\isasymcirc}\ f{\isacharparenright}\ x\ {\isacharequal}\ g{\isacharparenleft}f{\isacharparenleft}x{\isacharparenright}{\isacharparenright}{\isachardoublequoteclose}\ \isacommand{by}\isamarkupfalse%
\ simp\isanewline
\ \ \ \ \isacommand{also}\isamarkupfalse%
\ \isacommand{have}\isamarkupfalse%
\ {\isachardoublequoteopen}{\isasymdots}\ {\isacharequal}\ h{\isacharparenleft}f{\isacharparenleft}x{\isacharparenright}{\isacharparenright}{\isachardoublequoteclose}\ \isacommand{using}\isamarkupfalse%
\ {\isacharbackquoteopen}g\ {\isacharequal}\ h{\isacharbackquoteclose}\ \isacommand{by}\isamarkupfalse%
\ simp\isanewline
\ \ \ \ \isacommand{also}\isamarkupfalse%
\ \isacommand{have}\isamarkupfalse%
\ {\isachardoublequoteopen}{\isasymdots}\ {\isacharequal}\ {\isacharparenleft}h\ {\isasymcirc}\ f{\isacharparenright}\ x{\isachardoublequoteclose}\ \isacommand{by}\isamarkupfalse%
\ simp\isanewline
\ \ \ \ \isacommand{finally}\isamarkupfalse%
\ \isacommand{show}\isamarkupfalse%
\ {\isachardoublequoteopen}{\isacharparenleft}g\ {\isasymcirc}\ f{\isacharparenright}\ x\ {\isacharequal}\ {\isacharparenleft}h\ {\isasymcirc}\ f{\isacharparenright}\ x{\isachardoublequoteclose}\ \isacommand{by}\isamarkupfalse%
\ simp\isanewline
\ \ \isacommand{qed}\isamarkupfalse%
\isanewline
\isacommand{qed}\isamarkupfalse%
%
\endisatagproof
{\isafoldproof}%
%
\isadelimproof
%
\endisadelimproof
%
\begin{isamarkuptext}%
En la demostración hemos introducido: 
 \begin{itemize}
    \item[] \isa{\mbox{}\inferrule{\mbox{{\isasymexists}x{\isachardot}\ P\ x}\\\ \mbox{{\isasymAnd}x{\isachardot}\ \mbox{}\inferrule{\mbox{P\ x}}{\mbox{Q}}}}{\mbox{Q}}} 
      \hfill (\isa{rule\ exE}) 
  \end{itemize} 
 \begin{itemize}
    \item[] \isa{{\isasymlbrakk}P\ {\isasymLongrightarrow}\ Q{\isacharsemicolon}\ Q\ {\isasymLongrightarrow}\ P{\isasymrbrakk}\ {\isasymLongrightarrow}\ P\ {\isacharequal}\ Q} 
      \hfill (\isa{iffI})
  \end{itemize} 

La demostración aplicativa es:%
\end{isamarkuptext}\isamarkuptrue%
\isacommand{lemma}\isamarkupfalse%
\ {\isachardoublequoteopen}surj\ f\ {\isasymLongrightarrow}\ {\isacharparenleft}{\isacharparenleft}g\ {\isasymcirc}\ f{\isacharparenright}\ {\isacharequal}\ {\isacharparenleft}h\ {\isasymcirc}\ f{\isacharparenright}\ {\isacharparenright}\ {\isacharequal}\ {\isacharparenleft}g\ {\isacharequal}\ h{\isacharparenright}{\isachardoublequoteclose}\isanewline
%
\isadelimproof
\ \ %
\endisadelimproof
%
\isatagproof
\isacommand{apply}\isamarkupfalse%
\ {\isacharparenleft}simp\ add{\isacharcolon}\ surj{\isacharunderscore}def\ fun{\isacharunderscore}eq{\isacharunderscore}iff{\isacharparenright}\isanewline
\ \ \isacommand{apply}\isamarkupfalse%
\ {\isacharparenleft}rule\ iffI{\isacharparenright}\isanewline
\ \ \ \isacommand{prefer}\isamarkupfalse%
\ {\isadigit{2}}\isanewline
\ \ \isacommand{apply}\isamarkupfalse%
\ auto\isanewline
\ \isanewline
\ \ \isacommand{apply}\isamarkupfalse%
\ \ metis\isanewline
\isanewline
\ \ \isacommand{done}\isamarkupfalse%
%
\endisatagproof
{\isafoldproof}%
%
\isadelimproof
\isanewline
%
\endisadelimproof
\isanewline
\isacommand{lemma}\isamarkupfalse%
\ {\isachardoublequoteopen}surj\ f\ {\isasymLongrightarrow}\ {\isacharparenleft}{\isacharparenleft}g\ {\isasymcirc}\ f{\isacharparenright}\ {\isacharequal}\ {\isacharparenleft}h\ {\isasymcirc}\ f{\isacharparenright}\ {\isacharparenright}\ {\isacharequal}\ {\isacharparenleft}g\ {\isacharequal}\ h{\isacharparenright}{\isachardoublequoteclose}\isanewline
%
\isadelimproof
\ \ %
\endisadelimproof
%
\isatagproof
\isacommand{apply}\isamarkupfalse%
\ {\isacharparenleft}simp\ add{\isacharcolon}\ surj{\isacharunderscore}def\ fun{\isacharunderscore}eq{\isacharunderscore}iff\ {\isacharparenright}\ \isanewline
\ \ \isacommand{by}\isamarkupfalse%
\ metis%
\endisatagproof
{\isafoldproof}%
%
\isadelimproof
%
\endisadelimproof
%
\begin{isamarkuptext}%
En esta demostración hemos introducido:
 \begin{itemize}
    \item[] \isa{{\isacharparenleft}f\ {\isacharequal}\ g{\isacharparenright}\ {\isacharequal}\ {\isacharparenleft}{\isasymforall}x{\isachardot}\ f\ x\ {\isacharequal}\ g\ x{\isacharparenright}} 
      \hfill (\isa{fun{\isacharunderscore}eq{\isacharunderscore}iff})
  \end{itemize}%
\end{isamarkuptext}\isamarkuptrue%
%
\isadelimtheory
%
\endisadelimtheory
%
\isatagtheory
%
\endisatagtheory
{\isafoldtheory}%
%
\isadelimtheory
%
\endisadelimtheory
%
\end{isabellebody}%
\endinput
%:%file=~/Escritorio/TFG/EjerciciosDELMF/CancelacionSobreyectiva.thy%:%
%:%24=7%:%
%:%36=10%:%
%:%37=11%:%
%:%38=12%:%
%:%39=13%:%
%:%40=14%:%
%:%41=15%:%
%:%42=16%:%
%:%43=17%:%
%:%44=18%:%
%:%45=19%:%
%:%46=20%:%
%:%47=21%:%
%:%48=22%:%
%:%49=23%:%
%:%50=24%:%
%:%51=25%:%
%:%52=26%:%
%:%53=27%:%
%:%54=28%:%
%:%55=29%:%
%:%56=30%:%
%:%57=31%:%
%:%59=34%:%
%:%60=34%:%
%:%63=35%:%
%:%67=35%:%
%:%77=38%:%
%:%78=39%:%
%:%79=40%:%
%:%80=41%:%
%:%81=42%:%
%:%82=43%:%
%:%83=44%:%
%:%84=45%:%
%:%85=46%:%
%:%86=47%:%
%:%87=48%:%
%:%88=49%:%
%:%89=50%:%
%:%90=51%:%
%:%91=52%:%
%:%92=53%:%
%:%93=54%:%
%:%94=55%:%
%:%95=56%:%
%:%96=57%:%
%:%97=58%:%
%:%99=61%:%
%:%100=61%:%
%:%101=62%:%
%:%102=63%:%
%:%109=64%:%
%:%110=64%:%
%:%111=65%:%
%:%112=65%:%
%:%113=66%:%
%:%114=66%:%
%:%115=67%:%
%:%116=67%:%
%:%117=68%:%
%:%118=68%:%
%:%119=69%:%
%:%120=70%:%
%:%121=70%:%
%:%122=70%:%
%:%123=70%:%
%:%124=71%:%
%:%125=71%:%
%:%126=71%:%
%:%127=71%:%
%:%128=72%:%
%:%129=72%:%
%:%130=72%:%
%:%131=72%:%
%:%132=73%:%
%:%133=73%:%
%:%134=73%:%
%:%135=73%:%
%:%136=74%:%
%:%137=74%:%
%:%138=74%:%
%:%139=74%:%
%:%140=74%:%
%:%141=75%:%
%:%142=75%:%
%:%143=75%:%
%:%144=75%:%
%:%145=76%:%
%:%146=76%:%
%:%147=76%:%
%:%148=76%:%
%:%149=76%:%
%:%150=77%:%
%:%151=77%:%
%:%152=77%:%
%:%153=77%:%
%:%154=78%:%
%:%155=78%:%
%:%156=79%:%
%:%157=79%:%
%:%158=80%:%
%:%159=80%:%
%:%160=81%:%
%:%161=81%:%
%:%162=82%:%
%:%163=82%:%
%:%164=83%:%
%:%165=83%:%
%:%166=84%:%
%:%167=84%:%
%:%168=84%:%
%:%169=85%:%
%:%170=85%:%
%:%171=85%:%
%:%172=85%:%
%:%173=85%:%
%:%174=86%:%
%:%175=86%:%
%:%176=86%:%
%:%177=86%:%
%:%178=87%:%
%:%179=87%:%
%:%180=87%:%
%:%181=87%:%
%:%182=88%:%
%:%183=88%:%
%:%184=89%:%
%:%194=92%:%
%:%195=93%:%
%:%196=94%:%
%:%197=95%:%
%:%198=96%:%
%:%199=97%:%
%:%200=98%:%
%:%201=99%:%
%:%202=100%:%
%:%203=101%:%
%:%204=102%:%
%:%206=104%:%
%:%207=104%:%
%:%210=105%:%
%:%214=105%:%
%:%215=105%:%
%:%216=106%:%
%:%217=106%:%
%:%218=107%:%
%:%219=107%:%
%:%220=108%:%
%:%221=108%:%
%:%222=109%:%
%:%223=110%:%
%:%224=110%:%
%:%225=111%:%
%:%226=112%:%
%:%232=112%:%
%:%235=113%:%
%:%236=114%:%
%:%237=114%:%
%:%240=115%:%
%:%244=115%:%
%:%245=115%:%
%:%246=116%:%
%:%247=116%:%
%:%256=119%:%
%:%257=120%:%
%:%258=121%:%
%:%259=122%:%
%:%260=123%:%