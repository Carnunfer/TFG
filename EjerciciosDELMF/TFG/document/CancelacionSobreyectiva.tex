%
\begin{isabellebody}%
\setisabellecontext{CancelacionSobreyectiva}%
%
\isadelimtheory
%
\endisadelimtheory
%
\isatagtheory
%
\endisatagtheory
{\isafoldtheory}%
%
\isadelimtheory
%
\endisadelimtheory
%
\isadelimdocument
%
\endisadelimdocument
%
\isatagdocument
%
\isamarkupsection{Cancelación de las funciones sobreyectivas%
}
\isamarkuptrue%
%
\endisatagdocument
{\isafolddocument}%
%
\isadelimdocument
%
\endisadelimdocument
%
\begin{isamarkuptext}%
El siguiente teorema prueba una caracterización de las funciones
 sobreyectivas, en otras palabras, las funciones sobreyectivas son
 epimorfismos en la categoría de conjuntos. Donde un epimorfismo es un
 homomorfismo sobreyectivo y la categoría de conjuntos es la categoría
 donde los objetos son conjuntos.


\begin {teorema}
  f es sobreyectiva si y solo si  para todas funciones g y h tal que g o f
 = h o f se tiene que g = h.
\end {teorema}
 
El teorema lo podemos dividir en dos lemas, ya que el teorema se
 demuestra por una doble implicación, luego vamos a dividir el teorema
 en las dos implicaciones.

\begin {lema}
  f es sobreyectiva entonces  para todas funciones g y h tal que g o f
 = h o f se tiene que g = h.
\end {lema}
\begin {demostracion}
\begin {itemize}
\item Supongamos que tenemos que $g \circ  f = h \circ f$, queremos probar que $g =
 h.$ Usando la definición de sobreyectividad $(\forall y \in Y,
 \exists x | y = f(x))$ y nuestra hipótesis, tenemos que:
$$g(y) = g(f(x)) = (g o f) (x) = (h o f) (x) = h(f(x)) = h(y)$$
\item Supongamos que $g = h$, hay que probar que $g o f = h o f.$ Usando
nuestra hipótesis, tenemos que:
$$ (g o f)(x) = g(f(x)) = h(f(x)) = (h o f) (x).$$
\end {itemize}
.
\end {demostracion}

\begin {lema}
 Si  para todas funciones g y h tal que g o f  = h o f se tiene
 que g = h entonces f es sobreyectiva.
\end {lema}


Su especificación es la siguiente, que la dividiremos en dos al igual que 
en la demostración a mano:%
\end{isamarkuptext}\isamarkuptrue%
\isacommand{theorem}\isamarkupfalse%
\isanewline
\ {\isachardoublequoteopen}surj\ f\ {\isasymlongleftrightarrow}\ {\isacharparenleft}g\ {\isasymcirc}\ f\ {\isacharequal}\ h\ {\isasymcirc}\ f{\isacharparenright}\ {\isacharequal}\ {\isacharparenleft}g\ {\isacharequal}\ h{\isacharparenright}{\isachardoublequoteclose}\isanewline
%
\isadelimproof
\ \ %
\endisadelimproof
%
\isatagproof
\isacommand{oops}\isamarkupfalse%
%
\endisatagproof
{\isafoldproof}%
%
\isadelimproof
\isanewline
%
\endisadelimproof
\isanewline
\isacommand{lemma}\isamarkupfalse%
\ \isanewline
{\isachardoublequoteopen}surj\ f\ {\isasymLongrightarrow}\ \ {\isacharparenleft}g\ {\isasymcirc}\ f\ {\isacharequal}\ h\ {\isasymcirc}\ f{\isacharparenright}\ {\isacharequal}\ {\isacharparenleft}g\ {\isacharequal}\ h{\isacharparenright}{\isachardoublequoteclose}\isanewline
%
\isadelimproof
\ \ %
\endisadelimproof
%
\isatagproof
\isacommand{oops}\isamarkupfalse%
%
\endisatagproof
{\isafoldproof}%
%
\isadelimproof
\isanewline
%
\endisadelimproof
\isanewline
\isacommand{lemma}\isamarkupfalse%
\ \isanewline
{\isachardoublequoteopen}{\isasymforall}g\ h{\isachardot}\ {\isacharparenleft}g\ {\isasymcirc}\ f\ {\isacharequal}\ h\ {\isasymcirc}\ f\ {\isasymlongrightarrow}\ g\ {\isacharequal}\ h{\isacharparenright}\ {\isasymlongrightarrow}\ surj\ f{\isachardoublequoteclose}\isanewline
%
\isadelimproof
\ \ %
\endisadelimproof
%
\isatagproof
\isacommand{oops}\isamarkupfalse%
%
\endisatagproof
{\isafoldproof}%
%
\isadelimproof
%
\endisadelimproof
%
\begin{isamarkuptext}%
En la especificación anterior, \isa{surj\ f} es una abreviatura de 
  \isa{range\ f\ {\isacharequal}\ UNIV}, donde \isa{range\ f} es el rango o imagen
de la función f.
 Por otra parte, \isa{UNIV} es el conjunto universal definido en la 
  teoría \href{http://bit.ly/2XtHCW6}{Set.thy} como una abreviatura de 
  \isa{top} que, a su vez está definido en la teoría 
  \href{http://bit.ly/2Xyj9Pe}{Orderings.thy} mediante la siguiente
  propiedad 
  \begin{itemize}
    \item[] \isa{\mbox{}\inferrule{\mbox{ordering{\isacharunderscore}top\ less{\isacharunderscore}eq\ less\ top}}{\mbox{less{\isacharunderscore}eq\ a\ top}}} 
      \hfill (\isa{ordering{\isacharunderscore}top{\isachardot}extremum})
  \end{itemize} 
Además queda añadir que la teoría donde se encuentra definido \isa{surj\ f}
 es en \href{http://bit.ly/2XuPQx5}{Fun.thy}. Esta teoría contiene la
 definicion \isa{surj{\isacharunderscore}def}.
 \begin{itemize}
    \item[] \isa{surj\ f\ {\isacharequal}\ {\isacharparenleft}{\isasymforall}y{\isachardot}\ {\isasymexists}x{\isachardot}\ y\ {\isacharequal}\ f\ x{\isacharparenright}} \hfill (\isa{inj{\isacharunderscore}on{\isacharunderscore}def})
  \end{itemize} 

Presentaremos distintas demostraciones del teorema. La primera es la
 detallada:%
\end{isamarkuptext}\isamarkuptrue%
\isacommand{lemma}\isamarkupfalse%
\ \isanewline
\ \ \isakeyword{assumes}\ {\isachardoublequoteopen}surj\ f{\isachardoublequoteclose}\ \isanewline
\ \ \isakeyword{shows}\ {\isachardoublequoteopen}{\isacharparenleft}\ g\ {\isasymcirc}\ f\ {\isacharequal}\ h\ {\isasymcirc}\ f\ {\isacharparenright}\ {\isacharequal}\ {\isacharparenleft}g\ {\isacharequal}\ h{\isacharparenright}{\isachardoublequoteclose}\isanewline
%
\isadelimproof
%
\endisadelimproof
%
\isatagproof
\isacommand{proof}\isamarkupfalse%
\ {\isacharparenleft}rule\ iffI{\isacharparenright}\isanewline
\ \ \isacommand{assume}\isamarkupfalse%
\ {\isadigit{1}}{\isacharcolon}\ {\isachardoublequoteopen}\ g\ {\isasymcirc}\ f\ {\isacharequal}\ h\ {\isasymcirc}\ f\ {\isachardoublequoteclose}\isanewline
\ \ \isacommand{show}\isamarkupfalse%
\ {\isachardoublequoteopen}g\ {\isacharequal}\ h{\isachardoublequoteclose}\ \isanewline
\ \ \isacommand{proof}\isamarkupfalse%
\ \isanewline
\ \ \ \ \isacommand{fix}\isamarkupfalse%
\ x\isanewline
\isanewline
\ \ \ \ \isacommand{have}\isamarkupfalse%
\ {\isachardoublequoteopen}\ {\isasymexists}y\ {\isachardot}\ x\ {\isacharequal}\ f{\isacharparenleft}y{\isacharparenright}{\isachardoublequoteclose}\ \isacommand{using}\isamarkupfalse%
\ assms\ \isacommand{by}\isamarkupfalse%
\ {\isacharparenleft}simp\ add{\isacharcolon}surj{\isacharunderscore}def{\isacharparenright}\isanewline
\ \ \ \ \isacommand{then}\isamarkupfalse%
\ \isacommand{obtain}\isamarkupfalse%
\ {\isachardoublequoteopen}y{\isachardoublequoteclose}\ \isakeyword{where}\ {\isadigit{2}}{\isacharcolon}{\isachardoublequoteopen}x\ {\isacharequal}\ f{\isacharparenleft}y{\isacharparenright}{\isachardoublequoteclose}\ \isacommand{by}\isamarkupfalse%
\ {\isacharparenleft}rule\ exE{\isacharparenright}\isanewline
\ \ \ \ \isacommand{then}\isamarkupfalse%
\ \isacommand{have}\isamarkupfalse%
\ {\isachardoublequoteopen}g{\isacharparenleft}x{\isacharparenright}\ {\isacharequal}\ g{\isacharparenleft}f{\isacharparenleft}y{\isacharparenright}{\isacharparenright}{\isachardoublequoteclose}\ \isacommand{by}\isamarkupfalse%
\ simp\isanewline
\ \ \ \ \isacommand{also}\isamarkupfalse%
\ \isacommand{have}\isamarkupfalse%
\ {\isachardoublequoteopen}{\isachardot}{\isachardot}{\isachardot}\ {\isacharequal}\ {\isacharparenleft}g\ {\isasymcirc}\ f{\isacharparenright}\ {\isacharparenleft}y{\isacharparenright}\ \ {\isachardoublequoteclose}\ \isacommand{by}\isamarkupfalse%
\ simp\isanewline
\ \ \ \ \isacommand{also}\isamarkupfalse%
\ \isacommand{have}\isamarkupfalse%
\ {\isachardoublequoteopen}{\isachardot}{\isachardot}{\isachardot}\ {\isacharequal}\ {\isacharparenleft}h\ {\isasymcirc}\ f{\isacharparenright}\ {\isacharparenleft}y{\isacharparenright}{\isachardoublequoteclose}\ \isacommand{using}\isamarkupfalse%
\ {\isadigit{1}}\ \isacommand{by}\isamarkupfalse%
\ simp\isanewline
\ \ \ \ \isacommand{also}\isamarkupfalse%
\ \isacommand{have}\isamarkupfalse%
\ {\isachardoublequoteopen}{\isachardot}{\isachardot}{\isachardot}\ {\isacharequal}\ h{\isacharparenleft}f{\isacharparenleft}y{\isacharparenright}{\isacharparenright}{\isachardoublequoteclose}\ \isacommand{by}\isamarkupfalse%
\ simp\isanewline
\ \ \ \ \isacommand{also}\isamarkupfalse%
\ \isacommand{have}\isamarkupfalse%
\ {\isachardoublequoteopen}{\isachardot}{\isachardot}{\isachardot}\ {\isacharequal}\ h{\isacharparenleft}x{\isacharparenright}{\isachardoublequoteclose}\ \isacommand{using}\isamarkupfalse%
\ {\isadigit{2}}\ \ \ \isacommand{by}\isamarkupfalse%
\ {\isacharparenleft}simp\ add{\isacharcolon}\ {\isacartoucheopen}x\ {\isacharequal}\ f\ y{\isacartoucheclose}{\isacharparenright}\isanewline
\ \ \ \ \isacommand{finally}\isamarkupfalse%
\ \isacommand{show}\isamarkupfalse%
\ \ {\isachardoublequoteopen}\ g{\isacharparenleft}x{\isacharparenright}\ {\isacharequal}\ h{\isacharparenleft}x{\isacharparenright}\ {\isachardoublequoteclose}\ \isacommand{by}\isamarkupfalse%
\ simp\isanewline
\ \ \isacommand{qed}\isamarkupfalse%
\isanewline
\isacommand{next}\isamarkupfalse%
\isanewline
\ \ \isacommand{assume}\isamarkupfalse%
\ {\isachardoublequoteopen}g\ {\isacharequal}\ h{\isachardoublequoteclose}\ \isanewline
\ \ \isacommand{show}\isamarkupfalse%
\ {\isachardoublequoteopen}g\ {\isasymcirc}\ f\ {\isacharequal}\ h\ {\isasymcirc}\ f{\isachardoublequoteclose}\isanewline
\ \ \isacommand{proof}\isamarkupfalse%
\isanewline
\ \ \ \ \isacommand{fix}\isamarkupfalse%
\ x\isanewline
\ \ \ \ \isacommand{have}\isamarkupfalse%
\ {\isachardoublequoteopen}{\isacharparenleft}g\ {\isasymcirc}\ f{\isacharparenright}\ x\ {\isacharequal}\ g{\isacharparenleft}f{\isacharparenleft}x{\isacharparenright}{\isacharparenright}{\isachardoublequoteclose}\ \isacommand{by}\isamarkupfalse%
\ simp\isanewline
\ \ \ \ \isacommand{also}\isamarkupfalse%
\ \isacommand{have}\isamarkupfalse%
\ {\isachardoublequoteopen}{\isasymdots}\ {\isacharequal}\ h{\isacharparenleft}f{\isacharparenleft}x{\isacharparenright}{\isacharparenright}{\isachardoublequoteclose}\ \isacommand{using}\isamarkupfalse%
\ {\isacharbackquoteopen}g\ {\isacharequal}\ h{\isacharbackquoteclose}\ \isacommand{by}\isamarkupfalse%
\ simp\isanewline
\ \ \ \ \isacommand{also}\isamarkupfalse%
\ \isacommand{have}\isamarkupfalse%
\ {\isachardoublequoteopen}{\isasymdots}\ {\isacharequal}\ {\isacharparenleft}h\ {\isasymcirc}\ f{\isacharparenright}\ x{\isachardoublequoteclose}\ \isacommand{by}\isamarkupfalse%
\ simp\isanewline
\ \ \ \ \isacommand{finally}\isamarkupfalse%
\ \isacommand{show}\isamarkupfalse%
\ {\isachardoublequoteopen}{\isacharparenleft}g\ {\isasymcirc}\ f{\isacharparenright}\ x\ {\isacharequal}\ {\isacharparenleft}h\ {\isasymcirc}\ f{\isacharparenright}\ x{\isachardoublequoteclose}\ \isacommand{by}\isamarkupfalse%
\ simp\isanewline
\ \ \isacommand{qed}\isamarkupfalse%
\isanewline
\isacommand{qed}\isamarkupfalse%
%
\endisatagproof
{\isafoldproof}%
%
\isadelimproof
%
\endisadelimproof
%
\begin{isamarkuptext}%
En la demostración hemos introducido: 
 \begin{itemize}
    \item[] \isa{\mbox{}\inferrule{\mbox{{\isasymexists}x{\isachardot}\ P\ x}\\\ \mbox{{\isasymAnd}x{\isachardot}\ \mbox{}\inferrule{\mbox{P\ x}}{\mbox{Q}}}}{\mbox{Q}}} 
      \hfill (\isa{rule\ exE}) 
  \end{itemize} 
 \begin{itemize}
    \item[] \isa{{\isasymlbrakk}P\ {\isasymLongrightarrow}\ Q{\isacharsemicolon}\ Q\ {\isasymLongrightarrow}\ P{\isasymrbrakk}\ {\isasymLongrightarrow}\ P\ {\isacharequal}\ Q} 
      \hfill (\isa{iffI})
  \end{itemize} 

La demostración aplicativa es:%
\end{isamarkuptext}\isamarkuptrue%
\isacommand{lemma}\isamarkupfalse%
\ {\isachardoublequoteopen}surj\ f\ {\isasymLongrightarrow}\ {\isacharparenleft}{\isacharparenleft}g\ {\isasymcirc}\ f{\isacharparenright}\ {\isacharequal}\ {\isacharparenleft}h\ {\isasymcirc}\ f{\isacharparenright}\ {\isacharparenright}\ {\isacharequal}\ {\isacharparenleft}g\ {\isacharequal}\ h{\isacharparenright}{\isachardoublequoteclose}\isanewline
%
\isadelimproof
\ \ %
\endisadelimproof
%
\isatagproof
\isacommand{apply}\isamarkupfalse%
\ {\isacharparenleft}simp\ add{\isacharcolon}\ surj{\isacharunderscore}def\ fun{\isacharunderscore}eq{\isacharunderscore}iff{\isacharparenright}\isanewline
\ \ \isacommand{apply}\isamarkupfalse%
\ {\isacharparenleft}rule\ iffI{\isacharparenright}\isanewline
\ \ \ \isacommand{prefer}\isamarkupfalse%
\ {\isadigit{2}}\isanewline
\ \ \isacommand{apply}\isamarkupfalse%
\ auto\isanewline
\ \isanewline
\ \ \isacommand{apply}\isamarkupfalse%
\ \ metis\isanewline
\isanewline
\ \ \isacommand{done}\isamarkupfalse%
%
\endisatagproof
{\isafoldproof}%
%
\isadelimproof
\isanewline
%
\endisadelimproof
\isanewline
\isacommand{lemma}\isamarkupfalse%
\ {\isachardoublequoteopen}surj\ f\ {\isasymLongrightarrow}\ {\isacharparenleft}{\isacharparenleft}g\ {\isasymcirc}\ f{\isacharparenright}\ {\isacharequal}\ {\isacharparenleft}h\ {\isasymcirc}\ f{\isacharparenright}\ {\isacharparenright}\ {\isacharequal}\ {\isacharparenleft}g\ {\isacharequal}\ h{\isacharparenright}{\isachardoublequoteclose}\isanewline
%
\isadelimproof
\ \ %
\endisadelimproof
%
\isatagproof
\isacommand{apply}\isamarkupfalse%
\ {\isacharparenleft}simp\ add{\isacharcolon}\ surj{\isacharunderscore}def\ fun{\isacharunderscore}eq{\isacharunderscore}iff\ {\isacharparenright}\ \isanewline
\ \ \isacommand{by}\isamarkupfalse%
\ metis%
\endisatagproof
{\isafoldproof}%
%
\isadelimproof
%
\endisadelimproof
%
\begin{isamarkuptext}%
En esta demostración hemos introducido:
 \begin{itemize}
    \item[] \isa{{\isacharparenleft}f\ {\isacharequal}\ g{\isacharparenright}\ {\isacharequal}\ {\isacharparenleft}{\isasymforall}x{\isachardot}\ f\ x\ {\isacharequal}\ g\ x{\isacharparenright}} 
      \hfill (\isa{fun{\isacharunderscore}eq{\isacharunderscore}iff})
  \end{itemize}%
\end{isamarkuptext}\isamarkuptrue%
%
\isadelimtheory
%
\endisadelimtheory
%
\isatagtheory
%
\endisatagtheory
{\isafoldtheory}%
%
\isadelimtheory
%
\endisadelimtheory
%
\end{isabellebody}%
\endinput
%:%file=~/Escritorio/TFG/EjerciciosDELMF/CancelacionSobreyectiva.thy%:%
%:%24=7%:%
%:%36=10%:%
%:%37=11%:%
%:%38=12%:%
%:%39=13%:%
%:%40=14%:%
%:%41=15%:%
%:%42=16%:%
%:%43=17%:%
%:%44=18%:%
%:%45=19%:%
%:%46=20%:%
%:%47=21%:%
%:%48=22%:%
%:%49=23%:%
%:%50=24%:%
%:%51=25%:%
%:%52=26%:%
%:%53=27%:%
%:%54=28%:%
%:%55=29%:%
%:%56=30%:%
%:%57=31%:%
%:%58=32%:%
%:%59=33%:%
%:%60=34%:%
%:%61=35%:%
%:%62=36%:%
%:%63=37%:%
%:%64=38%:%
%:%65=39%:%
%:%66=40%:%
%:%67=41%:%
%:%68=42%:%
%:%69=43%:%
%:%70=44%:%
%:%71=45%:%
%:%72=46%:%
%:%73=47%:%
%:%74=48%:%
%:%75=49%:%
%:%76=50%:%
%:%78=53%:%
%:%79=53%:%
%:%80=54%:%
%:%83=55%:%
%:%87=55%:%
%:%93=55%:%
%:%96=56%:%
%:%97=57%:%
%:%98=57%:%
%:%99=58%:%
%:%102=59%:%
%:%106=59%:%
%:%112=59%:%
%:%115=60%:%
%:%116=61%:%
%:%117=61%:%
%:%118=62%:%
%:%121=63%:%
%:%125=63%:%
%:%135=67%:%
%:%136=68%:%
%:%137=69%:%
%:%138=70%:%
%:%139=71%:%
%:%140=72%:%
%:%141=73%:%
%:%142=74%:%
%:%143=75%:%
%:%144=76%:%
%:%145=77%:%
%:%146=78%:%
%:%147=79%:%
%:%148=80%:%
%:%149=81%:%
%:%150=82%:%
%:%151=83%:%
%:%152=84%:%
%:%153=85%:%
%:%154=86%:%
%:%155=87%:%
%:%157=90%:%
%:%158=90%:%
%:%159=91%:%
%:%160=92%:%
%:%167=93%:%
%:%168=93%:%
%:%169=94%:%
%:%170=94%:%
%:%171=95%:%
%:%172=95%:%
%:%173=96%:%
%:%174=96%:%
%:%175=97%:%
%:%176=97%:%
%:%177=98%:%
%:%178=99%:%
%:%179=99%:%
%:%180=99%:%
%:%181=99%:%
%:%182=100%:%
%:%183=100%:%
%:%184=100%:%
%:%185=100%:%
%:%186=101%:%
%:%187=101%:%
%:%188=101%:%
%:%189=101%:%
%:%190=102%:%
%:%191=102%:%
%:%192=102%:%
%:%193=102%:%
%:%194=103%:%
%:%195=103%:%
%:%196=103%:%
%:%197=103%:%
%:%198=103%:%
%:%199=104%:%
%:%200=104%:%
%:%201=104%:%
%:%202=104%:%
%:%203=105%:%
%:%204=105%:%
%:%205=105%:%
%:%206=105%:%
%:%207=105%:%
%:%208=106%:%
%:%209=106%:%
%:%210=106%:%
%:%211=106%:%
%:%212=107%:%
%:%213=107%:%
%:%214=108%:%
%:%215=108%:%
%:%216=109%:%
%:%217=109%:%
%:%218=110%:%
%:%219=110%:%
%:%220=111%:%
%:%221=111%:%
%:%222=112%:%
%:%223=112%:%
%:%224=113%:%
%:%225=113%:%
%:%226=113%:%
%:%227=114%:%
%:%228=114%:%
%:%229=114%:%
%:%230=114%:%
%:%231=114%:%
%:%232=115%:%
%:%233=115%:%
%:%234=115%:%
%:%235=115%:%
%:%236=116%:%
%:%237=116%:%
%:%238=116%:%
%:%239=116%:%
%:%240=117%:%
%:%241=117%:%
%:%242=118%:%
%:%252=121%:%
%:%253=122%:%
%:%254=123%:%
%:%255=124%:%
%:%256=125%:%
%:%257=126%:%
%:%258=127%:%
%:%259=128%:%
%:%260=129%:%
%:%261=130%:%
%:%262=131%:%
%:%264=133%:%
%:%265=133%:%
%:%268=134%:%
%:%272=134%:%
%:%273=134%:%
%:%274=135%:%
%:%275=135%:%
%:%276=136%:%
%:%277=136%:%
%:%278=137%:%
%:%279=137%:%
%:%280=138%:%
%:%281=139%:%
%:%282=139%:%
%:%283=140%:%
%:%284=141%:%
%:%290=141%:%
%:%293=142%:%
%:%294=143%:%
%:%295=143%:%
%:%298=144%:%
%:%302=144%:%
%:%303=144%:%
%:%304=145%:%
%:%305=145%:%
%:%314=148%:%
%:%315=149%:%
%:%316=150%:%
%:%317=151%:%
%:%318=152%:%