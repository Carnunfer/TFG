%
\begin{isabellebody}%
\setisabellecontext{SumaImpares}%
%
\isadelimtheory
%
\endisadelimtheory
%
\isatagtheory
%
\endisatagtheory
{\isafoldtheory}%
%
\isadelimtheory
%
\endisadelimtheory
%
\isadelimdocument
%
\endisadelimdocument
%
\isatagdocument
%
\isamarkupsection{Suma de los primeros números impares%
}
\isamarkuptrue%
%
\endisatagdocument
{\isafolddocument}%
%
\isadelimdocument
%
\endisadelimdocument
%
\begin{isamarkuptext}%
El primer teorema es una propiedad de los números naturales.

  \begin{teorema}
    La suma de los $n$ primeros números impares es $n^2$.
  \end{teorema}

  \begin{demostracion}
    La demostración la haremos en inducción sobre $n$.
\begin {itemize}
\item EL caso $n = 0$ es trivial, ya que $0 = 0$.
\item Supongamos que se verifica la hipótesis para $n$ y veamos para
 $n+1$. \\
Tenemos que demostrar que $\sum_{j=1}^{n+1} k_j = (n+1)^2$ siendo los
 $k_{j}$ el j-ésimo impar, es decir, $k_{j} = 2j - 1$.
$$\sum_{j = 1}^{n+1} k_{j} = k_{n+1} + \sum^{n}_{j=1} k_{j} = k_{n+1} +
 n^{2} = 2(n+1) - 1 + n^2 = n^2 + 2n + 1 = (n+1)^2$$ 
\end {itemize}
.
  \end{demostracion}

  Para especificar el teorema en Isabelle, se comienza definiendo 
  la función \isa{suma{\isacharunderscore}impares} tal que \isa{suma{\isacharunderscore}impares\ n} es la 
  suma de los $n$ primeros números impares%
\end{isamarkuptext}\isamarkuptrue%
\isacommand{fun}\isamarkupfalse%
\ suma{\isacharunderscore}impares\ {\isacharcolon}{\isacharcolon}\ {\isachardoublequoteopen}nat\ {\isasymRightarrow}\ nat{\isachardoublequoteclose}\ \isakeyword{where}\isanewline
\ \ {\isachardoublequoteopen}suma{\isacharunderscore}impares\ {\isadigit{0}}\ {\isacharequal}\ {\isadigit{0}}{\isachardoublequoteclose}\ \isanewline
{\isacharbar}\ {\isachardoublequoteopen}suma{\isacharunderscore}impares\ {\isacharparenleft}Suc\ n{\isacharparenright}\ {\isacharequal}\ {\isacharparenleft}{\isadigit{2}}{\isacharasterisk}{\isacharparenleft}Suc\ n{\isacharparenright}\ {\isacharminus}\ {\isadigit{1}}{\isacharparenright}\ {\isacharplus}\ suma{\isacharunderscore}impares\ n{\isachardoublequoteclose}%
\begin{isamarkuptext}%
El enunciado del teorema es el siguiente:%
\end{isamarkuptext}\isamarkuptrue%
\isacommand{lemma}\isamarkupfalse%
\ {\isachardoublequoteopen}suma{\isacharunderscore}impares\ n\ {\isacharequal}\ n\ {\isacharasterisk}\ n{\isachardoublequoteclose}\isanewline
%
\isadelimproof
%
\endisadelimproof
%
\isatagproof
\isacommand{oops}\isamarkupfalse%
%
\endisatagproof
{\isafoldproof}%
%
\isadelimproof
%
\endisadelimproof
%
\begin{isamarkuptext}%
En la demostración se usará la táctica \isa{induct} que hace
  uso del esquema de inducción sobre los naturales:
  \begin{itemize}
  \item[] \isa{\mbox{}\inferrule{\mbox{P\ {\isadigit{0}}}\\\ \mbox{{\isasymAnd}nat{\isachardot}\ \mbox{}\inferrule{\mbox{P\ nat}}{\mbox{P\ {\isacharparenleft}Suc\ nat{\isacharparenright}}}}}{\mbox{P\ nat}}} \hfill (\isa{nat{\isachardot}induct})
  \end{itemize}

  Vamos a presentar distintas demostraciones del teorema. La 
  primera es la demostración aplicativa%
\end{isamarkuptext}\isamarkuptrue%
\isacommand{lemma}\isamarkupfalse%
\ {\isachardoublequoteopen}suma{\isacharunderscore}impares\ n\ {\isacharequal}\ n\ {\isacharasterisk}\ n{\isachardoublequoteclose}\isanewline
%
\isadelimproof
\ \ %
\endisadelimproof
%
\isatagproof
\isacommand{apply}\isamarkupfalse%
\ {\isacharparenleft}induct\ n{\isacharparenright}\ \isanewline
\ \ \ \isacommand{apply}\isamarkupfalse%
\ simp{\isacharunderscore}all\isanewline
\ \ \isacommand{done}\isamarkupfalse%
%
\endisatagproof
{\isafoldproof}%
%
\isadelimproof
%
\endisadelimproof
%
\begin{isamarkuptext}%
La demostración automática es%
\end{isamarkuptext}\isamarkuptrue%
\isacommand{lemma}\isamarkupfalse%
\ {\isachardoublequoteopen}suma{\isacharunderscore}impares\ n\ {\isacharequal}\ n\ {\isacharasterisk}\ n{\isachardoublequoteclose}\isanewline
%
\isadelimproof
\ \ %
\endisadelimproof
%
\isatagproof
\isacommand{by}\isamarkupfalse%
\ {\isacharparenleft}induct\ n{\isacharparenright}\ simp{\isacharunderscore}all%
\endisatagproof
{\isafoldproof}%
%
\isadelimproof
%
\endisadelimproof
%
\begin{isamarkuptext}%
La demostración del lema anterior por inducción y razonamiento 
   ecuacional es%
\end{isamarkuptext}\isamarkuptrue%
\isacommand{lemma}\isamarkupfalse%
\ {\isachardoublequoteopen}suma{\isacharunderscore}impares\ n\ {\isacharequal}\ n\ {\isacharasterisk}\ n{\isachardoublequoteclose}\isanewline
%
\isadelimproof
%
\endisadelimproof
%
\isatagproof
\isacommand{proof}\isamarkupfalse%
\ {\isacharparenleft}induct\ n{\isacharparenright}\isanewline
\ \ \isacommand{show}\isamarkupfalse%
\ {\isachardoublequoteopen}suma{\isacharunderscore}impares\ {\isadigit{0}}\ {\isacharequal}\ {\isadigit{0}}\ {\isacharasterisk}\ {\isadigit{0}}{\isachardoublequoteclose}\ \isacommand{by}\isamarkupfalse%
\ simp\isanewline
\isacommand{next}\isamarkupfalse%
\isanewline
\ \ \isacommand{fix}\isamarkupfalse%
\ n\ \isacommand{assume}\isamarkupfalse%
\ HI{\isacharcolon}\ {\isachardoublequoteopen}suma{\isacharunderscore}impares\ n\ {\isacharequal}\ n\ {\isacharasterisk}\ n{\isachardoublequoteclose}\isanewline
\ \ \isacommand{have}\isamarkupfalse%
\ {\isachardoublequoteopen}suma{\isacharunderscore}impares\ {\isacharparenleft}Suc\ n{\isacharparenright}\ {\isacharequal}\ {\isacharparenleft}{\isadigit{2}}\ {\isacharasterisk}\ {\isacharparenleft}Suc\ n{\isacharparenright}\ {\isacharminus}\ {\isadigit{1}}{\isacharparenright}\ {\isacharplus}\ suma{\isacharunderscore}impares\ n{\isachardoublequoteclose}\ \isanewline
\ \ \ \ \isacommand{by}\isamarkupfalse%
\ simp\isanewline
\ \ \isacommand{also}\isamarkupfalse%
\ \isacommand{have}\isamarkupfalse%
\ {\isachardoublequoteopen}{\isasymdots}\ {\isacharequal}\ {\isacharparenleft}{\isadigit{2}}\ {\isacharasterisk}\ {\isacharparenleft}Suc\ n{\isacharparenright}\ {\isacharminus}\ {\isadigit{1}}{\isacharparenright}\ {\isacharplus}\ n\ {\isacharasterisk}\ n{\isachardoublequoteclose}\ \isacommand{using}\isamarkupfalse%
\ HI\ \isacommand{by}\isamarkupfalse%
\ simp\isanewline
\ \ \isacommand{also}\isamarkupfalse%
\ \isacommand{have}\isamarkupfalse%
\ {\isachardoublequoteopen}{\isasymdots}\ {\isacharequal}\ n\ {\isacharasterisk}\ n\ {\isacharplus}\ {\isadigit{2}}\ {\isacharasterisk}\ n\ {\isacharplus}\ {\isadigit{1}}{\isachardoublequoteclose}\ \isacommand{by}\isamarkupfalse%
\ simp\isanewline
\ \ \isacommand{finally}\isamarkupfalse%
\ \isacommand{show}\isamarkupfalse%
\ {\isachardoublequoteopen}suma{\isacharunderscore}impares\ {\isacharparenleft}Suc\ n{\isacharparenright}\ {\isacharequal}\ {\isacharparenleft}Suc\ n{\isacharparenright}\ {\isacharasterisk}\ {\isacharparenleft}Suc\ n{\isacharparenright}{\isachardoublequoteclose}\ \isacommand{by}\isamarkupfalse%
\ simp\isanewline
\isacommand{qed}\isamarkupfalse%
%
\endisatagproof
{\isafoldproof}%
%
\isadelimproof
%
\endisadelimproof
%
\begin{isamarkuptext}%
La demostración del lema anterior con patrones y razonamiento 
   ecuacional es%
\end{isamarkuptext}\isamarkuptrue%
\isacommand{lemma}\isamarkupfalse%
\ {\isachardoublequoteopen}suma{\isacharunderscore}impares\ n\ {\isacharequal}\ n\ {\isacharasterisk}\ n{\isachardoublequoteclose}\ {\isacharparenleft}\isakeyword{is}\ {\isachardoublequoteopen}{\isacharquery}P\ n{\isachardoublequoteclose}{\isacharparenright}\isanewline
%
\isadelimproof
%
\endisadelimproof
%
\isatagproof
\isacommand{proof}\isamarkupfalse%
\ {\isacharparenleft}induct\ n{\isacharparenright}\isanewline
\ \ \isacommand{show}\isamarkupfalse%
\ {\isachardoublequoteopen}{\isacharquery}P\ {\isadigit{0}}{\isachardoublequoteclose}\ \isacommand{by}\isamarkupfalse%
\ simp\isanewline
\isacommand{next}\isamarkupfalse%
\isanewline
\ \ \isacommand{fix}\isamarkupfalse%
\ n\ \isanewline
\ \ \isacommand{assume}\isamarkupfalse%
\ HI{\isacharcolon}\ {\isachardoublequoteopen}{\isacharquery}P\ n{\isachardoublequoteclose}\isanewline
\ \ \isacommand{have}\isamarkupfalse%
\ {\isachardoublequoteopen}suma{\isacharunderscore}impares\ {\isacharparenleft}Suc\ n{\isacharparenright}\ {\isacharequal}\ {\isacharparenleft}{\isadigit{2}}\ {\isacharasterisk}\ {\isacharparenleft}Suc\ n{\isacharparenright}\ {\isacharminus}\ {\isadigit{1}}{\isacharparenright}\ {\isacharplus}\ suma{\isacharunderscore}impares\ n{\isachardoublequoteclose}\ \isanewline
\ \ \ \ \isacommand{by}\isamarkupfalse%
\ simp\isanewline
\ \ \isacommand{also}\isamarkupfalse%
\ \isacommand{have}\isamarkupfalse%
\ {\isachardoublequoteopen}{\isasymdots}\ {\isacharequal}\ {\isacharparenleft}{\isadigit{2}}\ {\isacharasterisk}\ {\isacharparenleft}Suc\ n{\isacharparenright}\ {\isacharminus}\ {\isadigit{1}}{\isacharparenright}\ {\isacharplus}\ n\ {\isacharasterisk}\ n{\isachardoublequoteclose}\ \isacommand{using}\isamarkupfalse%
\ HI\ \isacommand{by}\isamarkupfalse%
\ simp\isanewline
\ \ \isacommand{also}\isamarkupfalse%
\ \isacommand{have}\isamarkupfalse%
\ {\isachardoublequoteopen}{\isasymdots}\ {\isacharequal}\ n\ {\isacharasterisk}\ n\ {\isacharplus}\ {\isadigit{2}}\ {\isacharasterisk}\ n\ {\isacharplus}\ {\isadigit{1}}{\isachardoublequoteclose}\ \isacommand{by}\isamarkupfalse%
\ simp\isanewline
\ \ \isacommand{finally}\isamarkupfalse%
\ \isacommand{show}\isamarkupfalse%
\ {\isachardoublequoteopen}{\isacharquery}P\ {\isacharparenleft}Suc\ n{\isacharparenright}{\isachardoublequoteclose}\ \isacommand{by}\isamarkupfalse%
\ simp\isanewline
\isacommand{qed}\isamarkupfalse%
%
\endisatagproof
{\isafoldproof}%
%
\isadelimproof
%
\endisadelimproof
%
\begin{isamarkuptext}%
La demostración usando patrones es%
\end{isamarkuptext}\isamarkuptrue%
\isacommand{lemma}\isamarkupfalse%
\ {\isachardoublequoteopen}suma{\isacharunderscore}impares\ n\ {\isacharequal}\ n\ {\isacharasterisk}\ n{\isachardoublequoteclose}\ {\isacharparenleft}\isakeyword{is}\ {\isachardoublequoteopen}{\isacharquery}P\ n{\isachardoublequoteclose}{\isacharparenright}\isanewline
%
\isadelimproof
%
\endisadelimproof
%
\isatagproof
\isacommand{proof}\isamarkupfalse%
\ {\isacharparenleft}induct\ n{\isacharparenright}\isanewline
\ \ \isacommand{show}\isamarkupfalse%
\ {\isachardoublequoteopen}{\isacharquery}P\ {\isadigit{0}}{\isachardoublequoteclose}\ \isacommand{by}\isamarkupfalse%
\ simp\isanewline
\isacommand{next}\isamarkupfalse%
\isanewline
\ \ \isacommand{fix}\isamarkupfalse%
\ n\ \isanewline
\ \ \isacommand{assume}\isamarkupfalse%
\ {\isachardoublequoteopen}{\isacharquery}P\ n{\isachardoublequoteclose}\isanewline
\ \ \isacommand{then}\isamarkupfalse%
\ \isacommand{show}\isamarkupfalse%
\ {\isachardoublequoteopen}{\isacharquery}P\ {\isacharparenleft}Suc\ n{\isacharparenright}{\isachardoublequoteclose}\ \isacommand{by}\isamarkupfalse%
\ simp\isanewline
\isacommand{qed}\isamarkupfalse%
\isanewline
%
\endisatagproof
{\isafoldproof}%
%
\isadelimproof
%
\endisadelimproof
%
\isadelimtheory
%
\endisadelimtheory
%
\isatagtheory
%
\endisatagtheory
{\isafoldtheory}%
%
\isadelimtheory
%
\endisadelimtheory
%
\end{isabellebody}%
\endinput
%:%file=~/Escritorio/TFG/EjerciciosDELMF/SumaImpares.thy%:%
%:%24=8%:%
%:%36=10%:%
%:%37=11%:%
%:%38=12%:%
%:%39=13%:%
%:%40=14%:%
%:%41=15%:%
%:%42=16%:%
%:%43=17%:%
%:%44=18%:%
%:%45=19%:%
%:%46=20%:%
%:%47=21%:%
%:%48=22%:%
%:%49=23%:%
%:%50=24%:%
%:%51=25%:%
%:%52=26%:%
%:%53=27%:%
%:%54=28%:%
%:%55=29%:%
%:%56=30%:%
%:%57=31%:%
%:%58=32%:%
%:%60=35%:%
%:%61=35%:%
%:%62=36%:%
%:%63=37%:%
%:%65=39%:%
%:%67=41%:%
%:%68=41%:%
%:%75=42%:%
%:%85=44%:%
%:%86=45%:%
%:%87=46%:%
%:%88=47%:%
%:%89=48%:%
%:%90=49%:%
%:%91=50%:%
%:%92=51%:%
%:%94=56%:%
%:%95=56%:%
%:%98=57%:%
%:%102=57%:%
%:%103=57%:%
%:%104=58%:%
%:%105=58%:%
%:%106=59%:%
%:%116=61%:%
%:%118=63%:%
%:%119=63%:%
%:%122=64%:%
%:%126=64%:%
%:%127=64%:%
%:%136=66%:%
%:%137=67%:%
%:%139=69%:%
%:%140=69%:%
%:%147=70%:%
%:%148=70%:%
%:%149=71%:%
%:%150=71%:%
%:%151=71%:%
%:%152=72%:%
%:%153=72%:%
%:%154=73%:%
%:%155=73%:%
%:%156=73%:%
%:%157=74%:%
%:%158=74%:%
%:%159=75%:%
%:%160=75%:%
%:%161=76%:%
%:%162=76%:%
%:%163=76%:%
%:%164=76%:%
%:%165=76%:%
%:%166=77%:%
%:%167=77%:%
%:%168=77%:%
%:%169=77%:%
%:%170=78%:%
%:%171=78%:%
%:%172=78%:%
%:%173=78%:%
%:%174=79%:%
%:%184=81%:%
%:%185=82%:%
%:%187=83%:%
%:%188=83%:%
%:%195=84%:%
%:%196=84%:%
%:%197=85%:%
%:%198=85%:%
%:%199=85%:%
%:%200=86%:%
%:%201=86%:%
%:%202=87%:%
%:%203=87%:%
%:%204=88%:%
%:%205=88%:%
%:%206=89%:%
%:%207=89%:%
%:%208=90%:%
%:%209=90%:%
%:%210=91%:%
%:%211=91%:%
%:%212=91%:%
%:%213=91%:%
%:%214=91%:%
%:%215=92%:%
%:%216=92%:%
%:%217=92%:%
%:%218=92%:%
%:%219=93%:%
%:%220=93%:%
%:%221=93%:%
%:%222=93%:%
%:%223=94%:%
%:%233=96%:%
%:%235=98%:%
%:%236=98%:%
%:%243=99%:%
%:%244=99%:%
%:%245=100%:%
%:%246=100%:%
%:%247=100%:%
%:%248=101%:%
%:%249=101%:%
%:%250=102%:%
%:%251=102%:%
%:%252=103%:%
%:%253=103%:%
%:%254=104%:%
%:%255=104%:%
%:%256=104%:%
%:%257=104%:%
%:%258=105%:%
%:%259=105%:%