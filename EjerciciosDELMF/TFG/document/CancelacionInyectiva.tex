%
\begin{isabellebody}%
\setisabellecontext{CancelacionInyectiva}%
%
\isadelimtheory
%
\endisadelimtheory
%
\isatagtheory
%
\endisatagtheory
{\isafoldtheory}%
%
\isadelimtheory
%
\endisadelimtheory
%
\isadelimdocument
%
\endisadelimdocument
%
\isatagdocument
%
\isamarkupsection{Cancelación de funciones inyectivas%
}
\isamarkuptrue%
%
\endisatagdocument
{\isafolddocument}%
%
\isadelimdocument
%
\endisadelimdocument
%
\begin{isamarkuptext}%
El siguiente teorema prueba una propiedad de las funciones
  inyectivas estudiado en el \href{http://bit.ly/2XBW6n2}{tema 10} del 
  curso. Su enunciado es el siguiente
  
  \begin{teorema}
    Las funciones inyectivas son cancelativas por la izquierda. Es
    decir, si $f$ es una función inyectiva entonces para todas $g$ y $h$
    tales que \isa{f\ {\isasymcirc}\ g\ {\isacharequal}\ f\ {\isasymcirc}\ h} se tiene que $g = h$. 
  \end{teorema}

  \begin{demostracion}
    La demostración la haremos por doble implicación: 
\begin {enumerate}
\item Supongamos que tenemos que $f \circ g = f \circ h$, queremos
 demostrar que $g = h$. \\
$$(f \circ g)(x) = (f \circ h)(x) \Longrightarrow f(g(x)) = f(h(x))
 \isa{Usando\ que\ f\ es\ inyectiva} \Longrightarrow g(x) = h(x)$$
\item Supongamos ahora que $g = h$, queremos demostrar que  $f \circ g
 = f \circ h$. \\
$$(f \circ g)(x) = f(g(x)) = f(h(x)) = (f \circ h)(x)$$
\end {enumerate}
.
  \end{demostracion}

  Su especificación es la siguiente:%
\end{isamarkuptext}\isamarkuptrue%
\isacommand{lemma}\isamarkupfalse%
\ \isanewline
\ \ {\isachardoublequoteopen}inj\ f\ {\isasymLongrightarrow}\ {\isacharparenleft}f\ {\isasymcirc}\ g\ {\isacharequal}\ f\ {\isasymcirc}\ h{\isacharparenright}\ {\isacharequal}\ {\isacharparenleft}g\ {\isacharequal}\ h{\isacharparenright}{\isachardoublequoteclose}\isanewline
%
\isadelimproof
%
\endisadelimproof
%
\isatagproof
\isacommand{oops}\isamarkupfalse%
%
\endisatagproof
{\isafoldproof}%
%
\isadelimproof
%
\endisadelimproof
%
\begin{isamarkuptext}%
En la especificación anterior, \isa{inj\ f} es una 
  abreviatura de \isa{inj\ f} definida en la teoría
  \href{http://bit.ly/2XuPQx5}{Fun.thy}. Además, contiene la definición
  de \isa{inj{\isacharunderscore}on}
  \begin{itemize}
    \item[] \isa{inj{\isacharunderscore}on\ f\ A\ {\isacharequal}\ {\isacharparenleft}{\isasymforall}x{\isasymin}A{\isachardot}\ {\isasymforall}y{\isasymin}A{\isachardot}\ f\ x\ {\isacharequal}\ f\ y\ {\isasymlongrightarrow}\ x\ {\isacharequal}\ y{\isacharparenright}} \hfill (\isa{inj{\isacharunderscore}on{\isacharunderscore}def})
  \end{itemize} 
  Por su parte, \isa{UNIV} es el conjunto universal definido en la 
  teoría \href{http://bit.ly/2XtHCW6}{Set.thy} como una abreviatura de 
  \isa{top} que, a su vez está definido en la teoría 
  \href{http://bit.ly/2Xyj9Pe}{Orderings.thy} mediante la siguiente
  propiedad 
  \begin{itemize}
    \item[] \isa{\mbox{}\inferrule{\mbox{ordering{\isacharunderscore}top\ less{\isacharunderscore}eq\ less\ top}}{\mbox{less{\isacharunderscore}eq\ a\ top}}} 
      \hfill (\isa{ordering{\isacharunderscore}top{\isachardot}extremum})
  \end{itemize} 
  En el caso de la teoría de conjuntos, la relación de orden es la
  inclusión de conjuntos.

  Presentaremos distintas demostraciones del teorema. La primera
  demostración es applicativa%
\end{isamarkuptext}\isamarkuptrue%
\isacommand{lemma}\isamarkupfalse%
\ \isanewline
\ \ {\isachardoublequoteopen}inj\ f\ {\isasymLongrightarrow}\ {\isacharparenleft}f\ {\isasymcirc}\ g\ {\isacharequal}\ f\ {\isasymcirc}\ h{\isacharparenright}\ {\isacharequal}\ {\isacharparenleft}g\ {\isacharequal}\ h{\isacharparenright}{\isachardoublequoteclose}\isanewline
%
\isadelimproof
\ \ %
\endisadelimproof
%
\isatagproof
\isacommand{apply}\isamarkupfalse%
\ {\isacharparenleft}simp\ add{\isacharcolon}\ inj{\isacharunderscore}on{\isacharunderscore}def\ fun{\isacharunderscore}eq{\isacharunderscore}iff{\isacharparenright}\ \isanewline
\ \ \isacommand{apply}\isamarkupfalse%
\ auto\isanewline
\ \ \isacommand{done}\isamarkupfalse%
%
\endisatagproof
{\isafoldproof}%
%
\isadelimproof
%
\endisadelimproof
%
\begin{isamarkuptext}%
En la demostración anterior se ha usado el siguiente lema
  \begin{itemize}
    \item[] \isa{{\isacharparenleft}f\ {\isacharequal}\ g{\isacharparenright}\ {\isacharequal}\ {\isacharparenleft}{\isasymforall}x{\isachardot}\ f\ x\ {\isacharequal}\ g\ x{\isacharparenright}} 
      \hfill (\isa{fun{\isacharunderscore}eq{\isacharunderscore}iff})
  \end{itemize} 

  La demostración applicativa sin auto es%
\end{isamarkuptext}\isamarkuptrue%
\isacommand{lemma}\isamarkupfalse%
\isanewline
\ \ {\isachardoublequoteopen}inj\ f\ {\isasymLongrightarrow}\ {\isacharparenleft}f\ {\isasymcirc}\ g\ {\isacharequal}\ f\ {\isasymcirc}\ h{\isacharparenright}\ {\isacharequal}\ {\isacharparenleft}g\ {\isacharequal}\ h{\isacharparenright}{\isachardoublequoteclose}\isanewline
%
\isadelimproof
\ \ %
\endisadelimproof
%
\isatagproof
\isacommand{apply}\isamarkupfalse%
\ {\isacharparenleft}unfold\ inj{\isacharunderscore}on{\isacharunderscore}def{\isacharparenright}\ \isanewline
\ \ \isacommand{apply}\isamarkupfalse%
\ {\isacharparenleft}unfold\ fun{\isacharunderscore}eq{\isacharunderscore}iff{\isacharparenright}\ \isanewline
\ \ \isacommand{apply}\isamarkupfalse%
\ {\isacharparenleft}unfold\ o{\isacharunderscore}apply{\isacharparenright}\isanewline
\ \ \isacommand{apply}\isamarkupfalse%
\ {\isacharparenleft}rule\ iffI{\isacharparenright}\isanewline
\ \ \ \isacommand{apply}\isamarkupfalse%
\ simp{\isacharplus}\isanewline
\ \ \isacommand{done}\isamarkupfalse%
%
\endisatagproof
{\isafoldproof}%
%
\isadelimproof
%
\endisadelimproof
%
\begin{isamarkuptext}%
En la demostración anterior se ha introducido los siguientes
  hechos
  \begin{itemize}
    \item \isa{{\isacharparenleft}f\ {\isasymcirc}\ g{\isacharparenright}\ x\ {\isacharequal}\ f\ {\isacharparenleft}g\ x{\isacharparenright}} \hfill (\isa{o{\isacharunderscore}apply})
    \item \isa{{\isasymlbrakk}P\ {\isasymLongrightarrow}\ Q{\isacharsemicolon}\ Q\ {\isasymLongrightarrow}\ P{\isasymrbrakk}\ {\isasymLongrightarrow}\ P\ {\isacharequal}\ Q} \hfill (\isa{iffI})
  \end{itemize} 

  La demostración automática es%
\end{isamarkuptext}\isamarkuptrue%
\isacommand{lemma}\isamarkupfalse%
\isanewline
\ \ \isakeyword{assumes}\ {\isachardoublequoteopen}inj\ f{\isachardoublequoteclose}\isanewline
\ \ \isakeyword{shows}\ {\isachardoublequoteopen}{\isacharparenleft}f\ {\isasymcirc}\ g\ {\isacharequal}\ f\ {\isasymcirc}\ h{\isacharparenright}\ {\isacharequal}\ {\isacharparenleft}g\ {\isacharequal}\ h{\isacharparenright}{\isachardoublequoteclose}\isanewline
%
\isadelimproof
\ \ %
\endisadelimproof
%
\isatagproof
\isacommand{using}\isamarkupfalse%
\ assms\isanewline
\ \ \isacommand{by}\isamarkupfalse%
\ {\isacharparenleft}auto\ simp\ add{\isacharcolon}\ inj{\isacharunderscore}on{\isacharunderscore}def\ fun{\isacharunderscore}eq{\isacharunderscore}iff{\isacharparenright}%
\endisatagproof
{\isafoldproof}%
%
\isadelimproof
%
\endisadelimproof
%
\begin{isamarkuptext}%
La demostración declarativa%
\end{isamarkuptext}\isamarkuptrue%
\isacommand{lemma}\isamarkupfalse%
\isanewline
\ \ \isakeyword{assumes}\ {\isachardoublequoteopen}inj\ f{\isachardoublequoteclose}\isanewline
\ \ \isakeyword{shows}\ {\isachardoublequoteopen}{\isacharparenleft}f\ {\isasymcirc}\ g\ {\isacharequal}\ f\ {\isasymcirc}\ h{\isacharparenright}\ {\isacharequal}\ {\isacharparenleft}g\ {\isacharequal}\ h{\isacharparenright}{\isachardoublequoteclose}\isanewline
%
\isadelimproof
%
\endisadelimproof
%
\isatagproof
\isacommand{proof}\isamarkupfalse%
\ \isanewline
\ \ \isacommand{assume}\isamarkupfalse%
\ {\isachardoublequoteopen}f\ {\isasymcirc}\ g\ {\isacharequal}\ f\ {\isasymcirc}\ h{\isachardoublequoteclose}\isanewline
\ \ \isacommand{show}\isamarkupfalse%
\ {\isachardoublequoteopen}g\ {\isacharequal}\ h{\isachardoublequoteclose}\isanewline
\ \ \isacommand{proof}\isamarkupfalse%
\isanewline
\ \ \ \ \isacommand{fix}\isamarkupfalse%
\ x\isanewline
\ \ \ \ \isacommand{have}\isamarkupfalse%
\ {\isachardoublequoteopen}{\isacharparenleft}f\ {\isasymcirc}\ g{\isacharparenright}{\isacharparenleft}x{\isacharparenright}\ {\isacharequal}\ {\isacharparenleft}f\ {\isasymcirc}\ h{\isacharparenright}{\isacharparenleft}x{\isacharparenright}{\isachardoublequoteclose}\ \isacommand{using}\isamarkupfalse%
\ {\isacharbackquoteopen}f\ {\isasymcirc}\ g\ {\isacharequal}\ f\ {\isasymcirc}\ h{\isacharbackquoteclose}\ \isacommand{by}\isamarkupfalse%
\ simp\isanewline
\ \ \ \ \isacommand{then}\isamarkupfalse%
\ \isacommand{have}\isamarkupfalse%
\ {\isachardoublequoteopen}f{\isacharparenleft}g{\isacharparenleft}x{\isacharparenright}{\isacharparenright}\ {\isacharequal}\ f{\isacharparenleft}h{\isacharparenleft}x{\isacharparenright}{\isacharparenright}{\isachardoublequoteclose}\ \isacommand{by}\isamarkupfalse%
\ simp\isanewline
\ \ \ \ \isacommand{then}\isamarkupfalse%
\ \isacommand{show}\isamarkupfalse%
\ {\isachardoublequoteopen}g{\isacharparenleft}x{\isacharparenright}\ {\isacharequal}\ h{\isacharparenleft}x{\isacharparenright}{\isachardoublequoteclose}\ \isacommand{using}\isamarkupfalse%
\ {\isacharbackquoteopen}inj\ f{\isacharbackquoteclose}\ \isacommand{by}\isamarkupfalse%
\ {\isacharparenleft}simp\ add{\isacharcolon}inj{\isacharunderscore}on{\isacharunderscore}def{\isacharparenright}\isanewline
\ \ \isacommand{qed}\isamarkupfalse%
\isanewline
\isacommand{next}\isamarkupfalse%
\isanewline
\ \ \isacommand{assume}\isamarkupfalse%
\ {\isachardoublequoteopen}g\ {\isacharequal}\ h{\isachardoublequoteclose}\isanewline
\ \ \isacommand{show}\isamarkupfalse%
\ {\isachardoublequoteopen}f\ {\isasymcirc}\ g\ {\isacharequal}\ f\ {\isasymcirc}\ h{\isachardoublequoteclose}\isanewline
\ \ \isacommand{proof}\isamarkupfalse%
\isanewline
\ \ \ \ \isacommand{fix}\isamarkupfalse%
\ x\isanewline
\ \ \ \ \isacommand{have}\isamarkupfalse%
\ {\isachardoublequoteopen}{\isacharparenleft}f\ {\isasymcirc}\ g{\isacharparenright}\ x\ {\isacharequal}\ f{\isacharparenleft}g{\isacharparenleft}x{\isacharparenright}{\isacharparenright}{\isachardoublequoteclose}\ \isacommand{by}\isamarkupfalse%
\ simp\isanewline
\ \ \ \ \isacommand{also}\isamarkupfalse%
\ \isacommand{have}\isamarkupfalse%
\ {\isachardoublequoteopen}{\isasymdots}\ {\isacharequal}\ f{\isacharparenleft}h{\isacharparenleft}x{\isacharparenright}{\isacharparenright}{\isachardoublequoteclose}\ \isacommand{using}\isamarkupfalse%
\ {\isacharbackquoteopen}g\ {\isacharequal}\ h{\isacharbackquoteclose}\ \isacommand{by}\isamarkupfalse%
\ simp\isanewline
\ \ \ \ \isacommand{also}\isamarkupfalse%
\ \isacommand{have}\isamarkupfalse%
\ {\isachardoublequoteopen}{\isasymdots}\ {\isacharequal}\ {\isacharparenleft}f\ {\isasymcirc}\ h{\isacharparenright}\ x{\isachardoublequoteclose}\ \isacommand{by}\isamarkupfalse%
\ simp\isanewline
\ \ \ \ \isacommand{finally}\isamarkupfalse%
\ \isacommand{show}\isamarkupfalse%
\ {\isachardoublequoteopen}{\isacharparenleft}f\ {\isasymcirc}\ g{\isacharparenright}\ x\ {\isacharequal}\ {\isacharparenleft}f\ {\isasymcirc}\ h{\isacharparenright}\ x{\isachardoublequoteclose}\ \isacommand{by}\isamarkupfalse%
\ simp\isanewline
\ \ \isacommand{qed}\isamarkupfalse%
\isanewline
\isacommand{qed}\isamarkupfalse%
%
\endisatagproof
{\isafoldproof}%
%
\isadelimproof
%
\endisadelimproof
%
\begin{isamarkuptext}%
Otra demostración declarativa es%
\end{isamarkuptext}\isamarkuptrue%
\isacommand{lemma}\isamarkupfalse%
\ \isanewline
\ \ \isakeyword{assumes}\ {\isachardoublequoteopen}inj\ f{\isachardoublequoteclose}\isanewline
\ \ \isakeyword{shows}\ {\isachardoublequoteopen}{\isacharparenleft}f\ {\isasymcirc}\ g\ {\isacharequal}\ f\ {\isasymcirc}\ h{\isacharparenright}\ {\isacharequal}\ {\isacharparenleft}g\ {\isacharequal}\ h{\isacharparenright}{\isachardoublequoteclose}\isanewline
%
\isadelimproof
%
\endisadelimproof
%
\isatagproof
\isacommand{proof}\isamarkupfalse%
\ \isanewline
\ \ \isacommand{assume}\isamarkupfalse%
\ {\isachardoublequoteopen}f\ {\isasymcirc}\ g\ {\isacharequal}\ f\ {\isasymcirc}\ h{\isachardoublequoteclose}\ \isanewline
\ \ \isacommand{then}\isamarkupfalse%
\ \isacommand{show}\isamarkupfalse%
\ {\isachardoublequoteopen}g\ {\isacharequal}\ h{\isachardoublequoteclose}\ \isacommand{using}\isamarkupfalse%
\ {\isacharbackquoteopen}inj\ f{\isacharbackquoteclose}\ \isacommand{by}\isamarkupfalse%
\ {\isacharparenleft}simp\ add{\isacharcolon}\ inj{\isacharunderscore}on{\isacharunderscore}def\ fun{\isacharunderscore}eq{\isacharunderscore}iff{\isacharparenright}\ \isanewline
\isacommand{next}\isamarkupfalse%
\isanewline
\ \ \isacommand{assume}\isamarkupfalse%
\ {\isachardoublequoteopen}g\ {\isacharequal}\ h{\isachardoublequoteclose}\ \isanewline
\ \ \isacommand{then}\isamarkupfalse%
\ \isacommand{show}\isamarkupfalse%
\ {\isachardoublequoteopen}f\ {\isasymcirc}\ g\ {\isacharequal}\ f\ {\isasymcirc}\ h{\isachardoublequoteclose}\ \isacommand{by}\isamarkupfalse%
\ simp\isanewline
\isacommand{qed}\isamarkupfalse%
\isanewline
%
\endisatagproof
{\isafoldproof}%
%
\isadelimproof
%
\endisadelimproof
%
\isadelimtheory
%
\endisadelimtheory
%
\isatagtheory
%
\endisatagtheory
{\isafoldtheory}%
%
\isadelimtheory
%
\endisadelimtheory
%
\end{isabellebody}%
\endinput
%:%file=/home/carlos/Escritorio/TFG-v1/EjerciciosDELMF/CancelacionInyectiva.thy%:%
%:%24=8%:%
%:%36=10%:%
%:%37=11%:%
%:%38=12%:%
%:%39=13%:%
%:%40=14%:%
%:%41=15%:%
%:%42=16%:%
%:%43=17%:%
%:%44=18%:%
%:%45=19%:%
%:%46=20%:%
%:%47=21%:%
%:%48=22%:%
%:%49=23%:%
%:%50=24%:%
%:%51=25%:%
%:%52=26%:%
%:%53=27%:%
%:%54=28%:%
%:%55=29%:%
%:%56=30%:%
%:%57=31%:%
%:%58=32%:%
%:%59=33%:%
%:%60=34%:%
%:%62=37%:%
%:%63=37%:%
%:%64=38%:%
%:%71=39%:%
%:%81=41%:%
%:%82=42%:%
%:%83=43%:%
%:%84=44%:%
%:%85=45%:%
%:%86=46%:%
%:%87=47%:%
%:%88=48%:%
%:%89=49%:%
%:%90=50%:%
%:%91=51%:%
%:%92=52%:%
%:%93=53%:%
%:%94=54%:%
%:%95=55%:%
%:%96=56%:%
%:%97=57%:%
%:%98=58%:%
%:%99=59%:%
%:%100=60%:%
%:%101=61%:%
%:%103=63%:%
%:%104=63%:%
%:%105=64%:%
%:%108=65%:%
%:%112=65%:%
%:%113=65%:%
%:%114=66%:%
%:%115=66%:%
%:%116=67%:%
%:%126=69%:%
%:%127=70%:%
%:%128=71%:%
%:%129=72%:%
%:%130=73%:%
%:%131=74%:%
%:%132=75%:%
%:%134=77%:%
%:%135=77%:%
%:%136=78%:%
%:%139=79%:%
%:%143=79%:%
%:%144=79%:%
%:%145=80%:%
%:%146=80%:%
%:%147=81%:%
%:%148=81%:%
%:%149=82%:%
%:%150=82%:%
%:%151=83%:%
%:%152=83%:%
%:%153=84%:%
%:%163=86%:%
%:%164=87%:%
%:%165=88%:%
%:%166=89%:%
%:%167=90%:%
%:%168=91%:%
%:%169=92%:%
%:%170=93%:%
%:%172=95%:%
%:%173=95%:%
%:%174=96%:%
%:%175=97%:%
%:%178=98%:%
%:%182=98%:%
%:%183=98%:%
%:%184=99%:%
%:%185=99%:%
%:%194=101%:%
%:%196=103%:%
%:%197=103%:%
%:%198=104%:%
%:%199=105%:%
%:%206=106%:%
%:%207=106%:%
%:%208=107%:%
%:%209=107%:%
%:%210=108%:%
%:%211=108%:%
%:%212=109%:%
%:%213=109%:%
%:%214=110%:%
%:%215=110%:%
%:%216=111%:%
%:%217=111%:%
%:%218=111%:%
%:%219=111%:%
%:%220=112%:%
%:%221=112%:%
%:%222=112%:%
%:%223=112%:%
%:%224=113%:%
%:%225=113%:%
%:%226=113%:%
%:%227=113%:%
%:%228=113%:%
%:%229=114%:%
%:%230=114%:%
%:%231=115%:%
%:%232=115%:%
%:%233=116%:%
%:%234=116%:%
%:%235=117%:%
%:%236=117%:%
%:%237=118%:%
%:%238=118%:%
%:%239=119%:%
%:%240=119%:%
%:%241=120%:%
%:%242=120%:%
%:%243=120%:%
%:%244=121%:%
%:%245=121%:%
%:%246=121%:%
%:%247=121%:%
%:%248=121%:%
%:%249=122%:%
%:%250=122%:%
%:%251=122%:%
%:%252=122%:%
%:%253=123%:%
%:%254=123%:%
%:%255=123%:%
%:%256=123%:%
%:%257=124%:%
%:%258=124%:%
%:%259=125%:%
%:%269=127%:%
%:%271=129%:%
%:%272=129%:%
%:%273=130%:%
%:%274=131%:%
%:%281=132%:%
%:%282=132%:%
%:%283=133%:%
%:%284=133%:%
%:%285=134%:%
%:%286=134%:%
%:%287=134%:%
%:%288=134%:%
%:%289=134%:%
%:%290=135%:%
%:%291=135%:%
%:%292=136%:%
%:%293=136%:%
%:%294=137%:%
%:%295=137%:%
%:%296=137%:%
%:%297=137%:%
%:%298=138%:%
%:%299=138%:%