\documentclass[12pt,a4paper,twoside]{book}
\usepackage{isabelle,isabellesym,latexsym}
\usepackage{ifthen,mathpartir}
\usepackage{amsmath}
\usepackage{textcomp}
\usepackage[only,bigsqcap]{stmaryrd}

% further packages required for unusual symbols (see also
% isabellesym.sty), use only when needed

% Personalización
\usepackage{float}
\usepackage{color,graphicx}        % Usa figuras.
\usepackage[utf8x]{inputenc}         % Acentos de UTF8
% \usepackage[T1]{fontenc}           % Codificación T1 con European Computer
% \usepackage[spanish]{babel}        % Castellanización.
% \usepackage{ucs}
\usepackage{mathpazo}              % Tipo de fuente
\usepackage[scaled=.90]{helvet}    % Tipo de fuente
\usepackage{a4wide}                % Márgenes
\linespread{1.05}                  % Distancia entre líneas
\setlength{\parindent}{2em}        % Indentación de comienzo de párrafo

\usepackage[colorinlistoftodos
           , backgroundcolor = yellow
           , textwidth = 4cm
           , shadow
           , spanish]{todonotes}

\setcounter{secnumdepth}{3}
           
\usepackage{amssymb}
  %for \<leadsto>, \<box>, \<diamond>, \<sqsupset>, \<mho>, \<Join>,
  %\<lhd>, \<lesssim>, \<greatersim>, \<lessapprox>, \<greaterapprox>,
  %\<triangleq>, \<yen>, \<lozenge>

%\usepackage{eurosym}
  %for \<euro>

%\usepackage[only,bigsqcap]{stmaryrd}
  %for \<Sqinter>

%\usepackage{eufrak}
  %for \<AA> ... \<ZZ>, \<aa> ... \<zz> (also included in amssymb)

% \usepackage{textcomp}
  %for \<onequarter>, \<onehalf>, \<threequarters>, \<degree>, \<cent>,
  %\<currency>

% this should be the last package used
\usepackage{pdfsetup}

% urls in roman style, theory text in math-similar italics
\urlstyle{rm}
\isabellestyle{it}

% for uniform font size
\renewcommand{\isastyle}{\isastyleminor}

% Nota: Definiciones
\input definiciones
\input castellano

% No ajusta los espacios verticales.
\raggedbottom

% Espacio entre párrafos
\parindent 2em\parskip 1ex
% Diagramas conmutativos 
%%%%%%%%%%%%%%%%%%%%%%%%%%%%%%%%%%%%%%%%%%%%%%%%%%%%%%%%%%%%%%%%%%%%%%%%%%%%%%
%% Cabeceras                                                              %%
%%%%%%%%%%%%%%%%%%%%%%%%%%%%%%%%%%%%%%%%%%%%%%%%%%%%%%%%%%%%%%%%%%%%%%%%%%%%%%

\usepackage{fancyhdr}

\addtolength{\headheight}{\baselineskip}

\pagestyle{fancy}

\cfoot{}

\fancyhead{}
\fancyhead[RE]{\slshape \nouppercase{\leftmark}}
\fancyhead[LO]{\slshape \nouppercase{\rightmark}}
\fancyhead[LE,RO]{\slshape \thepage}

%%%%%%%%%%%%%%%%%%%%%%%%%%%%%%%%%%%%%%%%%%%%%%%%%%%%%%%%%%%%%%%%%%%%%%%%%%%%%%
%% Documento
%%%%%%%%%%%%%%%%%%%%%%%%%%%%%%%%%%%%%%%%%%%%%%%%%%%%%%%%%%%%%%%%%%%%%%%%%%%%%%

\begin{document}

\title{Elementos de matemáticas formalizados en Isabelle/HOL}
\author{Carlos Núñez Fernández}
\date{\today}
\maketitle

% \begin{abstract}
%   En este trabajo vamos a presentar la formalización en Isabelle/HOL de
%   una selección de teoremas de distintos campos de las matemáticas.
% \end{abstract}

\tableofcontents

% sane default for proof documents
% \parindent 0pt\parskip 0.5ex
\parindent 2em\parskip 1ex

% generated text of all theories
% %
\begin{isabellebody}%
\setisabellecontext{SumaImpares}%
%
\isadelimtheory
%
\endisadelimtheory
%
\isatagtheory
%
\endisatagtheory
{\isafoldtheory}%
%
\isadelimtheory
%
\endisadelimtheory
%
\isadelimdocument
%
\endisadelimdocument
%
\isatagdocument
%
\isamarkupsection{Suma de los primeros números impares%
}
\isamarkuptrue%
%
\endisatagdocument
{\isafolddocument}%
%
\isadelimdocument
%
\endisadelimdocument
%
\begin{isamarkuptext}%
El primer teorema es una propiedad de los números naturales.

  \begin{teorema}
    La suma de los $n$ primeros números impares es $n^2$.
  \end{teorema}

  \begin{demostracion}
    La demostración la haremos en inducción sobre $n$.
\begin {itemize}
\item EL caso $n = 0$ es trivial, ya que $0 = 0$.
\item Supongamos que se verifica la hipótesis para $n$ y veamos para
 $n+1$. \\
Tenemos que demostrar que $\sum_{j=1}^{n+1} k_j = (n+1)^2$ siendo los
 $k_{j}$ el j-ésimo impar, es decir, $k_{j} = 2j - 1$.
$$\sum_{j = 1}^{n+1} k_{j} = k_{n+1} + \sum^{n}_{j=1} k_{j} = k_{n+1} +
 n^{2} = 2(n+1) - 1 + n^2 = n^2 + 2n + 1 = (n+1)^2$$ 
\end {itemize}
.
  \end{demostracion}

  Para especificar el teorema en Isabelle, se comienza definiendo 
  la función \isa{suma{\isacharunderscore}impares} tal que \isa{suma{\isacharunderscore}impares\ n} es la 
  suma de los $n$ primeros números impares%
\end{isamarkuptext}\isamarkuptrue%
\isacommand{fun}\isamarkupfalse%
\ suma{\isacharunderscore}impares\ {\isacharcolon}{\isacharcolon}\ {\isachardoublequoteopen}nat\ {\isasymRightarrow}\ nat{\isachardoublequoteclose}\ \isakeyword{where}\isanewline
\ \ {\isachardoublequoteopen}suma{\isacharunderscore}impares\ {\isadigit{0}}\ {\isacharequal}\ {\isadigit{0}}{\isachardoublequoteclose}\ \isanewline
{\isacharbar}\ {\isachardoublequoteopen}suma{\isacharunderscore}impares\ {\isacharparenleft}Suc\ n{\isacharparenright}\ {\isacharequal}\ {\isacharparenleft}{\isadigit{2}}{\isacharasterisk}{\isacharparenleft}Suc\ n{\isacharparenright}\ {\isacharminus}\ {\isadigit{1}}{\isacharparenright}\ {\isacharplus}\ suma{\isacharunderscore}impares\ n{\isachardoublequoteclose}%
\begin{isamarkuptext}%
El enunciado del teorema es el siguiente:%
\end{isamarkuptext}\isamarkuptrue%
\isacommand{lemma}\isamarkupfalse%
\ {\isachardoublequoteopen}suma{\isacharunderscore}impares\ n\ {\isacharequal}\ n\ {\isacharasterisk}\ n{\isachardoublequoteclose}\isanewline
%
\isadelimproof
%
\endisadelimproof
%
\isatagproof
\isacommand{oops}\isamarkupfalse%
%
\endisatagproof
{\isafoldproof}%
%
\isadelimproof
%
\endisadelimproof
%
\begin{isamarkuptext}%
En la demostración se usará la táctica \isa{induct} que hace
  uso del esquema de inducción sobre los naturales:
  \begin{itemize}
  \item[] \isa{\mbox{}\inferrule{\mbox{P\ {\isadigit{0}}}\\\ \mbox{{\isasymAnd}nat{\isachardot}\ \mbox{}\inferrule{\mbox{P\ nat}}{\mbox{P\ {\isacharparenleft}Suc\ nat{\isacharparenright}}}}}{\mbox{P\ nat}}} \hfill (\isa{nat{\isachardot}induct})
  \end{itemize}

  Vamos a presentar distintas demostraciones del teorema. La 
  primera es la demostración aplicativa%
\end{isamarkuptext}\isamarkuptrue%
\isacommand{lemma}\isamarkupfalse%
\ {\isachardoublequoteopen}suma{\isacharunderscore}impares\ n\ {\isacharequal}\ n\ {\isacharasterisk}\ n{\isachardoublequoteclose}\isanewline
%
\isadelimproof
\ \ %
\endisadelimproof
%
\isatagproof
\isacommand{apply}\isamarkupfalse%
\ {\isacharparenleft}induct\ n{\isacharparenright}\ \isanewline
\ \ \ \isacommand{apply}\isamarkupfalse%
\ simp{\isacharunderscore}all\isanewline
\ \ \isacommand{done}\isamarkupfalse%
%
\endisatagproof
{\isafoldproof}%
%
\isadelimproof
%
\endisadelimproof
%
\begin{isamarkuptext}%
La demostración automática es%
\end{isamarkuptext}\isamarkuptrue%
\isacommand{lemma}\isamarkupfalse%
\ {\isachardoublequoteopen}suma{\isacharunderscore}impares\ n\ {\isacharequal}\ n\ {\isacharasterisk}\ n{\isachardoublequoteclose}\isanewline
%
\isadelimproof
\ \ %
\endisadelimproof
%
\isatagproof
\isacommand{by}\isamarkupfalse%
\ {\isacharparenleft}induct\ n{\isacharparenright}\ simp{\isacharunderscore}all%
\endisatagproof
{\isafoldproof}%
%
\isadelimproof
%
\endisadelimproof
%
\begin{isamarkuptext}%
La demostración del lema anterior por inducción y razonamiento 
   ecuacional es%
\end{isamarkuptext}\isamarkuptrue%
\isacommand{lemma}\isamarkupfalse%
\ {\isachardoublequoteopen}suma{\isacharunderscore}impares\ n\ {\isacharequal}\ n\ {\isacharasterisk}\ n{\isachardoublequoteclose}\isanewline
%
\isadelimproof
%
\endisadelimproof
%
\isatagproof
\isacommand{proof}\isamarkupfalse%
\ {\isacharparenleft}induct\ n{\isacharparenright}\isanewline
\ \ \isacommand{show}\isamarkupfalse%
\ {\isachardoublequoteopen}suma{\isacharunderscore}impares\ {\isadigit{0}}\ {\isacharequal}\ {\isadigit{0}}\ {\isacharasterisk}\ {\isadigit{0}}{\isachardoublequoteclose}\ \isacommand{by}\isamarkupfalse%
\ simp\isanewline
\isacommand{next}\isamarkupfalse%
\isanewline
\ \ \isacommand{fix}\isamarkupfalse%
\ n\ \isacommand{assume}\isamarkupfalse%
\ HI{\isacharcolon}\ {\isachardoublequoteopen}suma{\isacharunderscore}impares\ n\ {\isacharequal}\ n\ {\isacharasterisk}\ n{\isachardoublequoteclose}\isanewline
\ \ \isacommand{have}\isamarkupfalse%
\ {\isachardoublequoteopen}suma{\isacharunderscore}impares\ {\isacharparenleft}Suc\ n{\isacharparenright}\ {\isacharequal}\ {\isacharparenleft}{\isadigit{2}}\ {\isacharasterisk}\ {\isacharparenleft}Suc\ n{\isacharparenright}\ {\isacharminus}\ {\isadigit{1}}{\isacharparenright}\ {\isacharplus}\ suma{\isacharunderscore}impares\ n{\isachardoublequoteclose}\ \isanewline
\ \ \ \ \isacommand{by}\isamarkupfalse%
\ simp\isanewline
\ \ \isacommand{also}\isamarkupfalse%
\ \isacommand{have}\isamarkupfalse%
\ {\isachardoublequoteopen}{\isasymdots}\ {\isacharequal}\ {\isacharparenleft}{\isadigit{2}}\ {\isacharasterisk}\ {\isacharparenleft}Suc\ n{\isacharparenright}\ {\isacharminus}\ {\isadigit{1}}{\isacharparenright}\ {\isacharplus}\ n\ {\isacharasterisk}\ n{\isachardoublequoteclose}\ \isacommand{using}\isamarkupfalse%
\ HI\ \isacommand{by}\isamarkupfalse%
\ simp\isanewline
\ \ \isacommand{also}\isamarkupfalse%
\ \isacommand{have}\isamarkupfalse%
\ {\isachardoublequoteopen}{\isasymdots}\ {\isacharequal}\ n\ {\isacharasterisk}\ n\ {\isacharplus}\ {\isadigit{2}}\ {\isacharasterisk}\ n\ {\isacharplus}\ {\isadigit{1}}{\isachardoublequoteclose}\ \isacommand{by}\isamarkupfalse%
\ simp\isanewline
\ \ \isacommand{finally}\isamarkupfalse%
\ \isacommand{show}\isamarkupfalse%
\ {\isachardoublequoteopen}suma{\isacharunderscore}impares\ {\isacharparenleft}Suc\ n{\isacharparenright}\ {\isacharequal}\ {\isacharparenleft}Suc\ n{\isacharparenright}\ {\isacharasterisk}\ {\isacharparenleft}Suc\ n{\isacharparenright}{\isachardoublequoteclose}\ \isacommand{by}\isamarkupfalse%
\ simp\isanewline
\isacommand{qed}\isamarkupfalse%
%
\endisatagproof
{\isafoldproof}%
%
\isadelimproof
%
\endisadelimproof
%
\begin{isamarkuptext}%
La demostración del lema anterior con patrones y razonamiento 
   ecuacional es%
\end{isamarkuptext}\isamarkuptrue%
\isacommand{lemma}\isamarkupfalse%
\ {\isachardoublequoteopen}suma{\isacharunderscore}impares\ n\ {\isacharequal}\ n\ {\isacharasterisk}\ n{\isachardoublequoteclose}\ {\isacharparenleft}\isakeyword{is}\ {\isachardoublequoteopen}{\isacharquery}P\ n{\isachardoublequoteclose}{\isacharparenright}\isanewline
%
\isadelimproof
%
\endisadelimproof
%
\isatagproof
\isacommand{proof}\isamarkupfalse%
\ {\isacharparenleft}induct\ n{\isacharparenright}\isanewline
\ \ \isacommand{show}\isamarkupfalse%
\ {\isachardoublequoteopen}{\isacharquery}P\ {\isadigit{0}}{\isachardoublequoteclose}\ \isacommand{by}\isamarkupfalse%
\ simp\isanewline
\isacommand{next}\isamarkupfalse%
\isanewline
\ \ \isacommand{fix}\isamarkupfalse%
\ n\ \isanewline
\ \ \isacommand{assume}\isamarkupfalse%
\ HI{\isacharcolon}\ {\isachardoublequoteopen}{\isacharquery}P\ n{\isachardoublequoteclose}\isanewline
\ \ \isacommand{have}\isamarkupfalse%
\ {\isachardoublequoteopen}suma{\isacharunderscore}impares\ {\isacharparenleft}Suc\ n{\isacharparenright}\ {\isacharequal}\ {\isacharparenleft}{\isadigit{2}}\ {\isacharasterisk}\ {\isacharparenleft}Suc\ n{\isacharparenright}\ {\isacharminus}\ {\isadigit{1}}{\isacharparenright}\ {\isacharplus}\ suma{\isacharunderscore}impares\ n{\isachardoublequoteclose}\ \isanewline
\ \ \ \ \isacommand{by}\isamarkupfalse%
\ simp\isanewline
\ \ \isacommand{also}\isamarkupfalse%
\ \isacommand{have}\isamarkupfalse%
\ {\isachardoublequoteopen}{\isasymdots}\ {\isacharequal}\ {\isacharparenleft}{\isadigit{2}}\ {\isacharasterisk}\ {\isacharparenleft}Suc\ n{\isacharparenright}\ {\isacharminus}\ {\isadigit{1}}{\isacharparenright}\ {\isacharplus}\ n\ {\isacharasterisk}\ n{\isachardoublequoteclose}\ \isacommand{using}\isamarkupfalse%
\ HI\ \isacommand{by}\isamarkupfalse%
\ simp\isanewline
\ \ \isacommand{also}\isamarkupfalse%
\ \isacommand{have}\isamarkupfalse%
\ {\isachardoublequoteopen}{\isasymdots}\ {\isacharequal}\ n\ {\isacharasterisk}\ n\ {\isacharplus}\ {\isadigit{2}}\ {\isacharasterisk}\ n\ {\isacharplus}\ {\isadigit{1}}{\isachardoublequoteclose}\ \isacommand{by}\isamarkupfalse%
\ simp\isanewline
\ \ \isacommand{finally}\isamarkupfalse%
\ \isacommand{show}\isamarkupfalse%
\ {\isachardoublequoteopen}{\isacharquery}P\ {\isacharparenleft}Suc\ n{\isacharparenright}{\isachardoublequoteclose}\ \isacommand{by}\isamarkupfalse%
\ simp\isanewline
\isacommand{qed}\isamarkupfalse%
%
\endisatagproof
{\isafoldproof}%
%
\isadelimproof
%
\endisadelimproof
%
\begin{isamarkuptext}%
La demostración usando patrones es%
\end{isamarkuptext}\isamarkuptrue%
\isacommand{lemma}\isamarkupfalse%
\ {\isachardoublequoteopen}suma{\isacharunderscore}impares\ n\ {\isacharequal}\ n\ {\isacharasterisk}\ n{\isachardoublequoteclose}\ {\isacharparenleft}\isakeyword{is}\ {\isachardoublequoteopen}{\isacharquery}P\ n{\isachardoublequoteclose}{\isacharparenright}\isanewline
%
\isadelimproof
%
\endisadelimproof
%
\isatagproof
\isacommand{proof}\isamarkupfalse%
\ {\isacharparenleft}induct\ n{\isacharparenright}\isanewline
\ \ \isacommand{show}\isamarkupfalse%
\ {\isachardoublequoteopen}{\isacharquery}P\ {\isadigit{0}}{\isachardoublequoteclose}\ \isacommand{by}\isamarkupfalse%
\ simp\isanewline
\isacommand{next}\isamarkupfalse%
\isanewline
\ \ \isacommand{fix}\isamarkupfalse%
\ n\ \isanewline
\ \ \isacommand{assume}\isamarkupfalse%
\ {\isachardoublequoteopen}{\isacharquery}P\ n{\isachardoublequoteclose}\isanewline
\ \ \isacommand{then}\isamarkupfalse%
\ \isacommand{show}\isamarkupfalse%
\ {\isachardoublequoteopen}{\isacharquery}P\ {\isacharparenleft}Suc\ n{\isacharparenright}{\isachardoublequoteclose}\ \isacommand{by}\isamarkupfalse%
\ simp\isanewline
\isacommand{qed}\isamarkupfalse%
\isanewline
%
\endisatagproof
{\isafoldproof}%
%
\isadelimproof
%
\endisadelimproof
%
\isadelimtheory
%
\endisadelimtheory
%
\isatagtheory
%
\endisatagtheory
{\isafoldtheory}%
%
\isadelimtheory
%
\endisadelimtheory
%
\end{isabellebody}%
\endinput
%:%file=~/Escritorio/TFG-v1/EjerciciosDELMF/SumaImpares.thy%:%
%:%24=8%:%
%:%36=10%:%
%:%37=11%:%
%:%38=12%:%
%:%39=13%:%
%:%40=14%:%
%:%41=15%:%
%:%42=16%:%
%:%43=17%:%
%:%44=18%:%
%:%45=19%:%
%:%46=20%:%
%:%47=21%:%
%:%48=22%:%
%:%49=23%:%
%:%50=24%:%
%:%51=25%:%
%:%52=26%:%
%:%53=27%:%
%:%54=28%:%
%:%55=29%:%
%:%56=30%:%
%:%57=31%:%
%:%58=32%:%
%:%60=35%:%
%:%61=35%:%
%:%62=36%:%
%:%63=37%:%
%:%65=39%:%
%:%67=41%:%
%:%68=41%:%
%:%75=42%:%
%:%85=44%:%
%:%86=45%:%
%:%87=46%:%
%:%88=47%:%
%:%89=48%:%
%:%90=49%:%
%:%91=50%:%
%:%92=51%:%
%:%94=56%:%
%:%95=56%:%
%:%98=57%:%
%:%102=57%:%
%:%103=57%:%
%:%104=58%:%
%:%105=58%:%
%:%106=59%:%
%:%116=61%:%
%:%118=63%:%
%:%119=63%:%
%:%122=64%:%
%:%126=64%:%
%:%127=64%:%
%:%136=66%:%
%:%137=67%:%
%:%139=69%:%
%:%140=69%:%
%:%147=70%:%
%:%148=70%:%
%:%149=71%:%
%:%150=71%:%
%:%151=71%:%
%:%152=72%:%
%:%153=72%:%
%:%154=73%:%
%:%155=73%:%
%:%156=73%:%
%:%157=74%:%
%:%158=74%:%
%:%159=75%:%
%:%160=75%:%
%:%161=76%:%
%:%162=76%:%
%:%163=76%:%
%:%164=76%:%
%:%165=76%:%
%:%166=77%:%
%:%167=77%:%
%:%168=77%:%
%:%169=77%:%
%:%170=78%:%
%:%171=78%:%
%:%172=78%:%
%:%173=78%:%
%:%174=79%:%
%:%184=81%:%
%:%185=82%:%
%:%187=83%:%
%:%188=83%:%
%:%195=84%:%
%:%196=84%:%
%:%197=85%:%
%:%198=85%:%
%:%199=85%:%
%:%200=86%:%
%:%201=86%:%
%:%202=87%:%
%:%203=87%:%
%:%204=88%:%
%:%205=88%:%
%:%206=89%:%
%:%207=89%:%
%:%208=90%:%
%:%209=90%:%
%:%210=91%:%
%:%211=91%:%
%:%212=91%:%
%:%213=91%:%
%:%214=91%:%
%:%215=92%:%
%:%216=92%:%
%:%217=92%:%
%:%218=92%:%
%:%219=93%:%
%:%220=93%:%
%:%221=93%:%
%:%222=93%:%
%:%223=94%:%
%:%233=96%:%
%:%235=98%:%
%:%236=98%:%
%:%243=99%:%
%:%244=99%:%
%:%245=100%:%
%:%246=100%:%
%:%247=100%:%
%:%248=101%:%
%:%249=101%:%
%:%250=102%:%
%:%251=102%:%
%:%252=103%:%
%:%253=103%:%
%:%254=104%:%
%:%255=104%:%
%:%256=104%:%
%:%257=104%:%
%:%258=105%:%
%:%259=105%:%

%
\begin{isabellebody}%
\setisabellecontext{CancelacionInyectiva}%
%
\isadelimtheory
%
\endisadelimtheory
%
\isatagtheory
%
\endisatagtheory
{\isafoldtheory}%
%
\isadelimtheory
%
\endisadelimtheory
%
\isadelimdocument
%
\endisadelimdocument
%
\isatagdocument
%
\isamarkupsection{Cancelación de funciones inyectivas%
}
\isamarkuptrue%
%
\endisatagdocument
{\isafolddocument}%
%
\isadelimdocument
%
\endisadelimdocument
%
\begin{isamarkuptext}%
El siguiente teorema prueba una caracterización de las funciones
 inyectivas, en otras palabras, las funciones inyectivas son
 monomorfismos en la categoría de conjuntos. Un monomorfismo es un
 homomorfismo inyectivo y la categoría de conjuntos es la categoría
 cuyos objetos son los conjuntos.
  
  \begin{teorema}
    $f$ es una función inyectiva, si y solo si, para todas funciones 
 $g$ y $h$  tales que  $f \circ g = f \circ h$ se tiene que $g = h$. 
  \end{teorema}

Vamos a hacer dos lemas de nuestro teorema, ya que podemos la doble 
implicación en dos implicaciones y demostrar cada una de ellas por
 separado.

\begin {lema}
$f$ es una función inyectiva si para todas funciones $g$ y $h$ tales que
 $f \circ g = f \circ h$ se tiene que $g = h.$
\end {lema}
  \begin{demostracion}
    La demostración la haremos por doble implicación: 
\begin {enumerate}
\item Supongamos que tenemos que $f \circ g = f \circ h$, queremos
 demostrar que $g = h$, usando que f es inyectiva tenemos que: \\
$$(f \circ g)(x) = (f \circ h)(x) \Longrightarrow f(g(x)) = f(h(x)) = 
g(x) = h(x)$$
\item Supongamos ahora que $g = h$, queremos demostrar que  $f \circ g
 = f \circ h$. \\
$$(f \circ g)(x) = f(g(x)) = f(h(x)) = (f \circ h)(x)$$
\end {enumerate}
.
  \end{demostracion}

\begin {lema} 
Si para toda $g$ y $h$ tales que $f \circ g =  f \circ h$ se tiene que $g
= h$ entonces f es inyectiva.
\end {lema} 

\begin {demostracion}


Supongamos que el dominio de nuestra función $f$ es distinto del vacío.
Tenemos que demostrar que $\forall a,b$ tales que $f(a) = f(b),$ esto
 implica que $a = b.$ \\
Sean $a,b$ tales que $f(a) = f(b)$, y definamos $g(x) = a  \ \forall x$
 y $h(x) = b \  \forall x$ entonces 
$$(f \circ g) = (f \circ h) \Longrightarrow  f(g(x)) = f(h(x)) \Longrightarrow f(a) = f(b)$$
Por hipótesis tenemos entonces que $a = b,$ como queríamos demostrar.
\end {demostracion}


  Su especificación es la siguiente, pero al igual que hemos hecho en la demostración
a mano vamos a demostrarlo a través de dos lemas:%
\end{isamarkuptext}\isamarkuptrue%
\isacommand{theorem}\isamarkupfalse%
\ caracterizacionineyctiva{\isacharcolon}\isanewline
\ \ {\isachardoublequoteopen}inj\ f\ {\isasymlongleftrightarrow}\ {\isacharparenleft}{\isasymforall}g\ h{\isachardot}\ {\isacharparenleft}f\ {\isasymcirc}\ g\ {\isacharequal}\ f\ {\isasymcirc}\ h{\isacharparenright}\ {\isasymlongrightarrow}\ {\isacharparenleft}g\ {\isacharequal}\ h{\isacharparenright}{\isacharparenright}{\isachardoublequoteclose}\isanewline
%
\isadelimproof
\ \ %
\endisadelimproof
%
\isatagproof
\isacommand{oops}\isamarkupfalse%
%
\endisatagproof
{\isafoldproof}%
%
\isadelimproof
%
\endisadelimproof
%
\begin{isamarkuptext}%
Sus lemas son los siguientes:%
\end{isamarkuptext}\isamarkuptrue%
\isacommand{lemma}\isamarkupfalse%
\ \isanewline
{\isachardoublequoteopen}{\isasymforall}g\ h{\isachardot}\ {\isacharparenleft}f\ {\isasymcirc}\ g\ {\isacharequal}\ f\ {\isasymcirc}\ h\ {\isasymlongrightarrow}\ g\ {\isacharequal}\ h{\isacharparenright}\ {\isasymLongrightarrow}\ inj\ f{\isachardoublequoteclose}\isanewline
%
\isadelimproof
\ \ %
\endisadelimproof
%
\isatagproof
\isacommand{oops}\isamarkupfalse%
%
\endisatagproof
{\isafoldproof}%
%
\isadelimproof
\isanewline
%
\endisadelimproof
\isanewline
\isacommand{lemma}\isamarkupfalse%
\ \isanewline
{\isachardoublequoteopen}inj\ f\ {\isasymLongrightarrow}\ {\isacharparenleft}{\isasymforall}g\ h{\isachardot}{\isacharparenleft}f\ {\isasymcirc}\ g\ {\isacharequal}\ f\ {\isasymcirc}\ h{\isacharparenright}\ {\isasymlongrightarrow}\ {\isacharparenleft}g\ {\isacharequal}\ h{\isacharparenright}{\isacharparenright}{\isachardoublequoteclose}\isanewline
%
\isadelimproof
\ \ %
\endisadelimproof
%
\isatagproof
\isacommand{oops}\isamarkupfalse%
%
\endisatagproof
{\isafoldproof}%
%
\isadelimproof
%
\endisadelimproof
%
\begin{isamarkuptext}%
En la especificación anterior, \isa{inj\ f} es una 
  abreviatura de \isa{inj\ f} definida en la teoría
  \href{http://bit.ly/2XuPQx5}{Fun.thy}. Además, contiene la definición
  de \isa{inj{\isacharunderscore}on}
  \begin{itemize}
    \item[] \isa{inj{\isacharunderscore}on\ f\ A\ {\isacharequal}\ {\isacharparenleft}{\isasymforall}x{\isasymin}A{\isachardot}\ {\isasymforall}y{\isasymin}A{\isachardot}\ f\ x\ {\isacharequal}\ f\ y\ {\isasymlongrightarrow}\ x\ {\isacharequal}\ y{\isacharparenright}} \hfill (\isa{inj{\isacharunderscore}on{\isacharunderscore}def})
  \end{itemize} 
  Por su parte, \isa{UNIV} es el conjunto universal definido en la 
  teoría \href{http://bit.ly/2XtHCW6}{Set.thy} como una abreviatura de 
  \isa{top} que, a su vez está definido en la teoría 
  \href{http://bit.ly/2Xyj9Pe}{Orderings.thy} mediante la siguiente
  propiedad 
  \begin{itemize}
    \item[] \isa{\mbox{}\inferrule{\mbox{ordering{\isacharunderscore}top\ less{\isacharunderscore}eq\ less\ top}}{\mbox{less{\isacharunderscore}eq\ a\ top}}} 
      \hfill (\isa{ordering{\isacharunderscore}top{\isachardot}extremum})
  \end{itemize} 
  En el caso de la teoría de conjuntos, la relación de orden es la
  inclusión de conjuntos.

  Presentaremos distintas demostraciones de los lemas. La primera
  demostración es applicativa:%
\end{isamarkuptext}\isamarkuptrue%
\isacommand{lemma}\isamarkupfalse%
\ inyectivapli{\isacharcolon}\isanewline
\ \ {\isachardoublequoteopen}inj\ f\ {\isasymLongrightarrow}\ {\isacharparenleft}{\isasymforall}g\ h{\isachardot}{\isacharparenleft}f\ {\isasymcirc}\ g\ {\isacharequal}\ f\ {\isasymcirc}\ h{\isacharparenright}\ {\isasymlongrightarrow}\ \ {\isacharparenleft}g\ {\isacharequal}\ h{\isacharparenright}{\isacharparenright}{\isachardoublequoteclose}\isanewline
%
\isadelimproof
\ \ %
\endisadelimproof
%
\isatagproof
\isacommand{apply}\isamarkupfalse%
\ {\isacharparenleft}simp\ add{\isacharcolon}\ inj{\isacharunderscore}on{\isacharunderscore}def\ fun{\isacharunderscore}eq{\isacharunderscore}iff{\isacharparenright}\ \isanewline
\ \ \isacommand{done}\isamarkupfalse%
%
\endisatagproof
{\isafoldproof}%
%
\isadelimproof
\ \isanewline
%
\endisadelimproof
\isanewline
\isacommand{lemma}\isamarkupfalse%
\ inyectivapli{\isadigit{2}}{\isacharcolon}\isanewline
{\isachardoublequoteopen}{\isasymforall}g\ h{\isachardot}\ {\isacharparenleft}f\ {\isasymcirc}\ g\ {\isacharequal}\ f\ {\isasymcirc}\ h\ {\isasymlongrightarrow}\ g\ {\isacharequal}\ h{\isacharparenright}\ {\isasymLongrightarrow}\ inj\ f{\isachardoublequoteclose}\isanewline
%
\isadelimproof
\ \ %
\endisadelimproof
%
\isatagproof
\isacommand{apply}\isamarkupfalse%
\ {\isacharparenleft}rule\ injI{\isacharparenright}\isanewline
\ \ \isacommand{by}\isamarkupfalse%
\ {\isacharparenleft}metis\ fun{\isacharunderscore}upd{\isacharunderscore}apply\ fun{\isacharunderscore}upd{\isacharunderscore}comp{\isacharparenright}%
\endisatagproof
{\isafoldproof}%
%
\isadelimproof
%
\endisadelimproof
%
\begin{isamarkuptext}%
En las demostraciones anteriores se han usado los siguientes
 lemas:
  \begin{itemize}
    \item[] \isa{{\isacharparenleft}f\ {\isacharequal}\ g{\isacharparenright}\ {\isacharequal}\ {\isacharparenleft}{\isasymforall}x{\isachardot}\ f\ x\ {\isacharequal}\ g\ x{\isacharparenright}} 
      \hfill (\isa{fun{\isacharunderscore}eq{\isacharunderscore}iff})
  \end{itemize} 
  \begin{itemize}
    \item[] \isa{{\isacharparenleft}f{\isacharparenleft}x\ {\isacharcolon}{\isacharequal}\ y{\isacharparenright}{\isacharparenright}\ z\ {\isacharequal}\ {\isacharparenleft}\textsf{if}\ z\ {\isacharequal}\ x\ \textsf{then}\ y\ \textsf{else}\ f\ z{\isacharparenright}} 
      \hfill (\isa{fun{\isacharunderscore}upd{\isacharunderscore}apply})
  \end{itemize} 
  \begin{itemize}
    \item[] \isa{{\isacharparenleft}f\ {\isacharequal}\ g{\isacharparenright}\ {\isacharequal}\ {\isacharparenleft}{\isasymforall}x{\isachardot}\ f\ x\ {\isacharequal}\ g\ x{\isacharparenright}} 
      \hfill (\isa{fun{\isacharunderscore}upd{\isacharunderscore}comp})
  \end{itemize} 

  La demostración applicativa1 sin auto es%
\end{isamarkuptext}\isamarkuptrue%
\isacommand{lemma}\isamarkupfalse%
\isanewline
\ \ {\isachardoublequoteopen}inj\ f\ {\isasymLongrightarrow}\ {\isasymforall}g\ h{\isachardot}\ {\isacharparenleft}f\ {\isasymcirc}\ g\ {\isacharequal}\ f\ {\isasymcirc}\ h{\isacharparenright}\ {\isasymlongrightarrow}\ \ {\isacharparenleft}g\ {\isacharequal}\ h{\isacharparenright}{\isachardoublequoteclose}\isanewline
%
\isadelimproof
\ \ %
\endisadelimproof
%
\isatagproof
\isacommand{apply}\isamarkupfalse%
\ {\isacharparenleft}unfold\ inj{\isacharunderscore}on{\isacharunderscore}def{\isacharparenright}\ \isanewline
\ \ \isacommand{apply}\isamarkupfalse%
\ {\isacharparenleft}unfold\ fun{\isacharunderscore}eq{\isacharunderscore}iff{\isacharparenright}\ \isanewline
\ \ \isacommand{apply}\isamarkupfalse%
\ {\isacharparenleft}unfold\ o{\isacharunderscore}apply{\isacharparenright}\isanewline
\ \ \ \isacommand{apply}\isamarkupfalse%
\ simp{\isacharplus}\isanewline
\ \ \isacommand{done}\isamarkupfalse%
%
\endisatagproof
{\isafoldproof}%
%
\isadelimproof
\isanewline
%
\endisadelimproof
\isanewline
\isacommand{lemma}\isamarkupfalse%
\ \isanewline
{\isachardoublequoteopen}{\isasymforall}g\ h{\isachardot}\ {\isacharparenleft}f\ {\isasymcirc}\ g\ {\isacharequal}\ f\ {\isasymcirc}\ h\ {\isasymlongrightarrow}\ g\ {\isacharequal}\ h{\isacharparenright}\ {\isasymLongrightarrow}\ inj\ f{\isachardoublequoteclose}\isanewline
%
\isadelimproof
\ \ %
\endisadelimproof
%
\isatagproof
\isacommand{oops}\isamarkupfalse%
%
\endisatagproof
{\isafoldproof}%
%
\isadelimproof
%
\endisadelimproof
%
\begin{isamarkuptext}%
En la demostración anterior se ha introducido los siguientes
  hechos
  \begin{itemize}
    \item \isa{{\isacharparenleft}f\ {\isasymcirc}\ g{\isacharparenright}\ x\ {\isacharequal}\ f\ {\isacharparenleft}g\ x{\isacharparenright}} \hfill (\isa{o{\isacharunderscore}apply})
    \item \isa{{\isasymlbrakk}P\ {\isasymLongrightarrow}\ Q{\isacharsemicolon}\ Q\ {\isasymLongrightarrow}\ P{\isasymrbrakk}\ {\isasymLongrightarrow}\ P\ {\isacharequal}\ Q} \hfill (\isa{iffI})
  \end{itemize} 

  La demostración automática es%
\end{isamarkuptext}\isamarkuptrue%
\isacommand{lemma}\isamarkupfalse%
\ inyectivaut{\isacharcolon}\isanewline
\ \ \isakeyword{assumes}\ {\isachardoublequoteopen}inj\ f{\isachardoublequoteclose}\isanewline
\ \ \isakeyword{shows}\ {\isachardoublequoteopen}{\isasymforall}g\ h{\isachardot}\ {\isacharparenleft}f\ {\isasymcirc}\ g\ {\isacharequal}\ f\ {\isasymcirc}\ h{\isacharparenright}\ {\isasymlongrightarrow}\ {\isacharparenleft}g\ {\isacharequal}\ h{\isacharparenright}{\isachardoublequoteclose}\isanewline
%
\isadelimproof
\ \ %
\endisadelimproof
%
\isatagproof
\isacommand{using}\isamarkupfalse%
\ assms\isanewline
\ \ \isacommand{by}\isamarkupfalse%
\ {\isacharparenleft}auto\ simp\ add{\isacharcolon}\ inj{\isacharunderscore}on{\isacharunderscore}def\ fun{\isacharunderscore}eq{\isacharunderscore}iff{\isacharparenright}%
\endisatagproof
{\isafoldproof}%
%
\isadelimproof
\ \isanewline
%
\endisadelimproof
\isanewline
\isacommand{lemma}\isamarkupfalse%
\ inyectivaut{\isadigit{2}}{\isacharcolon}\ \isanewline
\ \ \isakeyword{assumes}\ {\isachardoublequoteopen}{\isasymforall}g\ h{\isachardot}\ {\isacharparenleft}{\isacharparenleft}f\ {\isasymcirc}\ g\ {\isacharequal}\ f\ {\isasymcirc}\ h{\isacharparenright}\ {\isasymlongrightarrow}\ {\isacharparenleft}g\ {\isacharequal}\ h{\isacharparenright}{\isacharparenright}{\isachardoublequoteclose}\isanewline
\ \ \isakeyword{shows}\ {\isachardoublequoteopen}inj\ f{\isachardoublequoteclose}\isanewline
%
\isadelimproof
\ \ %
\endisadelimproof
%
\isatagproof
\isacommand{using}\isamarkupfalse%
\ assms\isanewline
\ \ \isacommand{oops}\isamarkupfalse%
%
\endisatagproof
{\isafoldproof}%
%
\isadelimproof
%
\endisadelimproof
%
\begin{isamarkuptext}%
La demostración declarativa%
\end{isamarkuptext}\isamarkuptrue%
\isacommand{lemma}\isamarkupfalse%
\ inyectdeclarada{\isacharcolon}\isanewline
\ \ \isakeyword{assumes}\ {\isachardoublequoteopen}inj\ f{\isachardoublequoteclose}\isanewline
\ \ \isakeyword{shows}\ {\isachardoublequoteopen}{\isasymforall}g\ h{\isachardot}{\isacharparenleft}f\ {\isasymcirc}\ g\ {\isacharequal}\ f\ {\isasymcirc}\ h{\isacharparenright}\ {\isasymlongrightarrow}\ {\isacharparenleft}g\ {\isacharequal}\ h{\isacharparenright}{\isachardoublequoteclose}\isanewline
%
\isadelimproof
%
\endisadelimproof
%
\isatagproof
\isacommand{proof}\isamarkupfalse%
\isanewline
\ \ \isacommand{fix}\isamarkupfalse%
\ g{\isacharcolon}{\isacharcolon}\ {\isachardoublequoteopen}{\isacharprime}c\ {\isasymRightarrow}\ {\isacharprime}a{\isachardoublequoteclose}\isanewline
\ \ \isacommand{show}\isamarkupfalse%
\ {\isachardoublequoteopen}{\isasymforall}h{\isachardot}{\isacharparenleft}f\ {\isasymcirc}\ g\ {\isacharequal}\ f\ {\isasymcirc}\ h{\isacharparenright}\ {\isasymlongrightarrow}\ {\isacharparenleft}g\ {\isacharequal}\ h{\isacharparenright}{\isachardoublequoteclose}\isanewline
\ \ \isacommand{proof}\isamarkupfalse%
\ {\isacharparenleft}rule\ allI{\isacharparenright}\isanewline
\ \ \ \ \isacommand{fix}\isamarkupfalse%
\ h\isanewline
\ \ \ \ \isacommand{show}\isamarkupfalse%
\ {\isachardoublequoteopen}f\ {\isasymcirc}\ g\ {\isacharequal}\ f\ {\isasymcirc}\ h\ {\isasymlongrightarrow}\ {\isacharparenleft}g\ {\isacharequal}\ h{\isacharparenright}{\isachardoublequoteclose}\isanewline
\ \ \ \ \isacommand{proof}\isamarkupfalse%
\ {\isacharparenleft}rule\ impI{\isacharparenright}\isanewline
\ \ \ \ \ \ \isacommand{assume}\isamarkupfalse%
\ {\isachardoublequoteopen}f\ {\isasymcirc}\ g\ {\isacharequal}\ f\ {\isasymcirc}\ h{\isachardoublequoteclose}\isanewline
\ \ \ \ \ \ \isacommand{show}\isamarkupfalse%
\ {\isachardoublequoteopen}g\ {\isacharequal}\ h{\isachardoublequoteclose}\isanewline
\ \ \ \ \ \ \isacommand{proof}\isamarkupfalse%
\ \isanewline
\ \ \ \ \ \ \ \ \isacommand{fix}\isamarkupfalse%
\ x\isanewline
\ \ \ \ \ \ \ \ \isacommand{have}\isamarkupfalse%
\ \ {\isachardoublequoteopen}{\isacharparenleft}f\ {\isasymcirc}\ g{\isacharparenright}{\isacharparenleft}x{\isacharparenright}\ {\isacharequal}\ {\isacharparenleft}f\ {\isasymcirc}\ h{\isacharparenright}{\isacharparenleft}x{\isacharparenright}{\isachardoublequoteclose}\ \isacommand{using}\isamarkupfalse%
\ {\isacharbackquoteopen}f\ {\isasymcirc}\ g\ {\isacharequal}\ f\ {\isasymcirc}\ h{\isacharbackquoteclose}\ \isacommand{by}\isamarkupfalse%
\ simp\isanewline
\ \ \ \ \ \ \ \ \isacommand{then}\isamarkupfalse%
\ \isacommand{have}\isamarkupfalse%
\ {\isachardoublequoteopen}f{\isacharparenleft}g{\isacharparenleft}x{\isacharparenright}{\isacharparenright}\ {\isacharequal}\ f{\isacharparenleft}h{\isacharparenleft}x{\isacharparenright}{\isacharparenright}{\isachardoublequoteclose}\ \isacommand{by}\isamarkupfalse%
\ simp\isanewline
\ \ \ \ \ \ \ \ \isacommand{thus}\isamarkupfalse%
\ \ {\isachardoublequoteopen}g{\isacharparenleft}x{\isacharparenright}\ {\isacharequal}\ h{\isacharparenleft}x{\isacharparenright}{\isachardoublequoteclose}\ \isacommand{using}\isamarkupfalse%
\ {\isacharbackquoteopen}inj\ f{\isacharbackquoteclose}\ \isacommand{by}\isamarkupfalse%
\ {\isacharparenleft}simp\ add{\isacharcolon}inj{\isacharunderscore}on{\isacharunderscore}def{\isacharparenright}\isanewline
\ \ \ \ \ \ \isacommand{qed}\isamarkupfalse%
\isanewline
\ \ \ \ \isacommand{qed}\isamarkupfalse%
\isanewline
\ \ \isacommand{qed}\isamarkupfalse%
\isanewline
\isacommand{qed}\isamarkupfalse%
%
\endisatagproof
{\isafoldproof}%
%
\isadelimproof
\isanewline
%
\endisadelimproof
\isanewline
\isacommand{declare}\isamarkupfalse%
\ {\isacharbrackleft}{\isacharbrackleft}show{\isacharunderscore}types{\isacharbrackright}{\isacharbrackright}\isanewline
\isanewline
\isacommand{lemma}\isamarkupfalse%
\ inyectdeclarada{\isadigit{2}}{\isacharcolon}\isanewline
\ \ \isakeyword{fixes}\ f\ {\isacharcolon}{\isacharcolon}\ {\isachardoublequoteopen}{\isacharprime}b\ {\isasymRightarrow}\ {\isacharprime}c{\isachardoublequoteclose}\ \isanewline
\ \ \isakeyword{assumes}\ {\isachardoublequoteopen}{\isasymforall}{\isacharparenleft}g\ {\isacharcolon}{\isacharcolon}\ {\isacharprime}a\ {\isasymRightarrow}\ {\isacharprime}b{\isacharparenright}\ {\isacharparenleft}h\ {\isacharcolon}{\isacharcolon}\ {\isacharprime}a\ {\isasymRightarrow}\ {\isacharprime}b{\isacharparenright}{\isachardot}\isanewline
\ \ \ \ \ \ \ \ \ {\isacharparenleft}f\ {\isasymcirc}\ g\ {\isacharequal}\ f\ {\isasymcirc}\ h\ {\isasymlongrightarrow}\ g\ {\isacharequal}\ h{\isacharparenright}{\isachardoublequoteclose}\isanewline
\isakeyword{shows}\ {\isachardoublequoteopen}\ inj\ f{\isachardoublequoteclose}\isanewline
%
\isadelimproof
%
\endisadelimproof
%
\isatagproof
\isacommand{proof}\isamarkupfalse%
\ {\isacharparenleft}rule\ injI{\isacharparenright}\isanewline
\ \ \isacommand{fix}\isamarkupfalse%
\ a\ b\ \isanewline
\ \ \isacommand{assume}\isamarkupfalse%
\ {\isadigit{3}}{\isacharcolon}\ {\isachardoublequoteopen}f\ a\ {\isacharequal}\ f\ b\ {\isachardoublequoteclose}\isanewline
\ \ \isacommand{let}\isamarkupfalse%
\ {\isacharquery}g\ {\isacharequal}\ {\isachardoublequoteopen}{\isasymlambda}x\ {\isacharcolon}{\isacharcolon}\ {\isacharprime}a{\isachardot}\ a{\isachardoublequoteclose}\isanewline
\ \ \isacommand{let}\isamarkupfalse%
\ {\isacharquery}h\ {\isacharequal}\ {\isachardoublequoteopen}{\isasymlambda}x\ {\isacharcolon}{\isacharcolon}\ {\isacharprime}a{\isachardot}\ b{\isachardoublequoteclose}\isanewline
\ \ \isacommand{have}\isamarkupfalse%
\ {\isachardoublequoteopen}{\isasymforall}{\isacharparenleft}h\ {\isacharcolon}{\isacharcolon}\ {\isacharprime}a\ {\isasymRightarrow}\ {\isacharprime}b{\isacharparenright}{\isachardot}\ {\isacharparenleft}f\ {\isasymcirc}\ {\isacharquery}g\ {\isacharequal}\ f\ {\isasymcirc}\ h\ {\isasymlongrightarrow}\ {\isacharquery}g\ {\isacharequal}\ h{\isacharparenright}{\isachardoublequoteclose}\isanewline
\ \ \ \ \isacommand{using}\isamarkupfalse%
\ assms\ \isacommand{by}\isamarkupfalse%
\ {\isacharparenleft}rule\ allE{\isacharparenright}\isanewline
\ \ \isacommand{hence}\isamarkupfalse%
\ {\isadigit{1}}{\isacharcolon}\ {\isachardoublequoteopen}\ {\isacharparenleft}f\ {\isasymcirc}\ {\isacharquery}g\ {\isacharequal}\ f\ {\isasymcirc}\ {\isacharquery}h\ {\isasymlongrightarrow}\ {\isacharquery}g\ {\isacharequal}\ {\isacharquery}h{\isacharparenright}{\isachardoublequoteclose}\ \ \isacommand{by}\isamarkupfalse%
\ {\isacharparenleft}rule\ allE{\isacharparenright}\ \isanewline
\ \ \isacommand{have}\isamarkupfalse%
\ {\isadigit{2}}{\isacharcolon}\ {\isachardoublequoteopen}f\ {\isasymcirc}\ {\isacharquery}g\ {\isacharequal}\ f\ {\isasymcirc}\ {\isacharquery}h{\isachardoublequoteclose}\ \isanewline
\ \ \isacommand{proof}\isamarkupfalse%
\ \isanewline
\ \ \ \ \isacommand{fix}\isamarkupfalse%
\ x\isanewline
\ \ \ \ \isacommand{have}\isamarkupfalse%
\ {\isachardoublequoteopen}\ {\isacharparenleft}f\ {\isasymcirc}\ {\isacharparenleft}{\isasymlambda}x\ {\isacharcolon}{\isacharcolon}\ {\isacharprime}a{\isachardot}\ a{\isacharparenright}{\isacharparenright}\ x\ {\isacharequal}\ f{\isacharparenleft}a{\isacharparenright}\ {\isachardoublequoteclose}\ \isacommand{by}\isamarkupfalse%
\ simp\isanewline
\ \ \ \ \isacommand{also}\isamarkupfalse%
\ \isacommand{have}\isamarkupfalse%
\ {\isachardoublequoteopen}{\isachardot}{\isachardot}{\isachardot}\ {\isacharequal}\ f{\isacharparenleft}b{\isacharparenright}{\isachardoublequoteclose}\ \isacommand{using}\isamarkupfalse%
\ {\isadigit{3}}\ \isacommand{by}\isamarkupfalse%
\ simp\isanewline
\ \ \ \ \isacommand{also}\isamarkupfalse%
\ \isacommand{have}\isamarkupfalse%
\ {\isachardoublequoteopen}{\isachardot}{\isachardot}{\isachardot}\ {\isacharequal}\ \ {\isacharparenleft}f\ {\isasymcirc}\ {\isacharparenleft}{\isasymlambda}x\ {\isacharcolon}{\isacharcolon}\ {\isacharprime}a{\isachardot}\ b{\isacharparenright}{\isacharparenright}\ x{\isachardoublequoteclose}\ \isacommand{by}\isamarkupfalse%
\ simp\isanewline
\ \ \ \ \isacommand{finally}\isamarkupfalse%
\ \isacommand{show}\isamarkupfalse%
\ {\isachardoublequoteopen}\ {\isacharparenleft}f\ {\isasymcirc}\ {\isacharparenleft}{\isasymlambda}x\ {\isacharcolon}{\isacharcolon}\ {\isacharprime}a{\isachardot}\ a{\isacharparenright}{\isacharparenright}\ x\ {\isacharequal}\ \ {\isacharparenleft}f\ {\isasymcirc}\ {\isacharparenleft}{\isasymlambda}x\ {\isacharcolon}{\isacharcolon}\ {\isacharprime}a{\isachardot}\ b{\isacharparenright}{\isacharparenright}\ x{\isachardoublequoteclose}\isanewline
\ \ \ \ \ \ \isacommand{by}\isamarkupfalse%
\ simp\isanewline
\ \ \isacommand{qed}\isamarkupfalse%
\isanewline
\ \ \isacommand{have}\isamarkupfalse%
\ {\isachardoublequoteopen}{\isacharquery}g\ {\isacharequal}\ {\isacharquery}h{\isachardoublequoteclose}\ \isacommand{using}\isamarkupfalse%
\ {\isadigit{1}}\ {\isadigit{2}}\ \isacommand{by}\isamarkupfalse%
\ {\isacharparenleft}rule\ mp{\isacharparenright}\isanewline
\ \ \isacommand{then}\isamarkupfalse%
\ \isacommand{show}\isamarkupfalse%
\ {\isachardoublequoteopen}\ a\ {\isacharequal}\ b{\isachardoublequoteclose}\ \isacommand{by}\isamarkupfalse%
\ meson\isanewline
\isacommand{qed}\isamarkupfalse%
%
\endisatagproof
{\isafoldproof}%
%
\isadelimproof
%
\endisadelimproof
%
\begin{isamarkuptext}%
Otra demostración declarativa es%
\end{isamarkuptext}\isamarkuptrue%
\isacommand{lemma}\isamarkupfalse%
\ inyectdetalladacorta{\isadigit{1}}{\isacharcolon}\isanewline
\ \ \isakeyword{assumes}\ {\isachardoublequoteopen}inj\ f{\isachardoublequoteclose}\isanewline
\ \ \isakeyword{shows}\ {\isachardoublequoteopen}{\isacharparenleft}f\ {\isasymcirc}\ g\ {\isacharequal}\ f\ {\isasymcirc}\ h{\isacharparenright}\ {\isasymlongrightarrow}{\isacharparenleft}g\ {\isacharequal}\ h{\isacharparenright}{\isachardoublequoteclose}\isanewline
%
\isadelimproof
%
\endisadelimproof
%
\isatagproof
\isacommand{proof}\isamarkupfalse%
\ \isanewline
\ \ \isacommand{assume}\isamarkupfalse%
\ {\isachardoublequoteopen}f\ {\isasymcirc}\ g\ {\isacharequal}\ f\ {\isasymcirc}\ h{\isachardoublequoteclose}\ \isanewline
\ \ \isacommand{then}\isamarkupfalse%
\ \isacommand{show}\isamarkupfalse%
\ {\isachardoublequoteopen}g\ {\isacharequal}\ h{\isachardoublequoteclose}\ \isacommand{using}\isamarkupfalse%
\ {\isacharbackquoteopen}inj\ f{\isacharbackquoteclose}\ \isacommand{by}\isamarkupfalse%
\ {\isacharparenleft}simp\ add{\isacharcolon}\ inj{\isacharunderscore}on{\isacharunderscore}def\ fun{\isacharunderscore}eq{\isacharunderscore}iff{\isacharparenright}\ \isanewline
\isacommand{qed}\isamarkupfalse%
%
\endisatagproof
{\isafoldproof}%
%
\isadelimproof
\isanewline
%
\endisadelimproof
\isanewline
\isacommand{lemma}\isamarkupfalse%
\ inyectdetalladacorta{\isadigit{2}}{\isacharcolon}\isanewline
\ \ \isakeyword{fixes}\ f\ {\isacharcolon}{\isacharcolon}\ {\isachardoublequoteopen}{\isacharprime}b\ {\isasymRightarrow}\ {\isacharprime}c{\isachardoublequoteclose}\ \isanewline
\ \ \isakeyword{assumes}\ {\isachardoublequoteopen}{\isasymforall}{\isacharparenleft}g\ {\isacharcolon}{\isacharcolon}\ {\isacharprime}a\ {\isasymRightarrow}\ {\isacharprime}b{\isacharparenright}\ {\isacharparenleft}h\ {\isacharcolon}{\isacharcolon}\ {\isacharprime}a\ {\isasymRightarrow}\ {\isacharprime}b{\isacharparenright}{\isachardot}\isanewline
\ \ \ \ \ \ \ \ \ {\isacharparenleft}f\ {\isasymcirc}\ g\ {\isacharequal}\ f\ {\isasymcirc}\ h\ {\isasymlongrightarrow}\ g\ {\isacharequal}\ h{\isacharparenright}{\isachardoublequoteclose}\isanewline
\ \ \isakeyword{shows}\ {\isachardoublequoteopen}\ inj\ f{\isachardoublequoteclose}\isanewline
%
\isadelimproof
%
\endisadelimproof
%
\isatagproof
\isacommand{proof}\isamarkupfalse%
\ {\isacharparenleft}rule\ injI{\isacharparenright}\isanewline
\ \ \isacommand{fix}\isamarkupfalse%
\ a\ b\ \isanewline
\ \ \isacommand{assume}\isamarkupfalse%
\ {\isadigit{1}}{\isacharcolon}\ {\isachardoublequoteopen}f\ a\ {\isacharequal}\ f\ b\ {\isachardoublequoteclose}\isanewline
\ \ \isacommand{let}\isamarkupfalse%
\ {\isacharquery}g\ {\isacharequal}\ {\isachardoublequoteopen}{\isasymlambda}x\ {\isacharcolon}{\isacharcolon}\ {\isacharprime}a{\isachardot}\ a{\isachardoublequoteclose}\isanewline
\ \ \isacommand{let}\isamarkupfalse%
\ {\isacharquery}h\ {\isacharequal}\ {\isachardoublequoteopen}{\isasymlambda}x\ {\isacharcolon}{\isacharcolon}\ {\isacharprime}a{\isachardot}\ b{\isachardoublequoteclose}\isanewline
\ \ \isacommand{have}\isamarkupfalse%
\ {\isadigit{2}}{\isacharcolon}\ {\isachardoublequoteopen}\ {\isacharparenleft}f\ {\isasymcirc}\ {\isacharquery}g\ {\isacharequal}\ f\ {\isasymcirc}\ {\isacharquery}h\ {\isasymlongrightarrow}\ {\isacharquery}g\ {\isacharequal}\ {\isacharquery}h{\isacharparenright}{\isachardoublequoteclose}\ \ \isacommand{using}\isamarkupfalse%
\ assms\ \isacommand{by}\isamarkupfalse%
\ blast\isanewline
\ \ \isacommand{have}\isamarkupfalse%
\ {\isadigit{3}}{\isacharcolon}\ {\isachardoublequoteopen}f\ {\isasymcirc}\ {\isacharquery}g\ {\isacharequal}\ f\ {\isasymcirc}\ {\isacharquery}h{\isachardoublequoteclose}\ \isanewline
\ \ \isacommand{proof}\isamarkupfalse%
\ \isanewline
\ \ \ \ \isacommand{fix}\isamarkupfalse%
\ x\isanewline
\ \ \ \ \isacommand{have}\isamarkupfalse%
\ {\isachardoublequoteopen}\ {\isacharparenleft}f\ {\isasymcirc}\ {\isacharparenleft}{\isasymlambda}x\ {\isacharcolon}{\isacharcolon}\ {\isacharprime}a{\isachardot}\ a{\isacharparenright}{\isacharparenright}\ x\ {\isacharequal}\ f{\isacharparenleft}a{\isacharparenright}\ {\isachardoublequoteclose}\ \isacommand{by}\isamarkupfalse%
\ simp\isanewline
\ \ \ \ \isacommand{also}\isamarkupfalse%
\ \isacommand{have}\isamarkupfalse%
\ {\isachardoublequoteopen}{\isachardot}{\isachardot}{\isachardot}\ {\isacharequal}\ f{\isacharparenleft}b{\isacharparenright}{\isachardoublequoteclose}\ \isacommand{using}\isamarkupfalse%
\ {\isadigit{1}}\ \isacommand{by}\isamarkupfalse%
\ simp\isanewline
\ \ \ \ \isacommand{also}\isamarkupfalse%
\ \isacommand{have}\isamarkupfalse%
\ {\isachardoublequoteopen}{\isachardot}{\isachardot}{\isachardot}\ {\isacharequal}\ \ {\isacharparenleft}f\ {\isasymcirc}\ {\isacharparenleft}{\isasymlambda}x\ {\isacharcolon}{\isacharcolon}\ {\isacharprime}a{\isachardot}\ b{\isacharparenright}{\isacharparenright}\ x{\isachardoublequoteclose}\ \isacommand{by}\isamarkupfalse%
\ simp\isanewline
\ \ \ \ \isacommand{finally}\isamarkupfalse%
\ \isacommand{show}\isamarkupfalse%
\ {\isachardoublequoteopen}\ {\isacharparenleft}f\ {\isasymcirc}\ {\isacharparenleft}{\isasymlambda}x\ {\isacharcolon}{\isacharcolon}\ {\isacharprime}a{\isachardot}\ a{\isacharparenright}{\isacharparenright}\ x\ {\isacharequal}\ \ {\isacharparenleft}f\ {\isasymcirc}\ {\isacharparenleft}{\isasymlambda}x\ {\isacharcolon}{\isacharcolon}\ {\isacharprime}a{\isachardot}\ b{\isacharparenright}{\isacharparenright}\ x{\isachardoublequoteclose}\isanewline
\ \ \ \ \ \ \isacommand{by}\isamarkupfalse%
\ simp\isanewline
\ \ \isacommand{qed}\isamarkupfalse%
\isanewline
\ \ \isacommand{show}\isamarkupfalse%
\ \ {\isachardoublequoteopen}\ a\ {\isacharequal}\ b{\isachardoublequoteclose}\ \isacommand{using}\isamarkupfalse%
\ {\isadigit{2}}\ {\isadigit{3}}\ \isacommand{by}\isamarkupfalse%
\ meson\isanewline
\isacommand{qed}\isamarkupfalse%
%
\endisatagproof
{\isafoldproof}%
%
\isadelimproof
%
\endisadelimproof
%
\begin{isamarkuptext}%
En consecuencia, la demostración de nuestro teorema:%
\end{isamarkuptext}\isamarkuptrue%
\isacommand{theorem}\isamarkupfalse%
\ caracterizacioninyectiva{\isacharcolon}\isanewline
\ \ {\isachardoublequoteopen}inj\ f\ {\isasymlongleftrightarrow}\ {\isacharparenleft}{\isasymforall}g\ h{\isachardot}\ {\isacharparenleft}f\ {\isasymcirc}\ g\ {\isacharequal}\ f\ {\isasymcirc}\ h{\isacharparenright}\ {\isasymlongrightarrow}\ {\isacharparenleft}g\ {\isacharequal}\ h{\isacharparenright}{\isacharparenright}{\isachardoublequoteclose}\isanewline
%
\isadelimproof
\ \ %
\endisadelimproof
%
\isatagproof
\isacommand{using}\isamarkupfalse%
\ inyectdetalladacorta{\isadigit{1}}\ inyectdetalladacorta{\isadigit{2}}\ \isacommand{by}\isamarkupfalse%
\ auto\isanewline
\isanewline
\isanewline
\isanewline
\isanewline
%
\endisatagproof
{\isafoldproof}%
%
\isadelimproof
%
\endisadelimproof
%
\isadelimtheory
%
\endisadelimtheory
%
\isatagtheory
%
\endisatagtheory
{\isafoldtheory}%
%
\isadelimtheory
%
\endisadelimtheory
%
\end{isabellebody}%
\endinput
%:%file=~/Escritorio/TFG/CancelacionInyectiva.thy%:%
%:%24=8%:%
%:%36=10%:%
%:%37=11%:%
%:%38=12%:%
%:%39=13%:%
%:%40=14%:%
%:%41=15%:%
%:%42=16%:%
%:%43=17%:%
%:%44=18%:%
%:%45=19%:%
%:%46=20%:%
%:%47=21%:%
%:%48=22%:%
%:%49=23%:%
%:%50=24%:%
%:%51=25%:%
%:%52=26%:%
%:%53=27%:%
%:%54=28%:%
%:%55=29%:%
%:%56=30%:%
%:%57=31%:%
%:%58=32%:%
%:%59=33%:%
%:%60=34%:%
%:%61=35%:%
%:%62=36%:%
%:%63=37%:%
%:%64=38%:%
%:%65=39%:%
%:%66=40%:%
%:%67=41%:%
%:%68=42%:%
%:%69=43%:%
%:%70=44%:%
%:%71=45%:%
%:%72=46%:%
%:%73=47%:%
%:%74=48%:%
%:%75=49%:%
%:%76=50%:%
%:%77=51%:%
%:%78=52%:%
%:%79=53%:%
%:%80=54%:%
%:%81=55%:%
%:%82=56%:%
%:%83=57%:%
%:%84=58%:%
%:%85=59%:%
%:%86=60%:%
%:%87=61%:%
%:%88=62%:%
%:%90=65%:%
%:%91=65%:%
%:%92=66%:%
%:%95=67%:%
%:%99=67%:%
%:%109=71%:%
%:%111=73%:%
%:%112=73%:%
%:%113=74%:%
%:%116=75%:%
%:%120=75%:%
%:%126=75%:%
%:%129=76%:%
%:%130=77%:%
%:%131=77%:%
%:%132=78%:%
%:%135=79%:%
%:%139=79%:%
%:%149=82%:%
%:%150=83%:%
%:%151=84%:%
%:%152=85%:%
%:%153=86%:%
%:%154=87%:%
%:%155=88%:%
%:%156=89%:%
%:%157=90%:%
%:%158=91%:%
%:%159=92%:%
%:%160=93%:%
%:%161=94%:%
%:%162=95%:%
%:%163=96%:%
%:%164=97%:%
%:%165=98%:%
%:%166=99%:%
%:%167=100%:%
%:%168=101%:%
%:%169=102%:%
%:%171=104%:%
%:%172=104%:%
%:%173=105%:%
%:%176=106%:%
%:%180=106%:%
%:%181=106%:%
%:%182=107%:%
%:%188=107%:%
%:%191=108%:%
%:%192=109%:%
%:%193=109%:%
%:%194=110%:%
%:%197=111%:%
%:%201=111%:%
%:%202=111%:%
%:%203=112%:%
%:%204=112%:%
%:%213=115%:%
%:%214=116%:%
%:%215=117%:%
%:%216=118%:%
%:%217=119%:%
%:%218=120%:%
%:%219=121%:%
%:%220=122%:%
%:%221=123%:%
%:%222=124%:%
%:%223=125%:%
%:%224=126%:%
%:%225=127%:%
%:%226=128%:%
%:%227=129%:%
%:%228=130%:%
%:%230=132%:%
%:%231=132%:%
%:%232=133%:%
%:%235=134%:%
%:%239=134%:%
%:%240=134%:%
%:%241=135%:%
%:%242=135%:%
%:%243=136%:%
%:%244=136%:%
%:%245=137%:%
%:%246=137%:%
%:%247=138%:%
%:%253=138%:%
%:%256=139%:%
%:%257=140%:%
%:%258=140%:%
%:%259=141%:%
%:%262=142%:%
%:%266=142%:%
%:%276=144%:%
%:%277=145%:%
%:%278=146%:%
%:%279=147%:%
%:%280=148%:%
%:%281=149%:%
%:%282=150%:%
%:%283=151%:%
%:%285=153%:%
%:%286=153%:%
%:%287=154%:%
%:%288=155%:%
%:%291=156%:%
%:%295=156%:%
%:%296=156%:%
%:%297=157%:%
%:%298=157%:%
%:%303=157%:%
%:%306=158%:%
%:%307=159%:%
%:%308=159%:%
%:%309=160%:%
%:%310=161%:%
%:%313=162%:%
%:%317=162%:%
%:%318=162%:%
%:%319=163%:%
%:%329=165%:%
%:%331=169%:%
%:%332=169%:%
%:%333=170%:%
%:%334=171%:%
%:%341=172%:%
%:%342=172%:%
%:%343=173%:%
%:%344=173%:%
%:%345=174%:%
%:%346=174%:%
%:%347=175%:%
%:%348=175%:%
%:%349=176%:%
%:%350=176%:%
%:%351=177%:%
%:%352=177%:%
%:%353=178%:%
%:%354=178%:%
%:%355=179%:%
%:%356=179%:%
%:%357=180%:%
%:%358=180%:%
%:%359=181%:%
%:%360=181%:%
%:%361=182%:%
%:%362=182%:%
%:%363=183%:%
%:%364=183%:%
%:%365=183%:%
%:%366=183%:%
%:%367=184%:%
%:%368=184%:%
%:%369=184%:%
%:%370=184%:%
%:%371=185%:%
%:%372=185%:%
%:%373=185%:%
%:%374=185%:%
%:%375=186%:%
%:%376=186%:%
%:%377=187%:%
%:%378=187%:%
%:%379=188%:%
%:%380=188%:%
%:%381=189%:%
%:%387=189%:%
%:%390=190%:%
%:%391=191%:%
%:%392=191%:%
%:%393=192%:%
%:%394=193%:%
%:%395=193%:%
%:%396=194%:%
%:%397=195%:%
%:%398=196%:%
%:%399=197%:%
%:%406=198%:%
%:%407=198%:%
%:%408=199%:%
%:%409=199%:%
%:%410=200%:%
%:%411=200%:%
%:%412=201%:%
%:%413=201%:%
%:%414=202%:%
%:%415=202%:%
%:%416=203%:%
%:%417=203%:%
%:%418=204%:%
%:%419=204%:%
%:%420=204%:%
%:%421=205%:%
%:%422=205%:%
%:%423=205%:%
%:%424=206%:%
%:%425=206%:%
%:%426=207%:%
%:%427=207%:%
%:%428=208%:%
%:%429=208%:%
%:%430=209%:%
%:%431=209%:%
%:%432=209%:%
%:%433=210%:%
%:%434=210%:%
%:%435=210%:%
%:%436=210%:%
%:%437=210%:%
%:%438=211%:%
%:%439=211%:%
%:%440=211%:%
%:%441=211%:%
%:%442=212%:%
%:%443=212%:%
%:%444=212%:%
%:%445=213%:%
%:%446=213%:%
%:%447=214%:%
%:%448=214%:%
%:%449=215%:%
%:%450=215%:%
%:%451=215%:%
%:%452=215%:%
%:%453=216%:%
%:%454=216%:%
%:%455=216%:%
%:%456=216%:%
%:%457=217%:%
%:%467=221%:%
%:%469=223%:%
%:%470=223%:%
%:%471=224%:%
%:%472=225%:%
%:%479=226%:%
%:%480=226%:%
%:%481=227%:%
%:%482=227%:%
%:%483=228%:%
%:%484=228%:%
%:%485=228%:%
%:%486=228%:%
%:%487=228%:%
%:%488=229%:%
%:%494=229%:%
%:%497=230%:%
%:%498=231%:%
%:%499=231%:%
%:%500=232%:%
%:%501=233%:%
%:%502=234%:%
%:%503=235%:%
%:%510=236%:%
%:%511=236%:%
%:%512=237%:%
%:%513=237%:%
%:%514=238%:%
%:%515=238%:%
%:%516=239%:%
%:%517=239%:%
%:%518=240%:%
%:%519=240%:%
%:%520=241%:%
%:%521=241%:%
%:%522=241%:%
%:%523=241%:%
%:%524=242%:%
%:%525=242%:%
%:%526=243%:%
%:%527=243%:%
%:%528=244%:%
%:%529=244%:%
%:%530=245%:%
%:%531=245%:%
%:%532=245%:%
%:%533=246%:%
%:%534=246%:%
%:%535=246%:%
%:%536=246%:%
%:%537=246%:%
%:%538=247%:%
%:%539=247%:%
%:%540=247%:%
%:%541=247%:%
%:%542=248%:%
%:%543=248%:%
%:%544=248%:%
%:%545=249%:%
%:%546=249%:%
%:%547=250%:%
%:%548=250%:%
%:%549=251%:%
%:%550=251%:%
%:%551=251%:%
%:%552=251%:%
%:%553=252%:%
%:%563=256%:%
%:%565=258%:%
%:%566=258%:%
%:%567=259%:%
%:%570=260%:%
%:%574=260%:%
%:%575=260%:%
%:%576=260%:%
%:%577=261%:%
%:%578=262%:%
%:%579=263%:%
%:%580=264%:%

%
\begin{isabellebody}%
\setisabellecontext{CancelacionSobreyectiva}%
%
\isadelimtheory
%
\endisadelimtheory
%
\isatagtheory
%
\endisatagtheory
{\isafoldtheory}%
%
\isadelimtheory
%
\endisadelimtheory
%
\isadelimdocument
%
\endisadelimdocument
%
\isatagdocument
%
\isamarkupsection{Cancelación de las funciones sobreyectivas%
}
\isamarkuptrue%
%
\endisatagdocument
{\isafolddocument}%
%
\isadelimdocument
%
\endisadelimdocument
%
\begin{isamarkuptext}%
El siguiente teorema prueba una caracterización de las funciones
 sobreyectivas, en otras palabras, las funciones sobreyectivas son
 epimorfismos en la categoría de conjuntos. Donde un epimorfismo es un
 homomorfismo sobreyectivo y la categoría de conjuntos es la categoría
 donde los objetos son conjuntos.


\begin {teorema}
  f es sobreyectiva si y solo si  para todas funciones g y h tal que g o f
 = h o f se tiene que g = h.
\end {teorema}
 
El teorema lo podemos dividir en dos lemas, ya que el teorema se
 demuestra por una doble implicación, luego vamos a dividir el teorema
 en las dos implicaciones.

\begin {lema}
  f es sobreyectiva entonces  para todas funciones g y h tal que g o f
 = h o f se tiene que g = h.
\end {lema}
\begin {demostracion}
\begin {itemize}
\item Supongamos que tenemos que $g \circ  f = h \circ f$, queremos probar que $g =
 h.$ Usando la definición de sobreyectividad $(\forall y \in Y,
 \exists x | y = f(x))$ y nuestra hipótesis, tenemos que:
$$g(y) = g(f(x)) = (g o f) (x) = (h o f) (x) = h(f(x)) = h(y)$$
\item Supongamos que $g = h$, hay que probar que $g o f = h o f.$ Usando
nuestra hipótesis, tenemos que:
$$ (g o f)(x) = g(f(x)) = h(f(x)) = (h o f) (x).$$
\end {itemize}
.
\end {demostracion}

\begin {lema}
 Si  para todas funciones g y h tal que g o f  = h o f se tiene
 que g = h entonces f es sobreyectiva.
\end {lema}


Su especificación es la siguiente, que la dividiremos en dos al igual que 
en la demostración a mano:%
\end{isamarkuptext}\isamarkuptrue%
\isacommand{theorem}\isamarkupfalse%
\isanewline
\ {\isachardoublequoteopen}surj\ f\ {\isasymlongleftrightarrow}\ {\isacharparenleft}g\ {\isasymcirc}\ f\ {\isacharequal}\ h\ {\isasymcirc}\ f{\isacharparenright}\ {\isacharequal}\ {\isacharparenleft}g\ {\isacharequal}\ h{\isacharparenright}{\isachardoublequoteclose}\isanewline
%
\isadelimproof
\ \ %
\endisadelimproof
%
\isatagproof
\isacommand{oops}\isamarkupfalse%
%
\endisatagproof
{\isafoldproof}%
%
\isadelimproof
\isanewline
%
\endisadelimproof
\isanewline
\isacommand{lemma}\isamarkupfalse%
\ \isanewline
{\isachardoublequoteopen}surj\ f\ {\isasymLongrightarrow}\ \ {\isacharparenleft}g\ {\isasymcirc}\ f\ {\isacharequal}\ h\ {\isasymcirc}\ f{\isacharparenright}\ {\isacharequal}\ {\isacharparenleft}g\ {\isacharequal}\ h{\isacharparenright}{\isachardoublequoteclose}\isanewline
%
\isadelimproof
\ \ %
\endisadelimproof
%
\isatagproof
\isacommand{oops}\isamarkupfalse%
%
\endisatagproof
{\isafoldproof}%
%
\isadelimproof
\isanewline
%
\endisadelimproof
\isanewline
\isacommand{lemma}\isamarkupfalse%
\ \isanewline
{\isachardoublequoteopen}{\isasymforall}g\ h{\isachardot}\ {\isacharparenleft}g\ {\isasymcirc}\ f\ {\isacharequal}\ h\ {\isasymcirc}\ f\ {\isasymlongrightarrow}\ g\ {\isacharequal}\ h{\isacharparenright}\ {\isasymlongrightarrow}\ surj\ f{\isachardoublequoteclose}\isanewline
%
\isadelimproof
\ \ %
\endisadelimproof
%
\isatagproof
\isacommand{oops}\isamarkupfalse%
%
\endisatagproof
{\isafoldproof}%
%
\isadelimproof
%
\endisadelimproof
%
\begin{isamarkuptext}%
En la especificación anterior, \isa{surj\ f} es una abreviatura de 
  \isa{range\ f\ {\isacharequal}\ UNIV}, donde \isa{range\ f} es el rango o imagen
de la función f.
 Por otra parte, \isa{UNIV} es el conjunto universal definido en la 
  teoría \href{http://bit.ly/2XtHCW6}{Set.thy} como una abreviatura de 
  \isa{top} que, a su vez está definido en la teoría 
  \href{http://bit.ly/2Xyj9Pe}{Orderings.thy} mediante la siguiente
  propiedad 
  \begin{itemize}
    \item[] \isa{\mbox{}\inferrule{\mbox{ordering{\isacharunderscore}top\ less{\isacharunderscore}eq\ less\ top}}{\mbox{less{\isacharunderscore}eq\ a\ top}}} 
      \hfill (\isa{ordering{\isacharunderscore}top{\isachardot}extremum})
  \end{itemize} 
Además queda añadir que la teoría donde se encuentra definido \isa{surj\ f}
 es en \href{http://bit.ly/2XuPQx5}{Fun.thy}. Esta teoría contiene la
 definicion \isa{surj{\isacharunderscore}def}.
 \begin{itemize}
    \item[] \isa{surj\ f\ {\isacharequal}\ {\isacharparenleft}{\isasymforall}y{\isachardot}\ {\isasymexists}x{\isachardot}\ y\ {\isacharequal}\ f\ x{\isacharparenright}} \hfill (\isa{inj{\isacharunderscore}on{\isacharunderscore}def})
  \end{itemize} 

Presentaremos distintas demostraciones del teorema. La primera es la
 detallada:%
\end{isamarkuptext}\isamarkuptrue%
\isacommand{lemma}\isamarkupfalse%
\ \isanewline
\ \ \isakeyword{assumes}\ {\isachardoublequoteopen}surj\ f{\isachardoublequoteclose}\ \isanewline
\ \ \isakeyword{shows}\ {\isachardoublequoteopen}{\isacharparenleft}\ g\ {\isasymcirc}\ f\ {\isacharequal}\ h\ {\isasymcirc}\ f\ {\isacharparenright}\ {\isacharequal}\ {\isacharparenleft}g\ {\isacharequal}\ h{\isacharparenright}{\isachardoublequoteclose}\isanewline
%
\isadelimproof
%
\endisadelimproof
%
\isatagproof
\isacommand{proof}\isamarkupfalse%
\ {\isacharparenleft}rule\ iffI{\isacharparenright}\isanewline
\ \ \isacommand{assume}\isamarkupfalse%
\ {\isadigit{1}}{\isacharcolon}\ {\isachardoublequoteopen}\ g\ {\isasymcirc}\ f\ {\isacharequal}\ h\ {\isasymcirc}\ f\ {\isachardoublequoteclose}\isanewline
\ \ \isacommand{show}\isamarkupfalse%
\ {\isachardoublequoteopen}g\ {\isacharequal}\ h{\isachardoublequoteclose}\ \isanewline
\ \ \isacommand{proof}\isamarkupfalse%
\ \isanewline
\ \ \ \ \isacommand{fix}\isamarkupfalse%
\ x\isanewline
\isanewline
\ \ \ \ \isacommand{have}\isamarkupfalse%
\ {\isachardoublequoteopen}\ {\isasymexists}y\ {\isachardot}\ x\ {\isacharequal}\ f{\isacharparenleft}y{\isacharparenright}{\isachardoublequoteclose}\ \isacommand{using}\isamarkupfalse%
\ assms\ \isacommand{by}\isamarkupfalse%
\ {\isacharparenleft}simp\ add{\isacharcolon}surj{\isacharunderscore}def{\isacharparenright}\isanewline
\ \ \ \ \isacommand{then}\isamarkupfalse%
\ \isacommand{obtain}\isamarkupfalse%
\ {\isachardoublequoteopen}y{\isachardoublequoteclose}\ \isakeyword{where}\ {\isadigit{2}}{\isacharcolon}{\isachardoublequoteopen}x\ {\isacharequal}\ f{\isacharparenleft}y{\isacharparenright}{\isachardoublequoteclose}\ \isacommand{by}\isamarkupfalse%
\ {\isacharparenleft}rule\ exE{\isacharparenright}\isanewline
\ \ \ \ \isacommand{then}\isamarkupfalse%
\ \isacommand{have}\isamarkupfalse%
\ {\isachardoublequoteopen}g{\isacharparenleft}x{\isacharparenright}\ {\isacharequal}\ g{\isacharparenleft}f{\isacharparenleft}y{\isacharparenright}{\isacharparenright}{\isachardoublequoteclose}\ \isacommand{by}\isamarkupfalse%
\ simp\isanewline
\ \ \ \ \isacommand{also}\isamarkupfalse%
\ \isacommand{have}\isamarkupfalse%
\ {\isachardoublequoteopen}{\isachardot}{\isachardot}{\isachardot}\ {\isacharequal}\ {\isacharparenleft}g\ {\isasymcirc}\ f{\isacharparenright}\ {\isacharparenleft}y{\isacharparenright}\ \ {\isachardoublequoteclose}\ \isacommand{by}\isamarkupfalse%
\ simp\isanewline
\ \ \ \ \isacommand{also}\isamarkupfalse%
\ \isacommand{have}\isamarkupfalse%
\ {\isachardoublequoteopen}{\isachardot}{\isachardot}{\isachardot}\ {\isacharequal}\ {\isacharparenleft}h\ {\isasymcirc}\ f{\isacharparenright}\ {\isacharparenleft}y{\isacharparenright}{\isachardoublequoteclose}\ \isacommand{using}\isamarkupfalse%
\ {\isadigit{1}}\ \isacommand{by}\isamarkupfalse%
\ simp\isanewline
\ \ \ \ \isacommand{also}\isamarkupfalse%
\ \isacommand{have}\isamarkupfalse%
\ {\isachardoublequoteopen}{\isachardot}{\isachardot}{\isachardot}\ {\isacharequal}\ h{\isacharparenleft}f{\isacharparenleft}y{\isacharparenright}{\isacharparenright}{\isachardoublequoteclose}\ \isacommand{by}\isamarkupfalse%
\ simp\isanewline
\ \ \ \ \isacommand{also}\isamarkupfalse%
\ \isacommand{have}\isamarkupfalse%
\ {\isachardoublequoteopen}{\isachardot}{\isachardot}{\isachardot}\ {\isacharequal}\ h{\isacharparenleft}x{\isacharparenright}{\isachardoublequoteclose}\ \isacommand{using}\isamarkupfalse%
\ {\isadigit{2}}\ \ \ \isacommand{by}\isamarkupfalse%
\ {\isacharparenleft}simp\ add{\isacharcolon}\ {\isacartoucheopen}x\ {\isacharequal}\ f\ y{\isacartoucheclose}{\isacharparenright}\isanewline
\ \ \ \ \isacommand{finally}\isamarkupfalse%
\ \isacommand{show}\isamarkupfalse%
\ \ {\isachardoublequoteopen}\ g{\isacharparenleft}x{\isacharparenright}\ {\isacharequal}\ h{\isacharparenleft}x{\isacharparenright}\ {\isachardoublequoteclose}\ \isacommand{by}\isamarkupfalse%
\ simp\isanewline
\ \ \isacommand{qed}\isamarkupfalse%
\isanewline
\isacommand{next}\isamarkupfalse%
\isanewline
\ \ \isacommand{assume}\isamarkupfalse%
\ {\isachardoublequoteopen}g\ {\isacharequal}\ h{\isachardoublequoteclose}\ \isanewline
\ \ \isacommand{show}\isamarkupfalse%
\ {\isachardoublequoteopen}g\ {\isasymcirc}\ f\ {\isacharequal}\ h\ {\isasymcirc}\ f{\isachardoublequoteclose}\isanewline
\ \ \isacommand{proof}\isamarkupfalse%
\isanewline
\ \ \ \ \isacommand{fix}\isamarkupfalse%
\ x\isanewline
\ \ \ \ \isacommand{have}\isamarkupfalse%
\ {\isachardoublequoteopen}{\isacharparenleft}g\ {\isasymcirc}\ f{\isacharparenright}\ x\ {\isacharequal}\ g{\isacharparenleft}f{\isacharparenleft}x{\isacharparenright}{\isacharparenright}{\isachardoublequoteclose}\ \isacommand{by}\isamarkupfalse%
\ simp\isanewline
\ \ \ \ \isacommand{also}\isamarkupfalse%
\ \isacommand{have}\isamarkupfalse%
\ {\isachardoublequoteopen}{\isasymdots}\ {\isacharequal}\ h{\isacharparenleft}f{\isacharparenleft}x{\isacharparenright}{\isacharparenright}{\isachardoublequoteclose}\ \isacommand{using}\isamarkupfalse%
\ {\isacharbackquoteopen}g\ {\isacharequal}\ h{\isacharbackquoteclose}\ \isacommand{by}\isamarkupfalse%
\ simp\isanewline
\ \ \ \ \isacommand{also}\isamarkupfalse%
\ \isacommand{have}\isamarkupfalse%
\ {\isachardoublequoteopen}{\isasymdots}\ {\isacharequal}\ {\isacharparenleft}h\ {\isasymcirc}\ f{\isacharparenright}\ x{\isachardoublequoteclose}\ \isacommand{by}\isamarkupfalse%
\ simp\isanewline
\ \ \ \ \isacommand{finally}\isamarkupfalse%
\ \isacommand{show}\isamarkupfalse%
\ {\isachardoublequoteopen}{\isacharparenleft}g\ {\isasymcirc}\ f{\isacharparenright}\ x\ {\isacharequal}\ {\isacharparenleft}h\ {\isasymcirc}\ f{\isacharparenright}\ x{\isachardoublequoteclose}\ \isacommand{by}\isamarkupfalse%
\ simp\isanewline
\ \ \isacommand{qed}\isamarkupfalse%
\isanewline
\isacommand{qed}\isamarkupfalse%
%
\endisatagproof
{\isafoldproof}%
%
\isadelimproof
%
\endisadelimproof
%
\begin{isamarkuptext}%
En la demostración hemos introducido: 
 \begin{itemize}
    \item[] \isa{\mbox{}\inferrule{\mbox{{\isasymexists}x{\isachardot}\ P\ x}\\\ \mbox{{\isasymAnd}x{\isachardot}\ \mbox{}\inferrule{\mbox{P\ x}}{\mbox{Q}}}}{\mbox{Q}}} 
      \hfill (\isa{rule\ exE}) 
  \end{itemize} 
 \begin{itemize}
    \item[] \isa{{\isasymlbrakk}P\ {\isasymLongrightarrow}\ Q{\isacharsemicolon}\ Q\ {\isasymLongrightarrow}\ P{\isasymrbrakk}\ {\isasymLongrightarrow}\ P\ {\isacharequal}\ Q} 
      \hfill (\isa{iffI})
  \end{itemize} 

La demostración aplicativa es:%
\end{isamarkuptext}\isamarkuptrue%
\isacommand{lemma}\isamarkupfalse%
\ {\isachardoublequoteopen}surj\ f\ {\isasymLongrightarrow}\ {\isacharparenleft}{\isacharparenleft}g\ {\isasymcirc}\ f{\isacharparenright}\ {\isacharequal}\ {\isacharparenleft}h\ {\isasymcirc}\ f{\isacharparenright}\ {\isacharparenright}\ {\isacharequal}\ {\isacharparenleft}g\ {\isacharequal}\ h{\isacharparenright}{\isachardoublequoteclose}\isanewline
%
\isadelimproof
\ \ %
\endisadelimproof
%
\isatagproof
\isacommand{apply}\isamarkupfalse%
\ {\isacharparenleft}simp\ add{\isacharcolon}\ surj{\isacharunderscore}def\ fun{\isacharunderscore}eq{\isacharunderscore}iff{\isacharparenright}\isanewline
\ \ \isacommand{apply}\isamarkupfalse%
\ {\isacharparenleft}rule\ iffI{\isacharparenright}\isanewline
\ \ \ \isacommand{prefer}\isamarkupfalse%
\ {\isadigit{2}}\isanewline
\ \ \isacommand{apply}\isamarkupfalse%
\ auto\isanewline
\ \isanewline
\ \ \isacommand{apply}\isamarkupfalse%
\ \ metis\isanewline
\isanewline
\ \ \isacommand{done}\isamarkupfalse%
%
\endisatagproof
{\isafoldproof}%
%
\isadelimproof
\isanewline
%
\endisadelimproof
\isanewline
\isacommand{lemma}\isamarkupfalse%
\ {\isachardoublequoteopen}surj\ f\ {\isasymLongrightarrow}\ {\isacharparenleft}{\isacharparenleft}g\ {\isasymcirc}\ f{\isacharparenright}\ {\isacharequal}\ {\isacharparenleft}h\ {\isasymcirc}\ f{\isacharparenright}\ {\isacharparenright}\ {\isacharequal}\ {\isacharparenleft}g\ {\isacharequal}\ h{\isacharparenright}{\isachardoublequoteclose}\isanewline
%
\isadelimproof
\ \ %
\endisadelimproof
%
\isatagproof
\isacommand{apply}\isamarkupfalse%
\ {\isacharparenleft}simp\ add{\isacharcolon}\ surj{\isacharunderscore}def\ fun{\isacharunderscore}eq{\isacharunderscore}iff\ {\isacharparenright}\ \isanewline
\ \ \isacommand{by}\isamarkupfalse%
\ metis%
\endisatagproof
{\isafoldproof}%
%
\isadelimproof
%
\endisadelimproof
%
\begin{isamarkuptext}%
En esta demostración hemos introducido:
 \begin{itemize}
    \item[] \isa{{\isacharparenleft}f\ {\isacharequal}\ g{\isacharparenright}\ {\isacharequal}\ {\isacharparenleft}{\isasymforall}x{\isachardot}\ f\ x\ {\isacharequal}\ g\ x{\isacharparenright}} 
      \hfill (\isa{fun{\isacharunderscore}eq{\isacharunderscore}iff})
  \end{itemize}%
\end{isamarkuptext}\isamarkuptrue%
%
\isadelimtheory
%
\endisadelimtheory
%
\isatagtheory
%
\endisatagtheory
{\isafoldtheory}%
%
\isadelimtheory
%
\endisadelimtheory
%
\end{isabellebody}%
\endinput
%:%file=~/Escritorio/TFG/EjerciciosDELMF/CancelacionSobreyectiva.thy%:%
%:%24=7%:%
%:%36=10%:%
%:%37=11%:%
%:%38=12%:%
%:%39=13%:%
%:%40=14%:%
%:%41=15%:%
%:%42=16%:%
%:%43=17%:%
%:%44=18%:%
%:%45=19%:%
%:%46=20%:%
%:%47=21%:%
%:%48=22%:%
%:%49=23%:%
%:%50=24%:%
%:%51=25%:%
%:%52=26%:%
%:%53=27%:%
%:%54=28%:%
%:%55=29%:%
%:%56=30%:%
%:%57=31%:%
%:%58=32%:%
%:%59=33%:%
%:%60=34%:%
%:%61=35%:%
%:%62=36%:%
%:%63=37%:%
%:%64=38%:%
%:%65=39%:%
%:%66=40%:%
%:%67=41%:%
%:%68=42%:%
%:%69=43%:%
%:%70=44%:%
%:%71=45%:%
%:%72=46%:%
%:%73=47%:%
%:%74=48%:%
%:%75=49%:%
%:%76=50%:%
%:%78=53%:%
%:%79=53%:%
%:%80=54%:%
%:%83=55%:%
%:%87=55%:%
%:%93=55%:%
%:%96=56%:%
%:%97=57%:%
%:%98=57%:%
%:%99=58%:%
%:%102=59%:%
%:%106=59%:%
%:%112=59%:%
%:%115=60%:%
%:%116=61%:%
%:%117=61%:%
%:%118=62%:%
%:%121=63%:%
%:%125=63%:%
%:%135=67%:%
%:%136=68%:%
%:%137=69%:%
%:%138=70%:%
%:%139=71%:%
%:%140=72%:%
%:%141=73%:%
%:%142=74%:%
%:%143=75%:%
%:%144=76%:%
%:%145=77%:%
%:%146=78%:%
%:%147=79%:%
%:%148=80%:%
%:%149=81%:%
%:%150=82%:%
%:%151=83%:%
%:%152=84%:%
%:%153=85%:%
%:%154=86%:%
%:%155=87%:%
%:%157=90%:%
%:%158=90%:%
%:%159=91%:%
%:%160=92%:%
%:%167=93%:%
%:%168=93%:%
%:%169=94%:%
%:%170=94%:%
%:%171=95%:%
%:%172=95%:%
%:%173=96%:%
%:%174=96%:%
%:%175=97%:%
%:%176=97%:%
%:%177=98%:%
%:%178=99%:%
%:%179=99%:%
%:%180=99%:%
%:%181=99%:%
%:%182=100%:%
%:%183=100%:%
%:%184=100%:%
%:%185=100%:%
%:%186=101%:%
%:%187=101%:%
%:%188=101%:%
%:%189=101%:%
%:%190=102%:%
%:%191=102%:%
%:%192=102%:%
%:%193=102%:%
%:%194=103%:%
%:%195=103%:%
%:%196=103%:%
%:%197=103%:%
%:%198=103%:%
%:%199=104%:%
%:%200=104%:%
%:%201=104%:%
%:%202=104%:%
%:%203=105%:%
%:%204=105%:%
%:%205=105%:%
%:%206=105%:%
%:%207=105%:%
%:%208=106%:%
%:%209=106%:%
%:%210=106%:%
%:%211=106%:%
%:%212=107%:%
%:%213=107%:%
%:%214=108%:%
%:%215=108%:%
%:%216=109%:%
%:%217=109%:%
%:%218=110%:%
%:%219=110%:%
%:%220=111%:%
%:%221=111%:%
%:%222=112%:%
%:%223=112%:%
%:%224=113%:%
%:%225=113%:%
%:%226=113%:%
%:%227=114%:%
%:%228=114%:%
%:%229=114%:%
%:%230=114%:%
%:%231=114%:%
%:%232=115%:%
%:%233=115%:%
%:%234=115%:%
%:%235=115%:%
%:%236=116%:%
%:%237=116%:%
%:%238=116%:%
%:%239=116%:%
%:%240=117%:%
%:%241=117%:%
%:%242=118%:%
%:%252=121%:%
%:%253=122%:%
%:%254=123%:%
%:%255=124%:%
%:%256=125%:%
%:%257=126%:%
%:%258=127%:%
%:%259=128%:%
%:%260=129%:%
%:%261=130%:%
%:%262=131%:%
%:%264=133%:%
%:%265=133%:%
%:%268=134%:%
%:%272=134%:%
%:%273=134%:%
%:%274=135%:%
%:%275=135%:%
%:%276=136%:%
%:%277=136%:%
%:%278=137%:%
%:%279=137%:%
%:%280=138%:%
%:%281=139%:%
%:%282=139%:%
%:%283=140%:%
%:%284=141%:%
%:%290=141%:%
%:%293=142%:%
%:%294=143%:%
%:%295=143%:%
%:%298=144%:%
%:%302=144%:%
%:%303=144%:%
%:%304=145%:%
%:%305=145%:%
%:%314=148%:%
%:%315=149%:%
%:%316=150%:%
%:%317=151%:%
%:%318=152%:%

%
\begin{isabellebody}%
\setisabellecontext{ConjuntosFinitos}%
%
\isadelimtheory
%
\endisadelimtheory
%
\isatagtheory
%
\endisatagtheory
{\isafoldtheory}%
%
\isadelimtheory
%
\endisadelimtheory
%
\isadelimdocument
%
\endisadelimdocument
%
\isatagdocument
%
\isamarkupsection{Propiedad de los conjuntos finitos de números naturales%
}
\isamarkuptrue%
%
\endisatagdocument
{\isafolddocument}%
%
\isadelimdocument
%
\endisadelimdocument
%
\begin{isamarkuptext}%
El siguiente teorema es una propiedad que verifican todos los conjuntos finitos de números
  naturales  estudiado en el \href{http://bit.ly/2XBW6n2}{tema 10} de la
asignatura de LMF. Su enunciado es el siguiente 

  \begin{teorema} 
    Sea S un conjunto finito de números naturales.  Entonces todos los
 elementos de S son menores o iguales que la suma de los elementos de
 S, es decir, $$\forall m , m \in S \Longrightarrow m \leq \sum S$$ 
\newline
donde $\sum S $ denota la suma de todos los elementos de S.
  \end{teorema} 

\begin{demostracion}
La demostración del teorema la haremos por inducción sobre conjuntos
 finitos naturales.\\
Primero veamos el caso base, es decir, supongamos que $S = \emptyset$:

Tenemos que: $$\forall n, n \in \emptyset \Longrightarrow n \leq \sum
 \emptyset.$$\\
Ya hemos probado el caso base, veamos ahora el paso inductivo:
\newline
Sea S un conjunto finito para el que se cumple la hipótesis, es decir,
 todos los elementos de S son menores o iguales que la suma de todos sus
elementos, sea $a$ un elemento tal que $a \notin S$, ya que si $a \in S$
entonces la demostración es trivial.\\
Hay que probar: $$\forall n , n \in S \cup \{ a \} \Longrightarrow n
 \leq \sum (S \cup \{ a \})$$
Vamos a distinguir dos casos:\\
Caso 1: $n = a$ \\
Si $n = a$ tenemos que $n = a \leq a + \sum S = \sum ( S \cup \{ a
 \})$\\
Caso 2: $n \neq a$\\
Si $n \neq a$ tenemos que $a \notin S$ y que $n \in S \cup \{ a \}$,
 luego esto implica que $n \in S$ y usando la hipótesis de inducción
$$n \in S \Longrightarrow n \leq \sum S \leq \sum S + a = \sum (S \cup
 \{ a \})$$
\end{demostracion}

En la demostración del teorema hemos usado un resultado, que vamos a
 probar en Isabelle después de la especificación del teorema,
 el resultado es $\sum S + a = \sum (S \cup \{ a
 \})$.%
\end{isamarkuptext}\isamarkuptrue%
%
\begin{isamarkuptext}%
Para la especificación del teorema en isabelle, primero debemos
 notar que  \isa{finite\ S} indica que nuestro conjunto $S$ es 
finito  y definir  la función \isa{sumaConj} tal que
 \isa{sumaConj\ n} esla suma de todos los elementos de S.%
\end{isamarkuptext}\isamarkuptrue%
\isacommand{definition}\isamarkupfalse%
\ sumaConj\ {\isacharcolon}{\isacharcolon}\ {\isachardoublequoteopen}nat\ set\ {\isasymRightarrow}\ nat{\isachardoublequoteclose}\ \isakeyword{where}\isanewline
\ \ {\isachardoublequoteopen}sumaConj\ S\ {\isasymequiv}\ {\isasymSum}S{\isachardoublequoteclose}%
\begin{isamarkuptext}%
El enunciado del teorema es el siguiente :%
\end{isamarkuptext}\isamarkuptrue%
\isacommand{lemma}\isamarkupfalse%
\ {\isachardoublequoteopen}finite\ S\ {\isasymLongrightarrow}\ {\isasymforall}x{\isasymin}S{\isachardot}\ x\ {\isasymle}\ sumaConj\ S{\isachardoublequoteclose}\isanewline
%
\isadelimproof
\isanewline
\ \ %
\endisadelimproof
%
\isatagproof
\isacommand{oops}\isamarkupfalse%
%
\endisatagproof
{\isafoldproof}%
%
\isadelimproof
%
\endisadelimproof
%
\begin{isamarkuptext}%
Vamos a demostrar primero el lema enunciado anteriormente%
\end{isamarkuptext}\isamarkuptrue%
\isacommand{lemma}\isamarkupfalse%
\ {\isachardoublequoteopen}\ x\ {\isasymnotin}\ S\ {\isasymand}\ finite\ S\ {\isasymlongrightarrow}\ sumaConj\ S\ {\isacharplus}\ x\ {\isacharequal}\ sumaConj{\isacharparenleft}insert\ x\ S{\isacharparenright}{\isachardoublequoteclose}\isanewline
%
\isadelimproof
\ \ %
\endisadelimproof
%
\isatagproof
\isacommand{by}\isamarkupfalse%
\ {\isacharparenleft}simp\ add{\isacharcolon}\ sumaConj{\isacharunderscore}def{\isacharparenright}%
\endisatagproof
{\isafoldproof}%
%
\isadelimproof
%
\endisadelimproof
%
\begin{isamarkuptext}%
La demostración del lema anterior se ha incluido
 \isa{sumConj{\isacharunderscore}def}, que hace referencia a la definición sumaConj que
 hemos hecho anteriormente. \\
En la demostración se usará la táctica \isa{induct} que hace
  uso del esquema de inducción sobre los conjuntos finitos naturales:
  \begin{itemize}
  \item[] \isa{{\isasymlbrakk}finite\ x{\isacharsemicolon}\ P\ {\isasymemptyset}{\isacharsemicolon}\ {\isasymAnd}A\ a{\isachardot}\ finite\ A\ {\isasymand}\ P\ A\ {\isasymLongrightarrow}\ P\ {\isacharparenleft}{\isacharbraceleft}a{\isacharbraceright}\ {\isasymunion}\ A{\isacharparenright}{\isasymrbrakk}\ {\isasymLongrightarrow}\ P\ x} \hfill (\isa{finite{\isachardot}induct})
  \end{itemize} 

Vamos a ver presentar las diferentes formas de demostración.


La demostración aplicativa es:%
\end{isamarkuptext}\isamarkuptrue%
\isacommand{lemma}\isamarkupfalse%
\ {\isachardoublequoteopen}finite\ S\ {\isasymLongrightarrow}\ {\isasymforall}x{\isasymin}S{\isachardot}\ x\ {\isasymle}\ sumaConj\ S{\isachardoublequoteclose}\isanewline
%
\isadelimproof
\ \ %
\endisadelimproof
%
\isatagproof
\isacommand{apply}\isamarkupfalse%
\ {\isacharparenleft}induct\ rule{\isacharcolon}\ finite{\isacharunderscore}induct{\isacharparenright}\isanewline
\ \ \ \isacommand{apply}\isamarkupfalse%
\ simp\isanewline
\ \ \isacommand{apply}\isamarkupfalse%
\ {\isacharparenleft}simp\ add{\isacharcolon}\ add{\isacharunderscore}increasing\ sumaConj{\isacharunderscore}def{\isacharparenright}\isanewline
\ \ \isacommand{done}\isamarkupfalse%
%
\endisatagproof
{\isafoldproof}%
%
\isadelimproof
%
\endisadelimproof
%
\begin{isamarkuptext}%
En la demostración anterior se ha introducido:
 \begin{itemize}
    \item[] \isa{\mbox{}\inferrule{\mbox{{\isacharparenleft}{\isadigit{0}}\ {\isacharcolon}{\isacharcolon}\ {\isacharprime}a{\isacharparenright}\ {\isasymle}\ a\ {\isasymand}\ b\ {\isasymle}\ c}}{\mbox{b\ {\isasymle}\ a\ {\isacharplus}\ c}}} 
      \hfill (\isa{add{\isacharunderscore}increasing})
  \end{itemize} 
 La demostración automática es:%
\end{isamarkuptext}\isamarkuptrue%
\isacommand{lemma}\isamarkupfalse%
\ {\isachardoublequoteopen}finite\ S\ {\isasymLongrightarrow}\ {\isasymforall}x{\isasymin}S{\isachardot}\ x\ {\isasymle}\ sumaConj\ S{\isachardoublequoteclose}\isanewline
%
\isadelimproof
\ \ %
\endisadelimproof
%
\isatagproof
\isacommand{by}\isamarkupfalse%
\ {\isacharparenleft}induct\ rule{\isacharcolon}\ finite{\isacharunderscore}induct{\isacharparenright}\isanewline
\ \ \ \ \ {\isacharparenleft}auto\ simp\ add{\isacharcolon}\ \ sumaConj{\isacharunderscore}def{\isacharparenright}%
\endisatagproof
{\isafoldproof}%
%
\isadelimproof
%
\endisadelimproof
%
\begin{isamarkuptext}%
La demostración declarativa es:%
\end{isamarkuptext}\isamarkuptrue%
\isacommand{lemma}\isamarkupfalse%
\ sumaConj{\isacharunderscore}acota{\isacharcolon}\ \isanewline
\ \ {\isachardoublequoteopen}finite\ S\ {\isasymLongrightarrow}\ {\isasymforall}x{\isasymin}S{\isachardot}\ x\ {\isasymle}\ sumaConj\ S{\isachardoublequoteclose}\isanewline
%
\isadelimproof
%
\endisadelimproof
%
\isatagproof
\isacommand{proof}\isamarkupfalse%
\ {\isacharparenleft}induct\ rule{\isacharcolon}\ finite{\isacharunderscore}induct{\isacharparenright}\isanewline
\ \ \isacommand{show}\isamarkupfalse%
\ {\isachardoublequoteopen}{\isasymforall}x\ {\isasymin}\ {\isacharbraceleft}{\isacharbraceright}{\isachardot}\ x\ {\isasymle}\ sumaConj\ {\isacharbraceleft}{\isacharbraceright}{\isachardoublequoteclose}\ \isacommand{by}\isamarkupfalse%
\ simp\isanewline
\isacommand{next}\isamarkupfalse%
\isanewline
\ \ \isacommand{fix}\isamarkupfalse%
\ x\ \isakeyword{and}\ F\isanewline
\ \ \isacommand{assume}\isamarkupfalse%
\ fF{\isacharcolon}\ {\isachardoublequoteopen}finite\ F{\isachardoublequoteclose}\ \isanewline
\ \ \ \ \ \isakeyword{and}\ xF{\isacharcolon}\ {\isachardoublequoteopen}x\ {\isasymnotin}\ F{\isachardoublequoteclose}\ \isanewline
\ \ \ \ \ \isakeyword{and}\ HI{\isacharcolon}\ {\isachardoublequoteopen}{\isasymforall}\ x{\isasymin}F{\isachardot}\ x\ {\isasymle}\ sumaConj\ F{\isachardoublequoteclose}\isanewline
\ \ \isacommand{show}\isamarkupfalse%
\ {\isachardoublequoteopen}{\isasymforall}y\ {\isasymin}\ insert\ x\ F{\isachardot}\ y\ {\isasymle}\ sumaConj\ {\isacharparenleft}insert\ x\ F{\isacharparenright}{\isachardoublequoteclose}\isanewline
\ \ \isacommand{proof}\isamarkupfalse%
\ \isanewline
\ \ \ \ \isacommand{fix}\isamarkupfalse%
\ y\ \isanewline
\ \ \ \ \isacommand{assume}\isamarkupfalse%
\ {\isachardoublequoteopen}y\ {\isasymin}\ insert\ x\ F{\isachardoublequoteclose}\isanewline
\ \ \ \ \isacommand{show}\isamarkupfalse%
\ {\isachardoublequoteopen}y\ {\isasymle}\ sumaConj\ {\isacharparenleft}insert\ x\ F{\isacharparenright}{\isachardoublequoteclose}\isanewline
\ \ \ \ \isacommand{proof}\isamarkupfalse%
\ {\isacharparenleft}cases\ {\isachardoublequoteopen}y\ {\isacharequal}\ x{\isachardoublequoteclose}{\isacharparenright}\isanewline
\ \ \ \ \ \ \isacommand{assume}\isamarkupfalse%
\ {\isachardoublequoteopen}y\ {\isacharequal}\ x{\isachardoublequoteclose}\isanewline
\ \ \ \ \ \ \isacommand{then}\isamarkupfalse%
\ \isacommand{have}\isamarkupfalse%
\ {\isachardoublequoteopen}y\ {\isasymle}\ x\ {\isacharplus}\ {\isacharparenleft}sumaConj\ F{\isacharparenright}{\isachardoublequoteclose}\ \isacommand{by}\isamarkupfalse%
\ simp\isanewline
\ \ \ \ \ \ \isacommand{also}\isamarkupfalse%
\ \isacommand{have}\isamarkupfalse%
\ {\isachardoublequoteopen}{\isasymdots}\ {\isacharequal}\ sumaConj\ {\isacharparenleft}insert\ x\ F{\isacharparenright}{\isachardoublequoteclose}\ \ \ \isacommand{by}\isamarkupfalse%
\ {\isacharparenleft}simp\ add{\isacharcolon}\ fF\ sumaConj{\isacharunderscore}def\ xF{\isacharparenright}\ \isanewline
\ \ \ \ \ \ \isacommand{finally}\isamarkupfalse%
\ \isacommand{show}\isamarkupfalse%
\ {\isacharquery}thesis\ \isacommand{{\isachardot}}\isamarkupfalse%
\isanewline
\ \ \ \ \isacommand{next}\isamarkupfalse%
\isanewline
\ \ \ \ \ \ \isacommand{assume}\isamarkupfalse%
\ {\isachardoublequoteopen}y\ {\isasymnoteq}\ x{\isachardoublequoteclose}\isanewline
\ \ \ \ \ \ \isacommand{then}\isamarkupfalse%
\ \isacommand{have}\isamarkupfalse%
\ {\isachardoublequoteopen}y\ {\isasymin}\ F{\isachardoublequoteclose}\ \isacommand{using}\isamarkupfalse%
\ {\isacharbackquoteopen}y\ {\isasymin}\ insert\ x\ F{\isacharbackquoteclose}\ \isacommand{by}\isamarkupfalse%
\ simp\isanewline
\ \ \ \ \ \ \isacommand{then}\isamarkupfalse%
\ \isacommand{have}\isamarkupfalse%
\ {\isachardoublequoteopen}y\ {\isasymle}\ sumaConj\ F{\isachardoublequoteclose}\ \isacommand{using}\isamarkupfalse%
\ HI\ \isacommand{by}\isamarkupfalse%
\ simp\isanewline
\ \ \ \ \ \ \isacommand{also}\isamarkupfalse%
\ \isacommand{have}\isamarkupfalse%
\ {\isachardoublequoteopen}{\isasymdots}\ {\isasymle}\ x\ {\isacharplus}\ {\isacharparenleft}sumaConj\ F{\isacharparenright}{\isachardoublequoteclose}\ \isacommand{by}\isamarkupfalse%
\ simp\isanewline
\ \ \ \ \ \ \isacommand{also}\isamarkupfalse%
\ \isacommand{have}\isamarkupfalse%
\ {\isachardoublequoteopen}{\isasymdots}\ {\isacharequal}\ sumaConj\ {\isacharparenleft}insert\ x\ F{\isacharparenright}{\isachardoublequoteclose}\ \isacommand{using}\isamarkupfalse%
\ fF\ xF\isanewline
\ \ \ \ \ \ \ \ \isacommand{by}\isamarkupfalse%
\ {\isacharparenleft}simp\ add{\isacharcolon}\ sumaConj{\isacharunderscore}def{\isacharparenright}\isanewline
\ \ \ \ \ \ \isacommand{finally}\isamarkupfalse%
\ \isacommand{show}\isamarkupfalse%
\ {\isacharquery}thesis\ \isacommand{{\isachardot}}\isamarkupfalse%
\isanewline
\ \ \ \ \isacommand{qed}\isamarkupfalse%
\isanewline
\ \ \isacommand{qed}\isamarkupfalse%
\isanewline
\isacommand{qed}\isamarkupfalse%
%
\endisatagproof
{\isafoldproof}%
%
\isadelimproof
%
\endisadelimproof
%
\isadelimtheory
%
\endisadelimtheory
%
\isatagtheory
%
\endisatagtheory
{\isafoldtheory}%
%
\isadelimtheory
%
\endisadelimtheory
%
\end{isabellebody}%
\endinput
%:%file=~/Escritorio/TFG-v1/EjerciciosDELMF/ConjuntosFinitos.thy%:%
%:%24=7%:%
%:%36=9%:%
%:%37=10%:%
%:%38=11%:%
%:%39=12%:%
%:%40=13%:%
%:%41=14%:%
%:%42=15%:%
%:%43=16%:%
%:%44=17%:%
%:%45=18%:%
%:%46=19%:%
%:%47=20%:%
%:%48=21%:%
%:%49=22%:%
%:%50=23%:%
%:%51=24%:%
%:%52=25%:%
%:%53=26%:%
%:%54=27%:%
%:%55=28%:%
%:%56=29%:%
%:%57=30%:%
%:%58=31%:%
%:%59=32%:%
%:%60=33%:%
%:%61=34%:%
%:%62=35%:%
%:%63=36%:%
%:%64=37%:%
%:%65=38%:%
%:%66=39%:%
%:%67=40%:%
%:%68=41%:%
%:%69=42%:%
%:%70=43%:%
%:%71=44%:%
%:%72=45%:%
%:%73=46%:%
%:%74=47%:%
%:%75=48%:%
%:%76=49%:%
%:%77=50%:%
%:%81=54%:%
%:%82=55%:%
%:%83=56%:%
%:%84=57%:%
%:%86=60%:%
%:%87=60%:%
%:%88=61%:%
%:%90=63%:%
%:%92=66%:%
%:%93=66%:%
%:%96=67%:%
%:%97=68%:%
%:%101=68%:%
%:%111=70%:%
%:%113=71%:%
%:%114=71%:%
%:%117=72%:%
%:%121=72%:%
%:%122=72%:%
%:%131=75%:%
%:%132=76%:%
%:%133=77%:%
%:%134=78%:%
%:%135=79%:%
%:%136=80%:%
%:%137=81%:%
%:%138=82%:%
%:%139=83%:%
%:%140=84%:%
%:%141=85%:%
%:%142=86%:%
%:%143=87%:%
%:%145=89%:%
%:%146=89%:%
%:%149=90%:%
%:%153=90%:%
%:%154=90%:%
%:%155=91%:%
%:%156=91%:%
%:%157=92%:%
%:%158=92%:%
%:%159=93%:%
%:%169=95%:%
%:%170=96%:%
%:%171=97%:%
%:%172=98%:%
%:%173=99%:%
%:%174=100%:%
%:%176=102%:%
%:%177=102%:%
%:%180=103%:%
%:%184=103%:%
%:%185=103%:%
%:%186=104%:%
%:%195=106%:%
%:%197=108%:%
%:%198=108%:%
%:%199=109%:%
%:%206=110%:%
%:%207=110%:%
%:%208=111%:%
%:%209=111%:%
%:%210=111%:%
%:%211=112%:%
%:%212=112%:%
%:%213=113%:%
%:%214=113%:%
%:%215=114%:%
%:%216=114%:%
%:%217=115%:%
%:%218=116%:%
%:%219=117%:%
%:%220=117%:%
%:%221=118%:%
%:%222=118%:%
%:%223=119%:%
%:%224=119%:%
%:%225=120%:%
%:%226=120%:%
%:%227=121%:%
%:%228=121%:%
%:%229=122%:%
%:%230=122%:%
%:%231=123%:%
%:%232=123%:%
%:%233=124%:%
%:%234=124%:%
%:%235=124%:%
%:%236=124%:%
%:%237=125%:%
%:%238=125%:%
%:%239=125%:%
%:%240=125%:%
%:%241=126%:%
%:%242=126%:%
%:%243=126%:%
%:%244=126%:%
%:%245=127%:%
%:%246=127%:%
%:%247=128%:%
%:%248=128%:%
%:%249=129%:%
%:%250=129%:%
%:%251=129%:%
%:%252=129%:%
%:%253=129%:%
%:%254=130%:%
%:%255=130%:%
%:%256=130%:%
%:%257=130%:%
%:%258=130%:%
%:%259=131%:%
%:%260=131%:%
%:%261=131%:%
%:%262=131%:%
%:%263=132%:%
%:%264=132%:%
%:%265=132%:%
%:%266=132%:%
%:%267=133%:%
%:%268=133%:%
%:%269=134%:%
%:%270=134%:%
%:%271=134%:%
%:%272=134%:%
%:%273=135%:%
%:%274=135%:%
%:%275=136%:%
%:%276=136%:%
%:%277=137%:%

%
\begin{isabellebody}%
\setisabellecontext{TeoremaCantor}%
%
\isadelimtheory
%
\endisadelimtheory
%
\isatagtheory
%
\endisatagtheory
{\isafoldtheory}%
%
\isadelimtheory
%
\endisadelimtheory
%
\isadelimdocument
%
\endisadelimdocument
%
\isatagdocument
%
\isamarkupsection{Teorema de Cantor%
}
\isamarkuptrue%
%
\endisatagdocument
{\isafolddocument}%
%
\isadelimdocument
%
\endisadelimdocument
%
\begin{isamarkuptext}%
El siguiente, denominado  teorema de Cantor por el matemático
 Georg Cantor, es un resultado importante de la teoría
 de conjuntos. 

El matemático Georg Ferdinand Ludwig Philipp Cantor fue un matemático y
lógico nacido en Rusia en el siglo XIX. Fue inventor junto con Dedekind
 y Frege de la teoría de conjuntos, que es la base de las matemáticas
 modernas.



Para la comprensión del teorema vamos a definir una serie de conceptos:

\begin {itemize}

\item Conjunto de potencia $A$  $(\mathcal{P}(A))$: conjunto formado por
todos los subconjuntos de $A$.
\item Cardinal del conjunto $A$ (Denotado $\# A$): número de elementos del propio
 conjunto.

\end {itemize}
El enunciado original del teorema es el siguiente : 


\begin {teorema}
El cardinal del conjunto potencia de cualquier conjunto A es
 estrictamente mayor que el cardinal de A, o lo que es lo mismo,
$\# \mathcal{P}(A) > \# A.$


\end {teorema}
Pero el enunciado del teorema lo podemos reformular como: 
\begin{teorema}
Dado un conjunto A, $\nexists  f : A \longrightarrow \mathcal{P}(A)$ que
sea sobreyectiva.

\end{teorema}

El teorema lo hemos podido reescribir de la anterior forma, ya que si
 suponemos que $\exists f$ tal que $f: A \longrightarrow \mathcal{P}(A)$
es sobreyectiva, entonces tenemos que $f(A) = \mathcal{P}(A)$ y por lo
 tanto, $\# f(A) \geq \# \mathcal{P}(A)$, de lo que se deduce esta
 reformulación. Reciprocamente, es trivial ver que esta reformulación
 implica la primera.
 con el teorema. \\
El teorema de Cantor es trivial para conjuntos finitos, ya que el
 conjunto potencia, de conjuntos finitos de n elementos tiene
 $2^n$ elementos.

Por ello,  vamos a realizar la prueba para conjuntos infinitos. 


\begin{demostracion}
 
Vamos a realizar la prueba por reducción al absurdo.\\
Supongamos que $\exists f : A \longrightarrow \mathcal{P}(A)$ sobreyectiva, es
 decir, $\forall C \in \rho(A) ,  \exists x \in A$ tal que $C = f(x)$.
En particular, tomemos el conjunto $$B = \{ x \in A : x \notin f(x) \}$$
 y  supongamos que $\exists a \in A : B = f(a)$, ya que $B$ es un
 subconjunto de A, luego podemos distinguir dos casos $:$ \\
$1.$ Si $a \in B$, entonces por definición del conjunto $B$ tenemos que
$a \notin B$, luego llegamos a una contradicción. \\
$2.$ Si $a \notin B$, entonces por definición de B tenemos que $a \in 
B$, luego hemos llegado a otra contradicción. 

En las dos hipótesis hemos llegado a una contradicción,
por lo que no existe $a$ y $f$ no es sobreyectiva.
\end{demostracion}


Para la especificación del teorema en Isabelle, primero debemos notar
 que $$f :: \, 'a \Rightarrow \,'a \: set$$
 significa que es una función 
de tipos, donde $'a$ significa un tipo y para poder denotar
el conjunto potencia tenemos que poner $'a \ set$ que significa que es
 de un tipo formado por conjuntos del tipo $'a$.




El enunciado del teorema es el siguiente :%
\end{isamarkuptext}\isamarkuptrue%
\isacommand{theorem}\isamarkupfalse%
\ Cantor{\isacharcolon}\ {\isachardoublequoteopen}{\isasymnexists}f\ {\isacharcolon}{\isacharcolon}\ {\isacharprime}a\ {\isasymRightarrow}\ {\isacharprime}a\ set{\isachardot}\ {\isasymforall}A{\isachardot}\ {\isasymexists}x{\isachardot}\ A\ {\isacharequal}\ f\ x{\isachardoublequoteclose}\isanewline
\isanewline
%
\isadelimproof
\isanewline
\ \ %
\endisadelimproof
%
\isatagproof
\isacommand{oops}\isamarkupfalse%
%
\endisatagproof
{\isafoldproof}%
%
\isadelimproof
%
\endisadelimproof
%
\begin{isamarkuptext}%
La demostración la haremos por la regla la introducción a la
negación, la cual es una simplificación de la regla de 
reducción al absurdo, cuyo esquema mostramos a continuación:   
 \begin{itemize}
  \item[] \isa{{\isacharparenleft}P\ {\isasymLongrightarrow}\ False{\isacharparenright}\ {\isasymLongrightarrow}\ {\isasymnot}\ P} \hfill (\isa{notI})
  \end{itemize}


Esta es la demostración detallada del teorema:%
\end{isamarkuptext}\isamarkuptrue%
\isacommand{theorem}\isamarkupfalse%
\ CantorDetallada{\isacharcolon}\ {\isachardoublequoteopen}{\isasymnexists}f\ {\isacharcolon}{\isacharcolon}\ {\isacharprime}a\ {\isasymRightarrow}\ {\isacharprime}a\ set{\isachardot}\ {\isasymforall}B{\isachardot}\ {\isasymexists}x{\isachardot}\ B\ {\isacharequal}\ f\ x{\isachardoublequoteclose}\isanewline
%
\isadelimproof
%
\endisadelimproof
%
\isatagproof
\isacommand{proof}\isamarkupfalse%
\ {\isacharparenleft}rule\ notI{\isacharparenright}\isanewline
\ \ \isacommand{assume}\isamarkupfalse%
\ {\isachardoublequoteopen}{\isasymexists}f\ {\isacharcolon}{\isacharcolon}\ {\isacharprime}a\ {\isasymRightarrow}\ {\isacharprime}a\ set{\isachardot}\ {\isasymforall}A{\isachardot}\ {\isasymexists}x{\isachardot}\ A\ {\isacharequal}\ f\ x{\isachardoublequoteclose}\isanewline
\ \ \isacommand{then}\isamarkupfalse%
\ \isacommand{obtain}\isamarkupfalse%
\ f\ {\isacharcolon}{\isacharcolon}\ {\isachardoublequoteopen}{\isacharprime}a\ {\isasymRightarrow}\ {\isacharprime}a\ set{\isachardoublequoteclose}\ \isakeyword{where}\ {\isacharasterisk}{\isacharcolon}\ {\isachardoublequoteopen}{\isasymforall}A{\isachardot}\ {\isasymexists}x{\isachardot}\ A\ {\isacharequal}\ f\ x{\isachardoublequoteclose}\ \isacommand{by}\isamarkupfalse%
\ {\isacharparenleft}rule\isanewline
\ \ \ \ \ \ \ \ exE{\isacharparenright}\isanewline
\ \ \isacommand{let}\isamarkupfalse%
\ {\isacharquery}B\ {\isacharequal}\ {\isachardoublequoteopen}{\isacharbraceleft}x{\isachardot}\ x\ {\isasymnotin}\ f\ x{\isacharbraceright}{\isachardoublequoteclose}\isanewline
\ \ \isacommand{from}\isamarkupfalse%
\ {\isacharasterisk}\ \isacommand{obtain}\isamarkupfalse%
\ {\isachardoublequoteopen}\ {\isasymexists}x{\isachardot}\ {\isacharquery}B\ {\isacharequal}\ f\ x\ {\isachardoublequoteclose}\ \isacommand{by}\isamarkupfalse%
\ {\isacharparenleft}rule\ allE{\isacharparenright}\isanewline
\ \ \isacommand{then}\isamarkupfalse%
\ \ \isacommand{obtain}\isamarkupfalse%
\ a\ \isakeyword{where}\ {\isadigit{1}}{\isacharcolon}{\isachardoublequoteopen}{\isacharquery}B\ {\isacharequal}\ f\ a{\isachardoublequoteclose}\ \isacommand{by}\isamarkupfalse%
\ {\isacharparenleft}rule\ exE{\isacharparenright}\isanewline
\ \ \isacommand{show}\isamarkupfalse%
\ False\isanewline
\ \ \isacommand{proof}\isamarkupfalse%
\ {\isacharparenleft}cases{\isacharparenright}\isanewline
\ \ \ \ \isacommand{assume}\isamarkupfalse%
\ {\isachardoublequoteopen}a\ {\isasymin}\ {\isacharquery}B{\isachardoublequoteclose}\ \ \isanewline
\ \ \ \ \isacommand{then}\isamarkupfalse%
\ \isacommand{show}\isamarkupfalse%
\ False\ \ \isacommand{using}\isamarkupfalse%
\ {\isadigit{1}}\ \isacommand{by}\isamarkupfalse%
\ blast\isanewline
\ \ \isacommand{next}\isamarkupfalse%
\ \isanewline
\ \ \ \ \isacommand{assume}\isamarkupfalse%
\ {\isachardoublequoteopen}a\ {\isasymnotin}\ {\isacharquery}B{\isachardoublequoteclose}\isanewline
\ \ \ \ \isacommand{thus}\isamarkupfalse%
\ False\ \isacommand{using}\isamarkupfalse%
\ {\isadigit{1}}\ \isacommand{by}\isamarkupfalse%
\ blast\isanewline
\ \ \isacommand{qed}\isamarkupfalse%
\isanewline
\isacommand{qed}\isamarkupfalse%
%
\endisatagproof
{\isafoldproof}%
%
\isadelimproof
%
\endisadelimproof
%
\begin{isamarkuptext}%
Esta es la demostración aplicativa del teorema:%
\end{isamarkuptext}\isamarkuptrue%
\isacommand{theorem}\isamarkupfalse%
\ CantorAplicativa\ {\isacharcolon}\isanewline
\ {\isachardoublequoteopen}{\isasymnexists}f\ {\isacharcolon}{\isacharcolon}\ {\isacharprime}a\ {\isasymRightarrow}\ {\isacharprime}a\ set{\isachardot}\ {\isasymforall}A{\isachardot}\ {\isasymexists}x{\isachardot}\ A\ {\isacharequal}\ f\ x{\isachardoublequoteclose}\isanewline
%
\isadelimproof
\ \ %
\endisadelimproof
%
\isatagproof
\isacommand{apply}\isamarkupfalse%
\ {\isacharparenleft}rule\ notI{\isacharparenright}\isanewline
\ \ \isacommand{apply}\isamarkupfalse%
\ {\isacharparenleft}erule\ exE{\isacharparenright}\isanewline
\ \ \isacommand{apply}\isamarkupfalse%
\ {\isacharparenleft}erule{\isacharunderscore}tac\ x\ {\isacharequal}\ {\isachardoublequoteopen}{\isacharbraceleft}x{\isachardot}\ x\ {\isasymnotin}\ f\ x{\isacharbraceright}{\isachardoublequoteclose}\ \isakeyword{in}\ allE{\isacharparenright}\isanewline
\ \ \isacommand{apply}\isamarkupfalse%
\ {\isacharparenleft}erule\ exE{\isacharparenright}\isanewline
\ \ \isacommand{apply}\isamarkupfalse%
\ \ blast\ \isanewline
\ \ \isacommand{done}\isamarkupfalse%
%
\endisatagproof
{\isafoldproof}%
%
\isadelimproof
%
\endisadelimproof
%
\begin{isamarkuptext}%
Esta es la demostración automática del teorema:%
\end{isamarkuptext}\isamarkuptrue%
\isacommand{theorem}\isamarkupfalse%
\ CantorAutomatic{\isacharcolon}\ {\isachardoublequoteopen}{\isasymnexists}f\ {\isacharcolon}{\isacharcolon}\ {\isacharprime}a\ {\isasymRightarrow}\ {\isacharprime}a\ set{\isachardot}\ {\isasymforall}B{\isachardot}\ {\isasymexists}x{\isachardot}\ B\ {\isacharequal}\ f\ x{\isachardoublequoteclose}\isanewline
%
\isadelimproof
\ \ %
\endisadelimproof
%
\isatagproof
\isacommand{by}\isamarkupfalse%
\ best%
\endisatagproof
{\isafoldproof}%
%
\isadelimproof
%
\endisadelimproof
%
\begin{isamarkuptext}%
En la demostración de isabelle hemos utilizado el método de prueba
rule con las siguientes reglas, tanto en la aplicativa como en la
 detallada:
 \begin{itemize}
  \item[] \isa{\mbox{}\inferrule{\mbox{\mbox{}\inferrule{\mbox{P}}{\mbox{False}}}}{\mbox{{\isasymnot}\ P}}} \hfill (\isa{notI})
  \end{itemize}
 \begin{itemize}
  \item[] \isa{\mbox{}\inferrule{\mbox{{\isasymexists}x{\isachardot}\ P\ x}\\\ \mbox{{\isasymAnd}x{\isachardot}\ \mbox{}\inferrule{\mbox{P\ x}}{\mbox{Q}}}}{\mbox{Q}}} \hfill (\isa{exE})
  \end{itemize}
 \begin{itemize}
  \item[] \isa{\mbox{}\inferrule{\mbox{{\isasymforall}x{\isachardot}\ P\ x}\\\ \mbox{\mbox{}\inferrule{\mbox{P\ x}}{\mbox{R}}}}{\mbox{R}}} \hfill (\isa{allE})
  \end{itemize}
También hacemos uso de blast, que es un conjunto de reglas lógicas y 
 la demostración automática la hacemos por medio de "best".%
\end{isamarkuptext}\isamarkuptrue%
%
\isadelimtheory
%
\endisadelimtheory
%
\isatagtheory
%
\endisatagtheory
{\isafoldtheory}%
%
\isadelimtheory
%
\endisadelimtheory
%
\end{isabellebody}%
\endinput
%:%file=~/Escritorio/TFG-v1/EjerciciosDELMF/TeoremaCantor.thy%:%
%:%24=7%:%
%:%36=9%:%
%:%37=10%:%
%:%38=11%:%
%:%39=12%:%
%:%40=13%:%
%:%41=14%:%
%:%42=15%:%
%:%43=16%:%
%:%44=17%:%
%:%45=18%:%
%:%46=19%:%
%:%47=20%:%
%:%48=21%:%
%:%49=22%:%
%:%50=23%:%
%:%51=24%:%
%:%52=25%:%
%:%53=26%:%
%:%54=27%:%
%:%55=28%:%
%:%56=29%:%
%:%57=30%:%
%:%58=31%:%
%:%59=32%:%
%:%60=33%:%
%:%61=34%:%
%:%62=35%:%
%:%63=36%:%
%:%64=37%:%
%:%65=38%:%
%:%66=39%:%
%:%67=40%:%
%:%68=41%:%
%:%69=42%:%
%:%70=43%:%
%:%71=44%:%
%:%72=45%:%
%:%73=46%:%
%:%74=47%:%
%:%75=48%:%
%:%76=49%:%
%:%77=50%:%
%:%78=51%:%
%:%79=52%:%
%:%80=53%:%
%:%81=54%:%
%:%82=55%:%
%:%83=56%:%
%:%84=57%:%
%:%85=58%:%
%:%86=59%:%
%:%87=60%:%
%:%88=61%:%
%:%89=62%:%
%:%90=63%:%
%:%91=64%:%
%:%92=65%:%
%:%93=66%:%
%:%94=67%:%
%:%95=68%:%
%:%96=69%:%
%:%97=70%:%
%:%98=71%:%
%:%99=72%:%
%:%100=73%:%
%:%101=74%:%
%:%102=75%:%
%:%103=76%:%
%:%104=77%:%
%:%105=78%:%
%:%106=79%:%
%:%107=80%:%
%:%108=81%:%
%:%109=82%:%
%:%110=83%:%
%:%111=84%:%
%:%112=85%:%
%:%113=86%:%
%:%114=87%:%
%:%115=88%:%
%:%116=89%:%
%:%118=91%:%
%:%119=91%:%
%:%120=92%:%
%:%123=93%:%
%:%124=94%:%
%:%128=94%:%
%:%138=96%:%
%:%139=97%:%
%:%140=98%:%
%:%141=99%:%
%:%142=100%:%
%:%143=101%:%
%:%144=102%:%
%:%145=103%:%
%:%146=104%:%
%:%148=106%:%
%:%149=106%:%
%:%156=107%:%
%:%157=107%:%
%:%158=108%:%
%:%159=108%:%
%:%160=109%:%
%:%161=109%:%
%:%162=109%:%
%:%163=109%:%
%:%164=110%:%
%:%165=111%:%
%:%166=111%:%
%:%167=112%:%
%:%168=112%:%
%:%169=112%:%
%:%170=112%:%
%:%171=113%:%
%:%172=113%:%
%:%173=113%:%
%:%174=113%:%
%:%175=114%:%
%:%176=114%:%
%:%177=115%:%
%:%178=115%:%
%:%179=116%:%
%:%180=116%:%
%:%181=117%:%
%:%182=117%:%
%:%183=117%:%
%:%184=117%:%
%:%185=117%:%
%:%186=118%:%
%:%187=118%:%
%:%188=119%:%
%:%189=119%:%
%:%190=120%:%
%:%191=120%:%
%:%192=120%:%
%:%193=120%:%
%:%194=121%:%
%:%195=121%:%
%:%196=122%:%
%:%206=124%:%
%:%208=127%:%
%:%209=127%:%
%:%210=128%:%
%:%213=129%:%
%:%217=129%:%
%:%218=129%:%
%:%219=130%:%
%:%220=130%:%
%:%221=131%:%
%:%222=131%:%
%:%223=132%:%
%:%224=132%:%
%:%225=133%:%
%:%226=133%:%
%:%227=134%:%
%:%237=136%:%
%:%239=137%:%
%:%240=137%:%
%:%243=138%:%
%:%247=138%:%
%:%248=138%:%
%:%257=140%:%
%:%258=141%:%
%:%259=142%:%
%:%260=143%:%
%:%261=144%:%
%:%262=145%:%
%:%263=146%:%
%:%264=147%:%
%:%265=148%:%
%:%266=149%:%
%:%267=150%:%
%:%268=151%:%
%:%269=152%:%
%:%270=153%:%

%
\begin{isabellebody}%
\setisabellecontext{Metodosdepruebasyreglas}%
%
\isadelimtheory
%
\endisadelimtheory
%
\isatagtheory
%
\endisatagtheory
{\isafoldtheory}%
%
\isadelimtheory
%
\endisadelimtheory
%
\isadelimdocument
%
\endisadelimdocument
%
\isatagdocument
%
\isamarkupsection{Métodos de pruebas y reglas%
}
\isamarkuptrue%
%
\endisatagdocument
{\isafolddocument}%
%
\isadelimdocument
%
\endisadelimdocument
%
\begin{isamarkuptext}%
Métodos de pruebas de demostraciones:

 \begin{itemize}
  \item[] \isa{{\isasymlbrakk}P\ {\isadigit{0}}{\isacharsemicolon}\ {\isasymAnd}nat{\isachardot}\ P\ nat\ {\isasymLongrightarrow}\ P\ {\isacharparenleft}Suc\ nat{\isacharparenright}{\isasymrbrakk}\ {\isasymLongrightarrow}\ P\ nat} \hfill (\isa{nat{\isachardot}induct})
  \end{itemize}

 \begin{itemize}
  \item[] \isa{{\isasymlbrakk}P\ {\isasymLongrightarrow}\ Q{\isacharsemicolon}\ Q\ {\isasymLongrightarrow}\ P{\isasymrbrakk}\ {\isasymLongrightarrow}\ P\ {\isacharequal}\ Q} \hfill (\isa{iffI})
  \end{itemize}

 \begin{itemize}
  \item[] \isa{{\isasymlbrakk}finite\ x{\isacharsemicolon}\ P\ {\isasymemptyset}{\isacharsemicolon}\ {\isasymAnd}A\ a{\isachardot}\ finite\ A\ {\isasymand}\ P\ A\ {\isasymLongrightarrow}\ P\ {\isacharparenleft}{\isacharbraceleft}a{\isacharbraceright}\ {\isasymunion}\ A{\isacharparenright}{\isasymrbrakk}\ {\isasymLongrightarrow}\ P\ x} \hfill (\isa{finite{\isachardot}induct})
  \end{itemize}

 \begin{itemize}
  \item[] \isa{{\isacharparenleft}P\ {\isasymLongrightarrow}\ False{\isacharparenright}\ {\isasymLongrightarrow}\ {\isasymnot}\ P} \hfill (\isa{notI})
  \end{itemize}


Reglas usadas:

 \begin{itemize}
  \item[] \isa{inj{\isacharunderscore}on\ f\ A\ {\isacharequal}\ {\isacharparenleft}{\isasymforall}x{\isasymin}A{\isachardot}\ {\isasymforall}y{\isasymin}A{\isachardot}\ f\ x\ {\isacharequal}\ f\ y\ {\isasymlongrightarrow}\ x\ {\isacharequal}\ y{\isacharparenright}} \hfill (\isa{inj{\isacharunderscore}on{\isacharunderscore}def})
  \end{itemize}
 \begin{itemize}
  \item[] \isa{\mbox{}\inferrule{\mbox{ordering{\isacharunderscore}top\ less{\isacharunderscore}eq\ less\ top}}{\mbox{less{\isacharunderscore}eq\ a\ top}}} \hfill
 (\isa{ordering{\isacharunderscore}top{\isachardot}extremum})
  \end{itemize}
 \begin{itemize}
  \item[] \isa{{\isacharparenleft}f\ {\isacharequal}\ g{\isacharparenright}\ {\isacharequal}\ {\isacharparenleft}{\isasymforall}x{\isachardot}\ f\ x\ {\isacharequal}\ g\ x{\isacharparenright}} \hfill (\isa{fun{\isacharunderscore}eq{\isacharunderscore}iff})
  \end{itemize}
 \begin{itemize}
  \item[] \isa{{\isacharparenleft}f\ {\isasymcirc}\ g{\isacharparenright}\ x\ {\isacharequal}\ f\ {\isacharparenleft}g\ x{\isacharparenright}} \hfill (\isa{o{\isacharunderscore}apply})
  \end{itemize}
 \begin{itemize}
  \item[] \isa{\mbox{}\inferrule{\mbox{\mbox{}\inferrule{\mbox{P}}{\mbox{Q}}}\\\ \mbox{\mbox{}\inferrule{\mbox{Q}}{\mbox{P}}}}{\mbox{P\ {\isacharequal}\ Q}}} \hfill (\isa{iffI})
  \end{itemize}
 \begin{itemize}
  \item[] \isa{\mbox{}\inferrule{\mbox{ListMem\ x\ xs}}{\mbox{ListMem\ x\ {\isacharparenleft}y\ {\isasymcdot}\ xs{\isacharparenright}}}} \hfill (\isa{insert})
  \end{itemize}
 \begin{itemize}
  \item[] \isa{\mbox{}\inferrule{\mbox{{\isasymexists}x{\isachardot}\ P\ x}\\\ \mbox{{\isasymAnd}x{\isachardot}\ \mbox{}\inferrule{\mbox{P\ x}}{\mbox{Q}}}}{\mbox{Q}}} \hfill (\isa{exE})
  \end{itemize}
 \begin{itemize}
  \item[] \isa{\mbox{}\inferrule{\mbox{{\isasymforall}x{\isachardot}\ P\ x}\\\ \mbox{\mbox{}\inferrule{\mbox{P\ x}}{\mbox{R}}}}{\mbox{R}}} \hfill (\isa{allE})
  \end{itemize}
 \begin{itemize}
  \item[] \isa{\mbox{}\inferrule{\mbox{\mbox{}\inferrule{\mbox{P}}{\mbox{False}}}}{\mbox{{\isasymnot}\ P}}} \hfill (\isa{notI})
  \end{itemize}
 \begin{itemize}
  \item[] \isa{{\isacharparenleft}{\isacharparenleft}P\ {\isasymlongrightarrow}\ Q{\isacharparenright}\ {\isasymand}\ {\isacharparenleft}{\isasymnot}\ P\ {\isasymlongrightarrow}\ Q{\isacharparenright}{\isacharparenright}\ {\isacharequal}\ Q} \hfill (\isa{cases})
  \end{itemize}%
\end{isamarkuptext}\isamarkuptrue%
%
\isadelimtheory
%
\endisadelimtheory
%
\isatagtheory
%
\endisatagtheory
{\isafoldtheory}%
%
\isadelimtheory
%
\endisadelimtheory
%
\end{isabellebody}%
\endinput
%:%file=~/Escritorio/TFG/Metodosdepruebasyreglas.thy%:%
%:%24=7%:%
%:%36=9%:%
%:%37=10%:%
%:%38=11%:%
%:%39=12%:%
%:%40=13%:%
%:%41=14%:%
%:%42=15%:%
%:%43=16%:%
%:%44=17%:%
%:%45=18%:%
%:%46=19%:%
%:%47=20%:%
%:%48=21%:%
%:%49=22%:%
%:%50=23%:%
%:%51=24%:%
%:%52=25%:%
%:%53=26%:%
%:%54=27%:%
%:%55=28%:%
%:%56=29%:%
%:%57=30%:%
%:%58=31%:%
%:%58=32%:%
%:%59=33%:%
%:%60=34%:%
%:%61=35%:%
%:%62=36%:%
%:%63=37%:%
%:%64=38%:%
%:%65=39%:%
%:%66=40%:%
%:%67=41%:%
%:%68=42%:%
%:%69=43%:%
%:%70=44%:%
%:%71=45%:%
%:%72=46%:%
%:%73=47%:%
%:%74=48%:%
%:%75=49%:%
%:%76=50%:%
%:%77=51%:%
%:%78=52%:%
%:%79=53%:%
%:%80=54%:%
%:%81=55%:%
%:%82=56%:%
%:%83=57%:%
%:%84=58%:%
%:%85=59%:%
%:%86=60%:%
%:%87=61%:%

%
\begin{isabellebody}%
\setisabellecontext{Soporte}%
%
\isadelimtheory
%
\endisadelimtheory
%
\isatagtheory
%
\endisatagtheory
{\isafoldtheory}%
%
\isadelimtheory
%
\endisadelimtheory
%
\begin{isamarkuptext}%
En este apéndice se recogen la lista de los lemas usados en
  el trabajo indicando la página del
  \href{http://bit.ly/2OMbjMM}{libro de HOL} donde se encuentra.%
\end{isamarkuptext}\isamarkuptrue%
%
\begin{isamarkuptext}%
\comentario{Añadir el libro de HOL a la bibliografía.}%
\end{isamarkuptext}\isamarkuptrue%
%
\begin{isamarkuptext}%
\comentario{Completar la lista de lemas usados.}%
\end{isamarkuptext}\isamarkuptrue%
%
\isadelimdocument
%
\endisadelimdocument
%
\isatagdocument
%
\isamarkupsection{Números naturales (16)%
}
\isamarkuptrue%
%
\isamarkupsubsection{Operaciones aritméticas (16.3)%
}
\isamarkuptrue%
%
\endisatagdocument
{\isafolddocument}%
%
\isadelimdocument
%
\endisadelimdocument
%
\begin{isamarkuptext}%
\begin{itemize}
  \item (p. 348) \isa{{\isadigit{0}}\ {\isacharasterisk}\ n\ {\isacharequal}\ {\isadigit{0}}}
    \hfill (\isa{mult{\isacharunderscore}{\isadigit{0}}}) 
  \item (p. 348) \isa{Suc\ m\ {\isacharasterisk}\ n\ {\isacharequal}\ n\ {\isacharplus}\ m\ {\isacharasterisk}\ n}
    \hfill (\isa{mult{\isacharunderscore}Suc}) 
  \item (p. 348) \isa{m\ {\isacharasterisk}\ Suc\ n\ {\isacharequal}\ m\ {\isacharplus}\ m\ {\isacharasterisk}\ n}
    \hfill (\isa{mult{\isacharunderscore}Suc{\isacharunderscore}right}) 
\end{itemize}%
\end{isamarkuptext}\isamarkuptrue%
%
\isadelimtheory
%
\endisadelimtheory
%
\isatagtheory
%
\endisatagtheory
{\isafoldtheory}%
%
\isadelimtheory
%
\endisadelimtheory
%
\end{isabellebody}%
\endinput
%:%file=~/ownCloud/alonso/curso-TFG/Carlos/TFG_de_Carlos/Soporte.thy%:%
%:%19=11%:%
%:%20=12%:%
%:%21=13%:%
%:%25=15%:%
%:%29=17%:%
%:%38=19%:%
%:%42=21%:%
%:%54=24%:%
%:%55=25%:%
%:%56=26%:%
%:%57=27%:%
%:%58=28%:%
%:%59=29%:%
%:%60=30%:%
%:%61=31%:%



\chapter*{Introducción}
\addcontentsline{toc}{chapter}{Introducción}
\input{Introduccion}
\chapter{Teoría de números}
%
\begin{isabellebody}%
\setisabellecontext{SumaImpares}%
%
\isadelimtheory
%
\endisadelimtheory
%
\isatagtheory
%
\endisatagtheory
{\isafoldtheory}%
%
\isadelimtheory
%
\endisadelimtheory
%
\isadelimdocument
%
\endisadelimdocument
%
\isatagdocument
%
\isamarkupsection{Suma de los primeros números impares%
}
\isamarkuptrue%
%
\endisatagdocument
{\isafolddocument}%
%
\isadelimdocument
%
\endisadelimdocument
%
\begin{isamarkuptext}%
El primer teorema es una propiedad de los números naturales.

  \begin{teorema}
    La suma de los $n$ primeros números impares es $n^2$.
  \end{teorema}

  \begin{demostracion}
    La demostración la haremos en inducción sobre $n$.
\begin {itemize}
\item EL caso $n = 0$ es trivial, ya que $0 = 0$.
\item Supongamos que se verifica la hipótesis para $n$ y veamos para
 $n+1$. \\
Tenemos que demostrar que $\sum_{j=1}^{n+1} k_j = (n+1)^2$ siendo los
 $k_{j}$ el j-ésimo impar, es decir, $k_{j} = 2j - 1$.
$$\sum_{j = 1}^{n+1} k_{j} = k_{n+1} + \sum^{n}_{j=1} k_{j} = k_{n+1} +
 n^{2} = 2(n+1) - 1 + n^2 = n^2 + 2n + 1 = (n+1)^2$$ 
\end {itemize}
.
  \end{demostracion}

  Para especificar el teorema en Isabelle, se comienza definiendo 
  la función \isa{suma{\isacharunderscore}impares} tal que \isa{suma{\isacharunderscore}impares\ n} es la 
  suma de los $n$ primeros números impares%
\end{isamarkuptext}\isamarkuptrue%
\isacommand{fun}\isamarkupfalse%
\ suma{\isacharunderscore}impares\ {\isacharcolon}{\isacharcolon}\ {\isachardoublequoteopen}nat\ {\isasymRightarrow}\ nat{\isachardoublequoteclose}\ \isakeyword{where}\isanewline
\ \ {\isachardoublequoteopen}suma{\isacharunderscore}impares\ {\isadigit{0}}\ {\isacharequal}\ {\isadigit{0}}{\isachardoublequoteclose}\ \isanewline
{\isacharbar}\ {\isachardoublequoteopen}suma{\isacharunderscore}impares\ {\isacharparenleft}Suc\ n{\isacharparenright}\ {\isacharequal}\ {\isacharparenleft}{\isadigit{2}}{\isacharasterisk}{\isacharparenleft}Suc\ n{\isacharparenright}\ {\isacharminus}\ {\isadigit{1}}{\isacharparenright}\ {\isacharplus}\ suma{\isacharunderscore}impares\ n{\isachardoublequoteclose}%
\begin{isamarkuptext}%
El enunciado del teorema es el siguiente:%
\end{isamarkuptext}\isamarkuptrue%
\isacommand{lemma}\isamarkupfalse%
\ {\isachardoublequoteopen}suma{\isacharunderscore}impares\ n\ {\isacharequal}\ n\ {\isacharasterisk}\ n{\isachardoublequoteclose}\isanewline
%
\isadelimproof
%
\endisadelimproof
%
\isatagproof
\isacommand{oops}\isamarkupfalse%
%
\endisatagproof
{\isafoldproof}%
%
\isadelimproof
%
\endisadelimproof
%
\begin{isamarkuptext}%
En la demostración se usará la táctica \isa{induct} que hace
  uso del esquema de inducción sobre los naturales:
  \begin{itemize}
  \item[] \isa{\mbox{}\inferrule{\mbox{P\ {\isadigit{0}}}\\\ \mbox{{\isasymAnd}nat{\isachardot}\ \mbox{}\inferrule{\mbox{P\ nat}}{\mbox{P\ {\isacharparenleft}Suc\ nat{\isacharparenright}}}}}{\mbox{P\ nat}}} \hfill (\isa{nat{\isachardot}induct})
  \end{itemize}

  Vamos a presentar distintas demostraciones del teorema. La 
  primera es la demostración aplicativa%
\end{isamarkuptext}\isamarkuptrue%
\isacommand{lemma}\isamarkupfalse%
\ {\isachardoublequoteopen}suma{\isacharunderscore}impares\ n\ {\isacharequal}\ n\ {\isacharasterisk}\ n{\isachardoublequoteclose}\isanewline
%
\isadelimproof
\ \ %
\endisadelimproof
%
\isatagproof
\isacommand{apply}\isamarkupfalse%
\ {\isacharparenleft}induct\ n{\isacharparenright}\ \isanewline
\ \ \ \isacommand{apply}\isamarkupfalse%
\ simp{\isacharunderscore}all\isanewline
\ \ \isacommand{done}\isamarkupfalse%
%
\endisatagproof
{\isafoldproof}%
%
\isadelimproof
%
\endisadelimproof
%
\begin{isamarkuptext}%
La demostración automática es%
\end{isamarkuptext}\isamarkuptrue%
\isacommand{lemma}\isamarkupfalse%
\ {\isachardoublequoteopen}suma{\isacharunderscore}impares\ n\ {\isacharequal}\ n\ {\isacharasterisk}\ n{\isachardoublequoteclose}\isanewline
%
\isadelimproof
\ \ %
\endisadelimproof
%
\isatagproof
\isacommand{by}\isamarkupfalse%
\ {\isacharparenleft}induct\ n{\isacharparenright}\ simp{\isacharunderscore}all%
\endisatagproof
{\isafoldproof}%
%
\isadelimproof
%
\endisadelimproof
%
\begin{isamarkuptext}%
La demostración del lema anterior por inducción y razonamiento 
   ecuacional es%
\end{isamarkuptext}\isamarkuptrue%
\isacommand{lemma}\isamarkupfalse%
\ {\isachardoublequoteopen}suma{\isacharunderscore}impares\ n\ {\isacharequal}\ n\ {\isacharasterisk}\ n{\isachardoublequoteclose}\isanewline
%
\isadelimproof
%
\endisadelimproof
%
\isatagproof
\isacommand{proof}\isamarkupfalse%
\ {\isacharparenleft}induct\ n{\isacharparenright}\isanewline
\ \ \isacommand{show}\isamarkupfalse%
\ {\isachardoublequoteopen}suma{\isacharunderscore}impares\ {\isadigit{0}}\ {\isacharequal}\ {\isadigit{0}}\ {\isacharasterisk}\ {\isadigit{0}}{\isachardoublequoteclose}\ \isacommand{by}\isamarkupfalse%
\ simp\isanewline
\isacommand{next}\isamarkupfalse%
\isanewline
\ \ \isacommand{fix}\isamarkupfalse%
\ n\ \isacommand{assume}\isamarkupfalse%
\ HI{\isacharcolon}\ {\isachardoublequoteopen}suma{\isacharunderscore}impares\ n\ {\isacharequal}\ n\ {\isacharasterisk}\ n{\isachardoublequoteclose}\isanewline
\ \ \isacommand{have}\isamarkupfalse%
\ {\isachardoublequoteopen}suma{\isacharunderscore}impares\ {\isacharparenleft}Suc\ n{\isacharparenright}\ {\isacharequal}\ {\isacharparenleft}{\isadigit{2}}\ {\isacharasterisk}\ {\isacharparenleft}Suc\ n{\isacharparenright}\ {\isacharminus}\ {\isadigit{1}}{\isacharparenright}\ {\isacharplus}\ suma{\isacharunderscore}impares\ n{\isachardoublequoteclose}\ \isanewline
\ \ \ \ \isacommand{by}\isamarkupfalse%
\ simp\isanewline
\ \ \isacommand{also}\isamarkupfalse%
\ \isacommand{have}\isamarkupfalse%
\ {\isachardoublequoteopen}{\isasymdots}\ {\isacharequal}\ {\isacharparenleft}{\isadigit{2}}\ {\isacharasterisk}\ {\isacharparenleft}Suc\ n{\isacharparenright}\ {\isacharminus}\ {\isadigit{1}}{\isacharparenright}\ {\isacharplus}\ n\ {\isacharasterisk}\ n{\isachardoublequoteclose}\ \isacommand{using}\isamarkupfalse%
\ HI\ \isacommand{by}\isamarkupfalse%
\ simp\isanewline
\ \ \isacommand{also}\isamarkupfalse%
\ \isacommand{have}\isamarkupfalse%
\ {\isachardoublequoteopen}{\isasymdots}\ {\isacharequal}\ n\ {\isacharasterisk}\ n\ {\isacharplus}\ {\isadigit{2}}\ {\isacharasterisk}\ n\ {\isacharplus}\ {\isadigit{1}}{\isachardoublequoteclose}\ \isacommand{by}\isamarkupfalse%
\ simp\isanewline
\ \ \isacommand{finally}\isamarkupfalse%
\ \isacommand{show}\isamarkupfalse%
\ {\isachardoublequoteopen}suma{\isacharunderscore}impares\ {\isacharparenleft}Suc\ n{\isacharparenright}\ {\isacharequal}\ {\isacharparenleft}Suc\ n{\isacharparenright}\ {\isacharasterisk}\ {\isacharparenleft}Suc\ n{\isacharparenright}{\isachardoublequoteclose}\ \isacommand{by}\isamarkupfalse%
\ simp\isanewline
\isacommand{qed}\isamarkupfalse%
%
\endisatagproof
{\isafoldproof}%
%
\isadelimproof
%
\endisadelimproof
%
\begin{isamarkuptext}%
La demostración del lema anterior con patrones y razonamiento 
   ecuacional es%
\end{isamarkuptext}\isamarkuptrue%
\isacommand{lemma}\isamarkupfalse%
\ {\isachardoublequoteopen}suma{\isacharunderscore}impares\ n\ {\isacharequal}\ n\ {\isacharasterisk}\ n{\isachardoublequoteclose}\ {\isacharparenleft}\isakeyword{is}\ {\isachardoublequoteopen}{\isacharquery}P\ n{\isachardoublequoteclose}{\isacharparenright}\isanewline
%
\isadelimproof
%
\endisadelimproof
%
\isatagproof
\isacommand{proof}\isamarkupfalse%
\ {\isacharparenleft}induct\ n{\isacharparenright}\isanewline
\ \ \isacommand{show}\isamarkupfalse%
\ {\isachardoublequoteopen}{\isacharquery}P\ {\isadigit{0}}{\isachardoublequoteclose}\ \isacommand{by}\isamarkupfalse%
\ simp\isanewline
\isacommand{next}\isamarkupfalse%
\isanewline
\ \ \isacommand{fix}\isamarkupfalse%
\ n\ \isanewline
\ \ \isacommand{assume}\isamarkupfalse%
\ HI{\isacharcolon}\ {\isachardoublequoteopen}{\isacharquery}P\ n{\isachardoublequoteclose}\isanewline
\ \ \isacommand{have}\isamarkupfalse%
\ {\isachardoublequoteopen}suma{\isacharunderscore}impares\ {\isacharparenleft}Suc\ n{\isacharparenright}\ {\isacharequal}\ {\isacharparenleft}{\isadigit{2}}\ {\isacharasterisk}\ {\isacharparenleft}Suc\ n{\isacharparenright}\ {\isacharminus}\ {\isadigit{1}}{\isacharparenright}\ {\isacharplus}\ suma{\isacharunderscore}impares\ n{\isachardoublequoteclose}\ \isanewline
\ \ \ \ \isacommand{by}\isamarkupfalse%
\ simp\isanewline
\ \ \isacommand{also}\isamarkupfalse%
\ \isacommand{have}\isamarkupfalse%
\ {\isachardoublequoteopen}{\isasymdots}\ {\isacharequal}\ {\isacharparenleft}{\isadigit{2}}\ {\isacharasterisk}\ {\isacharparenleft}Suc\ n{\isacharparenright}\ {\isacharminus}\ {\isadigit{1}}{\isacharparenright}\ {\isacharplus}\ n\ {\isacharasterisk}\ n{\isachardoublequoteclose}\ \isacommand{using}\isamarkupfalse%
\ HI\ \isacommand{by}\isamarkupfalse%
\ simp\isanewline
\ \ \isacommand{also}\isamarkupfalse%
\ \isacommand{have}\isamarkupfalse%
\ {\isachardoublequoteopen}{\isasymdots}\ {\isacharequal}\ n\ {\isacharasterisk}\ n\ {\isacharplus}\ {\isadigit{2}}\ {\isacharasterisk}\ n\ {\isacharplus}\ {\isadigit{1}}{\isachardoublequoteclose}\ \isacommand{by}\isamarkupfalse%
\ simp\isanewline
\ \ \isacommand{finally}\isamarkupfalse%
\ \isacommand{show}\isamarkupfalse%
\ {\isachardoublequoteopen}{\isacharquery}P\ {\isacharparenleft}Suc\ n{\isacharparenright}{\isachardoublequoteclose}\ \isacommand{by}\isamarkupfalse%
\ simp\isanewline
\isacommand{qed}\isamarkupfalse%
%
\endisatagproof
{\isafoldproof}%
%
\isadelimproof
%
\endisadelimproof
%
\begin{isamarkuptext}%
La demostración usando patrones es%
\end{isamarkuptext}\isamarkuptrue%
\isacommand{lemma}\isamarkupfalse%
\ {\isachardoublequoteopen}suma{\isacharunderscore}impares\ n\ {\isacharequal}\ n\ {\isacharasterisk}\ n{\isachardoublequoteclose}\ {\isacharparenleft}\isakeyword{is}\ {\isachardoublequoteopen}{\isacharquery}P\ n{\isachardoublequoteclose}{\isacharparenright}\isanewline
%
\isadelimproof
%
\endisadelimproof
%
\isatagproof
\isacommand{proof}\isamarkupfalse%
\ {\isacharparenleft}induct\ n{\isacharparenright}\isanewline
\ \ \isacommand{show}\isamarkupfalse%
\ {\isachardoublequoteopen}{\isacharquery}P\ {\isadigit{0}}{\isachardoublequoteclose}\ \isacommand{by}\isamarkupfalse%
\ simp\isanewline
\isacommand{next}\isamarkupfalse%
\isanewline
\ \ \isacommand{fix}\isamarkupfalse%
\ n\ \isanewline
\ \ \isacommand{assume}\isamarkupfalse%
\ {\isachardoublequoteopen}{\isacharquery}P\ n{\isachardoublequoteclose}\isanewline
\ \ \isacommand{then}\isamarkupfalse%
\ \isacommand{show}\isamarkupfalse%
\ {\isachardoublequoteopen}{\isacharquery}P\ {\isacharparenleft}Suc\ n{\isacharparenright}{\isachardoublequoteclose}\ \isacommand{by}\isamarkupfalse%
\ simp\isanewline
\isacommand{qed}\isamarkupfalse%
\isanewline
%
\endisatagproof
{\isafoldproof}%
%
\isadelimproof
%
\endisadelimproof
%
\isadelimtheory
%
\endisadelimtheory
%
\isatagtheory
%
\endisatagtheory
{\isafoldtheory}%
%
\isadelimtheory
%
\endisadelimtheory
%
\end{isabellebody}%
\endinput
%:%file=~/Escritorio/TFG-v1/EjerciciosDELMF/SumaImpares.thy%:%
%:%24=8%:%
%:%36=10%:%
%:%37=11%:%
%:%38=12%:%
%:%39=13%:%
%:%40=14%:%
%:%41=15%:%
%:%42=16%:%
%:%43=17%:%
%:%44=18%:%
%:%45=19%:%
%:%46=20%:%
%:%47=21%:%
%:%48=22%:%
%:%49=23%:%
%:%50=24%:%
%:%51=25%:%
%:%52=26%:%
%:%53=27%:%
%:%54=28%:%
%:%55=29%:%
%:%56=30%:%
%:%57=31%:%
%:%58=32%:%
%:%60=35%:%
%:%61=35%:%
%:%62=36%:%
%:%63=37%:%
%:%65=39%:%
%:%67=41%:%
%:%68=41%:%
%:%75=42%:%
%:%85=44%:%
%:%86=45%:%
%:%87=46%:%
%:%88=47%:%
%:%89=48%:%
%:%90=49%:%
%:%91=50%:%
%:%92=51%:%
%:%94=56%:%
%:%95=56%:%
%:%98=57%:%
%:%102=57%:%
%:%103=57%:%
%:%104=58%:%
%:%105=58%:%
%:%106=59%:%
%:%116=61%:%
%:%118=63%:%
%:%119=63%:%
%:%122=64%:%
%:%126=64%:%
%:%127=64%:%
%:%136=66%:%
%:%137=67%:%
%:%139=69%:%
%:%140=69%:%
%:%147=70%:%
%:%148=70%:%
%:%149=71%:%
%:%150=71%:%
%:%151=71%:%
%:%152=72%:%
%:%153=72%:%
%:%154=73%:%
%:%155=73%:%
%:%156=73%:%
%:%157=74%:%
%:%158=74%:%
%:%159=75%:%
%:%160=75%:%
%:%161=76%:%
%:%162=76%:%
%:%163=76%:%
%:%164=76%:%
%:%165=76%:%
%:%166=77%:%
%:%167=77%:%
%:%168=77%:%
%:%169=77%:%
%:%170=78%:%
%:%171=78%:%
%:%172=78%:%
%:%173=78%:%
%:%174=79%:%
%:%184=81%:%
%:%185=82%:%
%:%187=83%:%
%:%188=83%:%
%:%195=84%:%
%:%196=84%:%
%:%197=85%:%
%:%198=85%:%
%:%199=85%:%
%:%200=86%:%
%:%201=86%:%
%:%202=87%:%
%:%203=87%:%
%:%204=88%:%
%:%205=88%:%
%:%206=89%:%
%:%207=89%:%
%:%208=90%:%
%:%209=90%:%
%:%210=91%:%
%:%211=91%:%
%:%212=91%:%
%:%213=91%:%
%:%214=91%:%
%:%215=92%:%
%:%216=92%:%
%:%217=92%:%
%:%218=92%:%
%:%219=93%:%
%:%220=93%:%
%:%221=93%:%
%:%222=93%:%
%:%223=94%:%
%:%233=96%:%
%:%235=98%:%
%:%236=98%:%
%:%243=99%:%
%:%244=99%:%
%:%245=100%:%
%:%246=100%:%
%:%247=100%:%
%:%248=101%:%
%:%249=101%:%
%:%250=102%:%
%:%251=102%:%
%:%252=103%:%
%:%253=103%:%
%:%254=104%:%
%:%255=104%:%
%:%256=104%:%
%:%257=104%:%
%:%258=105%:%
%:%259=105%:%
%
\begin{isabellebody}%
\setisabellecontext{ConjuntosFinitos}%
%
\isadelimtheory
%
\endisadelimtheory
%
\isatagtheory
%
\endisatagtheory
{\isafoldtheory}%
%
\isadelimtheory
%
\endisadelimtheory
%
\isadelimdocument
%
\endisadelimdocument
%
\isatagdocument
%
\isamarkupsection{Propiedad de los conjuntos finitos de números naturales%
}
\isamarkuptrue%
%
\endisatagdocument
{\isafolddocument}%
%
\isadelimdocument
%
\endisadelimdocument
%
\begin{isamarkuptext}%
El siguiente teorema es una propiedad que verifican todos los conjuntos finitos de números
  naturales  estudiado en el \href{http://bit.ly/2XBW6n2}{tema 10} de la
asignatura de LMF. Su enunciado es el siguiente 

  \begin{teorema} 
    Sea S un conjunto finito de números naturales.  Entonces todos los
 elementos de S son menores o iguales que la suma de los elementos de
 S, es decir, $$\forall m , m \in S \Longrightarrow m \leq \sum S$$ 
\newline
donde $\sum S $ denota la suma de todos los elementos de S.
  \end{teorema} 

\begin{demostracion}
La demostración del teorema la haremos por inducción sobre conjuntos
 finitos naturales.\\
Primero veamos el caso base, es decir, supongamos que $S = \emptyset$:

Tenemos que: $$\forall n, n \in \emptyset \Longrightarrow n \leq \sum
 \emptyset.$$\\
Ya hemos probado el caso base, veamos ahora el paso inductivo:
\newline
Sea S un conjunto finito para el que se cumple la hipótesis, es decir,
 todos los elementos de S son menores o iguales que la suma de todos sus
elementos, sea $a$ un elemento tal que $a \notin S$, ya que si $a \in S$
entonces la demostración es trivial.\\
Hay que probar: $$\forall n , n \in S \cup \{ a \} \Longrightarrow n
 \leq \sum (S \cup \{ a \})$$
Vamos a distinguir dos casos:\\
Caso 1: $n = a$ \\
Si $n = a$ tenemos que $n = a \leq a + \sum S = \sum ( S \cup \{ a
 \})$\\
Caso 2: $n \neq a$\\
Si $n \neq a$ tenemos que $a \notin S$ y que $n \in S \cup \{ a \}$,
 luego esto implica que $n \in S$ y usando la hipótesis de inducción
$$n \in S \Longrightarrow n \leq \sum S \leq \sum S + a = \sum (S \cup
 \{ a \})$$
\end{demostracion}

En la demostración del teorema hemos usado un resultado, que vamos a
 probar en Isabelle después de la especificación del teorema,
 el resultado es $\sum S + a = \sum (S \cup \{ a
 \})$.%
\end{isamarkuptext}\isamarkuptrue%
%
\begin{isamarkuptext}%
Para la especificación del teorema en isabelle, primero debemos
 notar que  \isa{finite\ S} indica que nuestro conjunto $S$ es 
finito  y definir  la función \isa{sumaConj} tal que
 \isa{sumaConj\ n} esla suma de todos los elementos de S.%
\end{isamarkuptext}\isamarkuptrue%
\isacommand{definition}\isamarkupfalse%
\ sumaConj\ {\isacharcolon}{\isacharcolon}\ {\isachardoublequoteopen}nat\ set\ {\isasymRightarrow}\ nat{\isachardoublequoteclose}\ \isakeyword{where}\isanewline
\ \ {\isachardoublequoteopen}sumaConj\ S\ {\isasymequiv}\ {\isasymSum}S{\isachardoublequoteclose}%
\begin{isamarkuptext}%
El enunciado del teorema es el siguiente :%
\end{isamarkuptext}\isamarkuptrue%
\isacommand{lemma}\isamarkupfalse%
\ {\isachardoublequoteopen}finite\ S\ {\isasymLongrightarrow}\ {\isasymforall}x{\isasymin}S{\isachardot}\ x\ {\isasymle}\ sumaConj\ S{\isachardoublequoteclose}\isanewline
%
\isadelimproof
\isanewline
\ \ %
\endisadelimproof
%
\isatagproof
\isacommand{oops}\isamarkupfalse%
%
\endisatagproof
{\isafoldproof}%
%
\isadelimproof
%
\endisadelimproof
%
\begin{isamarkuptext}%
Vamos a demostrar primero el lema enunciado anteriormente%
\end{isamarkuptext}\isamarkuptrue%
\isacommand{lemma}\isamarkupfalse%
\ {\isachardoublequoteopen}\ x\ {\isasymnotin}\ S\ {\isasymand}\ finite\ S\ {\isasymlongrightarrow}\ sumaConj\ S\ {\isacharplus}\ x\ {\isacharequal}\ sumaConj{\isacharparenleft}insert\ x\ S{\isacharparenright}{\isachardoublequoteclose}\isanewline
%
\isadelimproof
\ \ %
\endisadelimproof
%
\isatagproof
\isacommand{by}\isamarkupfalse%
\ {\isacharparenleft}simp\ add{\isacharcolon}\ sumaConj{\isacharunderscore}def{\isacharparenright}%
\endisatagproof
{\isafoldproof}%
%
\isadelimproof
%
\endisadelimproof
%
\begin{isamarkuptext}%
La demostración del lema anterior se ha incluido
 \isa{sumConj{\isacharunderscore}def}, que hace referencia a la definición sumaConj que
 hemos hecho anteriormente. \\
En la demostración se usará la táctica \isa{induct} que hace
  uso del esquema de inducción sobre los conjuntos finitos naturales:
  \begin{itemize}
  \item[] \isa{{\isasymlbrakk}finite\ x{\isacharsemicolon}\ P\ {\isasymemptyset}{\isacharsemicolon}\ {\isasymAnd}A\ a{\isachardot}\ finite\ A\ {\isasymand}\ P\ A\ {\isasymLongrightarrow}\ P\ {\isacharparenleft}{\isacharbraceleft}a{\isacharbraceright}\ {\isasymunion}\ A{\isacharparenright}{\isasymrbrakk}\ {\isasymLongrightarrow}\ P\ x} \hfill (\isa{finite{\isachardot}induct})
  \end{itemize} 

Vamos a ver presentar las diferentes formas de demostración.


La demostración aplicativa es:%
\end{isamarkuptext}\isamarkuptrue%
\isacommand{lemma}\isamarkupfalse%
\ {\isachardoublequoteopen}finite\ S\ {\isasymLongrightarrow}\ {\isasymforall}x{\isasymin}S{\isachardot}\ x\ {\isasymle}\ sumaConj\ S{\isachardoublequoteclose}\isanewline
%
\isadelimproof
\ \ %
\endisadelimproof
%
\isatagproof
\isacommand{apply}\isamarkupfalse%
\ {\isacharparenleft}induct\ rule{\isacharcolon}\ finite{\isacharunderscore}induct{\isacharparenright}\isanewline
\ \ \ \isacommand{apply}\isamarkupfalse%
\ simp\isanewline
\ \ \isacommand{apply}\isamarkupfalse%
\ {\isacharparenleft}simp\ add{\isacharcolon}\ add{\isacharunderscore}increasing\ sumaConj{\isacharunderscore}def{\isacharparenright}\isanewline
\ \ \isacommand{done}\isamarkupfalse%
%
\endisatagproof
{\isafoldproof}%
%
\isadelimproof
%
\endisadelimproof
%
\begin{isamarkuptext}%
En la demostración anterior se ha introducido:
 \begin{itemize}
    \item[] \isa{\mbox{}\inferrule{\mbox{{\isacharparenleft}{\isadigit{0}}\ {\isacharcolon}{\isacharcolon}\ {\isacharprime}a{\isacharparenright}\ {\isasymle}\ a\ {\isasymand}\ b\ {\isasymle}\ c}}{\mbox{b\ {\isasymle}\ a\ {\isacharplus}\ c}}} 
      \hfill (\isa{add{\isacharunderscore}increasing})
  \end{itemize} 
 La demostración automática es:%
\end{isamarkuptext}\isamarkuptrue%
\isacommand{lemma}\isamarkupfalse%
\ {\isachardoublequoteopen}finite\ S\ {\isasymLongrightarrow}\ {\isasymforall}x{\isasymin}S{\isachardot}\ x\ {\isasymle}\ sumaConj\ S{\isachardoublequoteclose}\isanewline
%
\isadelimproof
\ \ %
\endisadelimproof
%
\isatagproof
\isacommand{by}\isamarkupfalse%
\ {\isacharparenleft}induct\ rule{\isacharcolon}\ finite{\isacharunderscore}induct{\isacharparenright}\isanewline
\ \ \ \ \ {\isacharparenleft}auto\ simp\ add{\isacharcolon}\ \ sumaConj{\isacharunderscore}def{\isacharparenright}%
\endisatagproof
{\isafoldproof}%
%
\isadelimproof
%
\endisadelimproof
%
\begin{isamarkuptext}%
La demostración declarativa es:%
\end{isamarkuptext}\isamarkuptrue%
\isacommand{lemma}\isamarkupfalse%
\ sumaConj{\isacharunderscore}acota{\isacharcolon}\ \isanewline
\ \ {\isachardoublequoteopen}finite\ S\ {\isasymLongrightarrow}\ {\isasymforall}x{\isasymin}S{\isachardot}\ x\ {\isasymle}\ sumaConj\ S{\isachardoublequoteclose}\isanewline
%
\isadelimproof
%
\endisadelimproof
%
\isatagproof
\isacommand{proof}\isamarkupfalse%
\ {\isacharparenleft}induct\ rule{\isacharcolon}\ finite{\isacharunderscore}induct{\isacharparenright}\isanewline
\ \ \isacommand{show}\isamarkupfalse%
\ {\isachardoublequoteopen}{\isasymforall}x\ {\isasymin}\ {\isacharbraceleft}{\isacharbraceright}{\isachardot}\ x\ {\isasymle}\ sumaConj\ {\isacharbraceleft}{\isacharbraceright}{\isachardoublequoteclose}\ \isacommand{by}\isamarkupfalse%
\ simp\isanewline
\isacommand{next}\isamarkupfalse%
\isanewline
\ \ \isacommand{fix}\isamarkupfalse%
\ x\ \isakeyword{and}\ F\isanewline
\ \ \isacommand{assume}\isamarkupfalse%
\ fF{\isacharcolon}\ {\isachardoublequoteopen}finite\ F{\isachardoublequoteclose}\ \isanewline
\ \ \ \ \ \isakeyword{and}\ xF{\isacharcolon}\ {\isachardoublequoteopen}x\ {\isasymnotin}\ F{\isachardoublequoteclose}\ \isanewline
\ \ \ \ \ \isakeyword{and}\ HI{\isacharcolon}\ {\isachardoublequoteopen}{\isasymforall}\ x{\isasymin}F{\isachardot}\ x\ {\isasymle}\ sumaConj\ F{\isachardoublequoteclose}\isanewline
\ \ \isacommand{show}\isamarkupfalse%
\ {\isachardoublequoteopen}{\isasymforall}y\ {\isasymin}\ insert\ x\ F{\isachardot}\ y\ {\isasymle}\ sumaConj\ {\isacharparenleft}insert\ x\ F{\isacharparenright}{\isachardoublequoteclose}\isanewline
\ \ \isacommand{proof}\isamarkupfalse%
\ \isanewline
\ \ \ \ \isacommand{fix}\isamarkupfalse%
\ y\ \isanewline
\ \ \ \ \isacommand{assume}\isamarkupfalse%
\ {\isachardoublequoteopen}y\ {\isasymin}\ insert\ x\ F{\isachardoublequoteclose}\isanewline
\ \ \ \ \isacommand{show}\isamarkupfalse%
\ {\isachardoublequoteopen}y\ {\isasymle}\ sumaConj\ {\isacharparenleft}insert\ x\ F{\isacharparenright}{\isachardoublequoteclose}\isanewline
\ \ \ \ \isacommand{proof}\isamarkupfalse%
\ {\isacharparenleft}cases\ {\isachardoublequoteopen}y\ {\isacharequal}\ x{\isachardoublequoteclose}{\isacharparenright}\isanewline
\ \ \ \ \ \ \isacommand{assume}\isamarkupfalse%
\ {\isachardoublequoteopen}y\ {\isacharequal}\ x{\isachardoublequoteclose}\isanewline
\ \ \ \ \ \ \isacommand{then}\isamarkupfalse%
\ \isacommand{have}\isamarkupfalse%
\ {\isachardoublequoteopen}y\ {\isasymle}\ x\ {\isacharplus}\ {\isacharparenleft}sumaConj\ F{\isacharparenright}{\isachardoublequoteclose}\ \isacommand{by}\isamarkupfalse%
\ simp\isanewline
\ \ \ \ \ \ \isacommand{also}\isamarkupfalse%
\ \isacommand{have}\isamarkupfalse%
\ {\isachardoublequoteopen}{\isasymdots}\ {\isacharequal}\ sumaConj\ {\isacharparenleft}insert\ x\ F{\isacharparenright}{\isachardoublequoteclose}\ \ \ \isacommand{by}\isamarkupfalse%
\ {\isacharparenleft}simp\ add{\isacharcolon}\ fF\ sumaConj{\isacharunderscore}def\ xF{\isacharparenright}\ \isanewline
\ \ \ \ \ \ \isacommand{finally}\isamarkupfalse%
\ \isacommand{show}\isamarkupfalse%
\ {\isacharquery}thesis\ \isacommand{{\isachardot}}\isamarkupfalse%
\isanewline
\ \ \ \ \isacommand{next}\isamarkupfalse%
\isanewline
\ \ \ \ \ \ \isacommand{assume}\isamarkupfalse%
\ {\isachardoublequoteopen}y\ {\isasymnoteq}\ x{\isachardoublequoteclose}\isanewline
\ \ \ \ \ \ \isacommand{then}\isamarkupfalse%
\ \isacommand{have}\isamarkupfalse%
\ {\isachardoublequoteopen}y\ {\isasymin}\ F{\isachardoublequoteclose}\ \isacommand{using}\isamarkupfalse%
\ {\isacharbackquoteopen}y\ {\isasymin}\ insert\ x\ F{\isacharbackquoteclose}\ \isacommand{by}\isamarkupfalse%
\ simp\isanewline
\ \ \ \ \ \ \isacommand{then}\isamarkupfalse%
\ \isacommand{have}\isamarkupfalse%
\ {\isachardoublequoteopen}y\ {\isasymle}\ sumaConj\ F{\isachardoublequoteclose}\ \isacommand{using}\isamarkupfalse%
\ HI\ \isacommand{by}\isamarkupfalse%
\ simp\isanewline
\ \ \ \ \ \ \isacommand{also}\isamarkupfalse%
\ \isacommand{have}\isamarkupfalse%
\ {\isachardoublequoteopen}{\isasymdots}\ {\isasymle}\ x\ {\isacharplus}\ {\isacharparenleft}sumaConj\ F{\isacharparenright}{\isachardoublequoteclose}\ \isacommand{by}\isamarkupfalse%
\ simp\isanewline
\ \ \ \ \ \ \isacommand{also}\isamarkupfalse%
\ \isacommand{have}\isamarkupfalse%
\ {\isachardoublequoteopen}{\isasymdots}\ {\isacharequal}\ sumaConj\ {\isacharparenleft}insert\ x\ F{\isacharparenright}{\isachardoublequoteclose}\ \isacommand{using}\isamarkupfalse%
\ fF\ xF\isanewline
\ \ \ \ \ \ \ \ \isacommand{by}\isamarkupfalse%
\ {\isacharparenleft}simp\ add{\isacharcolon}\ sumaConj{\isacharunderscore}def{\isacharparenright}\isanewline
\ \ \ \ \ \ \isacommand{finally}\isamarkupfalse%
\ \isacommand{show}\isamarkupfalse%
\ {\isacharquery}thesis\ \isacommand{{\isachardot}}\isamarkupfalse%
\isanewline
\ \ \ \ \isacommand{qed}\isamarkupfalse%
\isanewline
\ \ \isacommand{qed}\isamarkupfalse%
\isanewline
\isacommand{qed}\isamarkupfalse%
%
\endisatagproof
{\isafoldproof}%
%
\isadelimproof
%
\endisadelimproof
%
\isadelimtheory
%
\endisadelimtheory
%
\isatagtheory
%
\endisatagtheory
{\isafoldtheory}%
%
\isadelimtheory
%
\endisadelimtheory
%
\end{isabellebody}%
\endinput
%:%file=~/Escritorio/TFG-v1/EjerciciosDELMF/ConjuntosFinitos.thy%:%
%:%24=7%:%
%:%36=9%:%
%:%37=10%:%
%:%38=11%:%
%:%39=12%:%
%:%40=13%:%
%:%41=14%:%
%:%42=15%:%
%:%43=16%:%
%:%44=17%:%
%:%45=18%:%
%:%46=19%:%
%:%47=20%:%
%:%48=21%:%
%:%49=22%:%
%:%50=23%:%
%:%51=24%:%
%:%52=25%:%
%:%53=26%:%
%:%54=27%:%
%:%55=28%:%
%:%56=29%:%
%:%57=30%:%
%:%58=31%:%
%:%59=32%:%
%:%60=33%:%
%:%61=34%:%
%:%62=35%:%
%:%63=36%:%
%:%64=37%:%
%:%65=38%:%
%:%66=39%:%
%:%67=40%:%
%:%68=41%:%
%:%69=42%:%
%:%70=43%:%
%:%71=44%:%
%:%72=45%:%
%:%73=46%:%
%:%74=47%:%
%:%75=48%:%
%:%76=49%:%
%:%77=50%:%
%:%81=54%:%
%:%82=55%:%
%:%83=56%:%
%:%84=57%:%
%:%86=60%:%
%:%87=60%:%
%:%88=61%:%
%:%90=63%:%
%:%92=66%:%
%:%93=66%:%
%:%96=67%:%
%:%97=68%:%
%:%101=68%:%
%:%111=70%:%
%:%113=71%:%
%:%114=71%:%
%:%117=72%:%
%:%121=72%:%
%:%122=72%:%
%:%131=75%:%
%:%132=76%:%
%:%133=77%:%
%:%134=78%:%
%:%135=79%:%
%:%136=80%:%
%:%137=81%:%
%:%138=82%:%
%:%139=83%:%
%:%140=84%:%
%:%141=85%:%
%:%142=86%:%
%:%143=87%:%
%:%145=89%:%
%:%146=89%:%
%:%149=90%:%
%:%153=90%:%
%:%154=90%:%
%:%155=91%:%
%:%156=91%:%
%:%157=92%:%
%:%158=92%:%
%:%159=93%:%
%:%169=95%:%
%:%170=96%:%
%:%171=97%:%
%:%172=98%:%
%:%173=99%:%
%:%174=100%:%
%:%176=102%:%
%:%177=102%:%
%:%180=103%:%
%:%184=103%:%
%:%185=103%:%
%:%186=104%:%
%:%195=106%:%
%:%197=108%:%
%:%198=108%:%
%:%199=109%:%
%:%206=110%:%
%:%207=110%:%
%:%208=111%:%
%:%209=111%:%
%:%210=111%:%
%:%211=112%:%
%:%212=112%:%
%:%213=113%:%
%:%214=113%:%
%:%215=114%:%
%:%216=114%:%
%:%217=115%:%
%:%218=116%:%
%:%219=117%:%
%:%220=117%:%
%:%221=118%:%
%:%222=118%:%
%:%223=119%:%
%:%224=119%:%
%:%225=120%:%
%:%226=120%:%
%:%227=121%:%
%:%228=121%:%
%:%229=122%:%
%:%230=122%:%
%:%231=123%:%
%:%232=123%:%
%:%233=124%:%
%:%234=124%:%
%:%235=124%:%
%:%236=124%:%
%:%237=125%:%
%:%238=125%:%
%:%239=125%:%
%:%240=125%:%
%:%241=126%:%
%:%242=126%:%
%:%243=126%:%
%:%244=126%:%
%:%245=127%:%
%:%246=127%:%
%:%247=128%:%
%:%248=128%:%
%:%249=129%:%
%:%250=129%:%
%:%251=129%:%
%:%252=129%:%
%:%253=129%:%
%:%254=130%:%
%:%255=130%:%
%:%256=130%:%
%:%257=130%:%
%:%258=130%:%
%:%259=131%:%
%:%260=131%:%
%:%261=131%:%
%:%262=131%:%
%:%263=132%:%
%:%264=132%:%
%:%265=132%:%
%:%266=132%:%
%:%267=133%:%
%:%268=133%:%
%:%269=134%:%
%:%270=134%:%
%:%271=134%:%
%:%272=134%:%
%:%273=135%:%
%:%274=135%:%
%:%275=136%:%
%:%276=136%:%
%:%277=137%:%
\chapter{Teoría de funciones}
%
\begin{isabellebody}%
\setisabellecontext{CancelacionInyectiva}%
%
\isadelimtheory
%
\endisadelimtheory
%
\isatagtheory
%
\endisatagtheory
{\isafoldtheory}%
%
\isadelimtheory
%
\endisadelimtheory
%
\isadelimdocument
%
\endisadelimdocument
%
\isatagdocument
%
\isamarkupsection{Cancelación de funciones inyectivas%
}
\isamarkuptrue%
%
\endisatagdocument
{\isafolddocument}%
%
\isadelimdocument
%
\endisadelimdocument
%
\begin{isamarkuptext}%
El siguiente teorema prueba una caracterización de las funciones
 inyectivas, en otras palabras, las funciones inyectivas son
 monomorfismos en la categoría de conjuntos. Un monomorfismo es un
 homomorfismo inyectivo y la categoría de conjuntos es la categoría
 cuyos objetos son los conjuntos.
  
  \begin{teorema}
    $f$ es una función inyectiva, si y solo si, para todas funciones 
 $g$ y $h$  tales que  $f \circ g = f \circ h$ se tiene que $g = h$. 
  \end{teorema}

Vamos a hacer dos lemas de nuestro teorema, ya que podemos la doble 
implicación en dos implicaciones y demostrar cada una de ellas por
 separado.

\begin {lema}
$f$ es una función inyectiva si para todas funciones $g$ y $h$ tales que
 $f \circ g = f \circ h$ se tiene que $g = h.$
\end {lema}
  \begin{demostracion}
    La demostración la haremos por doble implicación: 
\begin {enumerate}
\item Supongamos que tenemos que $f \circ g = f \circ h$, queremos
 demostrar que $g = h$, usando que f es inyectiva tenemos que: \\
$$(f \circ g)(x) = (f \circ h)(x) \Longrightarrow f(g(x)) = f(h(x)) = 
g(x) = h(x)$$
\item Supongamos ahora que $g = h$, queremos demostrar que  $f \circ g
 = f \circ h$. \\
$$(f \circ g)(x) = f(g(x)) = f(h(x)) = (f \circ h)(x)$$
\end {enumerate}
.
  \end{demostracion}

\begin {lema} 
Si para toda $g$ y $h$ tales que $f \circ g =  f \circ h$ se tiene que $g
= h$ entonces f es inyectiva.
\end {lema} 

\begin {demostracion}


Supongamos que el dominio de nuestra función $f$ es distinto del vacío.
Tenemos que demostrar que $\forall a,b$ tales que $f(a) = f(b),$ esto
 implica que $a = b.$ \\
Sean $a,b$ tales que $f(a) = f(b)$, y definamos $g(x) = a  \ \forall x$
 y $h(x) = b \  \forall x$ entonces 
$$(f \circ g) = (f \circ h) \Longrightarrow  f(g(x)) = f(h(x)) \Longrightarrow f(a) = f(b)$$
Por hipótesis tenemos entonces que $a = b,$ como queríamos demostrar.
\end {demostracion}


  Su especificación es la siguiente, pero al igual que hemos hecho en la demostración
a mano vamos a demostrarlo a través de dos lemas:%
\end{isamarkuptext}\isamarkuptrue%
\isacommand{theorem}\isamarkupfalse%
\ caracterizacionineyctiva{\isacharcolon}\isanewline
\ \ {\isachardoublequoteopen}inj\ f\ {\isasymlongleftrightarrow}\ {\isacharparenleft}{\isasymforall}g\ h{\isachardot}\ {\isacharparenleft}f\ {\isasymcirc}\ g\ {\isacharequal}\ f\ {\isasymcirc}\ h{\isacharparenright}\ {\isasymlongrightarrow}\ {\isacharparenleft}g\ {\isacharequal}\ h{\isacharparenright}{\isacharparenright}{\isachardoublequoteclose}\isanewline
%
\isadelimproof
\ \ %
\endisadelimproof
%
\isatagproof
\isacommand{oops}\isamarkupfalse%
%
\endisatagproof
{\isafoldproof}%
%
\isadelimproof
%
\endisadelimproof
%
\begin{isamarkuptext}%
Sus lemas son los siguientes:%
\end{isamarkuptext}\isamarkuptrue%
\isacommand{lemma}\isamarkupfalse%
\ \isanewline
{\isachardoublequoteopen}{\isasymforall}g\ h{\isachardot}\ {\isacharparenleft}f\ {\isasymcirc}\ g\ {\isacharequal}\ f\ {\isasymcirc}\ h\ {\isasymlongrightarrow}\ g\ {\isacharequal}\ h{\isacharparenright}\ {\isasymLongrightarrow}\ inj\ f{\isachardoublequoteclose}\isanewline
%
\isadelimproof
\ \ %
\endisadelimproof
%
\isatagproof
\isacommand{oops}\isamarkupfalse%
%
\endisatagproof
{\isafoldproof}%
%
\isadelimproof
\isanewline
%
\endisadelimproof
\isanewline
\isacommand{lemma}\isamarkupfalse%
\ \isanewline
{\isachardoublequoteopen}inj\ f\ {\isasymLongrightarrow}\ {\isacharparenleft}{\isasymforall}g\ h{\isachardot}{\isacharparenleft}f\ {\isasymcirc}\ g\ {\isacharequal}\ f\ {\isasymcirc}\ h{\isacharparenright}\ {\isasymlongrightarrow}\ {\isacharparenleft}g\ {\isacharequal}\ h{\isacharparenright}{\isacharparenright}{\isachardoublequoteclose}\isanewline
%
\isadelimproof
\ \ %
\endisadelimproof
%
\isatagproof
\isacommand{oops}\isamarkupfalse%
%
\endisatagproof
{\isafoldproof}%
%
\isadelimproof
%
\endisadelimproof
%
\begin{isamarkuptext}%
En la especificación anterior, \isa{inj\ f} es una 
  abreviatura de \isa{inj\ f} definida en la teoría
  \href{http://bit.ly/2XuPQx5}{Fun.thy}. Además, contiene la definición
  de \isa{inj{\isacharunderscore}on}
  \begin{itemize}
    \item[] \isa{inj{\isacharunderscore}on\ f\ A\ {\isacharequal}\ {\isacharparenleft}{\isasymforall}x{\isasymin}A{\isachardot}\ {\isasymforall}y{\isasymin}A{\isachardot}\ f\ x\ {\isacharequal}\ f\ y\ {\isasymlongrightarrow}\ x\ {\isacharequal}\ y{\isacharparenright}} \hfill (\isa{inj{\isacharunderscore}on{\isacharunderscore}def})
  \end{itemize} 
  Por su parte, \isa{UNIV} es el conjunto universal definido en la 
  teoría \href{http://bit.ly/2XtHCW6}{Set.thy} como una abreviatura de 
  \isa{top} que, a su vez está definido en la teoría 
  \href{http://bit.ly/2Xyj9Pe}{Orderings.thy} mediante la siguiente
  propiedad 
  \begin{itemize}
    \item[] \isa{\mbox{}\inferrule{\mbox{ordering{\isacharunderscore}top\ less{\isacharunderscore}eq\ less\ top}}{\mbox{less{\isacharunderscore}eq\ a\ top}}} 
      \hfill (\isa{ordering{\isacharunderscore}top{\isachardot}extremum})
  \end{itemize} 
  En el caso de la teoría de conjuntos, la relación de orden es la
  inclusión de conjuntos.

  Presentaremos distintas demostraciones de los lemas. La primera
  demostración es applicativa:%
\end{isamarkuptext}\isamarkuptrue%
\isacommand{lemma}\isamarkupfalse%
\ inyectivapli{\isacharcolon}\isanewline
\ \ {\isachardoublequoteopen}inj\ f\ {\isasymLongrightarrow}\ {\isacharparenleft}{\isasymforall}g\ h{\isachardot}{\isacharparenleft}f\ {\isasymcirc}\ g\ {\isacharequal}\ f\ {\isasymcirc}\ h{\isacharparenright}\ {\isasymlongrightarrow}\ \ {\isacharparenleft}g\ {\isacharequal}\ h{\isacharparenright}{\isacharparenright}{\isachardoublequoteclose}\isanewline
%
\isadelimproof
\ \ %
\endisadelimproof
%
\isatagproof
\isacommand{apply}\isamarkupfalse%
\ {\isacharparenleft}simp\ add{\isacharcolon}\ inj{\isacharunderscore}on{\isacharunderscore}def\ fun{\isacharunderscore}eq{\isacharunderscore}iff{\isacharparenright}\ \isanewline
\ \ \isacommand{done}\isamarkupfalse%
%
\endisatagproof
{\isafoldproof}%
%
\isadelimproof
\ \isanewline
%
\endisadelimproof
\isanewline
\isacommand{lemma}\isamarkupfalse%
\ inyectivapli{\isadigit{2}}{\isacharcolon}\isanewline
{\isachardoublequoteopen}{\isasymforall}g\ h{\isachardot}\ {\isacharparenleft}f\ {\isasymcirc}\ g\ {\isacharequal}\ f\ {\isasymcirc}\ h\ {\isasymlongrightarrow}\ g\ {\isacharequal}\ h{\isacharparenright}\ {\isasymLongrightarrow}\ inj\ f{\isachardoublequoteclose}\isanewline
%
\isadelimproof
\ \ %
\endisadelimproof
%
\isatagproof
\isacommand{apply}\isamarkupfalse%
\ {\isacharparenleft}rule\ injI{\isacharparenright}\isanewline
\ \ \isacommand{by}\isamarkupfalse%
\ {\isacharparenleft}metis\ fun{\isacharunderscore}upd{\isacharunderscore}apply\ fun{\isacharunderscore}upd{\isacharunderscore}comp{\isacharparenright}%
\endisatagproof
{\isafoldproof}%
%
\isadelimproof
%
\endisadelimproof
%
\begin{isamarkuptext}%
En las demostraciones anteriores se han usado los siguientes
 lemas:
  \begin{itemize}
    \item[] \isa{{\isacharparenleft}f\ {\isacharequal}\ g{\isacharparenright}\ {\isacharequal}\ {\isacharparenleft}{\isasymforall}x{\isachardot}\ f\ x\ {\isacharequal}\ g\ x{\isacharparenright}} 
      \hfill (\isa{fun{\isacharunderscore}eq{\isacharunderscore}iff})
  \end{itemize} 
  \begin{itemize}
    \item[] \isa{{\isacharparenleft}f{\isacharparenleft}x\ {\isacharcolon}{\isacharequal}\ y{\isacharparenright}{\isacharparenright}\ z\ {\isacharequal}\ {\isacharparenleft}\textsf{if}\ z\ {\isacharequal}\ x\ \textsf{then}\ y\ \textsf{else}\ f\ z{\isacharparenright}} 
      \hfill (\isa{fun{\isacharunderscore}upd{\isacharunderscore}apply})
  \end{itemize} 
  \begin{itemize}
    \item[] \isa{{\isacharparenleft}f\ {\isacharequal}\ g{\isacharparenright}\ {\isacharequal}\ {\isacharparenleft}{\isasymforall}x{\isachardot}\ f\ x\ {\isacharequal}\ g\ x{\isacharparenright}} 
      \hfill (\isa{fun{\isacharunderscore}upd{\isacharunderscore}comp})
  \end{itemize} 

  La demostración applicativa1 sin auto es%
\end{isamarkuptext}\isamarkuptrue%
\isacommand{lemma}\isamarkupfalse%
\isanewline
\ \ {\isachardoublequoteopen}inj\ f\ {\isasymLongrightarrow}\ {\isasymforall}g\ h{\isachardot}\ {\isacharparenleft}f\ {\isasymcirc}\ g\ {\isacharequal}\ f\ {\isasymcirc}\ h{\isacharparenright}\ {\isasymlongrightarrow}\ \ {\isacharparenleft}g\ {\isacharequal}\ h{\isacharparenright}{\isachardoublequoteclose}\isanewline
%
\isadelimproof
\ \ %
\endisadelimproof
%
\isatagproof
\isacommand{apply}\isamarkupfalse%
\ {\isacharparenleft}unfold\ inj{\isacharunderscore}on{\isacharunderscore}def{\isacharparenright}\ \isanewline
\ \ \isacommand{apply}\isamarkupfalse%
\ {\isacharparenleft}unfold\ fun{\isacharunderscore}eq{\isacharunderscore}iff{\isacharparenright}\ \isanewline
\ \ \isacommand{apply}\isamarkupfalse%
\ {\isacharparenleft}unfold\ o{\isacharunderscore}apply{\isacharparenright}\isanewline
\ \ \ \isacommand{apply}\isamarkupfalse%
\ simp{\isacharplus}\isanewline
\ \ \isacommand{done}\isamarkupfalse%
%
\endisatagproof
{\isafoldproof}%
%
\isadelimproof
\isanewline
%
\endisadelimproof
\isanewline
\isacommand{lemma}\isamarkupfalse%
\ \isanewline
{\isachardoublequoteopen}{\isasymforall}g\ h{\isachardot}\ {\isacharparenleft}f\ {\isasymcirc}\ g\ {\isacharequal}\ f\ {\isasymcirc}\ h\ {\isasymlongrightarrow}\ g\ {\isacharequal}\ h{\isacharparenright}\ {\isasymLongrightarrow}\ inj\ f{\isachardoublequoteclose}\isanewline
%
\isadelimproof
\ \ %
\endisadelimproof
%
\isatagproof
\isacommand{oops}\isamarkupfalse%
%
\endisatagproof
{\isafoldproof}%
%
\isadelimproof
%
\endisadelimproof
%
\begin{isamarkuptext}%
En la demostración anterior se ha introducido los siguientes
  hechos
  \begin{itemize}
    \item \isa{{\isacharparenleft}f\ {\isasymcirc}\ g{\isacharparenright}\ x\ {\isacharequal}\ f\ {\isacharparenleft}g\ x{\isacharparenright}} \hfill (\isa{o{\isacharunderscore}apply})
    \item \isa{{\isasymlbrakk}P\ {\isasymLongrightarrow}\ Q{\isacharsemicolon}\ Q\ {\isasymLongrightarrow}\ P{\isasymrbrakk}\ {\isasymLongrightarrow}\ P\ {\isacharequal}\ Q} \hfill (\isa{iffI})
  \end{itemize} 

  La demostración automática es%
\end{isamarkuptext}\isamarkuptrue%
\isacommand{lemma}\isamarkupfalse%
\ inyectivaut{\isacharcolon}\isanewline
\ \ \isakeyword{assumes}\ {\isachardoublequoteopen}inj\ f{\isachardoublequoteclose}\isanewline
\ \ \isakeyword{shows}\ {\isachardoublequoteopen}{\isasymforall}g\ h{\isachardot}\ {\isacharparenleft}f\ {\isasymcirc}\ g\ {\isacharequal}\ f\ {\isasymcirc}\ h{\isacharparenright}\ {\isasymlongrightarrow}\ {\isacharparenleft}g\ {\isacharequal}\ h{\isacharparenright}{\isachardoublequoteclose}\isanewline
%
\isadelimproof
\ \ %
\endisadelimproof
%
\isatagproof
\isacommand{using}\isamarkupfalse%
\ assms\isanewline
\ \ \isacommand{by}\isamarkupfalse%
\ {\isacharparenleft}auto\ simp\ add{\isacharcolon}\ inj{\isacharunderscore}on{\isacharunderscore}def\ fun{\isacharunderscore}eq{\isacharunderscore}iff{\isacharparenright}%
\endisatagproof
{\isafoldproof}%
%
\isadelimproof
\ \isanewline
%
\endisadelimproof
\isanewline
\isacommand{lemma}\isamarkupfalse%
\ inyectivaut{\isadigit{2}}{\isacharcolon}\ \isanewline
\ \ \isakeyword{assumes}\ {\isachardoublequoteopen}{\isasymforall}g\ h{\isachardot}\ {\isacharparenleft}{\isacharparenleft}f\ {\isasymcirc}\ g\ {\isacharequal}\ f\ {\isasymcirc}\ h{\isacharparenright}\ {\isasymlongrightarrow}\ {\isacharparenleft}g\ {\isacharequal}\ h{\isacharparenright}{\isacharparenright}{\isachardoublequoteclose}\isanewline
\ \ \isakeyword{shows}\ {\isachardoublequoteopen}inj\ f{\isachardoublequoteclose}\isanewline
%
\isadelimproof
\ \ %
\endisadelimproof
%
\isatagproof
\isacommand{using}\isamarkupfalse%
\ assms\isanewline
\ \ \isacommand{oops}\isamarkupfalse%
%
\endisatagproof
{\isafoldproof}%
%
\isadelimproof
%
\endisadelimproof
%
\begin{isamarkuptext}%
La demostración declarativa%
\end{isamarkuptext}\isamarkuptrue%
\isacommand{lemma}\isamarkupfalse%
\ inyectdeclarada{\isacharcolon}\isanewline
\ \ \isakeyword{assumes}\ {\isachardoublequoteopen}inj\ f{\isachardoublequoteclose}\isanewline
\ \ \isakeyword{shows}\ {\isachardoublequoteopen}{\isasymforall}g\ h{\isachardot}{\isacharparenleft}f\ {\isasymcirc}\ g\ {\isacharequal}\ f\ {\isasymcirc}\ h{\isacharparenright}\ {\isasymlongrightarrow}\ {\isacharparenleft}g\ {\isacharequal}\ h{\isacharparenright}{\isachardoublequoteclose}\isanewline
%
\isadelimproof
%
\endisadelimproof
%
\isatagproof
\isacommand{proof}\isamarkupfalse%
\isanewline
\ \ \isacommand{fix}\isamarkupfalse%
\ g{\isacharcolon}{\isacharcolon}\ {\isachardoublequoteopen}{\isacharprime}c\ {\isasymRightarrow}\ {\isacharprime}a{\isachardoublequoteclose}\isanewline
\ \ \isacommand{show}\isamarkupfalse%
\ {\isachardoublequoteopen}{\isasymforall}h{\isachardot}{\isacharparenleft}f\ {\isasymcirc}\ g\ {\isacharequal}\ f\ {\isasymcirc}\ h{\isacharparenright}\ {\isasymlongrightarrow}\ {\isacharparenleft}g\ {\isacharequal}\ h{\isacharparenright}{\isachardoublequoteclose}\isanewline
\ \ \isacommand{proof}\isamarkupfalse%
\ {\isacharparenleft}rule\ allI{\isacharparenright}\isanewline
\ \ \ \ \isacommand{fix}\isamarkupfalse%
\ h\isanewline
\ \ \ \ \isacommand{show}\isamarkupfalse%
\ {\isachardoublequoteopen}f\ {\isasymcirc}\ g\ {\isacharequal}\ f\ {\isasymcirc}\ h\ {\isasymlongrightarrow}\ {\isacharparenleft}g\ {\isacharequal}\ h{\isacharparenright}{\isachardoublequoteclose}\isanewline
\ \ \ \ \isacommand{proof}\isamarkupfalse%
\ {\isacharparenleft}rule\ impI{\isacharparenright}\isanewline
\ \ \ \ \ \ \isacommand{assume}\isamarkupfalse%
\ {\isachardoublequoteopen}f\ {\isasymcirc}\ g\ {\isacharequal}\ f\ {\isasymcirc}\ h{\isachardoublequoteclose}\isanewline
\ \ \ \ \ \ \isacommand{show}\isamarkupfalse%
\ {\isachardoublequoteopen}g\ {\isacharequal}\ h{\isachardoublequoteclose}\isanewline
\ \ \ \ \ \ \isacommand{proof}\isamarkupfalse%
\ \isanewline
\ \ \ \ \ \ \ \ \isacommand{fix}\isamarkupfalse%
\ x\isanewline
\ \ \ \ \ \ \ \ \isacommand{have}\isamarkupfalse%
\ \ {\isachardoublequoteopen}{\isacharparenleft}f\ {\isasymcirc}\ g{\isacharparenright}{\isacharparenleft}x{\isacharparenright}\ {\isacharequal}\ {\isacharparenleft}f\ {\isasymcirc}\ h{\isacharparenright}{\isacharparenleft}x{\isacharparenright}{\isachardoublequoteclose}\ \isacommand{using}\isamarkupfalse%
\ {\isacharbackquoteopen}f\ {\isasymcirc}\ g\ {\isacharequal}\ f\ {\isasymcirc}\ h{\isacharbackquoteclose}\ \isacommand{by}\isamarkupfalse%
\ simp\isanewline
\ \ \ \ \ \ \ \ \isacommand{then}\isamarkupfalse%
\ \isacommand{have}\isamarkupfalse%
\ {\isachardoublequoteopen}f{\isacharparenleft}g{\isacharparenleft}x{\isacharparenright}{\isacharparenright}\ {\isacharequal}\ f{\isacharparenleft}h{\isacharparenleft}x{\isacharparenright}{\isacharparenright}{\isachardoublequoteclose}\ \isacommand{by}\isamarkupfalse%
\ simp\isanewline
\ \ \ \ \ \ \ \ \isacommand{thus}\isamarkupfalse%
\ \ {\isachardoublequoteopen}g{\isacharparenleft}x{\isacharparenright}\ {\isacharequal}\ h{\isacharparenleft}x{\isacharparenright}{\isachardoublequoteclose}\ \isacommand{using}\isamarkupfalse%
\ {\isacharbackquoteopen}inj\ f{\isacharbackquoteclose}\ \isacommand{by}\isamarkupfalse%
\ {\isacharparenleft}simp\ add{\isacharcolon}inj{\isacharunderscore}on{\isacharunderscore}def{\isacharparenright}\isanewline
\ \ \ \ \ \ \isacommand{qed}\isamarkupfalse%
\isanewline
\ \ \ \ \isacommand{qed}\isamarkupfalse%
\isanewline
\ \ \isacommand{qed}\isamarkupfalse%
\isanewline
\isacommand{qed}\isamarkupfalse%
%
\endisatagproof
{\isafoldproof}%
%
\isadelimproof
\isanewline
%
\endisadelimproof
\isanewline
\isacommand{declare}\isamarkupfalse%
\ {\isacharbrackleft}{\isacharbrackleft}show{\isacharunderscore}types{\isacharbrackright}{\isacharbrackright}\isanewline
\isanewline
\isacommand{lemma}\isamarkupfalse%
\ inyectdeclarada{\isadigit{2}}{\isacharcolon}\isanewline
\ \ \isakeyword{fixes}\ f\ {\isacharcolon}{\isacharcolon}\ {\isachardoublequoteopen}{\isacharprime}b\ {\isasymRightarrow}\ {\isacharprime}c{\isachardoublequoteclose}\ \isanewline
\ \ \isakeyword{assumes}\ {\isachardoublequoteopen}{\isasymforall}{\isacharparenleft}g\ {\isacharcolon}{\isacharcolon}\ {\isacharprime}a\ {\isasymRightarrow}\ {\isacharprime}b{\isacharparenright}\ {\isacharparenleft}h\ {\isacharcolon}{\isacharcolon}\ {\isacharprime}a\ {\isasymRightarrow}\ {\isacharprime}b{\isacharparenright}{\isachardot}\isanewline
\ \ \ \ \ \ \ \ \ {\isacharparenleft}f\ {\isasymcirc}\ g\ {\isacharequal}\ f\ {\isasymcirc}\ h\ {\isasymlongrightarrow}\ g\ {\isacharequal}\ h{\isacharparenright}{\isachardoublequoteclose}\isanewline
\isakeyword{shows}\ {\isachardoublequoteopen}\ inj\ f{\isachardoublequoteclose}\isanewline
%
\isadelimproof
%
\endisadelimproof
%
\isatagproof
\isacommand{proof}\isamarkupfalse%
\ {\isacharparenleft}rule\ injI{\isacharparenright}\isanewline
\ \ \isacommand{fix}\isamarkupfalse%
\ a\ b\ \isanewline
\ \ \isacommand{assume}\isamarkupfalse%
\ {\isadigit{3}}{\isacharcolon}\ {\isachardoublequoteopen}f\ a\ {\isacharequal}\ f\ b\ {\isachardoublequoteclose}\isanewline
\ \ \isacommand{let}\isamarkupfalse%
\ {\isacharquery}g\ {\isacharequal}\ {\isachardoublequoteopen}{\isasymlambda}x\ {\isacharcolon}{\isacharcolon}\ {\isacharprime}a{\isachardot}\ a{\isachardoublequoteclose}\isanewline
\ \ \isacommand{let}\isamarkupfalse%
\ {\isacharquery}h\ {\isacharequal}\ {\isachardoublequoteopen}{\isasymlambda}x\ {\isacharcolon}{\isacharcolon}\ {\isacharprime}a{\isachardot}\ b{\isachardoublequoteclose}\isanewline
\ \ \isacommand{have}\isamarkupfalse%
\ {\isachardoublequoteopen}{\isasymforall}{\isacharparenleft}h\ {\isacharcolon}{\isacharcolon}\ {\isacharprime}a\ {\isasymRightarrow}\ {\isacharprime}b{\isacharparenright}{\isachardot}\ {\isacharparenleft}f\ {\isasymcirc}\ {\isacharquery}g\ {\isacharequal}\ f\ {\isasymcirc}\ h\ {\isasymlongrightarrow}\ {\isacharquery}g\ {\isacharequal}\ h{\isacharparenright}{\isachardoublequoteclose}\isanewline
\ \ \ \ \isacommand{using}\isamarkupfalse%
\ assms\ \isacommand{by}\isamarkupfalse%
\ {\isacharparenleft}rule\ allE{\isacharparenright}\isanewline
\ \ \isacommand{hence}\isamarkupfalse%
\ {\isadigit{1}}{\isacharcolon}\ {\isachardoublequoteopen}\ {\isacharparenleft}f\ {\isasymcirc}\ {\isacharquery}g\ {\isacharequal}\ f\ {\isasymcirc}\ {\isacharquery}h\ {\isasymlongrightarrow}\ {\isacharquery}g\ {\isacharequal}\ {\isacharquery}h{\isacharparenright}{\isachardoublequoteclose}\ \ \isacommand{by}\isamarkupfalse%
\ {\isacharparenleft}rule\ allE{\isacharparenright}\ \isanewline
\ \ \isacommand{have}\isamarkupfalse%
\ {\isadigit{2}}{\isacharcolon}\ {\isachardoublequoteopen}f\ {\isasymcirc}\ {\isacharquery}g\ {\isacharequal}\ f\ {\isasymcirc}\ {\isacharquery}h{\isachardoublequoteclose}\ \isanewline
\ \ \isacommand{proof}\isamarkupfalse%
\ \isanewline
\ \ \ \ \isacommand{fix}\isamarkupfalse%
\ x\isanewline
\ \ \ \ \isacommand{have}\isamarkupfalse%
\ {\isachardoublequoteopen}\ {\isacharparenleft}f\ {\isasymcirc}\ {\isacharparenleft}{\isasymlambda}x\ {\isacharcolon}{\isacharcolon}\ {\isacharprime}a{\isachardot}\ a{\isacharparenright}{\isacharparenright}\ x\ {\isacharequal}\ f{\isacharparenleft}a{\isacharparenright}\ {\isachardoublequoteclose}\ \isacommand{by}\isamarkupfalse%
\ simp\isanewline
\ \ \ \ \isacommand{also}\isamarkupfalse%
\ \isacommand{have}\isamarkupfalse%
\ {\isachardoublequoteopen}{\isachardot}{\isachardot}{\isachardot}\ {\isacharequal}\ f{\isacharparenleft}b{\isacharparenright}{\isachardoublequoteclose}\ \isacommand{using}\isamarkupfalse%
\ {\isadigit{3}}\ \isacommand{by}\isamarkupfalse%
\ simp\isanewline
\ \ \ \ \isacommand{also}\isamarkupfalse%
\ \isacommand{have}\isamarkupfalse%
\ {\isachardoublequoteopen}{\isachardot}{\isachardot}{\isachardot}\ {\isacharequal}\ \ {\isacharparenleft}f\ {\isasymcirc}\ {\isacharparenleft}{\isasymlambda}x\ {\isacharcolon}{\isacharcolon}\ {\isacharprime}a{\isachardot}\ b{\isacharparenright}{\isacharparenright}\ x{\isachardoublequoteclose}\ \isacommand{by}\isamarkupfalse%
\ simp\isanewline
\ \ \ \ \isacommand{finally}\isamarkupfalse%
\ \isacommand{show}\isamarkupfalse%
\ {\isachardoublequoteopen}\ {\isacharparenleft}f\ {\isasymcirc}\ {\isacharparenleft}{\isasymlambda}x\ {\isacharcolon}{\isacharcolon}\ {\isacharprime}a{\isachardot}\ a{\isacharparenright}{\isacharparenright}\ x\ {\isacharequal}\ \ {\isacharparenleft}f\ {\isasymcirc}\ {\isacharparenleft}{\isasymlambda}x\ {\isacharcolon}{\isacharcolon}\ {\isacharprime}a{\isachardot}\ b{\isacharparenright}{\isacharparenright}\ x{\isachardoublequoteclose}\isanewline
\ \ \ \ \ \ \isacommand{by}\isamarkupfalse%
\ simp\isanewline
\ \ \isacommand{qed}\isamarkupfalse%
\isanewline
\ \ \isacommand{have}\isamarkupfalse%
\ {\isachardoublequoteopen}{\isacharquery}g\ {\isacharequal}\ {\isacharquery}h{\isachardoublequoteclose}\ \isacommand{using}\isamarkupfalse%
\ {\isadigit{1}}\ {\isadigit{2}}\ \isacommand{by}\isamarkupfalse%
\ {\isacharparenleft}rule\ mp{\isacharparenright}\isanewline
\ \ \isacommand{then}\isamarkupfalse%
\ \isacommand{show}\isamarkupfalse%
\ {\isachardoublequoteopen}\ a\ {\isacharequal}\ b{\isachardoublequoteclose}\ \isacommand{by}\isamarkupfalse%
\ meson\isanewline
\isacommand{qed}\isamarkupfalse%
%
\endisatagproof
{\isafoldproof}%
%
\isadelimproof
%
\endisadelimproof
%
\begin{isamarkuptext}%
Otra demostración declarativa es%
\end{isamarkuptext}\isamarkuptrue%
\isacommand{lemma}\isamarkupfalse%
\ inyectdetalladacorta{\isadigit{1}}{\isacharcolon}\isanewline
\ \ \isakeyword{assumes}\ {\isachardoublequoteopen}inj\ f{\isachardoublequoteclose}\isanewline
\ \ \isakeyword{shows}\ {\isachardoublequoteopen}{\isacharparenleft}f\ {\isasymcirc}\ g\ {\isacharequal}\ f\ {\isasymcirc}\ h{\isacharparenright}\ {\isasymlongrightarrow}{\isacharparenleft}g\ {\isacharequal}\ h{\isacharparenright}{\isachardoublequoteclose}\isanewline
%
\isadelimproof
%
\endisadelimproof
%
\isatagproof
\isacommand{proof}\isamarkupfalse%
\ \isanewline
\ \ \isacommand{assume}\isamarkupfalse%
\ {\isachardoublequoteopen}f\ {\isasymcirc}\ g\ {\isacharequal}\ f\ {\isasymcirc}\ h{\isachardoublequoteclose}\ \isanewline
\ \ \isacommand{then}\isamarkupfalse%
\ \isacommand{show}\isamarkupfalse%
\ {\isachardoublequoteopen}g\ {\isacharequal}\ h{\isachardoublequoteclose}\ \isacommand{using}\isamarkupfalse%
\ {\isacharbackquoteopen}inj\ f{\isacharbackquoteclose}\ \isacommand{by}\isamarkupfalse%
\ {\isacharparenleft}simp\ add{\isacharcolon}\ inj{\isacharunderscore}on{\isacharunderscore}def\ fun{\isacharunderscore}eq{\isacharunderscore}iff{\isacharparenright}\ \isanewline
\isacommand{qed}\isamarkupfalse%
%
\endisatagproof
{\isafoldproof}%
%
\isadelimproof
\isanewline
%
\endisadelimproof
\isanewline
\isacommand{lemma}\isamarkupfalse%
\ inyectdetalladacorta{\isadigit{2}}{\isacharcolon}\isanewline
\ \ \isakeyword{fixes}\ f\ {\isacharcolon}{\isacharcolon}\ {\isachardoublequoteopen}{\isacharprime}b\ {\isasymRightarrow}\ {\isacharprime}c{\isachardoublequoteclose}\ \isanewline
\ \ \isakeyword{assumes}\ {\isachardoublequoteopen}{\isasymforall}{\isacharparenleft}g\ {\isacharcolon}{\isacharcolon}\ {\isacharprime}a\ {\isasymRightarrow}\ {\isacharprime}b{\isacharparenright}\ {\isacharparenleft}h\ {\isacharcolon}{\isacharcolon}\ {\isacharprime}a\ {\isasymRightarrow}\ {\isacharprime}b{\isacharparenright}{\isachardot}\isanewline
\ \ \ \ \ \ \ \ \ {\isacharparenleft}f\ {\isasymcirc}\ g\ {\isacharequal}\ f\ {\isasymcirc}\ h\ {\isasymlongrightarrow}\ g\ {\isacharequal}\ h{\isacharparenright}{\isachardoublequoteclose}\isanewline
\ \ \isakeyword{shows}\ {\isachardoublequoteopen}\ inj\ f{\isachardoublequoteclose}\isanewline
%
\isadelimproof
%
\endisadelimproof
%
\isatagproof
\isacommand{proof}\isamarkupfalse%
\ {\isacharparenleft}rule\ injI{\isacharparenright}\isanewline
\ \ \isacommand{fix}\isamarkupfalse%
\ a\ b\ \isanewline
\ \ \isacommand{assume}\isamarkupfalse%
\ {\isadigit{1}}{\isacharcolon}\ {\isachardoublequoteopen}f\ a\ {\isacharequal}\ f\ b\ {\isachardoublequoteclose}\isanewline
\ \ \isacommand{let}\isamarkupfalse%
\ {\isacharquery}g\ {\isacharequal}\ {\isachardoublequoteopen}{\isasymlambda}x\ {\isacharcolon}{\isacharcolon}\ {\isacharprime}a{\isachardot}\ a{\isachardoublequoteclose}\isanewline
\ \ \isacommand{let}\isamarkupfalse%
\ {\isacharquery}h\ {\isacharequal}\ {\isachardoublequoteopen}{\isasymlambda}x\ {\isacharcolon}{\isacharcolon}\ {\isacharprime}a{\isachardot}\ b{\isachardoublequoteclose}\isanewline
\ \ \isacommand{have}\isamarkupfalse%
\ {\isadigit{2}}{\isacharcolon}\ {\isachardoublequoteopen}\ {\isacharparenleft}f\ {\isasymcirc}\ {\isacharquery}g\ {\isacharequal}\ f\ {\isasymcirc}\ {\isacharquery}h\ {\isasymlongrightarrow}\ {\isacharquery}g\ {\isacharequal}\ {\isacharquery}h{\isacharparenright}{\isachardoublequoteclose}\ \ \isacommand{using}\isamarkupfalse%
\ assms\ \isacommand{by}\isamarkupfalse%
\ blast\isanewline
\ \ \isacommand{have}\isamarkupfalse%
\ {\isadigit{3}}{\isacharcolon}\ {\isachardoublequoteopen}f\ {\isasymcirc}\ {\isacharquery}g\ {\isacharequal}\ f\ {\isasymcirc}\ {\isacharquery}h{\isachardoublequoteclose}\ \isanewline
\ \ \isacommand{proof}\isamarkupfalse%
\ \isanewline
\ \ \ \ \isacommand{fix}\isamarkupfalse%
\ x\isanewline
\ \ \ \ \isacommand{have}\isamarkupfalse%
\ {\isachardoublequoteopen}\ {\isacharparenleft}f\ {\isasymcirc}\ {\isacharparenleft}{\isasymlambda}x\ {\isacharcolon}{\isacharcolon}\ {\isacharprime}a{\isachardot}\ a{\isacharparenright}{\isacharparenright}\ x\ {\isacharequal}\ f{\isacharparenleft}a{\isacharparenright}\ {\isachardoublequoteclose}\ \isacommand{by}\isamarkupfalse%
\ simp\isanewline
\ \ \ \ \isacommand{also}\isamarkupfalse%
\ \isacommand{have}\isamarkupfalse%
\ {\isachardoublequoteopen}{\isachardot}{\isachardot}{\isachardot}\ {\isacharequal}\ f{\isacharparenleft}b{\isacharparenright}{\isachardoublequoteclose}\ \isacommand{using}\isamarkupfalse%
\ {\isadigit{1}}\ \isacommand{by}\isamarkupfalse%
\ simp\isanewline
\ \ \ \ \isacommand{also}\isamarkupfalse%
\ \isacommand{have}\isamarkupfalse%
\ {\isachardoublequoteopen}{\isachardot}{\isachardot}{\isachardot}\ {\isacharequal}\ \ {\isacharparenleft}f\ {\isasymcirc}\ {\isacharparenleft}{\isasymlambda}x\ {\isacharcolon}{\isacharcolon}\ {\isacharprime}a{\isachardot}\ b{\isacharparenright}{\isacharparenright}\ x{\isachardoublequoteclose}\ \isacommand{by}\isamarkupfalse%
\ simp\isanewline
\ \ \ \ \isacommand{finally}\isamarkupfalse%
\ \isacommand{show}\isamarkupfalse%
\ {\isachardoublequoteopen}\ {\isacharparenleft}f\ {\isasymcirc}\ {\isacharparenleft}{\isasymlambda}x\ {\isacharcolon}{\isacharcolon}\ {\isacharprime}a{\isachardot}\ a{\isacharparenright}{\isacharparenright}\ x\ {\isacharequal}\ \ {\isacharparenleft}f\ {\isasymcirc}\ {\isacharparenleft}{\isasymlambda}x\ {\isacharcolon}{\isacharcolon}\ {\isacharprime}a{\isachardot}\ b{\isacharparenright}{\isacharparenright}\ x{\isachardoublequoteclose}\isanewline
\ \ \ \ \ \ \isacommand{by}\isamarkupfalse%
\ simp\isanewline
\ \ \isacommand{qed}\isamarkupfalse%
\isanewline
\ \ \isacommand{show}\isamarkupfalse%
\ \ {\isachardoublequoteopen}\ a\ {\isacharequal}\ b{\isachardoublequoteclose}\ \isacommand{using}\isamarkupfalse%
\ {\isadigit{2}}\ {\isadigit{3}}\ \isacommand{by}\isamarkupfalse%
\ meson\isanewline
\isacommand{qed}\isamarkupfalse%
%
\endisatagproof
{\isafoldproof}%
%
\isadelimproof
%
\endisadelimproof
%
\begin{isamarkuptext}%
En consecuencia, la demostración de nuestro teorema:%
\end{isamarkuptext}\isamarkuptrue%
\isacommand{theorem}\isamarkupfalse%
\ caracterizacioninyectiva{\isacharcolon}\isanewline
\ \ {\isachardoublequoteopen}inj\ f\ {\isasymlongleftrightarrow}\ {\isacharparenleft}{\isasymforall}g\ h{\isachardot}\ {\isacharparenleft}f\ {\isasymcirc}\ g\ {\isacharequal}\ f\ {\isasymcirc}\ h{\isacharparenright}\ {\isasymlongrightarrow}\ {\isacharparenleft}g\ {\isacharequal}\ h{\isacharparenright}{\isacharparenright}{\isachardoublequoteclose}\isanewline
%
\isadelimproof
\ \ %
\endisadelimproof
%
\isatagproof
\isacommand{using}\isamarkupfalse%
\ inyectdetalladacorta{\isadigit{1}}\ inyectdetalladacorta{\isadigit{2}}\ \isacommand{by}\isamarkupfalse%
\ auto\isanewline
\isanewline
\isanewline
\isanewline
\isanewline
%
\endisatagproof
{\isafoldproof}%
%
\isadelimproof
%
\endisadelimproof
%
\isadelimtheory
%
\endisadelimtheory
%
\isatagtheory
%
\endisatagtheory
{\isafoldtheory}%
%
\isadelimtheory
%
\endisadelimtheory
%
\end{isabellebody}%
\endinput
%:%file=~/Escritorio/TFG/CancelacionInyectiva.thy%:%
%:%24=8%:%
%:%36=10%:%
%:%37=11%:%
%:%38=12%:%
%:%39=13%:%
%:%40=14%:%
%:%41=15%:%
%:%42=16%:%
%:%43=17%:%
%:%44=18%:%
%:%45=19%:%
%:%46=20%:%
%:%47=21%:%
%:%48=22%:%
%:%49=23%:%
%:%50=24%:%
%:%51=25%:%
%:%52=26%:%
%:%53=27%:%
%:%54=28%:%
%:%55=29%:%
%:%56=30%:%
%:%57=31%:%
%:%58=32%:%
%:%59=33%:%
%:%60=34%:%
%:%61=35%:%
%:%62=36%:%
%:%63=37%:%
%:%64=38%:%
%:%65=39%:%
%:%66=40%:%
%:%67=41%:%
%:%68=42%:%
%:%69=43%:%
%:%70=44%:%
%:%71=45%:%
%:%72=46%:%
%:%73=47%:%
%:%74=48%:%
%:%75=49%:%
%:%76=50%:%
%:%77=51%:%
%:%78=52%:%
%:%79=53%:%
%:%80=54%:%
%:%81=55%:%
%:%82=56%:%
%:%83=57%:%
%:%84=58%:%
%:%85=59%:%
%:%86=60%:%
%:%87=61%:%
%:%88=62%:%
%:%90=65%:%
%:%91=65%:%
%:%92=66%:%
%:%95=67%:%
%:%99=67%:%
%:%109=71%:%
%:%111=73%:%
%:%112=73%:%
%:%113=74%:%
%:%116=75%:%
%:%120=75%:%
%:%126=75%:%
%:%129=76%:%
%:%130=77%:%
%:%131=77%:%
%:%132=78%:%
%:%135=79%:%
%:%139=79%:%
%:%149=82%:%
%:%150=83%:%
%:%151=84%:%
%:%152=85%:%
%:%153=86%:%
%:%154=87%:%
%:%155=88%:%
%:%156=89%:%
%:%157=90%:%
%:%158=91%:%
%:%159=92%:%
%:%160=93%:%
%:%161=94%:%
%:%162=95%:%
%:%163=96%:%
%:%164=97%:%
%:%165=98%:%
%:%166=99%:%
%:%167=100%:%
%:%168=101%:%
%:%169=102%:%
%:%171=104%:%
%:%172=104%:%
%:%173=105%:%
%:%176=106%:%
%:%180=106%:%
%:%181=106%:%
%:%182=107%:%
%:%188=107%:%
%:%191=108%:%
%:%192=109%:%
%:%193=109%:%
%:%194=110%:%
%:%197=111%:%
%:%201=111%:%
%:%202=111%:%
%:%203=112%:%
%:%204=112%:%
%:%213=115%:%
%:%214=116%:%
%:%215=117%:%
%:%216=118%:%
%:%217=119%:%
%:%218=120%:%
%:%219=121%:%
%:%220=122%:%
%:%221=123%:%
%:%222=124%:%
%:%223=125%:%
%:%224=126%:%
%:%225=127%:%
%:%226=128%:%
%:%227=129%:%
%:%228=130%:%
%:%230=132%:%
%:%231=132%:%
%:%232=133%:%
%:%235=134%:%
%:%239=134%:%
%:%240=134%:%
%:%241=135%:%
%:%242=135%:%
%:%243=136%:%
%:%244=136%:%
%:%245=137%:%
%:%246=137%:%
%:%247=138%:%
%:%253=138%:%
%:%256=139%:%
%:%257=140%:%
%:%258=140%:%
%:%259=141%:%
%:%262=142%:%
%:%266=142%:%
%:%276=144%:%
%:%277=145%:%
%:%278=146%:%
%:%279=147%:%
%:%280=148%:%
%:%281=149%:%
%:%282=150%:%
%:%283=151%:%
%:%285=153%:%
%:%286=153%:%
%:%287=154%:%
%:%288=155%:%
%:%291=156%:%
%:%295=156%:%
%:%296=156%:%
%:%297=157%:%
%:%298=157%:%
%:%303=157%:%
%:%306=158%:%
%:%307=159%:%
%:%308=159%:%
%:%309=160%:%
%:%310=161%:%
%:%313=162%:%
%:%317=162%:%
%:%318=162%:%
%:%319=163%:%
%:%329=165%:%
%:%331=169%:%
%:%332=169%:%
%:%333=170%:%
%:%334=171%:%
%:%341=172%:%
%:%342=172%:%
%:%343=173%:%
%:%344=173%:%
%:%345=174%:%
%:%346=174%:%
%:%347=175%:%
%:%348=175%:%
%:%349=176%:%
%:%350=176%:%
%:%351=177%:%
%:%352=177%:%
%:%353=178%:%
%:%354=178%:%
%:%355=179%:%
%:%356=179%:%
%:%357=180%:%
%:%358=180%:%
%:%359=181%:%
%:%360=181%:%
%:%361=182%:%
%:%362=182%:%
%:%363=183%:%
%:%364=183%:%
%:%365=183%:%
%:%366=183%:%
%:%367=184%:%
%:%368=184%:%
%:%369=184%:%
%:%370=184%:%
%:%371=185%:%
%:%372=185%:%
%:%373=185%:%
%:%374=185%:%
%:%375=186%:%
%:%376=186%:%
%:%377=187%:%
%:%378=187%:%
%:%379=188%:%
%:%380=188%:%
%:%381=189%:%
%:%387=189%:%
%:%390=190%:%
%:%391=191%:%
%:%392=191%:%
%:%393=192%:%
%:%394=193%:%
%:%395=193%:%
%:%396=194%:%
%:%397=195%:%
%:%398=196%:%
%:%399=197%:%
%:%406=198%:%
%:%407=198%:%
%:%408=199%:%
%:%409=199%:%
%:%410=200%:%
%:%411=200%:%
%:%412=201%:%
%:%413=201%:%
%:%414=202%:%
%:%415=202%:%
%:%416=203%:%
%:%417=203%:%
%:%418=204%:%
%:%419=204%:%
%:%420=204%:%
%:%421=205%:%
%:%422=205%:%
%:%423=205%:%
%:%424=206%:%
%:%425=206%:%
%:%426=207%:%
%:%427=207%:%
%:%428=208%:%
%:%429=208%:%
%:%430=209%:%
%:%431=209%:%
%:%432=209%:%
%:%433=210%:%
%:%434=210%:%
%:%435=210%:%
%:%436=210%:%
%:%437=210%:%
%:%438=211%:%
%:%439=211%:%
%:%440=211%:%
%:%441=211%:%
%:%442=212%:%
%:%443=212%:%
%:%444=212%:%
%:%445=213%:%
%:%446=213%:%
%:%447=214%:%
%:%448=214%:%
%:%449=215%:%
%:%450=215%:%
%:%451=215%:%
%:%452=215%:%
%:%453=216%:%
%:%454=216%:%
%:%455=216%:%
%:%456=216%:%
%:%457=217%:%
%:%467=221%:%
%:%469=223%:%
%:%470=223%:%
%:%471=224%:%
%:%472=225%:%
%:%479=226%:%
%:%480=226%:%
%:%481=227%:%
%:%482=227%:%
%:%483=228%:%
%:%484=228%:%
%:%485=228%:%
%:%486=228%:%
%:%487=228%:%
%:%488=229%:%
%:%494=229%:%
%:%497=230%:%
%:%498=231%:%
%:%499=231%:%
%:%500=232%:%
%:%501=233%:%
%:%502=234%:%
%:%503=235%:%
%:%510=236%:%
%:%511=236%:%
%:%512=237%:%
%:%513=237%:%
%:%514=238%:%
%:%515=238%:%
%:%516=239%:%
%:%517=239%:%
%:%518=240%:%
%:%519=240%:%
%:%520=241%:%
%:%521=241%:%
%:%522=241%:%
%:%523=241%:%
%:%524=242%:%
%:%525=242%:%
%:%526=243%:%
%:%527=243%:%
%:%528=244%:%
%:%529=244%:%
%:%530=245%:%
%:%531=245%:%
%:%532=245%:%
%:%533=246%:%
%:%534=246%:%
%:%535=246%:%
%:%536=246%:%
%:%537=246%:%
%:%538=247%:%
%:%539=247%:%
%:%540=247%:%
%:%541=247%:%
%:%542=248%:%
%:%543=248%:%
%:%544=248%:%
%:%545=249%:%
%:%546=249%:%
%:%547=250%:%
%:%548=250%:%
%:%549=251%:%
%:%550=251%:%
%:%551=251%:%
%:%552=251%:%
%:%553=252%:%
%:%563=256%:%
%:%565=258%:%
%:%566=258%:%
%:%567=259%:%
%:%570=260%:%
%:%574=260%:%
%:%575=260%:%
%:%576=260%:%
%:%577=261%:%
%:%578=262%:%
%:%579=263%:%
%:%580=264%:%
%
\begin{isabellebody}%
\setisabellecontext{CancelacionSobreyectiva}%
%
\isadelimtheory
%
\endisadelimtheory
%
\isatagtheory
%
\endisatagtheory
{\isafoldtheory}%
%
\isadelimtheory
%
\endisadelimtheory
%
\isadelimdocument
%
\endisadelimdocument
%
\isatagdocument
%
\isamarkupsection{Cancelación de las funciones sobreyectivas%
}
\isamarkuptrue%
%
\endisatagdocument
{\isafolddocument}%
%
\isadelimdocument
%
\endisadelimdocument
%
\begin{isamarkuptext}%
El siguiente teorema prueba una caracterización de las funciones
 sobreyectivas, en otras palabras, las funciones sobreyectivas son
 epimorfismos en la categoría de conjuntos. Donde un epimorfismo es un
 homomorfismo sobreyectivo y la categoría de conjuntos es la categoría
 donde los objetos son conjuntos.


\begin {teorema}
  f es sobreyectiva si y solo si  para todas funciones g y h tal que g o f
 = h o f se tiene que g = h.
\end {teorema}
 
El teorema lo podemos dividir en dos lemas, ya que el teorema se
 demuestra por una doble implicación, luego vamos a dividir el teorema
 en las dos implicaciones.

\begin {lema}
  f es sobreyectiva entonces  para todas funciones g y h tal que g o f
 = h o f se tiene que g = h.
\end {lema}
\begin {demostracion}
\begin {itemize}
\item Supongamos que tenemos que $g \circ  f = h \circ f$, queremos probar que $g =
 h.$ Usando la definición de sobreyectividad $(\forall y \in Y,
 \exists x | y = f(x))$ y nuestra hipótesis, tenemos que:
$$g(y) = g(f(x)) = (g o f) (x) = (h o f) (x) = h(f(x)) = h(y)$$
\item Supongamos que $g = h$, hay que probar que $g o f = h o f.$ Usando
nuestra hipótesis, tenemos que:
$$ (g o f)(x) = g(f(x)) = h(f(x)) = (h o f) (x).$$
\end {itemize}
.
\end {demostracion}

\begin {lema}
 Si  para todas funciones g y h tal que g o f  = h o f se tiene
 que g = h entonces f es sobreyectiva.
\end {lema}


Su especificación es la siguiente, que la dividiremos en dos al igual que 
en la demostración a mano:%
\end{isamarkuptext}\isamarkuptrue%
\isacommand{theorem}\isamarkupfalse%
\isanewline
\ {\isachardoublequoteopen}surj\ f\ {\isasymlongleftrightarrow}\ {\isacharparenleft}g\ {\isasymcirc}\ f\ {\isacharequal}\ h\ {\isasymcirc}\ f{\isacharparenright}\ {\isacharequal}\ {\isacharparenleft}g\ {\isacharequal}\ h{\isacharparenright}{\isachardoublequoteclose}\isanewline
%
\isadelimproof
\ \ %
\endisadelimproof
%
\isatagproof
\isacommand{oops}\isamarkupfalse%
%
\endisatagproof
{\isafoldproof}%
%
\isadelimproof
\isanewline
%
\endisadelimproof
\isanewline
\isacommand{lemma}\isamarkupfalse%
\ \isanewline
{\isachardoublequoteopen}surj\ f\ {\isasymLongrightarrow}\ \ {\isacharparenleft}g\ {\isasymcirc}\ f\ {\isacharequal}\ h\ {\isasymcirc}\ f{\isacharparenright}\ {\isacharequal}\ {\isacharparenleft}g\ {\isacharequal}\ h{\isacharparenright}{\isachardoublequoteclose}\isanewline
%
\isadelimproof
\ \ %
\endisadelimproof
%
\isatagproof
\isacommand{oops}\isamarkupfalse%
%
\endisatagproof
{\isafoldproof}%
%
\isadelimproof
\isanewline
%
\endisadelimproof
\isanewline
\isacommand{lemma}\isamarkupfalse%
\ \isanewline
{\isachardoublequoteopen}{\isasymforall}g\ h{\isachardot}\ {\isacharparenleft}g\ {\isasymcirc}\ f\ {\isacharequal}\ h\ {\isasymcirc}\ f\ {\isasymlongrightarrow}\ g\ {\isacharequal}\ h{\isacharparenright}\ {\isasymlongrightarrow}\ surj\ f{\isachardoublequoteclose}\isanewline
%
\isadelimproof
\ \ %
\endisadelimproof
%
\isatagproof
\isacommand{oops}\isamarkupfalse%
%
\endisatagproof
{\isafoldproof}%
%
\isadelimproof
%
\endisadelimproof
%
\begin{isamarkuptext}%
En la especificación anterior, \isa{surj\ f} es una abreviatura de 
  \isa{range\ f\ {\isacharequal}\ UNIV}, donde \isa{range\ f} es el rango o imagen
de la función f.
 Por otra parte, \isa{UNIV} es el conjunto universal definido en la 
  teoría \href{http://bit.ly/2XtHCW6}{Set.thy} como una abreviatura de 
  \isa{top} que, a su vez está definido en la teoría 
  \href{http://bit.ly/2Xyj9Pe}{Orderings.thy} mediante la siguiente
  propiedad 
  \begin{itemize}
    \item[] \isa{\mbox{}\inferrule{\mbox{ordering{\isacharunderscore}top\ less{\isacharunderscore}eq\ less\ top}}{\mbox{less{\isacharunderscore}eq\ a\ top}}} 
      \hfill (\isa{ordering{\isacharunderscore}top{\isachardot}extremum})
  \end{itemize} 
Además queda añadir que la teoría donde se encuentra definido \isa{surj\ f}
 es en \href{http://bit.ly/2XuPQx5}{Fun.thy}. Esta teoría contiene la
 definicion \isa{surj{\isacharunderscore}def}.
 \begin{itemize}
    \item[] \isa{surj\ f\ {\isacharequal}\ {\isacharparenleft}{\isasymforall}y{\isachardot}\ {\isasymexists}x{\isachardot}\ y\ {\isacharequal}\ f\ x{\isacharparenright}} \hfill (\isa{inj{\isacharunderscore}on{\isacharunderscore}def})
  \end{itemize} 

Presentaremos distintas demostraciones del teorema. La primera es la
 detallada:%
\end{isamarkuptext}\isamarkuptrue%
\isacommand{lemma}\isamarkupfalse%
\ \isanewline
\ \ \isakeyword{assumes}\ {\isachardoublequoteopen}surj\ f{\isachardoublequoteclose}\ \isanewline
\ \ \isakeyword{shows}\ {\isachardoublequoteopen}{\isacharparenleft}\ g\ {\isasymcirc}\ f\ {\isacharequal}\ h\ {\isasymcirc}\ f\ {\isacharparenright}\ {\isacharequal}\ {\isacharparenleft}g\ {\isacharequal}\ h{\isacharparenright}{\isachardoublequoteclose}\isanewline
%
\isadelimproof
%
\endisadelimproof
%
\isatagproof
\isacommand{proof}\isamarkupfalse%
\ {\isacharparenleft}rule\ iffI{\isacharparenright}\isanewline
\ \ \isacommand{assume}\isamarkupfalse%
\ {\isadigit{1}}{\isacharcolon}\ {\isachardoublequoteopen}\ g\ {\isasymcirc}\ f\ {\isacharequal}\ h\ {\isasymcirc}\ f\ {\isachardoublequoteclose}\isanewline
\ \ \isacommand{show}\isamarkupfalse%
\ {\isachardoublequoteopen}g\ {\isacharequal}\ h{\isachardoublequoteclose}\ \isanewline
\ \ \isacommand{proof}\isamarkupfalse%
\ \isanewline
\ \ \ \ \isacommand{fix}\isamarkupfalse%
\ x\isanewline
\isanewline
\ \ \ \ \isacommand{have}\isamarkupfalse%
\ {\isachardoublequoteopen}\ {\isasymexists}y\ {\isachardot}\ x\ {\isacharequal}\ f{\isacharparenleft}y{\isacharparenright}{\isachardoublequoteclose}\ \isacommand{using}\isamarkupfalse%
\ assms\ \isacommand{by}\isamarkupfalse%
\ {\isacharparenleft}simp\ add{\isacharcolon}surj{\isacharunderscore}def{\isacharparenright}\isanewline
\ \ \ \ \isacommand{then}\isamarkupfalse%
\ \isacommand{obtain}\isamarkupfalse%
\ {\isachardoublequoteopen}y{\isachardoublequoteclose}\ \isakeyword{where}\ {\isadigit{2}}{\isacharcolon}{\isachardoublequoteopen}x\ {\isacharequal}\ f{\isacharparenleft}y{\isacharparenright}{\isachardoublequoteclose}\ \isacommand{by}\isamarkupfalse%
\ {\isacharparenleft}rule\ exE{\isacharparenright}\isanewline
\ \ \ \ \isacommand{then}\isamarkupfalse%
\ \isacommand{have}\isamarkupfalse%
\ {\isachardoublequoteopen}g{\isacharparenleft}x{\isacharparenright}\ {\isacharequal}\ g{\isacharparenleft}f{\isacharparenleft}y{\isacharparenright}{\isacharparenright}{\isachardoublequoteclose}\ \isacommand{by}\isamarkupfalse%
\ simp\isanewline
\ \ \ \ \isacommand{also}\isamarkupfalse%
\ \isacommand{have}\isamarkupfalse%
\ {\isachardoublequoteopen}{\isachardot}{\isachardot}{\isachardot}\ {\isacharequal}\ {\isacharparenleft}g\ {\isasymcirc}\ f{\isacharparenright}\ {\isacharparenleft}y{\isacharparenright}\ \ {\isachardoublequoteclose}\ \isacommand{by}\isamarkupfalse%
\ simp\isanewline
\ \ \ \ \isacommand{also}\isamarkupfalse%
\ \isacommand{have}\isamarkupfalse%
\ {\isachardoublequoteopen}{\isachardot}{\isachardot}{\isachardot}\ {\isacharequal}\ {\isacharparenleft}h\ {\isasymcirc}\ f{\isacharparenright}\ {\isacharparenleft}y{\isacharparenright}{\isachardoublequoteclose}\ \isacommand{using}\isamarkupfalse%
\ {\isadigit{1}}\ \isacommand{by}\isamarkupfalse%
\ simp\isanewline
\ \ \ \ \isacommand{also}\isamarkupfalse%
\ \isacommand{have}\isamarkupfalse%
\ {\isachardoublequoteopen}{\isachardot}{\isachardot}{\isachardot}\ {\isacharequal}\ h{\isacharparenleft}f{\isacharparenleft}y{\isacharparenright}{\isacharparenright}{\isachardoublequoteclose}\ \isacommand{by}\isamarkupfalse%
\ simp\isanewline
\ \ \ \ \isacommand{also}\isamarkupfalse%
\ \isacommand{have}\isamarkupfalse%
\ {\isachardoublequoteopen}{\isachardot}{\isachardot}{\isachardot}\ {\isacharequal}\ h{\isacharparenleft}x{\isacharparenright}{\isachardoublequoteclose}\ \isacommand{using}\isamarkupfalse%
\ {\isadigit{2}}\ \ \ \isacommand{by}\isamarkupfalse%
\ {\isacharparenleft}simp\ add{\isacharcolon}\ {\isacartoucheopen}x\ {\isacharequal}\ f\ y{\isacartoucheclose}{\isacharparenright}\isanewline
\ \ \ \ \isacommand{finally}\isamarkupfalse%
\ \isacommand{show}\isamarkupfalse%
\ \ {\isachardoublequoteopen}\ g{\isacharparenleft}x{\isacharparenright}\ {\isacharequal}\ h{\isacharparenleft}x{\isacharparenright}\ {\isachardoublequoteclose}\ \isacommand{by}\isamarkupfalse%
\ simp\isanewline
\ \ \isacommand{qed}\isamarkupfalse%
\isanewline
\isacommand{next}\isamarkupfalse%
\isanewline
\ \ \isacommand{assume}\isamarkupfalse%
\ {\isachardoublequoteopen}g\ {\isacharequal}\ h{\isachardoublequoteclose}\ \isanewline
\ \ \isacommand{show}\isamarkupfalse%
\ {\isachardoublequoteopen}g\ {\isasymcirc}\ f\ {\isacharequal}\ h\ {\isasymcirc}\ f{\isachardoublequoteclose}\isanewline
\ \ \isacommand{proof}\isamarkupfalse%
\isanewline
\ \ \ \ \isacommand{fix}\isamarkupfalse%
\ x\isanewline
\ \ \ \ \isacommand{have}\isamarkupfalse%
\ {\isachardoublequoteopen}{\isacharparenleft}g\ {\isasymcirc}\ f{\isacharparenright}\ x\ {\isacharequal}\ g{\isacharparenleft}f{\isacharparenleft}x{\isacharparenright}{\isacharparenright}{\isachardoublequoteclose}\ \isacommand{by}\isamarkupfalse%
\ simp\isanewline
\ \ \ \ \isacommand{also}\isamarkupfalse%
\ \isacommand{have}\isamarkupfalse%
\ {\isachardoublequoteopen}{\isasymdots}\ {\isacharequal}\ h{\isacharparenleft}f{\isacharparenleft}x{\isacharparenright}{\isacharparenright}{\isachardoublequoteclose}\ \isacommand{using}\isamarkupfalse%
\ {\isacharbackquoteopen}g\ {\isacharequal}\ h{\isacharbackquoteclose}\ \isacommand{by}\isamarkupfalse%
\ simp\isanewline
\ \ \ \ \isacommand{also}\isamarkupfalse%
\ \isacommand{have}\isamarkupfalse%
\ {\isachardoublequoteopen}{\isasymdots}\ {\isacharequal}\ {\isacharparenleft}h\ {\isasymcirc}\ f{\isacharparenright}\ x{\isachardoublequoteclose}\ \isacommand{by}\isamarkupfalse%
\ simp\isanewline
\ \ \ \ \isacommand{finally}\isamarkupfalse%
\ \isacommand{show}\isamarkupfalse%
\ {\isachardoublequoteopen}{\isacharparenleft}g\ {\isasymcirc}\ f{\isacharparenright}\ x\ {\isacharequal}\ {\isacharparenleft}h\ {\isasymcirc}\ f{\isacharparenright}\ x{\isachardoublequoteclose}\ \isacommand{by}\isamarkupfalse%
\ simp\isanewline
\ \ \isacommand{qed}\isamarkupfalse%
\isanewline
\isacommand{qed}\isamarkupfalse%
%
\endisatagproof
{\isafoldproof}%
%
\isadelimproof
%
\endisadelimproof
%
\begin{isamarkuptext}%
En la demostración hemos introducido: 
 \begin{itemize}
    \item[] \isa{\mbox{}\inferrule{\mbox{{\isasymexists}x{\isachardot}\ P\ x}\\\ \mbox{{\isasymAnd}x{\isachardot}\ \mbox{}\inferrule{\mbox{P\ x}}{\mbox{Q}}}}{\mbox{Q}}} 
      \hfill (\isa{rule\ exE}) 
  \end{itemize} 
 \begin{itemize}
    \item[] \isa{{\isasymlbrakk}P\ {\isasymLongrightarrow}\ Q{\isacharsemicolon}\ Q\ {\isasymLongrightarrow}\ P{\isasymrbrakk}\ {\isasymLongrightarrow}\ P\ {\isacharequal}\ Q} 
      \hfill (\isa{iffI})
  \end{itemize} 

La demostración aplicativa es:%
\end{isamarkuptext}\isamarkuptrue%
\isacommand{lemma}\isamarkupfalse%
\ {\isachardoublequoteopen}surj\ f\ {\isasymLongrightarrow}\ {\isacharparenleft}{\isacharparenleft}g\ {\isasymcirc}\ f{\isacharparenright}\ {\isacharequal}\ {\isacharparenleft}h\ {\isasymcirc}\ f{\isacharparenright}\ {\isacharparenright}\ {\isacharequal}\ {\isacharparenleft}g\ {\isacharequal}\ h{\isacharparenright}{\isachardoublequoteclose}\isanewline
%
\isadelimproof
\ \ %
\endisadelimproof
%
\isatagproof
\isacommand{apply}\isamarkupfalse%
\ {\isacharparenleft}simp\ add{\isacharcolon}\ surj{\isacharunderscore}def\ fun{\isacharunderscore}eq{\isacharunderscore}iff{\isacharparenright}\isanewline
\ \ \isacommand{apply}\isamarkupfalse%
\ {\isacharparenleft}rule\ iffI{\isacharparenright}\isanewline
\ \ \ \isacommand{prefer}\isamarkupfalse%
\ {\isadigit{2}}\isanewline
\ \ \isacommand{apply}\isamarkupfalse%
\ auto\isanewline
\ \isanewline
\ \ \isacommand{apply}\isamarkupfalse%
\ \ metis\isanewline
\isanewline
\ \ \isacommand{done}\isamarkupfalse%
%
\endisatagproof
{\isafoldproof}%
%
\isadelimproof
\isanewline
%
\endisadelimproof
\isanewline
\isacommand{lemma}\isamarkupfalse%
\ {\isachardoublequoteopen}surj\ f\ {\isasymLongrightarrow}\ {\isacharparenleft}{\isacharparenleft}g\ {\isasymcirc}\ f{\isacharparenright}\ {\isacharequal}\ {\isacharparenleft}h\ {\isasymcirc}\ f{\isacharparenright}\ {\isacharparenright}\ {\isacharequal}\ {\isacharparenleft}g\ {\isacharequal}\ h{\isacharparenright}{\isachardoublequoteclose}\isanewline
%
\isadelimproof
\ \ %
\endisadelimproof
%
\isatagproof
\isacommand{apply}\isamarkupfalse%
\ {\isacharparenleft}simp\ add{\isacharcolon}\ surj{\isacharunderscore}def\ fun{\isacharunderscore}eq{\isacharunderscore}iff\ {\isacharparenright}\ \isanewline
\ \ \isacommand{by}\isamarkupfalse%
\ metis%
\endisatagproof
{\isafoldproof}%
%
\isadelimproof
%
\endisadelimproof
%
\begin{isamarkuptext}%
En esta demostración hemos introducido:
 \begin{itemize}
    \item[] \isa{{\isacharparenleft}f\ {\isacharequal}\ g{\isacharparenright}\ {\isacharequal}\ {\isacharparenleft}{\isasymforall}x{\isachardot}\ f\ x\ {\isacharequal}\ g\ x{\isacharparenright}} 
      \hfill (\isa{fun{\isacharunderscore}eq{\isacharunderscore}iff})
  \end{itemize}%
\end{isamarkuptext}\isamarkuptrue%
%
\isadelimtheory
%
\endisadelimtheory
%
\isatagtheory
%
\endisatagtheory
{\isafoldtheory}%
%
\isadelimtheory
%
\endisadelimtheory
%
\end{isabellebody}%
\endinput
%:%file=~/Escritorio/TFG/EjerciciosDELMF/CancelacionSobreyectiva.thy%:%
%:%24=7%:%
%:%36=10%:%
%:%37=11%:%
%:%38=12%:%
%:%39=13%:%
%:%40=14%:%
%:%41=15%:%
%:%42=16%:%
%:%43=17%:%
%:%44=18%:%
%:%45=19%:%
%:%46=20%:%
%:%47=21%:%
%:%48=22%:%
%:%49=23%:%
%:%50=24%:%
%:%51=25%:%
%:%52=26%:%
%:%53=27%:%
%:%54=28%:%
%:%55=29%:%
%:%56=30%:%
%:%57=31%:%
%:%58=32%:%
%:%59=33%:%
%:%60=34%:%
%:%61=35%:%
%:%62=36%:%
%:%63=37%:%
%:%64=38%:%
%:%65=39%:%
%:%66=40%:%
%:%67=41%:%
%:%68=42%:%
%:%69=43%:%
%:%70=44%:%
%:%71=45%:%
%:%72=46%:%
%:%73=47%:%
%:%74=48%:%
%:%75=49%:%
%:%76=50%:%
%:%78=53%:%
%:%79=53%:%
%:%80=54%:%
%:%83=55%:%
%:%87=55%:%
%:%93=55%:%
%:%96=56%:%
%:%97=57%:%
%:%98=57%:%
%:%99=58%:%
%:%102=59%:%
%:%106=59%:%
%:%112=59%:%
%:%115=60%:%
%:%116=61%:%
%:%117=61%:%
%:%118=62%:%
%:%121=63%:%
%:%125=63%:%
%:%135=67%:%
%:%136=68%:%
%:%137=69%:%
%:%138=70%:%
%:%139=71%:%
%:%140=72%:%
%:%141=73%:%
%:%142=74%:%
%:%143=75%:%
%:%144=76%:%
%:%145=77%:%
%:%146=78%:%
%:%147=79%:%
%:%148=80%:%
%:%149=81%:%
%:%150=82%:%
%:%151=83%:%
%:%152=84%:%
%:%153=85%:%
%:%154=86%:%
%:%155=87%:%
%:%157=90%:%
%:%158=90%:%
%:%159=91%:%
%:%160=92%:%
%:%167=93%:%
%:%168=93%:%
%:%169=94%:%
%:%170=94%:%
%:%171=95%:%
%:%172=95%:%
%:%173=96%:%
%:%174=96%:%
%:%175=97%:%
%:%176=97%:%
%:%177=98%:%
%:%178=99%:%
%:%179=99%:%
%:%180=99%:%
%:%181=99%:%
%:%182=100%:%
%:%183=100%:%
%:%184=100%:%
%:%185=100%:%
%:%186=101%:%
%:%187=101%:%
%:%188=101%:%
%:%189=101%:%
%:%190=102%:%
%:%191=102%:%
%:%192=102%:%
%:%193=102%:%
%:%194=103%:%
%:%195=103%:%
%:%196=103%:%
%:%197=103%:%
%:%198=103%:%
%:%199=104%:%
%:%200=104%:%
%:%201=104%:%
%:%202=104%:%
%:%203=105%:%
%:%204=105%:%
%:%205=105%:%
%:%206=105%:%
%:%207=105%:%
%:%208=106%:%
%:%209=106%:%
%:%210=106%:%
%:%211=106%:%
%:%212=107%:%
%:%213=107%:%
%:%214=108%:%
%:%215=108%:%
%:%216=109%:%
%:%217=109%:%
%:%218=110%:%
%:%219=110%:%
%:%220=111%:%
%:%221=111%:%
%:%222=112%:%
%:%223=112%:%
%:%224=113%:%
%:%225=113%:%
%:%226=113%:%
%:%227=114%:%
%:%228=114%:%
%:%229=114%:%
%:%230=114%:%
%:%231=114%:%
%:%232=115%:%
%:%233=115%:%
%:%234=115%:%
%:%235=115%:%
%:%236=116%:%
%:%237=116%:%
%:%238=116%:%
%:%239=116%:%
%:%240=117%:%
%:%241=117%:%
%:%242=118%:%
%:%252=121%:%
%:%253=122%:%
%:%254=123%:%
%:%255=124%:%
%:%256=125%:%
%:%257=126%:%
%:%258=127%:%
%:%259=128%:%
%:%260=129%:%
%:%261=130%:%
%:%262=131%:%
%:%264=133%:%
%:%265=133%:%
%:%268=134%:%
%:%272=134%:%
%:%273=134%:%
%:%274=135%:%
%:%275=135%:%
%:%276=136%:%
%:%277=136%:%
%:%278=137%:%
%:%279=137%:%
%:%280=138%:%
%:%281=139%:%
%:%282=139%:%
%:%283=140%:%
%:%284=141%:%
%:%290=141%:%
%:%293=142%:%
%:%294=143%:%
%:%295=143%:%
%:%298=144%:%
%:%302=144%:%
%:%303=144%:%
%:%304=145%:%
%:%305=145%:%
%:%314=148%:%
%:%315=149%:%
%:%316=150%:%
%:%317=151%:%
%:%318=152%:%
\chapter{Teoría de conjuntos}
%
\begin{isabellebody}%
\setisabellecontext{TeoremaCantor}%
%
\isadelimtheory
%
\endisadelimtheory
%
\isatagtheory
%
\endisatagtheory
{\isafoldtheory}%
%
\isadelimtheory
%
\endisadelimtheory
%
\isadelimdocument
%
\endisadelimdocument
%
\isatagdocument
%
\isamarkupsection{Teorema de Cantor%
}
\isamarkuptrue%
%
\endisatagdocument
{\isafolddocument}%
%
\isadelimdocument
%
\endisadelimdocument
%
\begin{isamarkuptext}%
El siguiente, denominado  teorema de Cantor por el matemático
 Georg Cantor, es un resultado importante de la teoría
 de conjuntos. 

El matemático Georg Ferdinand Ludwig Philipp Cantor fue un matemático y
lógico nacido en Rusia en el siglo XIX. Fue inventor junto con Dedekind
 y Frege de la teoría de conjuntos, que es la base de las matemáticas
 modernas.



Para la comprensión del teorema vamos a definir una serie de conceptos:

\begin {itemize}

\item Conjunto de potencia $A$  $(\mathcal{P}(A))$: conjunto formado por
todos los subconjuntos de $A$.
\item Cardinal del conjunto $A$ (Denotado $\# A$): número de elementos del propio
 conjunto.

\end {itemize}
El enunciado original del teorema es el siguiente : 


\begin {teorema}
El cardinal del conjunto potencia de cualquier conjunto A es
 estrictamente mayor que el cardinal de A, o lo que es lo mismo,
$\# \mathcal{P}(A) > \# A.$


\end {teorema}
Pero el enunciado del teorema lo podemos reformular como: 
\begin{teorema}
Dado un conjunto A, $\nexists  f : A \longrightarrow \mathcal{P}(A)$ que
sea sobreyectiva.

\end{teorema}

El teorema lo hemos podido reescribir de la anterior forma, ya que si
 suponemos que $\exists f$ tal que $f: A \longrightarrow \mathcal{P}(A)$
es sobreyectiva, entonces tenemos que $f(A) = \mathcal{P}(A)$ y por lo
 tanto, $\# f(A) \geq \# \mathcal{P}(A)$, de lo que se deduce esta
 reformulación. Reciprocamente, es trivial ver que esta reformulación
 implica la primera.
 con el teorema. \\
El teorema de Cantor es trivial para conjuntos finitos, ya que el
 conjunto potencia, de conjuntos finitos de n elementos tiene
 $2^n$ elementos.

Por ello,  vamos a realizar la prueba para conjuntos infinitos. 


\begin{demostracion}
 
Vamos a realizar la prueba por reducción al absurdo.\\
Supongamos que $\exists f : A \longrightarrow \mathcal{P}(A)$ sobreyectiva, es
 decir, $\forall C \in \rho(A) ,  \exists x \in A$ tal que $C = f(x)$.
En particular, tomemos el conjunto $$B = \{ x \in A : x \notin f(x) \}$$
 y  supongamos que $\exists a \in A : B = f(a)$, ya que $B$ es un
 subconjunto de A, luego podemos distinguir dos casos $:$ \\
$1.$ Si $a \in B$, entonces por definición del conjunto $B$ tenemos que
$a \notin B$, luego llegamos a una contradicción. \\
$2.$ Si $a \notin B$, entonces por definición de B tenemos que $a \in 
B$, luego hemos llegado a otra contradicción. 

En las dos hipótesis hemos llegado a una contradicción,
por lo que no existe $a$ y $f$ no es sobreyectiva.
\end{demostracion}


Para la especificación del teorema en Isabelle, primero debemos notar
 que $$f :: \, 'a \Rightarrow \,'a \: set$$
 significa que es una función 
de tipos, donde $'a$ significa un tipo y para poder denotar
el conjunto potencia tenemos que poner $'a \ set$ que significa que es
 de un tipo formado por conjuntos del tipo $'a$.




El enunciado del teorema es el siguiente :%
\end{isamarkuptext}\isamarkuptrue%
\isacommand{theorem}\isamarkupfalse%
\ Cantor{\isacharcolon}\ {\isachardoublequoteopen}{\isasymnexists}f\ {\isacharcolon}{\isacharcolon}\ {\isacharprime}a\ {\isasymRightarrow}\ {\isacharprime}a\ set{\isachardot}\ {\isasymforall}A{\isachardot}\ {\isasymexists}x{\isachardot}\ A\ {\isacharequal}\ f\ x{\isachardoublequoteclose}\isanewline
\isanewline
%
\isadelimproof
\isanewline
\ \ %
\endisadelimproof
%
\isatagproof
\isacommand{oops}\isamarkupfalse%
%
\endisatagproof
{\isafoldproof}%
%
\isadelimproof
%
\endisadelimproof
%
\begin{isamarkuptext}%
La demostración la haremos por la regla la introducción a la
negación, la cual es una simplificación de la regla de 
reducción al absurdo, cuyo esquema mostramos a continuación:   
 \begin{itemize}
  \item[] \isa{{\isacharparenleft}P\ {\isasymLongrightarrow}\ False{\isacharparenright}\ {\isasymLongrightarrow}\ {\isasymnot}\ P} \hfill (\isa{notI})
  \end{itemize}


Esta es la demostración detallada del teorema:%
\end{isamarkuptext}\isamarkuptrue%
\isacommand{theorem}\isamarkupfalse%
\ CantorDetallada{\isacharcolon}\ {\isachardoublequoteopen}{\isasymnexists}f\ {\isacharcolon}{\isacharcolon}\ {\isacharprime}a\ {\isasymRightarrow}\ {\isacharprime}a\ set{\isachardot}\ {\isasymforall}B{\isachardot}\ {\isasymexists}x{\isachardot}\ B\ {\isacharequal}\ f\ x{\isachardoublequoteclose}\isanewline
%
\isadelimproof
%
\endisadelimproof
%
\isatagproof
\isacommand{proof}\isamarkupfalse%
\ {\isacharparenleft}rule\ notI{\isacharparenright}\isanewline
\ \ \isacommand{assume}\isamarkupfalse%
\ {\isachardoublequoteopen}{\isasymexists}f\ {\isacharcolon}{\isacharcolon}\ {\isacharprime}a\ {\isasymRightarrow}\ {\isacharprime}a\ set{\isachardot}\ {\isasymforall}A{\isachardot}\ {\isasymexists}x{\isachardot}\ A\ {\isacharequal}\ f\ x{\isachardoublequoteclose}\isanewline
\ \ \isacommand{then}\isamarkupfalse%
\ \isacommand{obtain}\isamarkupfalse%
\ f\ {\isacharcolon}{\isacharcolon}\ {\isachardoublequoteopen}{\isacharprime}a\ {\isasymRightarrow}\ {\isacharprime}a\ set{\isachardoublequoteclose}\ \isakeyword{where}\ {\isacharasterisk}{\isacharcolon}\ {\isachardoublequoteopen}{\isasymforall}A{\isachardot}\ {\isasymexists}x{\isachardot}\ A\ {\isacharequal}\ f\ x{\isachardoublequoteclose}\ \isacommand{by}\isamarkupfalse%
\ {\isacharparenleft}rule\isanewline
\ \ \ \ \ \ \ \ exE{\isacharparenright}\isanewline
\ \ \isacommand{let}\isamarkupfalse%
\ {\isacharquery}B\ {\isacharequal}\ {\isachardoublequoteopen}{\isacharbraceleft}x{\isachardot}\ x\ {\isasymnotin}\ f\ x{\isacharbraceright}{\isachardoublequoteclose}\isanewline
\ \ \isacommand{from}\isamarkupfalse%
\ {\isacharasterisk}\ \isacommand{obtain}\isamarkupfalse%
\ {\isachardoublequoteopen}\ {\isasymexists}x{\isachardot}\ {\isacharquery}B\ {\isacharequal}\ f\ x\ {\isachardoublequoteclose}\ \isacommand{by}\isamarkupfalse%
\ {\isacharparenleft}rule\ allE{\isacharparenright}\isanewline
\ \ \isacommand{then}\isamarkupfalse%
\ \ \isacommand{obtain}\isamarkupfalse%
\ a\ \isakeyword{where}\ {\isadigit{1}}{\isacharcolon}{\isachardoublequoteopen}{\isacharquery}B\ {\isacharequal}\ f\ a{\isachardoublequoteclose}\ \isacommand{by}\isamarkupfalse%
\ {\isacharparenleft}rule\ exE{\isacharparenright}\isanewline
\ \ \isacommand{show}\isamarkupfalse%
\ False\isanewline
\ \ \isacommand{proof}\isamarkupfalse%
\ {\isacharparenleft}cases{\isacharparenright}\isanewline
\ \ \ \ \isacommand{assume}\isamarkupfalse%
\ {\isachardoublequoteopen}a\ {\isasymin}\ {\isacharquery}B{\isachardoublequoteclose}\ \ \isanewline
\ \ \ \ \isacommand{then}\isamarkupfalse%
\ \isacommand{show}\isamarkupfalse%
\ False\ \ \isacommand{using}\isamarkupfalse%
\ {\isadigit{1}}\ \isacommand{by}\isamarkupfalse%
\ blast\isanewline
\ \ \isacommand{next}\isamarkupfalse%
\ \isanewline
\ \ \ \ \isacommand{assume}\isamarkupfalse%
\ {\isachardoublequoteopen}a\ {\isasymnotin}\ {\isacharquery}B{\isachardoublequoteclose}\isanewline
\ \ \ \ \isacommand{thus}\isamarkupfalse%
\ False\ \isacommand{using}\isamarkupfalse%
\ {\isadigit{1}}\ \isacommand{by}\isamarkupfalse%
\ blast\isanewline
\ \ \isacommand{qed}\isamarkupfalse%
\isanewline
\isacommand{qed}\isamarkupfalse%
%
\endisatagproof
{\isafoldproof}%
%
\isadelimproof
%
\endisadelimproof
%
\begin{isamarkuptext}%
Esta es la demostración aplicativa del teorema:%
\end{isamarkuptext}\isamarkuptrue%
\isacommand{theorem}\isamarkupfalse%
\ CantorAplicativa\ {\isacharcolon}\isanewline
\ {\isachardoublequoteopen}{\isasymnexists}f\ {\isacharcolon}{\isacharcolon}\ {\isacharprime}a\ {\isasymRightarrow}\ {\isacharprime}a\ set{\isachardot}\ {\isasymforall}A{\isachardot}\ {\isasymexists}x{\isachardot}\ A\ {\isacharequal}\ f\ x{\isachardoublequoteclose}\isanewline
%
\isadelimproof
\ \ %
\endisadelimproof
%
\isatagproof
\isacommand{apply}\isamarkupfalse%
\ {\isacharparenleft}rule\ notI{\isacharparenright}\isanewline
\ \ \isacommand{apply}\isamarkupfalse%
\ {\isacharparenleft}erule\ exE{\isacharparenright}\isanewline
\ \ \isacommand{apply}\isamarkupfalse%
\ {\isacharparenleft}erule{\isacharunderscore}tac\ x\ {\isacharequal}\ {\isachardoublequoteopen}{\isacharbraceleft}x{\isachardot}\ x\ {\isasymnotin}\ f\ x{\isacharbraceright}{\isachardoublequoteclose}\ \isakeyword{in}\ allE{\isacharparenright}\isanewline
\ \ \isacommand{apply}\isamarkupfalse%
\ {\isacharparenleft}erule\ exE{\isacharparenright}\isanewline
\ \ \isacommand{apply}\isamarkupfalse%
\ \ blast\ \isanewline
\ \ \isacommand{done}\isamarkupfalse%
%
\endisatagproof
{\isafoldproof}%
%
\isadelimproof
%
\endisadelimproof
%
\begin{isamarkuptext}%
Esta es la demostración automática del teorema:%
\end{isamarkuptext}\isamarkuptrue%
\isacommand{theorem}\isamarkupfalse%
\ CantorAutomatic{\isacharcolon}\ {\isachardoublequoteopen}{\isasymnexists}f\ {\isacharcolon}{\isacharcolon}\ {\isacharprime}a\ {\isasymRightarrow}\ {\isacharprime}a\ set{\isachardot}\ {\isasymforall}B{\isachardot}\ {\isasymexists}x{\isachardot}\ B\ {\isacharequal}\ f\ x{\isachardoublequoteclose}\isanewline
%
\isadelimproof
\ \ %
\endisadelimproof
%
\isatagproof
\isacommand{by}\isamarkupfalse%
\ best%
\endisatagproof
{\isafoldproof}%
%
\isadelimproof
%
\endisadelimproof
%
\begin{isamarkuptext}%
En la demostración de isabelle hemos utilizado el método de prueba
rule con las siguientes reglas, tanto en la aplicativa como en la
 detallada:
 \begin{itemize}
  \item[] \isa{\mbox{}\inferrule{\mbox{\mbox{}\inferrule{\mbox{P}}{\mbox{False}}}}{\mbox{{\isasymnot}\ P}}} \hfill (\isa{notI})
  \end{itemize}
 \begin{itemize}
  \item[] \isa{\mbox{}\inferrule{\mbox{{\isasymexists}x{\isachardot}\ P\ x}\\\ \mbox{{\isasymAnd}x{\isachardot}\ \mbox{}\inferrule{\mbox{P\ x}}{\mbox{Q}}}}{\mbox{Q}}} \hfill (\isa{exE})
  \end{itemize}
 \begin{itemize}
  \item[] \isa{\mbox{}\inferrule{\mbox{{\isasymforall}x{\isachardot}\ P\ x}\\\ \mbox{\mbox{}\inferrule{\mbox{P\ x}}{\mbox{R}}}}{\mbox{R}}} \hfill (\isa{allE})
  \end{itemize}
También hacemos uso de blast, que es un conjunto de reglas lógicas y 
 la demostración automática la hacemos por medio de "best".%
\end{isamarkuptext}\isamarkuptrue%
%
\isadelimtheory
%
\endisadelimtheory
%
\isatagtheory
%
\endisatagtheory
{\isafoldtheory}%
%
\isadelimtheory
%
\endisadelimtheory
%
\end{isabellebody}%
\endinput
%:%file=~/Escritorio/TFG-v1/EjerciciosDELMF/TeoremaCantor.thy%:%
%:%24=7%:%
%:%36=9%:%
%:%37=10%:%
%:%38=11%:%
%:%39=12%:%
%:%40=13%:%
%:%41=14%:%
%:%42=15%:%
%:%43=16%:%
%:%44=17%:%
%:%45=18%:%
%:%46=19%:%
%:%47=20%:%
%:%48=21%:%
%:%49=22%:%
%:%50=23%:%
%:%51=24%:%
%:%52=25%:%
%:%53=26%:%
%:%54=27%:%
%:%55=28%:%
%:%56=29%:%
%:%57=30%:%
%:%58=31%:%
%:%59=32%:%
%:%60=33%:%
%:%61=34%:%
%:%62=35%:%
%:%63=36%:%
%:%64=37%:%
%:%65=38%:%
%:%66=39%:%
%:%67=40%:%
%:%68=41%:%
%:%69=42%:%
%:%70=43%:%
%:%71=44%:%
%:%72=45%:%
%:%73=46%:%
%:%74=47%:%
%:%75=48%:%
%:%76=49%:%
%:%77=50%:%
%:%78=51%:%
%:%79=52%:%
%:%80=53%:%
%:%81=54%:%
%:%82=55%:%
%:%83=56%:%
%:%84=57%:%
%:%85=58%:%
%:%86=59%:%
%:%87=60%:%
%:%88=61%:%
%:%89=62%:%
%:%90=63%:%
%:%91=64%:%
%:%92=65%:%
%:%93=66%:%
%:%94=67%:%
%:%95=68%:%
%:%96=69%:%
%:%97=70%:%
%:%98=71%:%
%:%99=72%:%
%:%100=73%:%
%:%101=74%:%
%:%102=75%:%
%:%103=76%:%
%:%104=77%:%
%:%105=78%:%
%:%106=79%:%
%:%107=80%:%
%:%108=81%:%
%:%109=82%:%
%:%110=83%:%
%:%111=84%:%
%:%112=85%:%
%:%113=86%:%
%:%114=87%:%
%:%115=88%:%
%:%116=89%:%
%:%118=91%:%
%:%119=91%:%
%:%120=92%:%
%:%123=93%:%
%:%124=94%:%
%:%128=94%:%
%:%138=96%:%
%:%139=97%:%
%:%140=98%:%
%:%141=99%:%
%:%142=100%:%
%:%143=101%:%
%:%144=102%:%
%:%145=103%:%
%:%146=104%:%
%:%148=106%:%
%:%149=106%:%
%:%156=107%:%
%:%157=107%:%
%:%158=108%:%
%:%159=108%:%
%:%160=109%:%
%:%161=109%:%
%:%162=109%:%
%:%163=109%:%
%:%164=110%:%
%:%165=111%:%
%:%166=111%:%
%:%167=112%:%
%:%168=112%:%
%:%169=112%:%
%:%170=112%:%
%:%171=113%:%
%:%172=113%:%
%:%173=113%:%
%:%174=113%:%
%:%175=114%:%
%:%176=114%:%
%:%177=115%:%
%:%178=115%:%
%:%179=116%:%
%:%180=116%:%
%:%181=117%:%
%:%182=117%:%
%:%183=117%:%
%:%184=117%:%
%:%185=117%:%
%:%186=118%:%
%:%187=118%:%
%:%188=119%:%
%:%189=119%:%
%:%190=120%:%
%:%191=120%:%
%:%192=120%:%
%:%193=120%:%
%:%194=121%:%
%:%195=121%:%
%:%196=122%:%
%:%206=124%:%
%:%208=127%:%
%:%209=127%:%
%:%210=128%:%
%:%213=129%:%
%:%217=129%:%
%:%218=129%:%
%:%219=130%:%
%:%220=130%:%
%:%221=131%:%
%:%222=131%:%
%:%223=132%:%
%:%224=132%:%
%:%225=133%:%
%:%226=133%:%
%:%227=134%:%
%:%237=136%:%
%:%239=137%:%
%:%240=137%:%
%:%243=138%:%
%:%247=138%:%
%:%248=138%:%
%:%257=140%:%
%:%258=141%:%
%:%259=142%:%
%:%260=143%:%
%:%261=144%:%
%:%262=145%:%
%:%263=146%:%
%:%264=147%:%
%:%265=148%:%
%:%266=149%:%
%:%267=150%:%
%:%268=151%:%
%:%269=152%:%
%:%270=153%:%
\chapter{Teoría de retículos}
\input{TeoremaKnasterTarski}
\chapter{Teoría de geometría}
\input{Geometria}

\appendix

\chapter{Lemas de HOL usados}
%
\begin{isabellebody}%
\setisabellecontext{Soporte}%
%
\isadelimtheory
%
\endisadelimtheory
%
\isatagtheory
%
\endisatagtheory
{\isafoldtheory}%
%
\isadelimtheory
%
\endisadelimtheory
%
\begin{isamarkuptext}%
En este apéndice se recogen la lista de los lemas usados en
  el trabajo indicando la página del
  \href{http://bit.ly/2OMbjMM}{libro de HOL} donde se encuentra.%
\end{isamarkuptext}\isamarkuptrue%
%
\begin{isamarkuptext}%
\comentario{Añadir el libro de HOL a la bibliografía.}%
\end{isamarkuptext}\isamarkuptrue%
%
\begin{isamarkuptext}%
\comentario{Completar la lista de lemas usados.}%
\end{isamarkuptext}\isamarkuptrue%
%
\isadelimdocument
%
\endisadelimdocument
%
\isatagdocument
%
\isamarkupsection{Números naturales (16)%
}
\isamarkuptrue%
%
\isamarkupsubsection{Operaciones aritméticas (16.3)%
}
\isamarkuptrue%
%
\endisatagdocument
{\isafolddocument}%
%
\isadelimdocument
%
\endisadelimdocument
%
\begin{isamarkuptext}%
\begin{itemize}
  \item (p. 348) \isa{{\isadigit{0}}\ {\isacharasterisk}\ n\ {\isacharequal}\ {\isadigit{0}}}
    \hfill (\isa{mult{\isacharunderscore}{\isadigit{0}}}) 
  \item (p. 348) \isa{Suc\ m\ {\isacharasterisk}\ n\ {\isacharequal}\ n\ {\isacharplus}\ m\ {\isacharasterisk}\ n}
    \hfill (\isa{mult{\isacharunderscore}Suc}) 
  \item (p. 348) \isa{m\ {\isacharasterisk}\ Suc\ n\ {\isacharequal}\ m\ {\isacharplus}\ m\ {\isacharasterisk}\ n}
    \hfill (\isa{mult{\isacharunderscore}Suc{\isacharunderscore}right}) 
\end{itemize}%
\end{isamarkuptext}\isamarkuptrue%
%
\isadelimtheory
%
\endisadelimtheory
%
\isatagtheory
%
\endisatagtheory
{\isafoldtheory}%
%
\isadelimtheory
%
\endisadelimtheory
%
\end{isabellebody}%
\endinput
%:%file=~/ownCloud/alonso/curso-TFG/Carlos/TFG_de_Carlos/Soporte.thy%:%
%:%19=11%:%
%:%20=12%:%
%:%21=13%:%
%:%25=15%:%
%:%29=17%:%
%:%38=19%:%
%:%42=21%:%
%:%54=24%:%
%:%55=25%:%
%:%56=26%:%
%:%57=27%:%
%:%58=28%:%
%:%59=29%:%
%:%60=30%:%
%:%61=31%:%

% optional bibliography
\nocite{LMF, tutorial, Isabelle, Prooftheory, 100theorems, Provers, Isabelle/Isar,
 Mathematical, Naive, Lattices, ALgebra, Tarski, Simple, Finiteplane, Fanoplane}
\bibliographystyle{plain}
\bibliography{root}

% Pendientes
\todototoc
\listoftodos

\end{document}

%%% Local Variables:
%%% mode: latex
%%% TeX-master: t
%%% End:
