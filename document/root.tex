\documentclass[12pt,a4paper,twoside]{book}
\usepackage{isabelle,isabellesym}
\usepackage{ifthen,mathpartir}

% further packages required for unusual symbols (see also
% isabellesym.sty), use only when needed

% Personalización
% \usepackage{color,graphicx}        % Usa figuras.
\usepackage[utf8x]{inputenc}         % Acentos de UTF8
% \usepackage[T1]{fontenc}           % Codificación T1 con European Computer
% \usepackage[spanish]{babel}        % Castellanización.
% \usepackage{ucs}
\usepackage{mathpazo}              % Tipo de fuente
\usepackage[scaled=.90]{helvet}    % Tipo de fuente
\usepackage{a4wide}                % Márgenes
\linespread{1.05}                  % Distancia entre líneas
\setlength{\parindent}{2em}        % Indentación de comienzo de párrafo

\usepackage[colorinlistoftodos
           , backgroundcolor = yellow
           , textwidth = 4cm
           , shadow
           , spanish]{todonotes}

\setcounter{secnumdepth}{3}
           
\usepackage{amssymb}
  %for \<leadsto>, \<box>, \<diamond>, \<sqsupset>, \<mho>, \<Join>,
  %\<lhd>, \<lesssim>, \<greatersim>, \<lessapprox>, \<greaterapprox>,
  %\<triangleq>, \<yen>, \<lozenge>

%\usepackage{eurosym}
  %for \<euro>

%\usepackage[only,bigsqcap]{stmaryrd}
  %for \<Sqinter>

%\usepackage{eufrak}
  %for \<AA> ... \<ZZ>, \<aa> ... \<zz> (also included in amssymb)

% \usepackage{textcomp}
  %for \<onequarter>, \<onehalf>, \<threequarters>, \<degree>, \<cent>,
  %\<currency>

% this should be the last package used
\usepackage{pdfsetup}

% urls in roman style, theory text in math-similar italics
\urlstyle{rm}
\isabellestyle{it}

% for uniform font size
\renewcommand{\isastyle}{\isastyleminor}

% Nota: Definiciones
\input definiciones
\input castellano

% No ajusta los espacios verticales.
\raggedbottom

% Espacio entre párrafos
\parindent 2em\parskip 1ex
% Diagramas conmutativos 
%%%%%%%%%%%%%%%%%%%%%%%%%%%%%%%%%%%%%%%%%%%%%%%%%%%%%%%%%%%%%%%%%%%%%%%%%%%%%%
%% Cabeceras                                                              %%
%%%%%%%%%%%%%%%%%%%%%%%%%%%%%%%%%%%%%%%%%%%%%%%%%%%%%%%%%%%%%%%%%%%%%%%%%%%%%%

\usepackage{fancyhdr}

\addtolength{\headheight}{\baselineskip}

\pagestyle{fancy}

\cfoot{}

\fancyhead{}
\fancyhead[RE]{\slshape \nouppercase{\leftmark}}
\fancyhead[LO]{\slshape \nouppercase{\rightmark}}
\fancyhead[LE,RO]{\slshape \thepage}

%%%%%%%%%%%%%%%%%%%%%%%%%%%%%%%%%%%%%%%%%%%%%%%%%%%%%%%%%%%%%%%%%%%%%%%%%%%%%%
%% Documento
%%%%%%%%%%%%%%%%%%%%%%%%%%%%%%%%%%%%%%%%%%%%%%%%%%%%%%%%%%%%%%%%%%%%%%%%%%%%%%

\begin{document}

\title{Elementos de matemáticas formalizados en Isabelle/HOL}
\author{Carlos Núñez Fernández}
\date{18 de noviembre de 2019}
\maketitle

% \begin{abstract}
%   En este trabajo vamos a presentar la formalización en Isabelle/HOL de
%   una selección de teoremas de distintos campos de las matemáticas.
% \end{abstract}

\tableofcontents

% sane default for proof documents
% \parindent 0pt\parskip 0.5ex
\parindent 2em\parskip 1ex

% generated text of all theories
% %
\begin{isabellebody}%
\setisabellecontext{SumaImpares}%
%
\isadelimtheory
%
\endisadelimtheory
%
\isatagtheory
%
\endisatagtheory
{\isafoldtheory}%
%
\isadelimtheory
%
\endisadelimtheory
%
\isadelimdocument
%
\endisadelimdocument
%
\isatagdocument
%
\isamarkupsection{Suma de los primeros números impares%
}
\isamarkuptrue%
%
\endisatagdocument
{\isafolddocument}%
%
\isadelimdocument
%
\endisadelimdocument
%
\begin{isamarkuptext}%
El primer teorema es una propiedad de los números naturales.

  \begin{teorema}
    La suma de los $n$ primeros números impares es $n^2$.
  \end{teorema}

  \begin{demostracion}
    La demostración la haremos en inducción sobre $n$.
\begin {itemize}
\item EL caso $n = 0$ es trivial, ya que $0 = 0$.
\item Supongamos que se verifica la hipótesis para $n$ y veamos para
 $n+1$. \\
Tenemos que demostrar que $\sum_{j=1}^{n+1} k_j = (n+1)^2$ siendo los
 $k_{j}$ el j-ésimo impar, es decir, $k_{j} = 2j - 1$.
$$\sum_{j = 1}^{n+1} k_{j} = k_{n+1} + \sum^{n}_{j=1} k_{j} = k_{n+1} +
 n^{2} = 2(n+1) - 1 + n^2 = n^2 + 2n + 1 = (n+1)^2$$ 
\end {itemize}
.
  \end{demostracion}

  Para especificar el teorema en Isabelle, se comienza definiendo 
  la función \isa{suma{\isacharunderscore}impares} tal que \isa{suma{\isacharunderscore}impares\ n} es la 
  suma de los $n$ primeros números impares%
\end{isamarkuptext}\isamarkuptrue%
\isacommand{fun}\isamarkupfalse%
\ suma{\isacharunderscore}impares\ {\isacharcolon}{\isacharcolon}\ {\isachardoublequoteopen}nat\ {\isasymRightarrow}\ nat{\isachardoublequoteclose}\ \isakeyword{where}\isanewline
\ \ {\isachardoublequoteopen}suma{\isacharunderscore}impares\ {\isadigit{0}}\ {\isacharequal}\ {\isadigit{0}}{\isachardoublequoteclose}\ \isanewline
{\isacharbar}\ {\isachardoublequoteopen}suma{\isacharunderscore}impares\ {\isacharparenleft}Suc\ n{\isacharparenright}\ {\isacharequal}\ {\isacharparenleft}{\isadigit{2}}{\isacharasterisk}{\isacharparenleft}Suc\ n{\isacharparenright}\ {\isacharminus}\ {\isadigit{1}}{\isacharparenright}\ {\isacharplus}\ suma{\isacharunderscore}impares\ n{\isachardoublequoteclose}%
\begin{isamarkuptext}%
El enunciado del teorema es el siguiente:%
\end{isamarkuptext}\isamarkuptrue%
\isacommand{lemma}\isamarkupfalse%
\ {\isachardoublequoteopen}suma{\isacharunderscore}impares\ n\ {\isacharequal}\ n\ {\isacharasterisk}\ n{\isachardoublequoteclose}\isanewline
%
\isadelimproof
%
\endisadelimproof
%
\isatagproof
\isacommand{oops}\isamarkupfalse%
%
\endisatagproof
{\isafoldproof}%
%
\isadelimproof
%
\endisadelimproof
%
\begin{isamarkuptext}%
En la demostración se usará la táctica \isa{induct} que hace
  uso del esquema de inducción sobre los naturales:
  \begin{itemize}
  \item[] \isa{\mbox{}\inferrule{\mbox{P\ {\isadigit{0}}}\\\ \mbox{{\isasymAnd}nat{\isachardot}\ \mbox{}\inferrule{\mbox{P\ nat}}{\mbox{P\ {\isacharparenleft}Suc\ nat{\isacharparenright}}}}}{\mbox{P\ nat}}} \hfill (\isa{nat{\isachardot}induct})
  \end{itemize}

  Vamos a presentar distintas demostraciones del teorema. La 
  primera es la demostración aplicativa%
\end{isamarkuptext}\isamarkuptrue%
\isacommand{lemma}\isamarkupfalse%
\ {\isachardoublequoteopen}suma{\isacharunderscore}impares\ n\ {\isacharequal}\ n\ {\isacharasterisk}\ n{\isachardoublequoteclose}\isanewline
%
\isadelimproof
\ \ %
\endisadelimproof
%
\isatagproof
\isacommand{apply}\isamarkupfalse%
\ {\isacharparenleft}induct\ n{\isacharparenright}\ \isanewline
\ \ \ \isacommand{apply}\isamarkupfalse%
\ simp{\isacharunderscore}all\isanewline
\ \ \isacommand{done}\isamarkupfalse%
%
\endisatagproof
{\isafoldproof}%
%
\isadelimproof
%
\endisadelimproof
%
\begin{isamarkuptext}%
La demostración automática es%
\end{isamarkuptext}\isamarkuptrue%
\isacommand{lemma}\isamarkupfalse%
\ {\isachardoublequoteopen}suma{\isacharunderscore}impares\ n\ {\isacharequal}\ n\ {\isacharasterisk}\ n{\isachardoublequoteclose}\isanewline
%
\isadelimproof
\ \ %
\endisadelimproof
%
\isatagproof
\isacommand{by}\isamarkupfalse%
\ {\isacharparenleft}induct\ n{\isacharparenright}\ simp{\isacharunderscore}all%
\endisatagproof
{\isafoldproof}%
%
\isadelimproof
%
\endisadelimproof
%
\begin{isamarkuptext}%
La demostración del lema anterior por inducción y razonamiento 
   ecuacional es%
\end{isamarkuptext}\isamarkuptrue%
\isacommand{lemma}\isamarkupfalse%
\ {\isachardoublequoteopen}suma{\isacharunderscore}impares\ n\ {\isacharequal}\ n\ {\isacharasterisk}\ n{\isachardoublequoteclose}\isanewline
%
\isadelimproof
%
\endisadelimproof
%
\isatagproof
\isacommand{proof}\isamarkupfalse%
\ {\isacharparenleft}induct\ n{\isacharparenright}\isanewline
\ \ \isacommand{show}\isamarkupfalse%
\ {\isachardoublequoteopen}suma{\isacharunderscore}impares\ {\isadigit{0}}\ {\isacharequal}\ {\isadigit{0}}\ {\isacharasterisk}\ {\isadigit{0}}{\isachardoublequoteclose}\ \isacommand{by}\isamarkupfalse%
\ simp\isanewline
\isacommand{next}\isamarkupfalse%
\isanewline
\ \ \isacommand{fix}\isamarkupfalse%
\ n\ \isacommand{assume}\isamarkupfalse%
\ HI{\isacharcolon}\ {\isachardoublequoteopen}suma{\isacharunderscore}impares\ n\ {\isacharequal}\ n\ {\isacharasterisk}\ n{\isachardoublequoteclose}\isanewline
\ \ \isacommand{have}\isamarkupfalse%
\ {\isachardoublequoteopen}suma{\isacharunderscore}impares\ {\isacharparenleft}Suc\ n{\isacharparenright}\ {\isacharequal}\ {\isacharparenleft}{\isadigit{2}}\ {\isacharasterisk}\ {\isacharparenleft}Suc\ n{\isacharparenright}\ {\isacharminus}\ {\isadigit{1}}{\isacharparenright}\ {\isacharplus}\ suma{\isacharunderscore}impares\ n{\isachardoublequoteclose}\ \isanewline
\ \ \ \ \isacommand{by}\isamarkupfalse%
\ simp\isanewline
\ \ \isacommand{also}\isamarkupfalse%
\ \isacommand{have}\isamarkupfalse%
\ {\isachardoublequoteopen}{\isasymdots}\ {\isacharequal}\ {\isacharparenleft}{\isadigit{2}}\ {\isacharasterisk}\ {\isacharparenleft}Suc\ n{\isacharparenright}\ {\isacharminus}\ {\isadigit{1}}{\isacharparenright}\ {\isacharplus}\ n\ {\isacharasterisk}\ n{\isachardoublequoteclose}\ \isacommand{using}\isamarkupfalse%
\ HI\ \isacommand{by}\isamarkupfalse%
\ simp\isanewline
\ \ \isacommand{also}\isamarkupfalse%
\ \isacommand{have}\isamarkupfalse%
\ {\isachardoublequoteopen}{\isasymdots}\ {\isacharequal}\ n\ {\isacharasterisk}\ n\ {\isacharplus}\ {\isadigit{2}}\ {\isacharasterisk}\ n\ {\isacharplus}\ {\isadigit{1}}{\isachardoublequoteclose}\ \isacommand{by}\isamarkupfalse%
\ simp\isanewline
\ \ \isacommand{finally}\isamarkupfalse%
\ \isacommand{show}\isamarkupfalse%
\ {\isachardoublequoteopen}suma{\isacharunderscore}impares\ {\isacharparenleft}Suc\ n{\isacharparenright}\ {\isacharequal}\ {\isacharparenleft}Suc\ n{\isacharparenright}\ {\isacharasterisk}\ {\isacharparenleft}Suc\ n{\isacharparenright}{\isachardoublequoteclose}\ \isacommand{by}\isamarkupfalse%
\ simp\isanewline
\isacommand{qed}\isamarkupfalse%
%
\endisatagproof
{\isafoldproof}%
%
\isadelimproof
%
\endisadelimproof
%
\begin{isamarkuptext}%
La demostración del lema anterior con patrones y razonamiento 
   ecuacional es%
\end{isamarkuptext}\isamarkuptrue%
\isacommand{lemma}\isamarkupfalse%
\ {\isachardoublequoteopen}suma{\isacharunderscore}impares\ n\ {\isacharequal}\ n\ {\isacharasterisk}\ n{\isachardoublequoteclose}\ {\isacharparenleft}\isakeyword{is}\ {\isachardoublequoteopen}{\isacharquery}P\ n{\isachardoublequoteclose}{\isacharparenright}\isanewline
%
\isadelimproof
%
\endisadelimproof
%
\isatagproof
\isacommand{proof}\isamarkupfalse%
\ {\isacharparenleft}induct\ n{\isacharparenright}\isanewline
\ \ \isacommand{show}\isamarkupfalse%
\ {\isachardoublequoteopen}{\isacharquery}P\ {\isadigit{0}}{\isachardoublequoteclose}\ \isacommand{by}\isamarkupfalse%
\ simp\isanewline
\isacommand{next}\isamarkupfalse%
\isanewline
\ \ \isacommand{fix}\isamarkupfalse%
\ n\ \isanewline
\ \ \isacommand{assume}\isamarkupfalse%
\ HI{\isacharcolon}\ {\isachardoublequoteopen}{\isacharquery}P\ n{\isachardoublequoteclose}\isanewline
\ \ \isacommand{have}\isamarkupfalse%
\ {\isachardoublequoteopen}suma{\isacharunderscore}impares\ {\isacharparenleft}Suc\ n{\isacharparenright}\ {\isacharequal}\ {\isacharparenleft}{\isadigit{2}}\ {\isacharasterisk}\ {\isacharparenleft}Suc\ n{\isacharparenright}\ {\isacharminus}\ {\isadigit{1}}{\isacharparenright}\ {\isacharplus}\ suma{\isacharunderscore}impares\ n{\isachardoublequoteclose}\ \isanewline
\ \ \ \ \isacommand{by}\isamarkupfalse%
\ simp\isanewline
\ \ \isacommand{also}\isamarkupfalse%
\ \isacommand{have}\isamarkupfalse%
\ {\isachardoublequoteopen}{\isasymdots}\ {\isacharequal}\ {\isacharparenleft}{\isadigit{2}}\ {\isacharasterisk}\ {\isacharparenleft}Suc\ n{\isacharparenright}\ {\isacharminus}\ {\isadigit{1}}{\isacharparenright}\ {\isacharplus}\ n\ {\isacharasterisk}\ n{\isachardoublequoteclose}\ \isacommand{using}\isamarkupfalse%
\ HI\ \isacommand{by}\isamarkupfalse%
\ simp\isanewline
\ \ \isacommand{also}\isamarkupfalse%
\ \isacommand{have}\isamarkupfalse%
\ {\isachardoublequoteopen}{\isasymdots}\ {\isacharequal}\ n\ {\isacharasterisk}\ n\ {\isacharplus}\ {\isadigit{2}}\ {\isacharasterisk}\ n\ {\isacharplus}\ {\isadigit{1}}{\isachardoublequoteclose}\ \isacommand{by}\isamarkupfalse%
\ simp\isanewline
\ \ \isacommand{finally}\isamarkupfalse%
\ \isacommand{show}\isamarkupfalse%
\ {\isachardoublequoteopen}{\isacharquery}P\ {\isacharparenleft}Suc\ n{\isacharparenright}{\isachardoublequoteclose}\ \isacommand{by}\isamarkupfalse%
\ simp\isanewline
\isacommand{qed}\isamarkupfalse%
%
\endisatagproof
{\isafoldproof}%
%
\isadelimproof
%
\endisadelimproof
%
\begin{isamarkuptext}%
La demostración usando patrones es%
\end{isamarkuptext}\isamarkuptrue%
\isacommand{lemma}\isamarkupfalse%
\ {\isachardoublequoteopen}suma{\isacharunderscore}impares\ n\ {\isacharequal}\ n\ {\isacharasterisk}\ n{\isachardoublequoteclose}\ {\isacharparenleft}\isakeyword{is}\ {\isachardoublequoteopen}{\isacharquery}P\ n{\isachardoublequoteclose}{\isacharparenright}\isanewline
%
\isadelimproof
%
\endisadelimproof
%
\isatagproof
\isacommand{proof}\isamarkupfalse%
\ {\isacharparenleft}induct\ n{\isacharparenright}\isanewline
\ \ \isacommand{show}\isamarkupfalse%
\ {\isachardoublequoteopen}{\isacharquery}P\ {\isadigit{0}}{\isachardoublequoteclose}\ \isacommand{by}\isamarkupfalse%
\ simp\isanewline
\isacommand{next}\isamarkupfalse%
\isanewline
\ \ \isacommand{fix}\isamarkupfalse%
\ n\ \isanewline
\ \ \isacommand{assume}\isamarkupfalse%
\ {\isachardoublequoteopen}{\isacharquery}P\ n{\isachardoublequoteclose}\isanewline
\ \ \isacommand{then}\isamarkupfalse%
\ \isacommand{show}\isamarkupfalse%
\ {\isachardoublequoteopen}{\isacharquery}P\ {\isacharparenleft}Suc\ n{\isacharparenright}{\isachardoublequoteclose}\ \isacommand{by}\isamarkupfalse%
\ simp\isanewline
\isacommand{qed}\isamarkupfalse%
\isanewline
%
\endisatagproof
{\isafoldproof}%
%
\isadelimproof
%
\endisadelimproof
%
\isadelimtheory
%
\endisadelimtheory
%
\isatagtheory
%
\endisatagtheory
{\isafoldtheory}%
%
\isadelimtheory
%
\endisadelimtheory
%
\end{isabellebody}%
\endinput
%:%file=~/Escritorio/TFG-v1/EjerciciosDELMF/SumaImpares.thy%:%
%:%24=8%:%
%:%36=10%:%
%:%37=11%:%
%:%38=12%:%
%:%39=13%:%
%:%40=14%:%
%:%41=15%:%
%:%42=16%:%
%:%43=17%:%
%:%44=18%:%
%:%45=19%:%
%:%46=20%:%
%:%47=21%:%
%:%48=22%:%
%:%49=23%:%
%:%50=24%:%
%:%51=25%:%
%:%52=26%:%
%:%53=27%:%
%:%54=28%:%
%:%55=29%:%
%:%56=30%:%
%:%57=31%:%
%:%58=32%:%
%:%60=35%:%
%:%61=35%:%
%:%62=36%:%
%:%63=37%:%
%:%65=39%:%
%:%67=41%:%
%:%68=41%:%
%:%75=42%:%
%:%85=44%:%
%:%86=45%:%
%:%87=46%:%
%:%88=47%:%
%:%89=48%:%
%:%90=49%:%
%:%91=50%:%
%:%92=51%:%
%:%94=56%:%
%:%95=56%:%
%:%98=57%:%
%:%102=57%:%
%:%103=57%:%
%:%104=58%:%
%:%105=58%:%
%:%106=59%:%
%:%116=61%:%
%:%118=63%:%
%:%119=63%:%
%:%122=64%:%
%:%126=64%:%
%:%127=64%:%
%:%136=66%:%
%:%137=67%:%
%:%139=69%:%
%:%140=69%:%
%:%147=70%:%
%:%148=70%:%
%:%149=71%:%
%:%150=71%:%
%:%151=71%:%
%:%152=72%:%
%:%153=72%:%
%:%154=73%:%
%:%155=73%:%
%:%156=73%:%
%:%157=74%:%
%:%158=74%:%
%:%159=75%:%
%:%160=75%:%
%:%161=76%:%
%:%162=76%:%
%:%163=76%:%
%:%164=76%:%
%:%165=76%:%
%:%166=77%:%
%:%167=77%:%
%:%168=77%:%
%:%169=77%:%
%:%170=78%:%
%:%171=78%:%
%:%172=78%:%
%:%173=78%:%
%:%174=79%:%
%:%184=81%:%
%:%185=82%:%
%:%187=83%:%
%:%188=83%:%
%:%195=84%:%
%:%196=84%:%
%:%197=85%:%
%:%198=85%:%
%:%199=85%:%
%:%200=86%:%
%:%201=86%:%
%:%202=87%:%
%:%203=87%:%
%:%204=88%:%
%:%205=88%:%
%:%206=89%:%
%:%207=89%:%
%:%208=90%:%
%:%209=90%:%
%:%210=91%:%
%:%211=91%:%
%:%212=91%:%
%:%213=91%:%
%:%214=91%:%
%:%215=92%:%
%:%216=92%:%
%:%217=92%:%
%:%218=92%:%
%:%219=93%:%
%:%220=93%:%
%:%221=93%:%
%:%222=93%:%
%:%223=94%:%
%:%233=96%:%
%:%235=98%:%
%:%236=98%:%
%:%243=99%:%
%:%244=99%:%
%:%245=100%:%
%:%246=100%:%
%:%247=100%:%
%:%248=101%:%
%:%249=101%:%
%:%250=102%:%
%:%251=102%:%
%:%252=103%:%
%:%253=103%:%
%:%254=104%:%
%:%255=104%:%
%:%256=104%:%
%:%257=104%:%
%:%258=105%:%
%:%259=105%:%

%
\begin{isabellebody}%
\setisabellecontext{CancelacionInyectiva}%
%
\isadelimtheory
%
\endisadelimtheory
%
\isatagtheory
%
\endisatagtheory
{\isafoldtheory}%
%
\isadelimtheory
%
\endisadelimtheory
%
\isadelimdocument
%
\endisadelimdocument
%
\isatagdocument
%
\isamarkupsection{Cancelación de funciones inyectivas%
}
\isamarkuptrue%
%
\endisatagdocument
{\isafolddocument}%
%
\isadelimdocument
%
\endisadelimdocument
%
\begin{isamarkuptext}%
El siguiente teorema prueba una caracterización de las funciones
 inyectivas, en otras palabras, las funciones inyectivas son
 monomorfismos en la categoría de conjuntos. Un monomorfismo es un
 homomorfismo inyectivo y la categoría de conjuntos es la categoría
 cuyos objetos son los conjuntos.
  
  \begin{teorema}
    $f$ es una función inyectiva, si y solo si, para todas $g$ y $h$
    tales que \isa{f\ {\isasymcirc}\ g\ {\isacharequal}\ f\ {\isasymcirc}\ h} se tiene que $g = h$. 
  \end{teorema}

Vamos a hacer dos lemas de nuestro teorema, ya que podemos la doble 
implicación en dos implicaciones y demostrar cada una de ellas por
 separado.

\begin {lema}
$f$ es una función inyectiva si para todas $g$ y $h$ tales que $f \circ
 g = f \circ h$ se tiene que $g = h.$
\end {lema}
  \begin{demostracion}
    La demostración la haremos por doble implicación: 
\begin {enumerate}
\item Supongamos que tenemos que $f \circ g = f \circ h$, queremos
 demostrar que $g = h$, usando que f es inyectiva tenemos que: \\
$$(f \circ g)(x) = (f \circ h)(x) \Longrightarrow f(g(x)) = f(h(x)) = 
g(x) = h(x)$$
\item Supongamos ahora que $g = h$, queremos demostrar que  $f \circ g
 = f \circ h$. \\
$$(f \circ g)(x) = f(g(x)) = f(h(x)) = (f \circ h)(x)$$
\end {enumerate}
.
  \end{demostracion}

\begin {lema} 
Si para toda $g$ y $h$ tales que $f \circ g =  f \circ h$ se tiene que $g
= h$ entonces f es inyectiva.
\end {lema} 

\begin {demostracion}
Supongamos que el dominio de nuestra función $f$ es distinto del vacío.
Tenemos que demostrar que $\forall a,b$ tales que $f(a) = f(b),$ esto
 implica que $a = b.$ \\
Sean $a,b$ tales que $f(a) = f(b)$, sean ahora $g(x) = a \forall x$ y
 $h(x) = b \forall x$ entonces 
$$(f \circ g) = (f \circ h) \Longrightarrow  f(g(x)) = f(h(x)) \Longrightarrow f(a) = f(b)$$
Por hipótesis tenemos entonces que $a = b,$ como queríamos demostrar.
\end {demostracion}


  Su especificación es la siguiente, pero al igual que hemos hecho en la demostración
a mano vamos a demostrarlo a través de dos lemas:%
\end{isamarkuptext}\isamarkuptrue%
\isacommand{theorem}\isamarkupfalse%
\ \isanewline
\ \ {\isachardoublequoteopen}inj\ f\ {\isasymlongleftrightarrow}\ {\isacharparenleft}f\ {\isasymcirc}\ g\ {\isacharequal}\ f\ {\isasymcirc}\ h{\isacharparenright}\ {\isacharequal}\ {\isacharparenleft}g\ {\isacharequal}\ h{\isacharparenright}{\isachardoublequoteclose}\isanewline
%
\isadelimproof
\ \ %
\endisadelimproof
%
\isatagproof
\isacommand{oops}\isamarkupfalse%
%
\endisatagproof
{\isafoldproof}%
%
\isadelimproof
%
\endisadelimproof
%
\begin{isamarkuptext}%
Sus lemas son los siguientes:%
\end{isamarkuptext}\isamarkuptrue%
\isacommand{lemma}\isamarkupfalse%
\ \isanewline
{\isachardoublequoteopen}{\isasymforall}g\ h{\isachardot}\ {\isacharparenleft}f\ {\isasymcirc}\ g\ {\isacharequal}\ f\ {\isasymcirc}\ h\ {\isasymlongrightarrow}\ g\ {\isacharequal}\ h{\isacharparenright}\ {\isasymLongrightarrow}\ inj\ f{\isachardoublequoteclose}\isanewline
%
\isadelimproof
\ \ %
\endisadelimproof
%
\isatagproof
\isacommand{oops}\isamarkupfalse%
%
\endisatagproof
{\isafoldproof}%
%
\isadelimproof
\isanewline
%
\endisadelimproof
\isanewline
\isacommand{lemma}\isamarkupfalse%
\ \isanewline
{\isachardoublequoteopen}inj\ f\ {\isasymLongrightarrow}\ {\isacharparenleft}f\ {\isasymcirc}\ g\ {\isacharequal}\ f\ {\isasymcirc}\ h{\isacharparenright}\ {\isacharequal}\ {\isacharparenleft}g\ {\isacharequal}\ h{\isacharparenright}{\isachardoublequoteclose}\isanewline
%
\isadelimproof
\ \ %
\endisadelimproof
%
\isatagproof
\isacommand{oops}\isamarkupfalse%
%
\endisatagproof
{\isafoldproof}%
%
\isadelimproof
%
\endisadelimproof
%
\begin{isamarkuptext}%
En la especificación anterior, \isa{inj\ f} es una 
  abreviatura de \isa{inj\ f} definida en la teoría
  \href{http://bit.ly/2XuPQx5}{Fun.thy}. Además, contiene la definición
  de \isa{inj{\isacharunderscore}on}
  \begin{itemize}
    \item[] \isa{inj{\isacharunderscore}on\ f\ A\ {\isacharequal}\ {\isacharparenleft}{\isasymforall}x{\isasymin}A{\isachardot}\ {\isasymforall}y{\isasymin}A{\isachardot}\ f\ x\ {\isacharequal}\ f\ y\ {\isasymlongrightarrow}\ x\ {\isacharequal}\ y{\isacharparenright}} \hfill (\isa{inj{\isacharunderscore}on{\isacharunderscore}def})
  \end{itemize} 
  Por su parte, \isa{UNIV} es el conjunto universal definido en la 
  teoría \href{http://bit.ly/2XtHCW6}{Set.thy} como una abreviatura de 
  \isa{top} que, a su vez está definido en la teoría 
  \href{http://bit.ly/2Xyj9Pe}{Orderings.thy} mediante la siguiente
  propiedad 
  \begin{itemize}
    \item[] \isa{\mbox{}\inferrule{\mbox{ordering{\isacharunderscore}top\ less{\isacharunderscore}eq\ less\ top}}{\mbox{less{\isacharunderscore}eq\ a\ top}}} 
      \hfill (\isa{ordering{\isacharunderscore}top{\isachardot}extremum})
  \end{itemize} 
  En el caso de la teoría de conjuntos, la relación de orden es la
  inclusión de conjuntos.

  Presentaremos distintas demostraciones de los lemas. La primera
  demostración es applicativa:%
\end{isamarkuptext}\isamarkuptrue%
\isacommand{lemma}\isamarkupfalse%
\ inyectivapli{\isacharcolon}\isanewline
\ \ {\isachardoublequoteopen}inj\ f\ {\isasymLongrightarrow}\ {\isacharparenleft}f\ {\isasymcirc}\ g\ {\isacharequal}\ f\ {\isasymcirc}\ h{\isacharparenright}\ {\isacharequal}\ {\isacharparenleft}g\ {\isacharequal}\ h{\isacharparenright}{\isachardoublequoteclose}\isanewline
%
\isadelimproof
\ \ %
\endisadelimproof
%
\isatagproof
\isacommand{apply}\isamarkupfalse%
\ {\isacharparenleft}simp\ add{\isacharcolon}\ inj{\isacharunderscore}on{\isacharunderscore}def\ fun{\isacharunderscore}eq{\isacharunderscore}iff{\isacharparenright}\ \isanewline
\ \ \isacommand{apply}\isamarkupfalse%
\ auto\isanewline
\ \ \isacommand{done}\isamarkupfalse%
%
\endisatagproof
{\isafoldproof}%
%
\isadelimproof
\ \isanewline
%
\endisadelimproof
\isanewline
\isacommand{lemma}\isamarkupfalse%
\ inyectivapli{\isadigit{2}}{\isacharcolon}\isanewline
{\isachardoublequoteopen}{\isasymforall}g\ h{\isachardot}\ {\isacharparenleft}f\ {\isasymcirc}\ g\ {\isacharequal}\ f\ {\isasymcirc}\ h\ {\isasymlongrightarrow}\ g\ {\isacharequal}\ h{\isacharparenright}\ {\isasymLongrightarrow}\ inj\ f{\isachardoublequoteclose}\isanewline
%
\isadelimproof
\ \ %
\endisadelimproof
%
\isatagproof
\isacommand{apply}\isamarkupfalse%
\ {\isacharparenleft}rule\ injI{\isacharparenright}\isanewline
\ \ \isacommand{by}\isamarkupfalse%
\ {\isacharparenleft}metis\ fun{\isacharunderscore}upd{\isacharunderscore}apply\ fun{\isacharunderscore}upd{\isacharunderscore}comp{\isacharparenright}%
\endisatagproof
{\isafoldproof}%
%
\isadelimproof
%
\endisadelimproof
%
\begin{isamarkuptext}%
En las demostraciones anteriores se han usado los siguientes
 lemas:
  \begin{itemize}
    \item[] \isa{{\isacharparenleft}f\ {\isacharequal}\ g{\isacharparenright}\ {\isacharequal}\ {\isacharparenleft}{\isasymforall}x{\isachardot}\ f\ x\ {\isacharequal}\ g\ x{\isacharparenright}} 
      \hfill (\isa{fun{\isacharunderscore}eq{\isacharunderscore}iff})
  \end{itemize} 
  \begin{itemize}
    \item[] \isa{{\isacharparenleft}f{\isacharparenleft}x\ {\isacharcolon}{\isacharequal}\ y{\isacharparenright}{\isacharparenright}\ z\ {\isacharequal}\ {\isacharparenleft}\textsf{if}\ z\ {\isacharequal}\ x\ \textsf{then}\ y\ \textsf{else}\ f\ z{\isacharparenright}} 
      \hfill (\isa{fun{\isacharunderscore}upd{\isacharunderscore}apply})
  \end{itemize} 
  \begin{itemize}
    \item[] \isa{{\isacharparenleft}f\ {\isacharequal}\ g{\isacharparenright}\ {\isacharequal}\ {\isacharparenleft}{\isasymforall}x{\isachardot}\ f\ x\ {\isacharequal}\ g\ x{\isacharparenright}} 
      \hfill (\isa{fun{\isacharunderscore}upd{\isacharunderscore}comp})
  \end{itemize} 

  La demostración applicativa sin auto es%
\end{isamarkuptext}\isamarkuptrue%
\isacommand{lemma}\isamarkupfalse%
\isanewline
\ \ {\isachardoublequoteopen}inj\ f\ {\isasymLongrightarrow}\ {\isacharparenleft}f\ {\isasymcirc}\ g\ {\isacharequal}\ f\ {\isasymcirc}\ h{\isacharparenright}\ {\isacharequal}\ {\isacharparenleft}g\ {\isacharequal}\ h{\isacharparenright}{\isachardoublequoteclose}\isanewline
%
\isadelimproof
\ \ %
\endisadelimproof
%
\isatagproof
\isacommand{apply}\isamarkupfalse%
\ {\isacharparenleft}unfold\ inj{\isacharunderscore}on{\isacharunderscore}def{\isacharparenright}\ \isanewline
\ \ \isacommand{apply}\isamarkupfalse%
\ {\isacharparenleft}unfold\ fun{\isacharunderscore}eq{\isacharunderscore}iff{\isacharparenright}\ \isanewline
\ \ \isacommand{apply}\isamarkupfalse%
\ {\isacharparenleft}unfold\ o{\isacharunderscore}apply{\isacharparenright}\isanewline
\ \ \isacommand{apply}\isamarkupfalse%
\ {\isacharparenleft}rule\ iffI{\isacharparenright}\isanewline
\ \ \ \isacommand{apply}\isamarkupfalse%
\ simp{\isacharplus}\isanewline
\ \ \isacommand{done}\isamarkupfalse%
%
\endisatagproof
{\isafoldproof}%
%
\isadelimproof
%
\endisadelimproof
%
\begin{isamarkuptext}%
En la demostración anterior se ha introducido los siguientes
  hechos
  \begin{itemize}
    \item \isa{{\isacharparenleft}f\ {\isasymcirc}\ g{\isacharparenright}\ x\ {\isacharequal}\ f\ {\isacharparenleft}g\ x{\isacharparenright}} \hfill (\isa{o{\isacharunderscore}apply})
    \item \isa{{\isasymlbrakk}P\ {\isasymLongrightarrow}\ Q{\isacharsemicolon}\ Q\ {\isasymLongrightarrow}\ P{\isasymrbrakk}\ {\isasymLongrightarrow}\ P\ {\isacharequal}\ Q} \hfill (\isa{iffI})
  \end{itemize} 

  La demostración automática es%
\end{isamarkuptext}\isamarkuptrue%
\isacommand{lemma}\isamarkupfalse%
\isanewline
\ \ \isakeyword{assumes}\ {\isachardoublequoteopen}inj\ f{\isachardoublequoteclose}\isanewline
\ \ \isakeyword{shows}\ {\isachardoublequoteopen}{\isacharparenleft}f\ {\isasymcirc}\ g\ {\isacharequal}\ f\ {\isasymcirc}\ h{\isacharparenright}\ {\isacharequal}\ {\isacharparenleft}g\ {\isacharequal}\ h{\isacharparenright}{\isachardoublequoteclose}\isanewline
%
\isadelimproof
\ \ %
\endisadelimproof
%
\isatagproof
\isacommand{using}\isamarkupfalse%
\ assms\isanewline
\ \ \isacommand{by}\isamarkupfalse%
\ {\isacharparenleft}auto\ simp\ add{\isacharcolon}\ inj{\isacharunderscore}on{\isacharunderscore}def\ fun{\isacharunderscore}eq{\isacharunderscore}iff{\isacharparenright}%
\endisatagproof
{\isafoldproof}%
%
\isadelimproof
%
\endisadelimproof
%
\begin{isamarkuptext}%
La demostración declarativa%
\end{isamarkuptext}\isamarkuptrue%
\isacommand{lemma}\isamarkupfalse%
\isanewline
\ \ \isakeyword{assumes}\ {\isachardoublequoteopen}inj\ f{\isachardoublequoteclose}\isanewline
\ \ \isakeyword{shows}\ {\isachardoublequoteopen}{\isacharparenleft}f\ {\isasymcirc}\ g\ {\isacharequal}\ f\ {\isasymcirc}\ h{\isacharparenright}\ {\isacharequal}\ {\isacharparenleft}g\ {\isacharequal}\ h{\isacharparenright}{\isachardoublequoteclose}\isanewline
%
\isadelimproof
%
\endisadelimproof
%
\isatagproof
\isacommand{proof}\isamarkupfalse%
\ \isanewline
\ \ \isacommand{assume}\isamarkupfalse%
\ {\isachardoublequoteopen}f\ {\isasymcirc}\ g\ {\isacharequal}\ f\ {\isasymcirc}\ h{\isachardoublequoteclose}\isanewline
\ \ \isacommand{show}\isamarkupfalse%
\ {\isachardoublequoteopen}g\ {\isacharequal}\ h{\isachardoublequoteclose}\isanewline
\ \ \isacommand{proof}\isamarkupfalse%
\isanewline
\ \ \ \ \isacommand{fix}\isamarkupfalse%
\ x\isanewline
\ \ \ \ \isacommand{have}\isamarkupfalse%
\ {\isachardoublequoteopen}{\isacharparenleft}f\ {\isasymcirc}\ g{\isacharparenright}{\isacharparenleft}x{\isacharparenright}\ {\isacharequal}\ {\isacharparenleft}f\ {\isasymcirc}\ h{\isacharparenright}{\isacharparenleft}x{\isacharparenright}{\isachardoublequoteclose}\ \isacommand{using}\isamarkupfalse%
\ {\isacharbackquoteopen}f\ {\isasymcirc}\ g\ {\isacharequal}\ f\ {\isasymcirc}\ h{\isacharbackquoteclose}\ \isacommand{by}\isamarkupfalse%
\ simp\isanewline
\ \ \ \ \isacommand{then}\isamarkupfalse%
\ \isacommand{have}\isamarkupfalse%
\ {\isachardoublequoteopen}f{\isacharparenleft}g{\isacharparenleft}x{\isacharparenright}{\isacharparenright}\ {\isacharequal}\ f{\isacharparenleft}h{\isacharparenleft}x{\isacharparenright}{\isacharparenright}{\isachardoublequoteclose}\ \isacommand{by}\isamarkupfalse%
\ simp\isanewline
\ \ \ \ \isacommand{then}\isamarkupfalse%
\ \isacommand{show}\isamarkupfalse%
\ {\isachardoublequoteopen}g{\isacharparenleft}x{\isacharparenright}\ {\isacharequal}\ h{\isacharparenleft}x{\isacharparenright}{\isachardoublequoteclose}\ \isacommand{using}\isamarkupfalse%
\ {\isacharbackquoteopen}inj\ f{\isacharbackquoteclose}\ \isacommand{by}\isamarkupfalse%
\ {\isacharparenleft}simp\ add{\isacharcolon}inj{\isacharunderscore}on{\isacharunderscore}def{\isacharparenright}\isanewline
\ \ \isacommand{qed}\isamarkupfalse%
\isanewline
\isacommand{next}\isamarkupfalse%
\isanewline
\ \ \isacommand{assume}\isamarkupfalse%
\ {\isachardoublequoteopen}g\ {\isacharequal}\ h{\isachardoublequoteclose}\isanewline
\ \ \isacommand{show}\isamarkupfalse%
\ {\isachardoublequoteopen}f\ {\isasymcirc}\ g\ {\isacharequal}\ f\ {\isasymcirc}\ h{\isachardoublequoteclose}\isanewline
\ \ \isacommand{proof}\isamarkupfalse%
\isanewline
\ \ \ \ \isacommand{fix}\isamarkupfalse%
\ x\isanewline
\ \ \ \ \isacommand{have}\isamarkupfalse%
\ {\isachardoublequoteopen}{\isacharparenleft}f\ {\isasymcirc}\ g{\isacharparenright}\ x\ {\isacharequal}\ f{\isacharparenleft}g{\isacharparenleft}x{\isacharparenright}{\isacharparenright}{\isachardoublequoteclose}\ \isacommand{by}\isamarkupfalse%
\ simp\isanewline
\ \ \ \ \isacommand{also}\isamarkupfalse%
\ \isacommand{have}\isamarkupfalse%
\ {\isachardoublequoteopen}{\isasymdots}\ {\isacharequal}\ f{\isacharparenleft}h{\isacharparenleft}x{\isacharparenright}{\isacharparenright}{\isachardoublequoteclose}\ \isacommand{using}\isamarkupfalse%
\ {\isacharbackquoteopen}g\ {\isacharequal}\ h{\isacharbackquoteclose}\ \isacommand{by}\isamarkupfalse%
\ simp\isanewline
\ \ \ \ \isacommand{also}\isamarkupfalse%
\ \isacommand{have}\isamarkupfalse%
\ {\isachardoublequoteopen}{\isasymdots}\ {\isacharequal}\ {\isacharparenleft}f\ {\isasymcirc}\ h{\isacharparenright}\ x{\isachardoublequoteclose}\ \isacommand{by}\isamarkupfalse%
\ simp\isanewline
\ \ \ \ \isacommand{finally}\isamarkupfalse%
\ \isacommand{show}\isamarkupfalse%
\ {\isachardoublequoteopen}{\isacharparenleft}f\ {\isasymcirc}\ g{\isacharparenright}\ x\ {\isacharequal}\ {\isacharparenleft}f\ {\isasymcirc}\ h{\isacharparenright}\ x{\isachardoublequoteclose}\ \isacommand{by}\isamarkupfalse%
\ simp\isanewline
\ \ \isacommand{qed}\isamarkupfalse%
\isanewline
\isacommand{qed}\isamarkupfalse%
%
\endisatagproof
{\isafoldproof}%
%
\isadelimproof
%
\endisadelimproof
%
\begin{isamarkuptext}%
Otra demostración declarativa es%
\end{isamarkuptext}\isamarkuptrue%
\isacommand{lemma}\isamarkupfalse%
\ \isanewline
\ \ \isakeyword{assumes}\ {\isachardoublequoteopen}inj\ f{\isachardoublequoteclose}\isanewline
\ \ \isakeyword{shows}\ {\isachardoublequoteopen}{\isacharparenleft}f\ {\isasymcirc}\ g\ {\isacharequal}\ f\ {\isasymcirc}\ h{\isacharparenright}\ {\isacharequal}\ {\isacharparenleft}g\ {\isacharequal}\ h{\isacharparenright}{\isachardoublequoteclose}\isanewline
%
\isadelimproof
%
\endisadelimproof
%
\isatagproof
\isacommand{proof}\isamarkupfalse%
\ \isanewline
\ \ \isacommand{assume}\isamarkupfalse%
\ {\isachardoublequoteopen}f\ {\isasymcirc}\ g\ {\isacharequal}\ f\ {\isasymcirc}\ h{\isachardoublequoteclose}\ \isanewline
\ \ \isacommand{then}\isamarkupfalse%
\ \isacommand{show}\isamarkupfalse%
\ {\isachardoublequoteopen}g\ {\isacharequal}\ h{\isachardoublequoteclose}\ \isacommand{using}\isamarkupfalse%
\ {\isacharbackquoteopen}inj\ f{\isacharbackquoteclose}\ \isacommand{by}\isamarkupfalse%
\ {\isacharparenleft}simp\ add{\isacharcolon}\ inj{\isacharunderscore}on{\isacharunderscore}def\ fun{\isacharunderscore}eq{\isacharunderscore}iff{\isacharparenright}\ \isanewline
\isacommand{next}\isamarkupfalse%
\isanewline
\ \ \isacommand{assume}\isamarkupfalse%
\ {\isachardoublequoteopen}g\ {\isacharequal}\ h{\isachardoublequoteclose}\ \isanewline
\ \ \isacommand{then}\isamarkupfalse%
\ \isacommand{show}\isamarkupfalse%
\ {\isachardoublequoteopen}f\ {\isasymcirc}\ g\ {\isacharequal}\ f\ {\isasymcirc}\ h{\isachardoublequoteclose}\ \isacommand{by}\isamarkupfalse%
\ simp\isanewline
\isacommand{qed}\isamarkupfalse%
%
\endisatagproof
{\isafoldproof}%
%
\isadelimproof
%
\endisadelimproof
%
\begin{isamarkuptext}%
En consecuencia, la demostración de nuestro teorema:%
\end{isamarkuptext}\isamarkuptrue%
\isacommand{theorem}\isamarkupfalse%
\ \isanewline
{\isachardoublequoteopen}{\isasymforall}g\ h{\isachardot}\ {\isacharparenleft}f\ {\isasymcirc}\ g\ {\isacharequal}\ f\ {\isasymcirc}\ h\ {\isasymlongrightarrow}\ g\ {\isacharequal}\ h{\isacharparenright}\ {\isasymlongleftrightarrow}\ inj\ f{\isachardoublequoteclose}\isanewline
%
\isadelimproof
\ \ %
\endisadelimproof
%
\isatagproof
\isacommand{oops}\isamarkupfalse%
\isanewline
%
\endisatagproof
{\isafoldproof}%
%
\isadelimproof
%
\endisadelimproof
%
\isadelimtheory
%
\endisadelimtheory
%
\isatagtheory
%
\endisatagtheory
{\isafoldtheory}%
%
\isadelimtheory
%
\endisadelimtheory
%
\end{isabellebody}%
\endinput
%:%file=~/ownCloud/alonso/curso-TFG/Carlos/TFG/CancelacionInyectiva.thy%:%
%:%24=8%:%
%:%36=10%:%
%:%37=11%:%
%:%38=12%:%
%:%39=13%:%
%:%40=14%:%
%:%41=15%:%
%:%42=16%:%
%:%43=17%:%
%:%44=18%:%
%:%45=19%:%
%:%46=20%:%
%:%47=21%:%
%:%48=22%:%
%:%49=23%:%
%:%50=24%:%
%:%51=25%:%
%:%52=26%:%
%:%53=27%:%
%:%54=28%:%
%:%55=29%:%
%:%56=30%:%
%:%57=31%:%
%:%58=32%:%
%:%59=33%:%
%:%60=34%:%
%:%61=35%:%
%:%62=36%:%
%:%63=37%:%
%:%64=38%:%
%:%65=39%:%
%:%66=40%:%
%:%67=41%:%
%:%68=42%:%
%:%69=43%:%
%:%70=44%:%
%:%71=45%:%
%:%72=46%:%
%:%73=47%:%
%:%74=48%:%
%:%75=49%:%
%:%76=50%:%
%:%77=51%:%
%:%78=52%:%
%:%79=53%:%
%:%80=54%:%
%:%81=55%:%
%:%82=56%:%
%:%83=57%:%
%:%84=58%:%
%:%85=59%:%
%:%86=60%:%
%:%88=63%:%
%:%89=63%:%
%:%90=64%:%
%:%93=65%:%
%:%97=65%:%
%:%107=68%:%
%:%109=70%:%
%:%110=70%:%
%:%111=71%:%
%:%114=72%:%
%:%118=72%:%
%:%124=72%:%
%:%127=73%:%
%:%128=74%:%
%:%129=74%:%
%:%130=75%:%
%:%133=76%:%
%:%137=76%:%
%:%147=79%:%
%:%148=80%:%
%:%149=81%:%
%:%150=82%:%
%:%151=83%:%
%:%152=84%:%
%:%153=85%:%
%:%154=86%:%
%:%155=87%:%
%:%156=88%:%
%:%157=89%:%
%:%158=90%:%
%:%159=91%:%
%:%160=92%:%
%:%161=93%:%
%:%162=94%:%
%:%163=95%:%
%:%164=96%:%
%:%165=97%:%
%:%166=98%:%
%:%167=99%:%
%:%169=101%:%
%:%170=101%:%
%:%171=102%:%
%:%174=103%:%
%:%178=103%:%
%:%179=103%:%
%:%180=104%:%
%:%181=104%:%
%:%182=105%:%
%:%188=105%:%
%:%191=106%:%
%:%192=107%:%
%:%193=107%:%
%:%194=108%:%
%:%197=109%:%
%:%201=109%:%
%:%202=109%:%
%:%203=110%:%
%:%204=110%:%
%:%213=113%:%
%:%214=114%:%
%:%215=115%:%
%:%216=116%:%
%:%217=117%:%
%:%218=118%:%
%:%219=119%:%
%:%220=120%:%
%:%221=121%:%
%:%222=122%:%
%:%223=123%:%
%:%224=124%:%
%:%225=125%:%
%:%226=126%:%
%:%227=127%:%
%:%228=128%:%
%:%230=130%:%
%:%231=130%:%
%:%232=131%:%
%:%235=132%:%
%:%239=132%:%
%:%240=132%:%
%:%241=133%:%
%:%242=133%:%
%:%243=134%:%
%:%244=134%:%
%:%245=135%:%
%:%246=135%:%
%:%247=136%:%
%:%248=136%:%
%:%249=137%:%
%:%259=139%:%
%:%260=140%:%
%:%261=141%:%
%:%262=142%:%
%:%263=143%:%
%:%264=144%:%
%:%265=145%:%
%:%266=146%:%
%:%268=148%:%
%:%269=148%:%
%:%270=149%:%
%:%271=150%:%
%:%274=151%:%
%:%278=151%:%
%:%279=151%:%
%:%280=152%:%
%:%281=152%:%
%:%290=154%:%
%:%292=156%:%
%:%293=156%:%
%:%294=157%:%
%:%295=158%:%
%:%302=159%:%
%:%303=159%:%
%:%304=160%:%
%:%305=160%:%
%:%306=161%:%
%:%307=161%:%
%:%308=162%:%
%:%309=162%:%
%:%310=163%:%
%:%311=163%:%
%:%312=164%:%
%:%313=164%:%
%:%314=164%:%
%:%315=164%:%
%:%316=165%:%
%:%317=165%:%
%:%318=165%:%
%:%319=165%:%
%:%320=166%:%
%:%321=166%:%
%:%322=166%:%
%:%323=166%:%
%:%324=166%:%
%:%325=167%:%
%:%326=167%:%
%:%327=168%:%
%:%328=168%:%
%:%329=169%:%
%:%330=169%:%
%:%331=170%:%
%:%332=170%:%
%:%333=171%:%
%:%334=171%:%
%:%335=172%:%
%:%336=172%:%
%:%337=173%:%
%:%338=173%:%
%:%339=173%:%
%:%340=174%:%
%:%341=174%:%
%:%342=174%:%
%:%343=174%:%
%:%344=174%:%
%:%345=175%:%
%:%346=175%:%
%:%347=175%:%
%:%348=175%:%
%:%349=176%:%
%:%350=176%:%
%:%351=176%:%
%:%352=176%:%
%:%353=177%:%
%:%354=177%:%
%:%355=178%:%
%:%365=180%:%
%:%367=182%:%
%:%368=182%:%
%:%369=183%:%
%:%370=184%:%
%:%377=185%:%
%:%378=185%:%
%:%379=186%:%
%:%380=186%:%
%:%381=187%:%
%:%382=187%:%
%:%383=187%:%
%:%384=187%:%
%:%385=187%:%
%:%386=188%:%
%:%387=188%:%
%:%388=189%:%
%:%389=189%:%
%:%390=190%:%
%:%391=190%:%
%:%392=190%:%
%:%393=190%:%
%:%394=191%:%
%:%404=193%:%
%:%406=195%:%
%:%407=195%:%
%:%408=196%:%
%:%411=197%:%
%:%415=197%:%
%:%416=197%:%

%
\begin{isabellebody}%
\setisabellecontext{CancelacionSobreyectiva}%
%
\isadelimtheory
%
\endisadelimtheory
%
\isatagtheory
%
\endisatagtheory
{\isafoldtheory}%
%
\isadelimtheory
%
\endisadelimtheory
%
\isadelimdocument
%
\endisadelimdocument
%
\isatagdocument
%
\isamarkupsection{Cancelación de las funciones sobreyectivas%
}
\isamarkuptrue%
%
\endisatagdocument
{\isafolddocument}%
%
\isadelimdocument
%
\endisadelimdocument
%
\begin{isamarkuptext}%
El siguiente teorema prueba una propiedad de las funciones
 sobreyectivas. El enunciado es el siguiente: 
\begin {teorema}
Las funciones sobreyectivas son cancelativas por la derecha. Es decir,
 si f es sobreyectiva entonces para todas funciones g y h tal que g o f
 = h o f se tiene que g = h.
\end {teorema}
 
\begin {demostracion}
\begin {itemize}
\item Supongamos que tenemos que $g o f = h o f$, queremos probar que $g =
 h.$ Usando la definición de sobreyectividad $(\forall y \in Y,
 \exists x | y = f(x))$ y nuestra hipótesis, tenemos que:
$$g(y) = g(f(x)) = (g o f) (x) = (h o f) (x) = h(f(x)) = h(y)$$
\item Supongamos que $g = h$, hay que probar que $g o f = h o f.$ Usando
nuestra hipótesis, tenemos que:
$$ (g o f)(x) = g(f(x)) = h(f(x)) = (h o f) (x).$$
\end {itemize}
.
\end {demostracion}

Su especificación es la siguiente:%
\end{isamarkuptext}\isamarkuptrue%
\isacommand{lemma}\isamarkupfalse%
\ {\isachardoublequoteopen}surj\ f\ {\isasymLongrightarrow}\ {\isacharparenleft}\ g\ {\isasymo}\ f\ {\isacharequal}\ h\ {\isasymo}\ f\ {\isacharparenright}\ {\isacharequal}\ {\isacharparenleft}g\ {\isacharequal}\ h{\isacharparenright}{\isachardoublequoteclose}\isanewline
%
\isadelimproof
\ \ %
\endisadelimproof
%
\isatagproof
\isacommand{oops}\isamarkupfalse%
%
\endisatagproof
{\isafoldproof}%
%
\isadelimproof
%
\endisadelimproof
%
\begin{isamarkuptext}%
En la especificación anterior, \isa{surj\ f} es una abreviatura de 
  \isa{surj\ f}, donde \isa{range\ f} es el rango o imagen
de la función f.
 Por otra parte, \isa{UNIV} es el conjunto universal definido en la 
  teoría \href{http://bit.ly/2XtHCW6}{Set.thy} como una abreviatura de 
  \isa{top} que, a su vez está definido en la teoría 
  \href{http://bit.ly/2Xyj9Pe}{Orderings.thy} mediante la siguiente
  propiedad 
  \begin{itemize}
    \item[] \isa{\mbox{}\inferrule{\mbox{ordering{\isacharunderscore}top\ less{\isacharunderscore}eq\ less\ top}}{\mbox{less{\isacharunderscore}eq\ a\ top}}} 
      \hfill (\isa{ordering{\isacharunderscore}top{\isachardot}extremum})
  \end{itemize} 
Además queda añadir que la teoría donde se encuentra definido \isa{surj\ f}
 es en \href{http://bit.ly/2XuPQx5}{Fun.thy}. Esta teoría contiene la
 definicion \isa{surj{\isacharunderscore}def}.
 \begin{itemize}
    \item[] \isa{surj\ f\ {\isacharequal}\ {\isacharparenleft}{\isasymforall}y{\isachardot}\ {\isasymexists}x{\isachardot}\ y\ {\isacharequal}\ f\ x{\isacharparenright}} \hfill (\isa{inj{\isacharunderscore}on{\isacharunderscore}def})
  \end{itemize} 

Presentaremos distintas demostraciones del teorema. La primera es la
 detallada:%
\end{isamarkuptext}\isamarkuptrue%
\isacommand{lemma}\isamarkupfalse%
\ \isanewline
\ \ \isakeyword{assumes}\ {\isachardoublequoteopen}surj\ f{\isachardoublequoteclose}\ \isanewline
\ \ \isakeyword{shows}\ {\isachardoublequoteopen}{\isacharparenleft}\ g\ {\isasymcirc}\ f\ {\isacharequal}\ h\ {\isasymcirc}\ f\ {\isacharparenright}\ {\isacharequal}\ {\isacharparenleft}g\ {\isacharequal}\ h{\isacharparenright}{\isachardoublequoteclose}\isanewline
%
\isadelimproof
%
\endisadelimproof
%
\isatagproof
\isacommand{proof}\isamarkupfalse%
\ {\isacharparenleft}rule\ iffI{\isacharparenright}\isanewline
\ \ \isacommand{assume}\isamarkupfalse%
\ {\isadigit{1}}{\isacharcolon}\ {\isachardoublequoteopen}\ g\ {\isasymcirc}\ f\ {\isacharequal}\ h\ {\isasymcirc}\ f\ {\isachardoublequoteclose}\isanewline
\ \ \isacommand{show}\isamarkupfalse%
\ {\isachardoublequoteopen}g\ {\isacharequal}\ h{\isachardoublequoteclose}\ \isanewline
\ \ \isacommand{proof}\isamarkupfalse%
\ \isanewline
\ \ \ \ \isacommand{fix}\isamarkupfalse%
\ x\ \isanewline
\ \ \ \ \isacommand{have}\isamarkupfalse%
\ {\isachardoublequoteopen}\ {\isasymexists}y\ {\isachardot}\ x\ {\isacharequal}\ f{\isacharparenleft}y{\isacharparenright}{\isachardoublequoteclose}\ \isacommand{using}\isamarkupfalse%
\ assms\ \isacommand{by}\isamarkupfalse%
\ {\isacharparenleft}simp\ add{\isacharcolon}surj{\isacharunderscore}def{\isacharparenright}\isanewline
\ \ \ \ \isacommand{then}\isamarkupfalse%
\ \isacommand{obtain}\isamarkupfalse%
\ {\isachardoublequoteopen}y{\isachardoublequoteclose}\ \isakeyword{where}\ {\isadigit{2}}{\isacharcolon}{\isachardoublequoteopen}x\ {\isacharequal}\ f{\isacharparenleft}y{\isacharparenright}{\isachardoublequoteclose}\ \isacommand{by}\isamarkupfalse%
\ {\isacharparenleft}rule\ exE{\isacharparenright}\isanewline
\ \ \ \ \isacommand{then}\isamarkupfalse%
\ \isacommand{have}\isamarkupfalse%
\ {\isachardoublequoteopen}g{\isacharparenleft}x{\isacharparenright}\ {\isacharequal}\ g{\isacharparenleft}f{\isacharparenleft}y{\isacharparenright}{\isacharparenright}{\isachardoublequoteclose}\ \isacommand{by}\isamarkupfalse%
\ simp\isanewline
\ \ \ \ \isacommand{then}\isamarkupfalse%
\ \isacommand{have}\isamarkupfalse%
\ {\isachardoublequoteopen}{\isachardot}{\isachardot}{\isachardot}\ {\isacharequal}\ {\isacharparenleft}g\ {\isasymcirc}\ f{\isacharparenright}\ {\isacharparenleft}y{\isacharparenright}\ \ {\isachardoublequoteclose}\ \isacommand{by}\isamarkupfalse%
\ simp\isanewline
\ \ \ \ \isacommand{then}\isamarkupfalse%
\ \isacommand{have}\isamarkupfalse%
\ {\isachardoublequoteopen}{\isachardot}{\isachardot}{\isachardot}\ {\isacharequal}\ {\isacharparenleft}h\ o\ f{\isacharparenright}\ {\isacharparenleft}y{\isacharparenright}{\isachardoublequoteclose}\ \isacommand{using}\isamarkupfalse%
\ {\isadigit{1}}\ \isacommand{by}\isamarkupfalse%
\ simp\isanewline
\ \ \ \ \isacommand{then}\isamarkupfalse%
\ \isacommand{have}\isamarkupfalse%
\ {\isachardoublequoteopen}{\isachardot}{\isachardot}{\isachardot}\ {\isacharequal}\ h{\isacharparenleft}f{\isacharparenleft}y{\isacharparenright}{\isacharparenright}{\isachardoublequoteclose}\ \isacommand{by}\isamarkupfalse%
\ simp\isanewline
\ \ \ \ \isacommand{then}\isamarkupfalse%
\ \isacommand{have}\isamarkupfalse%
\ {\isachardoublequoteopen}{\isachardot}{\isachardot}{\isachardot}\ {\isacharequal}\ h{\isacharparenleft}x{\isacharparenright}{\isachardoublequoteclose}\ \isacommand{using}\isamarkupfalse%
\ {\isadigit{2}}\ \ \ \isacommand{by}\isamarkupfalse%
\ {\isacharparenleft}simp\ add{\isacharcolon}\ {\isacartoucheopen}x\ {\isacharequal}\ f\ y{\isacartoucheclose}{\isacharparenright}\isanewline
\ \ \ \ \isacommand{then}\isamarkupfalse%
\ \isacommand{show}\isamarkupfalse%
\ {\isachardoublequoteopen}\ g{\isacharparenleft}x{\isacharparenright}\ {\isacharequal}\ h{\isacharparenleft}x{\isacharparenright}\ {\isachardoublequoteclose}\ \isanewline
\ \ \ \ \ \ \isacommand{using}\isamarkupfalse%
\ {\isacartoucheopen}{\isacharparenleft}g\ {\isasymcirc}\ f{\isacharparenright}\ y\ {\isacharequal}\ {\isacharparenleft}h\ {\isasymcirc}\ f{\isacharparenright}\ y{\isacartoucheclose}\ {\isacartoucheopen}{\isacharparenleft}h\ {\isasymcirc}\ f{\isacharparenright}\ y\ {\isacharequal}\ h\ {\isacharparenleft}f\ y{\isacharparenright}{\isacartoucheclose}\isanewline
\ \ \ \ {\isacartoucheopen}g\ {\isacharparenleft}f\ y{\isacharparenright}\ {\isacharequal}\ {\isacharparenleft}g\ {\isasymcirc}\ f{\isacharparenright}\ y{\isacartoucheclose}\ {\isacartoucheopen}g\ x\ {\isacharequal}\ g\ {\isacharparenleft}f\ y{\isacharparenright}{\isacartoucheclose}\ {\isacartoucheopen}h\ {\isacharparenleft}f\ y{\isacharparenright}\ {\isacharequal}\ h\ x{\isacartoucheclose}\ \isacommand{by}\isamarkupfalse%
\ presburger\isanewline
\ \ \isacommand{qed}\isamarkupfalse%
\isanewline
\isacommand{next}\isamarkupfalse%
\isanewline
\ \ \isacommand{assume}\isamarkupfalse%
\ {\isachardoublequoteopen}g\ {\isacharequal}\ h{\isachardoublequoteclose}\ \isanewline
\ \ \isacommand{show}\isamarkupfalse%
\ {\isachardoublequoteopen}g\ {\isasymcirc}\ f\ {\isacharequal}\ h\ {\isasymcirc}\ f{\isachardoublequoteclose}\isanewline
\ \ \isacommand{proof}\isamarkupfalse%
\isanewline
\ \ \ \ \isacommand{fix}\isamarkupfalse%
\ x\isanewline
\ \ \ \ \isacommand{have}\isamarkupfalse%
\ {\isachardoublequoteopen}{\isacharparenleft}g\ {\isasymcirc}\ f{\isacharparenright}\ x\ {\isacharequal}\ g{\isacharparenleft}f{\isacharparenleft}x{\isacharparenright}{\isacharparenright}{\isachardoublequoteclose}\ \isacommand{by}\isamarkupfalse%
\ simp\isanewline
\ \ \ \ \isacommand{also}\isamarkupfalse%
\ \isacommand{have}\isamarkupfalse%
\ {\isachardoublequoteopen}{\isasymdots}\ {\isacharequal}\ h{\isacharparenleft}f{\isacharparenleft}x{\isacharparenright}{\isacharparenright}{\isachardoublequoteclose}\ \isacommand{using}\isamarkupfalse%
\ {\isacharbackquoteopen}g\ {\isacharequal}\ h{\isacharbackquoteclose}\ \isacommand{by}\isamarkupfalse%
\ simp\isanewline
\ \ \ \ \isacommand{also}\isamarkupfalse%
\ \isacommand{have}\isamarkupfalse%
\ {\isachardoublequoteopen}{\isasymdots}\ {\isacharequal}\ {\isacharparenleft}h\ {\isasymcirc}\ f{\isacharparenright}\ x{\isachardoublequoteclose}\ \isacommand{by}\isamarkupfalse%
\ simp\isanewline
\ \ \ \ \isacommand{finally}\isamarkupfalse%
\ \isacommand{show}\isamarkupfalse%
\ {\isachardoublequoteopen}{\isacharparenleft}g\ {\isasymcirc}\ f{\isacharparenright}\ x\ {\isacharequal}\ {\isacharparenleft}h\ {\isasymcirc}\ f{\isacharparenright}\ x{\isachardoublequoteclose}\ \isacommand{by}\isamarkupfalse%
\ simp\isanewline
\ \ \isacommand{qed}\isamarkupfalse%
\isanewline
\isacommand{qed}\isamarkupfalse%
%
\endisatagproof
{\isafoldproof}%
%
\isadelimproof
%
\endisadelimproof
%
\begin{isamarkuptext}%
En la demostración hemos introducido: 
 \begin{itemize}
    \item[] \isa{\mbox{}\inferrule{\mbox{{\isasymexists}x{\isachardot}\ P\ x}\\\ \mbox{{\isasymAnd}x{\isachardot}\ \mbox{}\inferrule{\mbox{P\ x}}{\mbox{Q}}}}{\mbox{Q}}} 
      \hfill (\isa{rule\ exE}) 
  \end{itemize} 
 \begin{itemize}
    \item[] \isa{{\isasymlbrakk}P\ {\isasymLongrightarrow}\ Q{\isacharsemicolon}\ Q\ {\isasymLongrightarrow}\ P{\isasymrbrakk}\ {\isasymLongrightarrow}\ P\ {\isacharequal}\ Q} 
      \hfill (\isa{iffI})
  \end{itemize} 

La demostración aplicativa es:%
\end{isamarkuptext}\isamarkuptrue%
\isacommand{lemma}\isamarkupfalse%
\ {\isachardoublequoteopen}surj\ f\ {\isasymLongrightarrow}\ {\isacharparenleft}{\isacharparenleft}g\ {\isasymcirc}\ f{\isacharparenright}\ {\isacharequal}\ {\isacharparenleft}h\ {\isasymcirc}\ f{\isacharparenright}\ {\isacharparenright}\ {\isacharequal}\ {\isacharparenleft}g\ {\isacharequal}\ h{\isacharparenright}{\isachardoublequoteclose}\isanewline
%
\isadelimproof
\ \ %
\endisadelimproof
%
\isatagproof
\isacommand{apply}\isamarkupfalse%
\ {\isacharparenleft}simp\ add{\isacharcolon}\ surj{\isacharunderscore}def\ fun{\isacharunderscore}eq{\isacharunderscore}iff{\isacharparenright}\isanewline
\ \ \isacommand{apply}\isamarkupfalse%
\ \ metis\isanewline
\ \ \isacommand{done}\isamarkupfalse%
%
\endisatagproof
{\isafoldproof}%
%
\isadelimproof
%
\endisadelimproof
%
\begin{isamarkuptext}%
En esta demostración hemos introducido:
 \begin{itemize}
    \item[] \isa{{\isacharparenleft}f\ {\isacharequal}\ g{\isacharparenright}\ {\isacharequal}\ {\isacharparenleft}{\isasymforall}x{\isachardot}\ f\ x\ {\isacharequal}\ g\ x{\isacharparenright}} 
      \hfill (\isa{fun{\isacharunderscore}eq{\isacharunderscore}iff})
  \end{itemize}%
\end{isamarkuptext}\isamarkuptrue%
%
\isadelimtheory
%
\endisadelimtheory
%
\isatagtheory
%
\endisatagtheory
{\isafoldtheory}%
%
\isadelimtheory
%
\endisadelimtheory
%
\end{isabellebody}%
\endinput
%:%file=/home/carlos/Escritorio/TFG-v1/EjerciciosDELMF/CancelacionSobreyectiva.thy%:%
%:%24=7%:%
%:%36=10%:%
%:%37=11%:%
%:%38=12%:%
%:%39=13%:%
%:%40=14%:%
%:%41=15%:%
%:%42=16%:%
%:%43=17%:%
%:%44=18%:%
%:%45=19%:%
%:%46=20%:%
%:%47=21%:%
%:%48=22%:%
%:%49=23%:%
%:%50=24%:%
%:%51=25%:%
%:%52=26%:%
%:%53=27%:%
%:%54=28%:%
%:%55=29%:%
%:%56=30%:%
%:%57=31%:%
%:%59=34%:%
%:%60=34%:%
%:%63=35%:%
%:%67=35%:%
%:%77=38%:%
%:%78=39%:%
%:%79=40%:%
%:%80=41%:%
%:%81=42%:%
%:%82=43%:%
%:%83=44%:%
%:%84=45%:%
%:%85=46%:%
%:%86=47%:%
%:%87=48%:%
%:%88=49%:%
%:%89=50%:%
%:%90=51%:%
%:%91=52%:%
%:%92=53%:%
%:%93=54%:%
%:%94=55%:%
%:%95=56%:%
%:%96=57%:%
%:%97=58%:%
%:%99=61%:%
%:%100=61%:%
%:%101=62%:%
%:%102=63%:%
%:%109=64%:%
%:%110=64%:%
%:%111=65%:%
%:%112=65%:%
%:%113=66%:%
%:%114=66%:%
%:%115=67%:%
%:%116=67%:%
%:%117=68%:%
%:%118=68%:%
%:%119=69%:%
%:%120=69%:%
%:%121=69%:%
%:%122=69%:%
%:%123=70%:%
%:%124=70%:%
%:%125=70%:%
%:%126=70%:%
%:%127=71%:%
%:%128=71%:%
%:%129=71%:%
%:%130=71%:%
%:%131=72%:%
%:%132=72%:%
%:%133=72%:%
%:%134=72%:%
%:%135=73%:%
%:%136=73%:%
%:%137=73%:%
%:%138=73%:%
%:%139=73%:%
%:%140=74%:%
%:%141=74%:%
%:%142=74%:%
%:%143=74%:%
%:%144=75%:%
%:%145=75%:%
%:%146=75%:%
%:%147=75%:%
%:%148=75%:%
%:%149=76%:%
%:%150=76%:%
%:%151=76%:%
%:%152=77%:%
%:%153=77%:%
%:%154=78%:%
%:%155=78%:%
%:%156=79%:%
%:%157=79%:%
%:%158=80%:%
%:%159=80%:%
%:%160=81%:%
%:%161=81%:%
%:%162=82%:%
%:%163=82%:%
%:%164=83%:%
%:%165=83%:%
%:%166=84%:%
%:%167=84%:%
%:%168=85%:%
%:%169=85%:%
%:%170=85%:%
%:%171=86%:%
%:%172=86%:%
%:%173=86%:%
%:%174=86%:%
%:%175=86%:%
%:%176=87%:%
%:%177=87%:%
%:%178=87%:%
%:%179=87%:%
%:%180=88%:%
%:%181=88%:%
%:%182=88%:%
%:%183=88%:%
%:%184=89%:%
%:%185=89%:%
%:%186=90%:%
%:%196=93%:%
%:%197=94%:%
%:%198=95%:%
%:%199=96%:%
%:%200=97%:%
%:%201=98%:%
%:%202=99%:%
%:%203=100%:%
%:%204=101%:%
%:%205=102%:%
%:%206=103%:%
%:%208=105%:%
%:%209=105%:%
%:%212=106%:%
%:%216=106%:%
%:%217=106%:%
%:%218=107%:%
%:%219=107%:%
%:%220=108%:%
%:%230=110%:%
%:%231=111%:%
%:%232=112%:%
%:%233=113%:%
%:%234=114%:%

%
\begin{isabellebody}%
\setisabellecontext{ConjuntosFinitos}%
%
\isadelimtheory
\isanewline
%
\endisadelimtheory
%
\isatagtheory
%
\endisatagtheory
{\isafoldtheory}%
%
\isadelimtheory
%
\endisadelimtheory
%
\begin{isamarkuptext}%
\comentario{Estructurar en secciones.}%
\end{isamarkuptext}\isamarkuptrue%
%
\begin{isamarkuptext}%
\comentario{Hacer demostraciones detalladas.}%
\end{isamarkuptext}\isamarkuptrue%
%
\begin{isamarkuptext}%
\comentario{Añadir lemas usados al Soporte.}%
\end{isamarkuptext}\isamarkuptrue%
%
\isadelimdocument
%
\endisadelimdocument
%
\isatagdocument
%
\isamarkupsection{Demostración en lenguaje natural%
}
\isamarkuptrue%
%
\endisatagdocument
{\isafolddocument}%
%
\isadelimdocument
%
\endisadelimdocument
%
\begin{isamarkuptext}%
El siguiente teorema es una propiedad que verifican todos los 
conjuntos finitos de números naturales  estudiado en el 
\href{http://bit.ly/2XBW6n2}{tema 10} de la
asignatura de LMF de tercer curso del grado en Matemáticas. Su enunciado
 es el siguiente 

  \begin{teorema} 
    Sea S un conjunto finito de números naturales.  Entonces todos los
 elementos de S son menores o iguales que la suma de los elementos de
 S, es decir,

 $$\forall m \in S \Longrightarrow m \leq \sum S$$ 

donde $\sum S $ denota la suma de todos los elementos de S.
  \end{teorema} 

\begin{demostracion}
La demostración del teorema la haremos por inducción sobre conjuntos
 finitos.

  
 (Base de la inducción) El caso $S = \emptyset$ es trivial.

 (Paso de la inducción) Supongamos que se verifica el teorema para un
 conjunto finito de números naturales, que se denotará por $S.$ 
 
Sea $a \in \Bbb{N}$ tal que $a \notin S,$ Ya que si $a \in S$ se tendría
probado el teorema. Luego hay que probar que: 

$$\forall n \in S \cup \{a\} \Longrightarrow n \leq \sum (S \cup
 \{a\})$$

Distingamos dos casos ahora:

Caso 1: $n = a$.

Si $n = a$, se tiene que:

$$n = a \leq a + \sum S = \sum (S \cup \{a\}).$$

Caso 2: $n \neq a.$

Si $n \neq a,$ tenemos que $n \in S,$ luego usando la hipótesis de
 inducción:

$$n \leq \sum S \leq \sum S + a = \sum (S \cup \{a\}).$$
\end{demostracion}

En la demostración del teorema hemos usado un resultado, que vamos a
 probar en Isabelle después de la especificación del teorema,
 el resultado es $\sum S + a = \sum (S \cup \{ a\})$.%
\end{isamarkuptext}\isamarkuptrue%
%
\isadelimdocument
%
\endisadelimdocument
%
\isatagdocument
%
\isamarkupsection{Especificación en Isabelle/HOL%
}
\isamarkuptrue%
%
\endisatagdocument
{\isafolddocument}%
%
\isadelimdocument
%
\endisadelimdocument
%
\begin{isamarkuptext}%
Para la especificación del teorema en Isabelle, primero debemos
 notar que  \isa{finite\ S} indica que un conjunto $S$ es 
finito  y definir  la función \isa{sumaConj} tal que
 \isa{sumaConj\ n} es la suma de todos los elementos de S.%
\end{isamarkuptext}\isamarkuptrue%
\isacommand{definition}\isamarkupfalse%
\ sumaConj\ {\isacharcolon}{\isacharcolon}\ {\isachardoublequoteopen}nat\ set\ {\isasymRightarrow}\ nat{\isachardoublequoteclose}\ \isakeyword{where}\isanewline
\ \ {\isachardoublequoteopen}sumaConj\ S\ {\isasymequiv}\ {\isasymSum}S{\isachardoublequoteclose}%
\begin{isamarkuptext}%
El enunciado del teorema es el siguiente :%
\end{isamarkuptext}\isamarkuptrue%
\isacommand{lemma}\isamarkupfalse%
\ {\isachardoublequoteopen}finite\ S\ {\isasymLongrightarrow}\ {\isasymforall}x\ {\isasymin}\ S{\isachardot}\ x\ {\isasymle}\ sumaConj\ S{\isachardoublequoteclose}\isanewline
%
\isadelimproof
\isanewline
\ \ %
\endisadelimproof
%
\isatagproof
\isacommand{oops}\isamarkupfalse%
%
\endisatagproof
{\isafoldproof}%
%
\isadelimproof
%
\endisadelimproof
%
\begin{isamarkuptext}%
Vamos a demostrar primero el lema enunciado anteriormente%
\end{isamarkuptext}\isamarkuptrue%
\isacommand{lemma}\isamarkupfalse%
\ aux{\isacharunderscore}propiedad{\isacharunderscore}conjuntos{\isacharunderscore}finitos{\isacharcolon}\isanewline
\ {\isachardoublequoteopen}\ x\ {\isasymnotin}\ S\ {\isasymand}\ finite\ S\ {\isasymlongrightarrow}\ sumaConj\ S\ {\isacharplus}\ x\ {\isacharequal}\ sumaConj{\isacharparenleft}insert\ x\ S{\isacharparenright}{\isachardoublequoteclose}\isanewline
%
\isadelimproof
\ \ %
\endisadelimproof
%
\isatagproof
\isacommand{by}\isamarkupfalse%
\ {\isacharparenleft}simp\ add{\isacharcolon}\ sumaConj{\isacharunderscore}def{\isacharparenright}%
\endisatagproof
{\isafoldproof}%
%
\isadelimproof
%
\endisadelimproof
%
\begin{isamarkuptext}%
La demostración del lema anterior se ha incluido
 \isa{sumConj{\isacharunderscore}def}, que hace referencia a la definición sumaConj que
 hemos hecho anteriormente.


En la demostración se usará la táctica \isa{induct} que hace
  uso del esquema de inducción sobre los conjuntos finitos:
  \begin{itemize}
  \item[] \isa{{\isasymlbrakk}finite\ x{\isacharsemicolon}\ P\ {\isasymemptyset}{\isacharsemicolon}\ {\isasymAnd}A\ a{\isachardot}\ finite\ A\ {\isasymand}\ P\ A\ {\isasymLongrightarrow}\ P\ {\isacharparenleft}{\isacharbraceleft}a{\isacharbraceright}\ {\isasymunion}\ A{\isacharparenright}{\isasymrbrakk}\ {\isasymLongrightarrow}\ P\ x}
 \hfill (\isa{finite{\isachardot}induct})
  \end{itemize} 

Vamos a presentar diferentes formas de demostración:%
\end{isamarkuptext}\isamarkuptrue%
%
\isadelimdocument
%
\endisadelimdocument
%
\isatagdocument
%
\isamarkupsection{Demostración aplicativa%
}
\isamarkuptrue%
%
\endisatagdocument
{\isafolddocument}%
%
\isadelimdocument
%
\endisadelimdocument
%
\begin{isamarkuptext}%
La demostración aplicativa del teorema es:%
\end{isamarkuptext}\isamarkuptrue%
\isacommand{lemma}\isamarkupfalse%
\ {\isachardoublequoteopen}finite\ S\ {\isasymLongrightarrow}\ {\isasymforall}x{\isasymin}S{\isachardot}\ x\ {\isasymle}\ sumaConj\ S{\isachardoublequoteclose}\isanewline
%
\isadelimproof
\ \ %
\endisadelimproof
%
\isatagproof
\isacommand{apply}\isamarkupfalse%
\ {\isacharparenleft}induct\ rule{\isacharcolon}\ finite{\isacharunderscore}induct{\isacharparenright}\isanewline
\ \ \ \isacommand{apply}\isamarkupfalse%
\ simp\isanewline
\ \ \isacommand{apply}\isamarkupfalse%
\ {\isacharparenleft}simp\ add{\isacharcolon}\ add{\isacharunderscore}increasing\ sumaConj{\isacharunderscore}def{\isacharparenright}\isanewline
\ \ \isacommand{done}\isamarkupfalse%
%
\endisatagproof
{\isafoldproof}%
%
\isadelimproof
%
\endisadelimproof
%
\begin{isamarkuptext}%
En la demostración anterior se ha introducido:
 \begin{itemize}
    \item[] \isa{\mbox{}\inferrule{\mbox{{\isacharparenleft}{\isadigit{0}}\ {\isacharcolon}{\isacharcolon}\ {\isacharprime}a{\isacharparenright}\ {\isasymle}\ a\ {\isasymand}\ b\ {\isasymle}\ c}}{\mbox{b\ {\isasymle}\ a\ {\isacharplus}\ c}}} 
      \hfill (\isa{add{\isacharunderscore}increasing})
  \end{itemize}%
\end{isamarkuptext}\isamarkuptrue%
%
\isadelimdocument
%
\endisadelimdocument
%
\isatagdocument
%
\isamarkupsection{Demostración automática%
}
\isamarkuptrue%
%
\endisatagdocument
{\isafolddocument}%
%
\isadelimdocument
%
\endisadelimdocument
%
\begin{isamarkuptext}%
La demostración automática es:%
\end{isamarkuptext}\isamarkuptrue%
\isacommand{lemma}\isamarkupfalse%
\ {\isachardoublequoteopen}finite\ S\ {\isasymLongrightarrow}\ {\isasymforall}x{\isasymin}S{\isachardot}\ x\ {\isasymle}\ sumaConj\ S{\isachardoublequoteclose}\isanewline
%
\isadelimproof
\ \ %
\endisadelimproof
%
\isatagproof
\isacommand{by}\isamarkupfalse%
\ {\isacharparenleft}induct\ rule{\isacharcolon}\ finite{\isacharunderscore}induct{\isacharparenright}\isanewline
\ \ \ \ \ {\isacharparenleft}auto\ simp\ add{\isacharcolon}\ \ sumaConj{\isacharunderscore}def{\isacharparenright}%
\endisatagproof
{\isafoldproof}%
%
\isadelimproof
%
\endisadelimproof
%
\isadelimdocument
%
\endisadelimdocument
%
\isatagdocument
%
\isamarkupsection{Demostración detallada%
}
\isamarkuptrue%
%
\endisatagdocument
{\isafolddocument}%
%
\isadelimdocument
%
\endisadelimdocument
%
\begin{isamarkuptext}%
La demostración declarativa es:%
\end{isamarkuptext}\isamarkuptrue%
\isacommand{lemma}\isamarkupfalse%
\ sumaConj{\isacharunderscore}acota{\isacharcolon}\ \isanewline
\ \ {\isachardoublequoteopen}finite\ S\ {\isasymLongrightarrow}\ {\isasymforall}x{\isasymin}S{\isachardot}\ x\ {\isasymle}\ sumaConj\ S{\isachardoublequoteclose}\isanewline
%
\isadelimproof
%
\endisadelimproof
%
\isatagproof
\isacommand{proof}\isamarkupfalse%
\ {\isacharparenleft}induct\ rule{\isacharcolon}\ finite{\isacharunderscore}induct{\isacharparenright}\isanewline
\ \ \isacommand{show}\isamarkupfalse%
\ {\isachardoublequoteopen}{\isasymforall}x\ {\isasymin}\ {\isacharbraceleft}{\isacharbraceright}{\isachardot}\ x\ {\isasymle}\ sumaConj\ {\isacharbraceleft}{\isacharbraceright}{\isachardoublequoteclose}\ \isacommand{by}\isamarkupfalse%
\ simp\isanewline
\isacommand{next}\isamarkupfalse%
\isanewline
\ \ \isacommand{fix}\isamarkupfalse%
\ x\ \isakeyword{and}\ F\isanewline
\ \ \isacommand{assume}\isamarkupfalse%
\ fF{\isacharcolon}\ {\isachardoublequoteopen}finite\ F{\isachardoublequoteclose}\ \isanewline
\ \ \ \ \ \isakeyword{and}\ xF{\isacharcolon}\ {\isachardoublequoteopen}x\ {\isasymnotin}\ F{\isachardoublequoteclose}\ \isanewline
\ \ \ \ \ \isakeyword{and}\ HI{\isacharcolon}\ {\isachardoublequoteopen}{\isasymforall}\ x{\isasymin}F{\isachardot}\ x\ {\isasymle}\ sumaConj\ F{\isachardoublequoteclose}\isanewline
\ \ \isacommand{show}\isamarkupfalse%
\ {\isachardoublequoteopen}{\isasymforall}y\ {\isasymin}\ insert\ x\ F{\isachardot}\ y\ {\isasymle}\ sumaConj\ {\isacharparenleft}insert\ x\ F{\isacharparenright}{\isachardoublequoteclose}\isanewline
\ \ \isacommand{proof}\isamarkupfalse%
\ \isanewline
\ \ \ \ \isacommand{fix}\isamarkupfalse%
\ y\ \isanewline
\ \ \ \ \isacommand{assume}\isamarkupfalse%
\ {\isachardoublequoteopen}y\ {\isasymin}\ insert\ x\ F{\isachardoublequoteclose}\isanewline
\ \ \ \ \isacommand{show}\isamarkupfalse%
\ {\isachardoublequoteopen}y\ {\isasymle}\ sumaConj\ {\isacharparenleft}insert\ x\ F{\isacharparenright}{\isachardoublequoteclose}\isanewline
\ \ \ \ \isacommand{proof}\isamarkupfalse%
\ {\isacharparenleft}cases\ {\isachardoublequoteopen}y\ {\isacharequal}\ x{\isachardoublequoteclose}{\isacharparenright}\isanewline
\ \ \ \ \ \ \isacommand{assume}\isamarkupfalse%
\ {\isachardoublequoteopen}y\ {\isacharequal}\ x{\isachardoublequoteclose}\isanewline
\ \ \ \ \ \ \isacommand{then}\isamarkupfalse%
\ \isacommand{have}\isamarkupfalse%
\ {\isachardoublequoteopen}y\ {\isasymle}\ x\ {\isacharplus}\ {\isacharparenleft}sumaConj\ F{\isacharparenright}{\isachardoublequoteclose}\ \isacommand{by}\isamarkupfalse%
\ simp\isanewline
\ \ \ \ \ \ \isacommand{also}\isamarkupfalse%
\ \isacommand{have}\isamarkupfalse%
\ {\isachardoublequoteopen}{\isasymdots}\ {\isacharequal}\ sumaConj\ {\isacharparenleft}insert\ x\ F{\isacharparenright}{\isachardoublequoteclose}\ \isanewline
\ \ \ \ \ \ \ \ \isacommand{by}\isamarkupfalse%
\ {\isacharparenleft}simp\ add{\isacharcolon}\ fF\ sumaConj{\isacharunderscore}def\ xF{\isacharparenright}\ \isanewline
\ \ \ \ \ \ \isacommand{finally}\isamarkupfalse%
\ \isacommand{show}\isamarkupfalse%
\ {\isacharquery}thesis\ \isacommand{{\isachardot}}\isamarkupfalse%
\isanewline
\ \ \ \ \isacommand{next}\isamarkupfalse%
\isanewline
\ \ \ \ \ \ \isacommand{assume}\isamarkupfalse%
\ {\isachardoublequoteopen}y\ {\isasymnoteq}\ x{\isachardoublequoteclose}\isanewline
\ \ \ \ \ \ \isacommand{then}\isamarkupfalse%
\ \isacommand{have}\isamarkupfalse%
\ {\isachardoublequoteopen}y\ {\isasymin}\ F{\isachardoublequoteclose}\ \isacommand{using}\isamarkupfalse%
\ {\isacharbackquoteopen}y\ {\isasymin}\ insert\ x\ F{\isacharbackquoteclose}\ \isacommand{by}\isamarkupfalse%
\ simp\isanewline
\ \ \ \ \ \ \isacommand{then}\isamarkupfalse%
\ \isacommand{have}\isamarkupfalse%
\ {\isachardoublequoteopen}y\ {\isasymle}\ sumaConj\ F{\isachardoublequoteclose}\ \isacommand{using}\isamarkupfalse%
\ HI\ \isacommand{by}\isamarkupfalse%
\ simp\isanewline
\ \ \ \ \ \ \isacommand{also}\isamarkupfalse%
\ \isacommand{have}\isamarkupfalse%
\ {\isachardoublequoteopen}{\isasymdots}\ {\isasymle}\ x\ {\isacharplus}\ {\isacharparenleft}sumaConj\ F{\isacharparenright}{\isachardoublequoteclose}\ \isacommand{by}\isamarkupfalse%
\ simp\isanewline
\ \ \ \ \ \ \isacommand{also}\isamarkupfalse%
\ \isacommand{have}\isamarkupfalse%
\ {\isachardoublequoteopen}{\isasymdots}\ {\isacharequal}\ sumaConj\ {\isacharparenleft}insert\ x\ F{\isacharparenright}{\isachardoublequoteclose}\ \isacommand{using}\isamarkupfalse%
\ fF\ xF\isanewline
\ \ \ \ \ \ \ \ \isacommand{by}\isamarkupfalse%
\ {\isacharparenleft}simp\ add{\isacharcolon}\ sumaConj{\isacharunderscore}def{\isacharparenright}\isanewline
\ \ \ \ \ \ \isacommand{finally}\isamarkupfalse%
\ \isacommand{show}\isamarkupfalse%
\ {\isacharquery}thesis\ \isacommand{{\isachardot}}\isamarkupfalse%
\isanewline
\ \ \ \ \isacommand{qed}\isamarkupfalse%
\isanewline
\ \ \isacommand{qed}\isamarkupfalse%
\isanewline
\isacommand{qed}\isamarkupfalse%
%
\endisatagproof
{\isafoldproof}%
%
\isadelimproof
%
\endisadelimproof
%
\begin{isamarkuptext}%
En esta última demostración hemos usado el método de prueba por
 casos,el método blast y también el simp("simplificador") añadiéndole 
\isa{sumaConj{\isacharunderscore}def}.%
\end{isamarkuptext}\isamarkuptrue%
%
\isadelimtheory
%
\endisadelimtheory
%
\isatagtheory
%
\endisatagtheory
{\isafoldtheory}%
%
\isadelimtheory
%
\endisadelimtheory
%
\end{isabellebody}%
\endinput
%:%file=~/Escritorio/TFG/ConjuntosFinitos.thy%:%
%:%6=1%:%
%:%20=9%:%
%:%24=11%:%
%:%28=13%:%
%:%37=15%:%
%:%49=18%:%
%:%50=19%:%
%:%51=20%:%
%:%52=21%:%
%:%53=22%:%
%:%54=23%:%
%:%55=24%:%
%:%56=25%:%
%:%57=26%:%
%:%58=27%:%
%:%59=28%:%
%:%60=29%:%
%:%61=30%:%
%:%62=31%:%
%:%63=32%:%
%:%64=33%:%
%:%65=34%:%
%:%66=35%:%
%:%67=36%:%
%:%68=37%:%
%:%69=38%:%
%:%70=39%:%
%:%71=40%:%
%:%72=41%:%
%:%73=42%:%
%:%74=43%:%
%:%75=44%:%
%:%76=45%:%
%:%77=46%:%
%:%78=47%:%
%:%79=48%:%
%:%80=49%:%
%:%81=50%:%
%:%82=51%:%
%:%83=52%:%
%:%84=53%:%
%:%85=54%:%
%:%86=55%:%
%:%87=56%:%
%:%88=57%:%
%:%89=58%:%
%:%90=59%:%
%:%91=60%:%
%:%92=61%:%
%:%93=62%:%
%:%94=63%:%
%:%95=64%:%
%:%96=65%:%
%:%97=66%:%
%:%98=67%:%
%:%99=68%:%
%:%108=71%:%
%:%120=73%:%
%:%121=74%:%
%:%122=75%:%
%:%123=76%:%
%:%125=79%:%
%:%126=79%:%
%:%127=80%:%
%:%129=82%:%
%:%131=85%:%
%:%132=85%:%
%:%135=86%:%
%:%136=87%:%
%:%140=87%:%
%:%150=89%:%
%:%152=91%:%
%:%153=91%:%
%:%154=92%:%
%:%157=93%:%
%:%161=93%:%
%:%162=93%:%
%:%171=96%:%
%:%172=97%:%
%:%173=98%:%
%:%174=99%:%
%:%175=100%:%
%:%176=101%:%
%:%177=102%:%
%:%178=103%:%
%:%179=104%:%
%:%180=105%:%
%:%181=106%:%
%:%182=107%:%
%:%183=108%:%
%:%192=111%:%
%:%204=113%:%
%:%206=115%:%
%:%207=115%:%
%:%210=116%:%
%:%214=116%:%
%:%215=116%:%
%:%216=117%:%
%:%217=117%:%
%:%218=118%:%
%:%219=118%:%
%:%220=119%:%
%:%230=121%:%
%:%231=122%:%
%:%232=123%:%
%:%233=124%:%
%:%234=125%:%
%:%243=128%:%
%:%255=130%:%
%:%257=132%:%
%:%258=132%:%
%:%261=133%:%
%:%265=133%:%
%:%266=133%:%
%:%267=134%:%
%:%281=136%:%
%:%293=138%:%
%:%295=140%:%
%:%296=140%:%
%:%297=141%:%
%:%304=142%:%
%:%305=142%:%
%:%306=143%:%
%:%307=143%:%
%:%308=143%:%
%:%309=144%:%
%:%310=144%:%
%:%311=145%:%
%:%312=145%:%
%:%313=146%:%
%:%314=146%:%
%:%315=147%:%
%:%316=148%:%
%:%317=149%:%
%:%318=149%:%
%:%319=150%:%
%:%320=150%:%
%:%321=151%:%
%:%322=151%:%
%:%323=152%:%
%:%324=152%:%
%:%325=153%:%
%:%326=153%:%
%:%327=154%:%
%:%328=154%:%
%:%329=155%:%
%:%330=155%:%
%:%331=156%:%
%:%332=156%:%
%:%333=156%:%
%:%334=156%:%
%:%335=157%:%
%:%336=157%:%
%:%337=157%:%
%:%338=158%:%
%:%339=158%:%
%:%340=159%:%
%:%341=159%:%
%:%342=159%:%
%:%343=159%:%
%:%344=160%:%
%:%345=160%:%
%:%346=161%:%
%:%347=161%:%
%:%348=162%:%
%:%349=162%:%
%:%350=162%:%
%:%351=162%:%
%:%352=162%:%
%:%353=163%:%
%:%354=163%:%
%:%355=163%:%
%:%356=163%:%
%:%357=163%:%
%:%358=164%:%
%:%359=164%:%
%:%360=164%:%
%:%361=164%:%
%:%362=165%:%
%:%363=165%:%
%:%364=165%:%
%:%365=165%:%
%:%366=166%:%
%:%367=166%:%
%:%368=167%:%
%:%369=167%:%
%:%370=167%:%
%:%371=167%:%
%:%372=168%:%
%:%373=168%:%
%:%374=169%:%
%:%375=169%:%
%:%376=170%:%
%:%386=173%:%
%:%387=174%:%
%:%388=175%:%

%
\begin{isabellebody}%
\setisabellecontext{TeoremaCantor}%
%
\isadelimtheory
\isanewline
%
\endisadelimtheory
%
\isatagtheory
%
\endisatagtheory
{\isafoldtheory}%
%
\isadelimtheory
%
\endisadelimtheory
%
\begin{isamarkuptext}%
\comentario{Estructurar en secciones.}%
\end{isamarkuptext}\isamarkuptrue%
%
\begin{isamarkuptext}%
\comentario{Hacer demostraciones detalladas.}%
\end{isamarkuptext}\isamarkuptrue%
%
\begin{isamarkuptext}%
\comentario{Añadir lemas usados al Soporte.}%
\end{isamarkuptext}\isamarkuptrue%
%
\begin{isamarkuptext}%
El siguiente, denominado  teorema de Cantor por el matemático
 Georg Cantor, es un resultado importante de la teoría
 de conjuntos. 

El matemático Georg Ferdinand Ludwig Philipp Cantor fue un matemático y
lógico nacido en Rusia en el siglo XIX. Fue inventor junto con Dedekind
 y Frege de la teoría de conjuntos, que es la base de las matemáticas
 modernas.



Para la comprensión del teorema vamos a definir una serie de conceptos:

\begin {itemize}

\item Conjunto de potencia $A$  $(\mathcal{P}(A))$: conjunto formado por
todos los subconjuntos de $A$.
\item Cardinal del conjunto $A$ (Denotado $\# A$): número de elementos del propio
 conjunto.

\end {itemize}
El enunciado original del teorema es el siguiente : 


\begin {teorema}
El cardinal del conjunto potencia de cualquier conjunto A es
 estrictamente mayor que el cardinal de A, o lo que es lo mismo,
$\# \mathcal{P}(A) > \# A.$


\end {teorema}
Pero el enunciado del teorema lo podemos reformular como: 
\begin{teorema}
Dado un conjunto A, $\nexists  f : A \longrightarrow \mathcal{P}(A)$ que
sea sobreyectiva.

\end{teorema}

El teorema lo hemos podido reescribir de la anterior forma, ya que si
 suponemos que $\exists f$ tal que $f: A \longrightarrow \mathcal{P}(A)$
es sobreyectiva, entonces tenemos que $f(A) = \mathcal{P}(A)$ y por lo
 tanto, $\# f(A) \geq \# \mathcal{P}(A)$, de lo que se deduce esta
 reformulación. Reciprocamente, es trivial ver que esta reformulación
 implica la primera.
 con el teorema. \\
El teorema de Cantor es trivial para conjuntos finitos, ya que el
 conjunto potencia, de conjuntos finitos de n elementos tiene
 $2^n$ elementos.

Por ello,  vamos a realizar la prueba para conjuntos infinitos. 


\begin{demostracion}
 
Vamos a realizar la prueba por reducción al absurdo.\\
Supongamos que $\exists f : A \longrightarrow \mathcal{P}(A)$ sobreyectiva, es
 decir, $\forall C \in \rho(A) ,  \exists x \in A$ tal que $C = f(x)$.
En particular, tomemos el conjunto $$B = \{ x \in A : x \notin f(x) \}$$
 y  supongamos que $\exists a \in A : B = f(a)$, ya que $B$ es un
 subconjunto de A, luego podemos distinguir dos casos $:$ \\
$1.$ Si $a \in B$, entonces por definición del conjunto $B$ tenemos que
$a \notin B$, luego llegamos a una contradicción. \\
$2.$ Si $a \notin B$, entonces por definición de B tenemos que $a \in 
B$, luego hemos llegado a otra contradicción. 

En las dos hipótesis hemos llegado a una contradicción,
por lo que no existe $a$ y $f$ no es sobreyectiva.
\end{demostracion}


Para la especificación del teorema en Isabelle, primero debemos notar
 que $$f :: \, 'a \Rightarrow \,'a \: set$$
 significa que es una función 
de tipos, donde $'a$ significa un tipo y para poder denotar
el conjunto potencia tenemos que poner $'a \ set$ que significa que es
 de un tipo formado por conjuntos del tipo $'a$.




El enunciado del teorema es el siguiente :%
\end{isamarkuptext}\isamarkuptrue%
\isacommand{theorem}\isamarkupfalse%
\ Cantor{\isacharcolon}\ {\isachardoublequoteopen}{\isasymnexists}f\ {\isacharcolon}{\isacharcolon}\ {\isacharprime}a\ {\isasymRightarrow}\ {\isacharprime}a\ set{\isachardot}\ {\isasymforall}A{\isachardot}\ {\isasymexists}x{\isachardot}\ A\ {\isacharequal}\ f\ x{\isachardoublequoteclose}\isanewline
\isanewline
%
\isadelimproof
\isanewline
\ \ %
\endisadelimproof
%
\isatagproof
\isacommand{oops}\isamarkupfalse%
%
\endisatagproof
{\isafoldproof}%
%
\isadelimproof
%
\endisadelimproof
%
\begin{isamarkuptext}%
La demostración la haremos por la regla la introducción a la
negación, la cual es una simplificación de la regla de 
reducción al absurdo, cuyo esquema mostramos a continuación:   
 \begin{itemize}
  \item[] \isa{{\isacharparenleft}P\ {\isasymLongrightarrow}\ False{\isacharparenright}\ {\isasymLongrightarrow}\ {\isasymnot}\ P} \hfill (\isa{notI})
  \end{itemize}


Esta es la demostración detallada del teorema:%
\end{isamarkuptext}\isamarkuptrue%
\isacommand{theorem}\isamarkupfalse%
\ CantorDetallada{\isacharcolon}\ {\isachardoublequoteopen}{\isasymnexists}f\ {\isacharcolon}{\isacharcolon}\ {\isacharprime}a\ {\isasymRightarrow}\ {\isacharprime}a\ set{\isachardot}\ {\isasymforall}B{\isachardot}\ {\isasymexists}x{\isachardot}\ B\ {\isacharequal}\ f\ x{\isachardoublequoteclose}\isanewline
%
\isadelimproof
%
\endisadelimproof
%
\isatagproof
\isacommand{proof}\isamarkupfalse%
\ {\isacharparenleft}rule\ notI{\isacharparenright}\isanewline
\ \ \isacommand{assume}\isamarkupfalse%
\ {\isachardoublequoteopen}{\isasymexists}f\ {\isacharcolon}{\isacharcolon}\ {\isacharprime}a\ {\isasymRightarrow}\ {\isacharprime}a\ set{\isachardot}\ {\isasymforall}A{\isachardot}\ {\isasymexists}x{\isachardot}\ A\ {\isacharequal}\ f\ x{\isachardoublequoteclose}\isanewline
\ \ \isacommand{then}\isamarkupfalse%
\ \isacommand{obtain}\isamarkupfalse%
\ f\ {\isacharcolon}{\isacharcolon}\ {\isachardoublequoteopen}{\isacharprime}a\ {\isasymRightarrow}\ {\isacharprime}a\ set{\isachardoublequoteclose}\ \isakeyword{where}\ {\isacharasterisk}{\isacharcolon}\ {\isachardoublequoteopen}{\isasymforall}A{\isachardot}\ {\isasymexists}x{\isachardot}\ A\ {\isacharequal}\ f\ x{\isachardoublequoteclose}\ \isacommand{by}\isamarkupfalse%
\ {\isacharparenleft}rule\isanewline
\ \ \ \ \ \ \ \ exE{\isacharparenright}\isanewline
\ \ \isacommand{let}\isamarkupfalse%
\ {\isacharquery}B\ {\isacharequal}\ {\isachardoublequoteopen}{\isacharbraceleft}x{\isachardot}\ x\ {\isasymnotin}\ f\ x{\isacharbraceright}{\isachardoublequoteclose}\isanewline
\ \ \isacommand{from}\isamarkupfalse%
\ {\isacharasterisk}\ \isacommand{obtain}\isamarkupfalse%
\ {\isachardoublequoteopen}\ {\isasymexists}x{\isachardot}\ {\isacharquery}B\ {\isacharequal}\ f\ x\ {\isachardoublequoteclose}\ \isacommand{by}\isamarkupfalse%
\ {\isacharparenleft}rule\ allE{\isacharparenright}\isanewline
\ \ \isacommand{then}\isamarkupfalse%
\ \ \isacommand{obtain}\isamarkupfalse%
\ a\ \isakeyword{where}\ {\isadigit{1}}{\isacharcolon}{\isachardoublequoteopen}{\isacharquery}B\ {\isacharequal}\ f\ a{\isachardoublequoteclose}\ \isacommand{by}\isamarkupfalse%
\ {\isacharparenleft}rule\ exE{\isacharparenright}\isanewline
\ \ \isacommand{show}\isamarkupfalse%
\ False\isanewline
\ \ \isacommand{proof}\isamarkupfalse%
\ {\isacharparenleft}cases{\isacharparenright}\isanewline
\ \ \ \ \isacommand{assume}\isamarkupfalse%
\ {\isachardoublequoteopen}a\ {\isasymin}\ {\isacharquery}B{\isachardoublequoteclose}\ \ \isanewline
\ \ \ \ \isacommand{then}\isamarkupfalse%
\ \isacommand{show}\isamarkupfalse%
\ False\ \ \isacommand{using}\isamarkupfalse%
\ {\isadigit{1}}\ \isacommand{by}\isamarkupfalse%
\ blast\isanewline
\ \ \isacommand{next}\isamarkupfalse%
\ \isanewline
\ \ \ \ \isacommand{assume}\isamarkupfalse%
\ {\isachardoublequoteopen}a\ {\isasymnotin}\ {\isacharquery}B{\isachardoublequoteclose}\isanewline
\ \ \ \ \isacommand{thus}\isamarkupfalse%
\ False\ \isacommand{using}\isamarkupfalse%
\ {\isadigit{1}}\ \isacommand{by}\isamarkupfalse%
\ blast\isanewline
\ \ \isacommand{qed}\isamarkupfalse%
\isanewline
\isacommand{qed}\isamarkupfalse%
%
\endisatagproof
{\isafoldproof}%
%
\isadelimproof
%
\endisadelimproof
%
\begin{isamarkuptext}%
Esta es la demostración aplicativa del teorema:%
\end{isamarkuptext}\isamarkuptrue%
\isacommand{theorem}\isamarkupfalse%
\ CantorAplicativa\ {\isacharcolon}\isanewline
\ {\isachardoublequoteopen}{\isasymnexists}f\ {\isacharcolon}{\isacharcolon}\ {\isacharprime}a\ {\isasymRightarrow}\ {\isacharprime}a\ set{\isachardot}\ {\isasymforall}A{\isachardot}\ {\isasymexists}x{\isachardot}\ A\ {\isacharequal}\ f\ x{\isachardoublequoteclose}\isanewline
%
\isadelimproof
\ \ %
\endisadelimproof
%
\isatagproof
\isacommand{apply}\isamarkupfalse%
\ {\isacharparenleft}rule\ notI{\isacharparenright}\isanewline
\ \ \isacommand{apply}\isamarkupfalse%
\ {\isacharparenleft}erule\ exE{\isacharparenright}\isanewline
\ \ \isacommand{apply}\isamarkupfalse%
\ {\isacharparenleft}erule{\isacharunderscore}tac\ x\ {\isacharequal}\ {\isachardoublequoteopen}{\isacharbraceleft}x{\isachardot}\ x\ {\isasymnotin}\ f\ x{\isacharbraceright}{\isachardoublequoteclose}\ \isakeyword{in}\ allE{\isacharparenright}\isanewline
\ \ \isacommand{apply}\isamarkupfalse%
\ {\isacharparenleft}erule\ exE{\isacharparenright}\isanewline
\ \ \isacommand{apply}\isamarkupfalse%
\ \ blast\ \isanewline
\ \ \isacommand{done}\isamarkupfalse%
%
\endisatagproof
{\isafoldproof}%
%
\isadelimproof
%
\endisadelimproof
%
\begin{isamarkuptext}%
Esta es la demostración automática del teorema:%
\end{isamarkuptext}\isamarkuptrue%
\isacommand{theorem}\isamarkupfalse%
\ CantorAutomatic{\isacharcolon}\ {\isachardoublequoteopen}{\isasymnexists}f\ {\isacharcolon}{\isacharcolon}\ {\isacharprime}a\ {\isasymRightarrow}\ {\isacharprime}a\ set{\isachardot}\ {\isasymforall}B{\isachardot}\ {\isasymexists}x{\isachardot}\ B\ {\isacharequal}\ f\ x{\isachardoublequoteclose}\isanewline
%
\isadelimproof
\ \ %
\endisadelimproof
%
\isatagproof
\isacommand{by}\isamarkupfalse%
\ best%
\endisatagproof
{\isafoldproof}%
%
\isadelimproof
%
\endisadelimproof
%
\begin{isamarkuptext}%
En la demostración de isabelle hemos utilizado el método de prueba
rule con las siguientes reglas, tanto en la aplicativa como en la
 detallada:
 \begin{itemize}
  \item[] \isa{\mbox{}\inferrule{\mbox{\mbox{}\inferrule{\mbox{P}}{\mbox{False}}}}{\mbox{{\isasymnot}\ P}}} \hfill (\isa{notI})
  \end{itemize}
 \begin{itemize}
  \item[] \isa{\mbox{}\inferrule{\mbox{{\isasymexists}x{\isachardot}\ P\ x}\\\ \mbox{{\isasymAnd}x{\isachardot}\ \mbox{}\inferrule{\mbox{P\ x}}{\mbox{Q}}}}{\mbox{Q}}} \hfill (\isa{exE})
  \end{itemize}
 \begin{itemize}
  \item[] \isa{\mbox{}\inferrule{\mbox{{\isasymforall}x{\isachardot}\ P\ x}\\\ \mbox{\mbox{}\inferrule{\mbox{P\ x}}{\mbox{R}}}}{\mbox{R}}} \hfill (\isa{allE})
  \end{itemize}
También hacemos uso de blast, que es un conjunto de reglas lógicas y 
 la demostración automática la hacemos por medio de "best".%
\end{isamarkuptext}\isamarkuptrue%
%
\isadelimtheory
%
\endisadelimtheory
%
\isatagtheory
%
\endisatagtheory
{\isafoldtheory}%
%
\isadelimtheory
%
\endisadelimtheory
%
\end{isabellebody}%
\endinput
%:%file=~/ownCloud/alonso/curso-TFG/Carlos/TFG_de_Carlos/TeoremaCantor.thy%:%
%:%6=1%:%
%:%20=9%:%
%:%24=11%:%
%:%28=13%:%
%:%32=15%:%
%:%33=16%:%
%:%34=17%:%
%:%35=18%:%
%:%36=19%:%
%:%37=20%:%
%:%38=21%:%
%:%39=22%:%
%:%40=23%:%
%:%41=24%:%
%:%42=25%:%
%:%43=26%:%
%:%44=27%:%
%:%45=28%:%
%:%46=29%:%
%:%47=30%:%
%:%48=31%:%
%:%49=32%:%
%:%50=33%:%
%:%51=34%:%
%:%52=35%:%
%:%53=36%:%
%:%54=37%:%
%:%55=38%:%
%:%56=39%:%
%:%57=40%:%
%:%58=41%:%
%:%59=42%:%
%:%60=43%:%
%:%61=44%:%
%:%62=45%:%
%:%63=46%:%
%:%64=47%:%
%:%65=48%:%
%:%66=49%:%
%:%67=50%:%
%:%68=51%:%
%:%69=52%:%
%:%70=53%:%
%:%71=54%:%
%:%72=55%:%
%:%73=56%:%
%:%74=57%:%
%:%75=58%:%
%:%76=59%:%
%:%77=60%:%
%:%78=61%:%
%:%79=62%:%
%:%80=63%:%
%:%81=64%:%
%:%82=65%:%
%:%83=66%:%
%:%84=67%:%
%:%85=68%:%
%:%86=69%:%
%:%87=70%:%
%:%88=71%:%
%:%89=72%:%
%:%90=73%:%
%:%91=74%:%
%:%92=75%:%
%:%93=76%:%
%:%94=77%:%
%:%95=78%:%
%:%96=79%:%
%:%97=80%:%
%:%98=81%:%
%:%99=82%:%
%:%100=83%:%
%:%101=84%:%
%:%102=85%:%
%:%103=86%:%
%:%104=87%:%
%:%105=88%:%
%:%106=89%:%
%:%107=90%:%
%:%108=91%:%
%:%109=92%:%
%:%110=93%:%
%:%111=94%:%
%:%112=95%:%
%:%114=97%:%
%:%115=97%:%
%:%116=98%:%
%:%119=99%:%
%:%120=100%:%
%:%124=100%:%
%:%134=102%:%
%:%135=103%:%
%:%136=104%:%
%:%137=105%:%
%:%138=106%:%
%:%139=107%:%
%:%140=108%:%
%:%141=109%:%
%:%142=110%:%
%:%144=112%:%
%:%145=112%:%
%:%152=113%:%
%:%153=113%:%
%:%154=114%:%
%:%155=114%:%
%:%156=115%:%
%:%157=115%:%
%:%158=115%:%
%:%159=115%:%
%:%160=116%:%
%:%161=117%:%
%:%162=117%:%
%:%163=118%:%
%:%164=118%:%
%:%165=118%:%
%:%166=118%:%
%:%167=119%:%
%:%168=119%:%
%:%169=119%:%
%:%170=119%:%
%:%171=120%:%
%:%172=120%:%
%:%173=121%:%
%:%174=121%:%
%:%175=122%:%
%:%176=122%:%
%:%177=123%:%
%:%178=123%:%
%:%179=123%:%
%:%180=123%:%
%:%181=123%:%
%:%182=124%:%
%:%183=124%:%
%:%184=125%:%
%:%185=125%:%
%:%186=126%:%
%:%187=126%:%
%:%188=126%:%
%:%189=126%:%
%:%190=127%:%
%:%191=127%:%
%:%192=128%:%
%:%202=130%:%
%:%204=133%:%
%:%205=133%:%
%:%206=134%:%
%:%209=135%:%
%:%213=135%:%
%:%214=135%:%
%:%215=136%:%
%:%216=136%:%
%:%217=137%:%
%:%218=137%:%
%:%219=138%:%
%:%220=138%:%
%:%221=139%:%
%:%222=139%:%
%:%223=140%:%
%:%233=142%:%
%:%235=143%:%
%:%236=143%:%
%:%239=144%:%
%:%243=144%:%
%:%244=144%:%
%:%253=146%:%
%:%254=147%:%
%:%255=148%:%
%:%256=149%:%
%:%257=150%:%
%:%258=151%:%
%:%259=152%:%
%:%260=153%:%
%:%261=154%:%
%:%262=155%:%
%:%263=156%:%
%:%264=157%:%
%:%265=158%:%
%:%266=159%:%

%
\begin{isabellebody}%
\setisabellecontext{Metodosdepruebasyreglas}%
%
\isadelimtheory
\isanewline
%
\endisadelimtheory
%
\isatagtheory
%
\endisatagtheory
{\isafoldtheory}%
%
\isadelimtheory
%
\endisadelimtheory
%
\begin{isamarkuptext}%
Métodos de pruebas de demostraciones:

 \begin{itemize}
  \item[] \isa{{\isasymlbrakk}P\ {\isadigit{0}}{\isacharsemicolon}\ {\isasymAnd}nat{\isachardot}\ P\ nat\ {\isasymLongrightarrow}\ P\ {\isacharparenleft}Suc\ nat{\isacharparenright}{\isasymrbrakk}\ {\isasymLongrightarrow}\ P\ nat} \hfill (\isa{nat{\isachardot}induct})
  \end{itemize}

 \begin{itemize}
  \item[] \isa{{\isasymlbrakk}P\ {\isasymLongrightarrow}\ Q{\isacharsemicolon}\ Q\ {\isasymLongrightarrow}\ P{\isasymrbrakk}\ {\isasymLongrightarrow}\ P\ {\isacharequal}\ Q} \hfill (\isa{iffI})
  \end{itemize}

 \begin{itemize}
  \item[] \isa{{\isasymlbrakk}finite\ x{\isacharsemicolon}\ P\ {\isasymemptyset}{\isacharsemicolon}\ {\isasymAnd}A\ a{\isachardot}\ finite\ A\ {\isasymand}\ P\ A\ {\isasymLongrightarrow}\ P\ {\isacharparenleft}{\isacharbraceleft}a{\isacharbraceright}\ {\isasymunion}\ A{\isacharparenright}{\isasymrbrakk}\ {\isasymLongrightarrow}\ P\ x} \hfill (\isa{finite{\isachardot}induct})
  \end{itemize}

 \begin{itemize}
  \item[] \isa{{\isacharparenleft}P\ {\isasymLongrightarrow}\ False{\isacharparenright}\ {\isasymLongrightarrow}\ {\isasymnot}\ P} \hfill (\isa{notI})
  \end{itemize}


Reglas usadas:

 \begin{itemize}
  \item[] \isa{inj{\isacharunderscore}on\ f\ A\ {\isacharequal}\ {\isacharparenleft}{\isasymforall}x{\isasymin}A{\isachardot}\ {\isasymforall}y{\isasymin}A{\isachardot}\ f\ x\ {\isacharequal}\ f\ y\ {\isasymlongrightarrow}\ x\ {\isacharequal}\ y{\isacharparenright}} \hfill (\isa{inj{\isacharunderscore}on{\isacharunderscore}def})
  \end{itemize}
 \begin{itemize}
  \item[] \isa{\mbox{}\inferrule{\mbox{ordering{\isacharunderscore}top\ less{\isacharunderscore}eq\ less\ top}}{\mbox{less{\isacharunderscore}eq\ a\ top}}} \hfill
 (\isa{ordering{\isacharunderscore}top{\isachardot}extremum})
  \end{itemize}
 \begin{itemize}
  \item[] \isa{{\isacharparenleft}f\ {\isacharequal}\ g{\isacharparenright}\ {\isacharequal}\ {\isacharparenleft}{\isasymforall}x{\isachardot}\ f\ x\ {\isacharequal}\ g\ x{\isacharparenright}} \hfill (\isa{fun{\isacharunderscore}eq{\isacharunderscore}iff})
  \end{itemize}
 \begin{itemize}
  \item[] \isa{{\isacharparenleft}f\ {\isasymcirc}\ g{\isacharparenright}\ x\ {\isacharequal}\ f\ {\isacharparenleft}g\ x{\isacharparenright}} \hfill (\isa{o{\isacharunderscore}apply})
  \end{itemize}
 \begin{itemize}
  \item[] \isa{\mbox{}\inferrule{\mbox{\mbox{}\inferrule{\mbox{P}}{\mbox{Q}}}\\\ \mbox{\mbox{}\inferrule{\mbox{Q}}{\mbox{P}}}}{\mbox{P\ {\isacharequal}\ Q}}} \hfill (\isa{iffI})
  \end{itemize}
 \begin{itemize}
  \item[] \isa{\mbox{}\inferrule{\mbox{ListMem\ x\ xs}}{\mbox{ListMem\ x\ {\isacharparenleft}y\ {\isasymcdot}\ xs{\isacharparenright}}}} \hfill (\isa{insert})
  \end{itemize}
 \begin{itemize}
  \item[] \isa{\mbox{}\inferrule{\mbox{{\isasymexists}x{\isachardot}\ P\ x}\\\ \mbox{{\isasymAnd}x{\isachardot}\ \mbox{}\inferrule{\mbox{P\ x}}{\mbox{Q}}}}{\mbox{Q}}} \hfill (\isa{exE})
  \end{itemize}
 \begin{itemize}
  \item[] \isa{\mbox{}\inferrule{\mbox{{\isasymforall}x{\isachardot}\ P\ x}\\\ \mbox{\mbox{}\inferrule{\mbox{P\ x}}{\mbox{R}}}}{\mbox{R}}} \hfill (\isa{allE})
  \end{itemize}
 \begin{itemize}
  \item[] \isa{\mbox{}\inferrule{\mbox{\mbox{}\inferrule{\mbox{P}}{\mbox{False}}}}{\mbox{{\isasymnot}\ P}}} \hfill (\isa{notI})
  \end{itemize}
 \begin{itemize}
  \item[] \isa{{\isacharparenleft}{\isacharparenleft}P\ {\isasymlongrightarrow}\ Q{\isacharparenright}\ {\isasymand}\ {\isacharparenleft}{\isasymnot}\ P\ {\isasymlongrightarrow}\ Q{\isacharparenright}{\isacharparenright}\ {\isacharequal}\ Q} \hfill (\isa{cases})
  \end{itemize}%
\end{isamarkuptext}\isamarkuptrue%
%
\isadelimtheory
%
\endisadelimtheory
%
\isatagtheory
%
\endisatagtheory
{\isafoldtheory}%
%
\isadelimtheory
%
\endisadelimtheory
%
\end{isabellebody}%
\endinput
%:%file=~/Escritorio/TFG/Metodosdepruebasyreglas.thy%:%
%:%6=1%:%
%:%20=9%:%
%:%21=10%:%
%:%22=11%:%
%:%23=12%:%
%:%24=13%:%
%:%25=14%:%
%:%26=15%:%
%:%27=16%:%
%:%28=17%:%
%:%29=18%:%
%:%30=19%:%
%:%31=20%:%
%:%32=21%:%
%:%33=22%:%
%:%34=23%:%
%:%35=24%:%
%:%36=25%:%
%:%37=26%:%
%:%38=27%:%
%:%39=28%:%
%:%40=29%:%
%:%41=30%:%
%:%42=31%:%
%:%42=32%:%
%:%43=33%:%
%:%44=34%:%
%:%45=35%:%
%:%46=36%:%
%:%47=37%:%
%:%48=38%:%
%:%49=39%:%
%:%50=40%:%
%:%51=41%:%
%:%52=42%:%
%:%53=43%:%
%:%54=44%:%
%:%55=45%:%
%:%56=46%:%
%:%57=47%:%
%:%58=48%:%
%:%59=49%:%
%:%60=50%:%
%:%61=51%:%
%:%62=52%:%
%:%63=53%:%
%:%64=54%:%
%:%65=55%:%
%:%66=56%:%
%:%67=57%:%
%:%68=58%:%
%:%69=59%:%
%:%70=60%:%
%:%71=61%:%




\chapter{Teoría de números}
%
\begin{isabellebody}%
\setisabellecontext{SumaImpares}%
%
\isadelimtheory
%
\endisadelimtheory
%
\isatagtheory
%
\endisatagtheory
{\isafoldtheory}%
%
\isadelimtheory
%
\endisadelimtheory
%
\isadelimdocument
%
\endisadelimdocument
%
\isatagdocument
%
\isamarkupsection{Suma de los primeros números impares%
}
\isamarkuptrue%
%
\endisatagdocument
{\isafolddocument}%
%
\isadelimdocument
%
\endisadelimdocument
%
\begin{isamarkuptext}%
El primer teorema es una propiedad de los números naturales.

  \begin{teorema}
    La suma de los $n$ primeros números impares es $n^2$.
  \end{teorema}

  \begin{demostracion}
    La demostración la haremos en inducción sobre $n$.
\begin {itemize}
\item EL caso $n = 0$ es trivial, ya que $0 = 0$.
\item Supongamos que se verifica la hipótesis para $n$ y veamos para
 $n+1$. \\
Tenemos que demostrar que $\sum_{j=1}^{n+1} k_j = (n+1)^2$ siendo los
 $k_{j}$ el j-ésimo impar, es decir, $k_{j} = 2j - 1$.
$$\sum_{j = 1}^{n+1} k_{j} = k_{n+1} + \sum^{n}_{j=1} k_{j} = k_{n+1} +
 n^{2} = 2(n+1) - 1 + n^2 = n^2 + 2n + 1 = (n+1)^2$$ 
\end {itemize}
.
  \end{demostracion}

  Para especificar el teorema en Isabelle, se comienza definiendo 
  la función \isa{suma{\isacharunderscore}impares} tal que \isa{suma{\isacharunderscore}impares\ n} es la 
  suma de los $n$ primeros números impares%
\end{isamarkuptext}\isamarkuptrue%
\isacommand{fun}\isamarkupfalse%
\ suma{\isacharunderscore}impares\ {\isacharcolon}{\isacharcolon}\ {\isachardoublequoteopen}nat\ {\isasymRightarrow}\ nat{\isachardoublequoteclose}\ \isakeyword{where}\isanewline
\ \ {\isachardoublequoteopen}suma{\isacharunderscore}impares\ {\isadigit{0}}\ {\isacharequal}\ {\isadigit{0}}{\isachardoublequoteclose}\ \isanewline
{\isacharbar}\ {\isachardoublequoteopen}suma{\isacharunderscore}impares\ {\isacharparenleft}Suc\ n{\isacharparenright}\ {\isacharequal}\ {\isacharparenleft}{\isadigit{2}}{\isacharasterisk}{\isacharparenleft}Suc\ n{\isacharparenright}\ {\isacharminus}\ {\isadigit{1}}{\isacharparenright}\ {\isacharplus}\ suma{\isacharunderscore}impares\ n{\isachardoublequoteclose}%
\begin{isamarkuptext}%
El enunciado del teorema es el siguiente:%
\end{isamarkuptext}\isamarkuptrue%
\isacommand{lemma}\isamarkupfalse%
\ {\isachardoublequoteopen}suma{\isacharunderscore}impares\ n\ {\isacharequal}\ n\ {\isacharasterisk}\ n{\isachardoublequoteclose}\isanewline
%
\isadelimproof
%
\endisadelimproof
%
\isatagproof
\isacommand{oops}\isamarkupfalse%
%
\endisatagproof
{\isafoldproof}%
%
\isadelimproof
%
\endisadelimproof
%
\begin{isamarkuptext}%
En la demostración se usará la táctica \isa{induct} que hace
  uso del esquema de inducción sobre los naturales:
  \begin{itemize}
  \item[] \isa{\mbox{}\inferrule{\mbox{P\ {\isadigit{0}}}\\\ \mbox{{\isasymAnd}nat{\isachardot}\ \mbox{}\inferrule{\mbox{P\ nat}}{\mbox{P\ {\isacharparenleft}Suc\ nat{\isacharparenright}}}}}{\mbox{P\ nat}}} \hfill (\isa{nat{\isachardot}induct})
  \end{itemize}

  Vamos a presentar distintas demostraciones del teorema. La 
  primera es la demostración aplicativa%
\end{isamarkuptext}\isamarkuptrue%
\isacommand{lemma}\isamarkupfalse%
\ {\isachardoublequoteopen}suma{\isacharunderscore}impares\ n\ {\isacharequal}\ n\ {\isacharasterisk}\ n{\isachardoublequoteclose}\isanewline
%
\isadelimproof
\ \ %
\endisadelimproof
%
\isatagproof
\isacommand{apply}\isamarkupfalse%
\ {\isacharparenleft}induct\ n{\isacharparenright}\ \isanewline
\ \ \ \isacommand{apply}\isamarkupfalse%
\ simp{\isacharunderscore}all\isanewline
\ \ \isacommand{done}\isamarkupfalse%
%
\endisatagproof
{\isafoldproof}%
%
\isadelimproof
%
\endisadelimproof
%
\begin{isamarkuptext}%
La demostración automática es%
\end{isamarkuptext}\isamarkuptrue%
\isacommand{lemma}\isamarkupfalse%
\ {\isachardoublequoteopen}suma{\isacharunderscore}impares\ n\ {\isacharequal}\ n\ {\isacharasterisk}\ n{\isachardoublequoteclose}\isanewline
%
\isadelimproof
\ \ %
\endisadelimproof
%
\isatagproof
\isacommand{by}\isamarkupfalse%
\ {\isacharparenleft}induct\ n{\isacharparenright}\ simp{\isacharunderscore}all%
\endisatagproof
{\isafoldproof}%
%
\isadelimproof
%
\endisadelimproof
%
\begin{isamarkuptext}%
La demostración del lema anterior por inducción y razonamiento 
   ecuacional es%
\end{isamarkuptext}\isamarkuptrue%
\isacommand{lemma}\isamarkupfalse%
\ {\isachardoublequoteopen}suma{\isacharunderscore}impares\ n\ {\isacharequal}\ n\ {\isacharasterisk}\ n{\isachardoublequoteclose}\isanewline
%
\isadelimproof
%
\endisadelimproof
%
\isatagproof
\isacommand{proof}\isamarkupfalse%
\ {\isacharparenleft}induct\ n{\isacharparenright}\isanewline
\ \ \isacommand{show}\isamarkupfalse%
\ {\isachardoublequoteopen}suma{\isacharunderscore}impares\ {\isadigit{0}}\ {\isacharequal}\ {\isadigit{0}}\ {\isacharasterisk}\ {\isadigit{0}}{\isachardoublequoteclose}\ \isacommand{by}\isamarkupfalse%
\ simp\isanewline
\isacommand{next}\isamarkupfalse%
\isanewline
\ \ \isacommand{fix}\isamarkupfalse%
\ n\ \isacommand{assume}\isamarkupfalse%
\ HI{\isacharcolon}\ {\isachardoublequoteopen}suma{\isacharunderscore}impares\ n\ {\isacharequal}\ n\ {\isacharasterisk}\ n{\isachardoublequoteclose}\isanewline
\ \ \isacommand{have}\isamarkupfalse%
\ {\isachardoublequoteopen}suma{\isacharunderscore}impares\ {\isacharparenleft}Suc\ n{\isacharparenright}\ {\isacharequal}\ {\isacharparenleft}{\isadigit{2}}\ {\isacharasterisk}\ {\isacharparenleft}Suc\ n{\isacharparenright}\ {\isacharminus}\ {\isadigit{1}}{\isacharparenright}\ {\isacharplus}\ suma{\isacharunderscore}impares\ n{\isachardoublequoteclose}\ \isanewline
\ \ \ \ \isacommand{by}\isamarkupfalse%
\ simp\isanewline
\ \ \isacommand{also}\isamarkupfalse%
\ \isacommand{have}\isamarkupfalse%
\ {\isachardoublequoteopen}{\isasymdots}\ {\isacharequal}\ {\isacharparenleft}{\isadigit{2}}\ {\isacharasterisk}\ {\isacharparenleft}Suc\ n{\isacharparenright}\ {\isacharminus}\ {\isadigit{1}}{\isacharparenright}\ {\isacharplus}\ n\ {\isacharasterisk}\ n{\isachardoublequoteclose}\ \isacommand{using}\isamarkupfalse%
\ HI\ \isacommand{by}\isamarkupfalse%
\ simp\isanewline
\ \ \isacommand{also}\isamarkupfalse%
\ \isacommand{have}\isamarkupfalse%
\ {\isachardoublequoteopen}{\isasymdots}\ {\isacharequal}\ n\ {\isacharasterisk}\ n\ {\isacharplus}\ {\isadigit{2}}\ {\isacharasterisk}\ n\ {\isacharplus}\ {\isadigit{1}}{\isachardoublequoteclose}\ \isacommand{by}\isamarkupfalse%
\ simp\isanewline
\ \ \isacommand{finally}\isamarkupfalse%
\ \isacommand{show}\isamarkupfalse%
\ {\isachardoublequoteopen}suma{\isacharunderscore}impares\ {\isacharparenleft}Suc\ n{\isacharparenright}\ {\isacharequal}\ {\isacharparenleft}Suc\ n{\isacharparenright}\ {\isacharasterisk}\ {\isacharparenleft}Suc\ n{\isacharparenright}{\isachardoublequoteclose}\ \isacommand{by}\isamarkupfalse%
\ simp\isanewline
\isacommand{qed}\isamarkupfalse%
%
\endisatagproof
{\isafoldproof}%
%
\isadelimproof
%
\endisadelimproof
%
\begin{isamarkuptext}%
La demostración del lema anterior con patrones y razonamiento 
   ecuacional es%
\end{isamarkuptext}\isamarkuptrue%
\isacommand{lemma}\isamarkupfalse%
\ {\isachardoublequoteopen}suma{\isacharunderscore}impares\ n\ {\isacharequal}\ n\ {\isacharasterisk}\ n{\isachardoublequoteclose}\ {\isacharparenleft}\isakeyword{is}\ {\isachardoublequoteopen}{\isacharquery}P\ n{\isachardoublequoteclose}{\isacharparenright}\isanewline
%
\isadelimproof
%
\endisadelimproof
%
\isatagproof
\isacommand{proof}\isamarkupfalse%
\ {\isacharparenleft}induct\ n{\isacharparenright}\isanewline
\ \ \isacommand{show}\isamarkupfalse%
\ {\isachardoublequoteopen}{\isacharquery}P\ {\isadigit{0}}{\isachardoublequoteclose}\ \isacommand{by}\isamarkupfalse%
\ simp\isanewline
\isacommand{next}\isamarkupfalse%
\isanewline
\ \ \isacommand{fix}\isamarkupfalse%
\ n\ \isanewline
\ \ \isacommand{assume}\isamarkupfalse%
\ HI{\isacharcolon}\ {\isachardoublequoteopen}{\isacharquery}P\ n{\isachardoublequoteclose}\isanewline
\ \ \isacommand{have}\isamarkupfalse%
\ {\isachardoublequoteopen}suma{\isacharunderscore}impares\ {\isacharparenleft}Suc\ n{\isacharparenright}\ {\isacharequal}\ {\isacharparenleft}{\isadigit{2}}\ {\isacharasterisk}\ {\isacharparenleft}Suc\ n{\isacharparenright}\ {\isacharminus}\ {\isadigit{1}}{\isacharparenright}\ {\isacharplus}\ suma{\isacharunderscore}impares\ n{\isachardoublequoteclose}\ \isanewline
\ \ \ \ \isacommand{by}\isamarkupfalse%
\ simp\isanewline
\ \ \isacommand{also}\isamarkupfalse%
\ \isacommand{have}\isamarkupfalse%
\ {\isachardoublequoteopen}{\isasymdots}\ {\isacharequal}\ {\isacharparenleft}{\isadigit{2}}\ {\isacharasterisk}\ {\isacharparenleft}Suc\ n{\isacharparenright}\ {\isacharminus}\ {\isadigit{1}}{\isacharparenright}\ {\isacharplus}\ n\ {\isacharasterisk}\ n{\isachardoublequoteclose}\ \isacommand{using}\isamarkupfalse%
\ HI\ \isacommand{by}\isamarkupfalse%
\ simp\isanewline
\ \ \isacommand{also}\isamarkupfalse%
\ \isacommand{have}\isamarkupfalse%
\ {\isachardoublequoteopen}{\isasymdots}\ {\isacharequal}\ n\ {\isacharasterisk}\ n\ {\isacharplus}\ {\isadigit{2}}\ {\isacharasterisk}\ n\ {\isacharplus}\ {\isadigit{1}}{\isachardoublequoteclose}\ \isacommand{by}\isamarkupfalse%
\ simp\isanewline
\ \ \isacommand{finally}\isamarkupfalse%
\ \isacommand{show}\isamarkupfalse%
\ {\isachardoublequoteopen}{\isacharquery}P\ {\isacharparenleft}Suc\ n{\isacharparenright}{\isachardoublequoteclose}\ \isacommand{by}\isamarkupfalse%
\ simp\isanewline
\isacommand{qed}\isamarkupfalse%
%
\endisatagproof
{\isafoldproof}%
%
\isadelimproof
%
\endisadelimproof
%
\begin{isamarkuptext}%
La demostración usando patrones es%
\end{isamarkuptext}\isamarkuptrue%
\isacommand{lemma}\isamarkupfalse%
\ {\isachardoublequoteopen}suma{\isacharunderscore}impares\ n\ {\isacharequal}\ n\ {\isacharasterisk}\ n{\isachardoublequoteclose}\ {\isacharparenleft}\isakeyword{is}\ {\isachardoublequoteopen}{\isacharquery}P\ n{\isachardoublequoteclose}{\isacharparenright}\isanewline
%
\isadelimproof
%
\endisadelimproof
%
\isatagproof
\isacommand{proof}\isamarkupfalse%
\ {\isacharparenleft}induct\ n{\isacharparenright}\isanewline
\ \ \isacommand{show}\isamarkupfalse%
\ {\isachardoublequoteopen}{\isacharquery}P\ {\isadigit{0}}{\isachardoublequoteclose}\ \isacommand{by}\isamarkupfalse%
\ simp\isanewline
\isacommand{next}\isamarkupfalse%
\isanewline
\ \ \isacommand{fix}\isamarkupfalse%
\ n\ \isanewline
\ \ \isacommand{assume}\isamarkupfalse%
\ {\isachardoublequoteopen}{\isacharquery}P\ n{\isachardoublequoteclose}\isanewline
\ \ \isacommand{then}\isamarkupfalse%
\ \isacommand{show}\isamarkupfalse%
\ {\isachardoublequoteopen}{\isacharquery}P\ {\isacharparenleft}Suc\ n{\isacharparenright}{\isachardoublequoteclose}\ \isacommand{by}\isamarkupfalse%
\ simp\isanewline
\isacommand{qed}\isamarkupfalse%
\isanewline
%
\endisatagproof
{\isafoldproof}%
%
\isadelimproof
%
\endisadelimproof
%
\isadelimtheory
%
\endisadelimtheory
%
\isatagtheory
%
\endisatagtheory
{\isafoldtheory}%
%
\isadelimtheory
%
\endisadelimtheory
%
\end{isabellebody}%
\endinput
%:%file=~/Escritorio/TFG-v1/EjerciciosDELMF/SumaImpares.thy%:%
%:%24=8%:%
%:%36=10%:%
%:%37=11%:%
%:%38=12%:%
%:%39=13%:%
%:%40=14%:%
%:%41=15%:%
%:%42=16%:%
%:%43=17%:%
%:%44=18%:%
%:%45=19%:%
%:%46=20%:%
%:%47=21%:%
%:%48=22%:%
%:%49=23%:%
%:%50=24%:%
%:%51=25%:%
%:%52=26%:%
%:%53=27%:%
%:%54=28%:%
%:%55=29%:%
%:%56=30%:%
%:%57=31%:%
%:%58=32%:%
%:%60=35%:%
%:%61=35%:%
%:%62=36%:%
%:%63=37%:%
%:%65=39%:%
%:%67=41%:%
%:%68=41%:%
%:%75=42%:%
%:%85=44%:%
%:%86=45%:%
%:%87=46%:%
%:%88=47%:%
%:%89=48%:%
%:%90=49%:%
%:%91=50%:%
%:%92=51%:%
%:%94=56%:%
%:%95=56%:%
%:%98=57%:%
%:%102=57%:%
%:%103=57%:%
%:%104=58%:%
%:%105=58%:%
%:%106=59%:%
%:%116=61%:%
%:%118=63%:%
%:%119=63%:%
%:%122=64%:%
%:%126=64%:%
%:%127=64%:%
%:%136=66%:%
%:%137=67%:%
%:%139=69%:%
%:%140=69%:%
%:%147=70%:%
%:%148=70%:%
%:%149=71%:%
%:%150=71%:%
%:%151=71%:%
%:%152=72%:%
%:%153=72%:%
%:%154=73%:%
%:%155=73%:%
%:%156=73%:%
%:%157=74%:%
%:%158=74%:%
%:%159=75%:%
%:%160=75%:%
%:%161=76%:%
%:%162=76%:%
%:%163=76%:%
%:%164=76%:%
%:%165=76%:%
%:%166=77%:%
%:%167=77%:%
%:%168=77%:%
%:%169=77%:%
%:%170=78%:%
%:%171=78%:%
%:%172=78%:%
%:%173=78%:%
%:%174=79%:%
%:%184=81%:%
%:%185=82%:%
%:%187=83%:%
%:%188=83%:%
%:%195=84%:%
%:%196=84%:%
%:%197=85%:%
%:%198=85%:%
%:%199=85%:%
%:%200=86%:%
%:%201=86%:%
%:%202=87%:%
%:%203=87%:%
%:%204=88%:%
%:%205=88%:%
%:%206=89%:%
%:%207=89%:%
%:%208=90%:%
%:%209=90%:%
%:%210=91%:%
%:%211=91%:%
%:%212=91%:%
%:%213=91%:%
%:%214=91%:%
%:%215=92%:%
%:%216=92%:%
%:%217=92%:%
%:%218=92%:%
%:%219=93%:%
%:%220=93%:%
%:%221=93%:%
%:%222=93%:%
%:%223=94%:%
%:%233=96%:%
%:%235=98%:%
%:%236=98%:%
%:%243=99%:%
%:%244=99%:%
%:%245=100%:%
%:%246=100%:%
%:%247=100%:%
%:%248=101%:%
%:%249=101%:%
%:%250=102%:%
%:%251=102%:%
%:%252=103%:%
%:%253=103%:%
%:%254=104%:%
%:%255=104%:%
%:%256=104%:%
%:%257=104%:%
%:%258=105%:%
%:%259=105%:%
%
\begin{isabellebody}%
\setisabellecontext{ConjuntosFinitos}%
%
\isadelimtheory
\isanewline
%
\endisadelimtheory
%
\isatagtheory
%
\endisatagtheory
{\isafoldtheory}%
%
\isadelimtheory
%
\endisadelimtheory
%
\begin{isamarkuptext}%
\comentario{Estructurar en secciones.}%
\end{isamarkuptext}\isamarkuptrue%
%
\begin{isamarkuptext}%
\comentario{Hacer demostraciones detalladas.}%
\end{isamarkuptext}\isamarkuptrue%
%
\begin{isamarkuptext}%
\comentario{Añadir lemas usados al Soporte.}%
\end{isamarkuptext}\isamarkuptrue%
%
\isadelimdocument
%
\endisadelimdocument
%
\isatagdocument
%
\isamarkupsection{Demostración en lenguaje natural%
}
\isamarkuptrue%
%
\endisatagdocument
{\isafolddocument}%
%
\isadelimdocument
%
\endisadelimdocument
%
\begin{isamarkuptext}%
El siguiente teorema es una propiedad que verifican todos los 
conjuntos finitos de números naturales  estudiado en el 
\href{http://bit.ly/2XBW6n2}{tema 10} de la
asignatura de LMF de tercer curso del grado en Matemáticas. Su enunciado
 es el siguiente 

  \begin{teorema} 
    Sea S un conjunto finito de números naturales.  Entonces todos los
 elementos de S son menores o iguales que la suma de los elementos de
 S, es decir,

 $$\forall m \in S \Longrightarrow m \leq \sum S$$ 

donde $\sum S $ denota la suma de todos los elementos de S.
  \end{teorema} 

\begin{demostracion}
La demostración del teorema la haremos por inducción sobre conjuntos
 finitos.

  
 (Base de la inducción) El caso $S = \emptyset$ es trivial.

 (Paso de la inducción) Supongamos que se verifica el teorema para un
 conjunto finito de números naturales, que se denotará por $S.$ 
 
Sea $a \in \Bbb{N}$ tal que $a \notin S,$ Ya que si $a \in S$ se tendría
probado el teorema. Luego hay que probar que: 

$$\forall n \in S \cup \{a\} \Longrightarrow n \leq \sum (S \cup
 \{a\})$$

Distingamos dos casos ahora:

Caso 1: $n = a$.

Si $n = a$, se tiene que:

$$n = a \leq a + \sum S = \sum (S \cup \{a\}).$$

Caso 2: $n \neq a.$

Si $n \neq a,$ tenemos que $n \in S,$ luego usando la hipótesis de
 inducción:

$$n \leq \sum S \leq \sum S + a = \sum (S \cup \{a\}).$$
\end{demostracion}

En la demostración del teorema hemos usado un resultado, que vamos a
 probar en Isabelle después de la especificación del teorema,
 el resultado es $\sum S + a = \sum (S \cup \{ a\})$.%
\end{isamarkuptext}\isamarkuptrue%
%
\isadelimdocument
%
\endisadelimdocument
%
\isatagdocument
%
\isamarkupsection{Especificación en Isabelle/HOL%
}
\isamarkuptrue%
%
\endisatagdocument
{\isafolddocument}%
%
\isadelimdocument
%
\endisadelimdocument
%
\begin{isamarkuptext}%
Para la especificación del teorema en Isabelle, primero debemos
 notar que  \isa{finite\ S} indica que un conjunto $S$ es 
finito  y definir  la función \isa{sumaConj} tal que
 \isa{sumaConj\ n} es la suma de todos los elementos de S.%
\end{isamarkuptext}\isamarkuptrue%
\isacommand{definition}\isamarkupfalse%
\ sumaConj\ {\isacharcolon}{\isacharcolon}\ {\isachardoublequoteopen}nat\ set\ {\isasymRightarrow}\ nat{\isachardoublequoteclose}\ \isakeyword{where}\isanewline
\ \ {\isachardoublequoteopen}sumaConj\ S\ {\isasymequiv}\ {\isasymSum}S{\isachardoublequoteclose}%
\begin{isamarkuptext}%
El enunciado del teorema es el siguiente :%
\end{isamarkuptext}\isamarkuptrue%
\isacommand{lemma}\isamarkupfalse%
\ {\isachardoublequoteopen}finite\ S\ {\isasymLongrightarrow}\ {\isasymforall}x\ {\isasymin}\ S{\isachardot}\ x\ {\isasymle}\ sumaConj\ S{\isachardoublequoteclose}\isanewline
%
\isadelimproof
\isanewline
\ \ %
\endisadelimproof
%
\isatagproof
\isacommand{oops}\isamarkupfalse%
%
\endisatagproof
{\isafoldproof}%
%
\isadelimproof
%
\endisadelimproof
%
\begin{isamarkuptext}%
Vamos a demostrar primero el lema enunciado anteriormente%
\end{isamarkuptext}\isamarkuptrue%
\isacommand{lemma}\isamarkupfalse%
\ aux{\isacharunderscore}propiedad{\isacharunderscore}conjuntos{\isacharunderscore}finitos{\isacharcolon}\isanewline
\ {\isachardoublequoteopen}\ x\ {\isasymnotin}\ S\ {\isasymand}\ finite\ S\ {\isasymlongrightarrow}\ sumaConj\ S\ {\isacharplus}\ x\ {\isacharequal}\ sumaConj{\isacharparenleft}insert\ x\ S{\isacharparenright}{\isachardoublequoteclose}\isanewline
%
\isadelimproof
\ \ %
\endisadelimproof
%
\isatagproof
\isacommand{by}\isamarkupfalse%
\ {\isacharparenleft}simp\ add{\isacharcolon}\ sumaConj{\isacharunderscore}def{\isacharparenright}%
\endisatagproof
{\isafoldproof}%
%
\isadelimproof
%
\endisadelimproof
%
\begin{isamarkuptext}%
La demostración del lema anterior se ha incluido
 \isa{sumConj{\isacharunderscore}def}, que hace referencia a la definición sumaConj que
 hemos hecho anteriormente.


En la demostración se usará la táctica \isa{induct} que hace
  uso del esquema de inducción sobre los conjuntos finitos:
  \begin{itemize}
  \item[] \isa{{\isasymlbrakk}finite\ x{\isacharsemicolon}\ P\ {\isasymemptyset}{\isacharsemicolon}\ {\isasymAnd}A\ a{\isachardot}\ finite\ A\ {\isasymand}\ P\ A\ {\isasymLongrightarrow}\ P\ {\isacharparenleft}{\isacharbraceleft}a{\isacharbraceright}\ {\isasymunion}\ A{\isacharparenright}{\isasymrbrakk}\ {\isasymLongrightarrow}\ P\ x}
 \hfill (\isa{finite{\isachardot}induct})
  \end{itemize} 

Vamos a presentar diferentes formas de demostración:%
\end{isamarkuptext}\isamarkuptrue%
%
\isadelimdocument
%
\endisadelimdocument
%
\isatagdocument
%
\isamarkupsection{Demostración aplicativa%
}
\isamarkuptrue%
%
\endisatagdocument
{\isafolddocument}%
%
\isadelimdocument
%
\endisadelimdocument
%
\begin{isamarkuptext}%
La demostración aplicativa del teorema es:%
\end{isamarkuptext}\isamarkuptrue%
\isacommand{lemma}\isamarkupfalse%
\ {\isachardoublequoteopen}finite\ S\ {\isasymLongrightarrow}\ {\isasymforall}x{\isasymin}S{\isachardot}\ x\ {\isasymle}\ sumaConj\ S{\isachardoublequoteclose}\isanewline
%
\isadelimproof
\ \ %
\endisadelimproof
%
\isatagproof
\isacommand{apply}\isamarkupfalse%
\ {\isacharparenleft}induct\ rule{\isacharcolon}\ finite{\isacharunderscore}induct{\isacharparenright}\isanewline
\ \ \ \isacommand{apply}\isamarkupfalse%
\ simp\isanewline
\ \ \isacommand{apply}\isamarkupfalse%
\ {\isacharparenleft}simp\ add{\isacharcolon}\ add{\isacharunderscore}increasing\ sumaConj{\isacharunderscore}def{\isacharparenright}\isanewline
\ \ \isacommand{done}\isamarkupfalse%
%
\endisatagproof
{\isafoldproof}%
%
\isadelimproof
%
\endisadelimproof
%
\begin{isamarkuptext}%
En la demostración anterior se ha introducido:
 \begin{itemize}
    \item[] \isa{\mbox{}\inferrule{\mbox{{\isacharparenleft}{\isadigit{0}}\ {\isacharcolon}{\isacharcolon}\ {\isacharprime}a{\isacharparenright}\ {\isasymle}\ a\ {\isasymand}\ b\ {\isasymle}\ c}}{\mbox{b\ {\isasymle}\ a\ {\isacharplus}\ c}}} 
      \hfill (\isa{add{\isacharunderscore}increasing})
  \end{itemize}%
\end{isamarkuptext}\isamarkuptrue%
%
\isadelimdocument
%
\endisadelimdocument
%
\isatagdocument
%
\isamarkupsection{Demostración automática%
}
\isamarkuptrue%
%
\endisatagdocument
{\isafolddocument}%
%
\isadelimdocument
%
\endisadelimdocument
%
\begin{isamarkuptext}%
La demostración automática es:%
\end{isamarkuptext}\isamarkuptrue%
\isacommand{lemma}\isamarkupfalse%
\ {\isachardoublequoteopen}finite\ S\ {\isasymLongrightarrow}\ {\isasymforall}x{\isasymin}S{\isachardot}\ x\ {\isasymle}\ sumaConj\ S{\isachardoublequoteclose}\isanewline
%
\isadelimproof
\ \ %
\endisadelimproof
%
\isatagproof
\isacommand{by}\isamarkupfalse%
\ {\isacharparenleft}induct\ rule{\isacharcolon}\ finite{\isacharunderscore}induct{\isacharparenright}\isanewline
\ \ \ \ \ {\isacharparenleft}auto\ simp\ add{\isacharcolon}\ \ sumaConj{\isacharunderscore}def{\isacharparenright}%
\endisatagproof
{\isafoldproof}%
%
\isadelimproof
%
\endisadelimproof
%
\isadelimdocument
%
\endisadelimdocument
%
\isatagdocument
%
\isamarkupsection{Demostración detallada%
}
\isamarkuptrue%
%
\endisatagdocument
{\isafolddocument}%
%
\isadelimdocument
%
\endisadelimdocument
%
\begin{isamarkuptext}%
La demostración declarativa es:%
\end{isamarkuptext}\isamarkuptrue%
\isacommand{lemma}\isamarkupfalse%
\ sumaConj{\isacharunderscore}acota{\isacharcolon}\ \isanewline
\ \ {\isachardoublequoteopen}finite\ S\ {\isasymLongrightarrow}\ {\isasymforall}x{\isasymin}S{\isachardot}\ x\ {\isasymle}\ sumaConj\ S{\isachardoublequoteclose}\isanewline
%
\isadelimproof
%
\endisadelimproof
%
\isatagproof
\isacommand{proof}\isamarkupfalse%
\ {\isacharparenleft}induct\ rule{\isacharcolon}\ finite{\isacharunderscore}induct{\isacharparenright}\isanewline
\ \ \isacommand{show}\isamarkupfalse%
\ {\isachardoublequoteopen}{\isasymforall}x\ {\isasymin}\ {\isacharbraceleft}{\isacharbraceright}{\isachardot}\ x\ {\isasymle}\ sumaConj\ {\isacharbraceleft}{\isacharbraceright}{\isachardoublequoteclose}\ \isacommand{by}\isamarkupfalse%
\ simp\isanewline
\isacommand{next}\isamarkupfalse%
\isanewline
\ \ \isacommand{fix}\isamarkupfalse%
\ x\ \isakeyword{and}\ F\isanewline
\ \ \isacommand{assume}\isamarkupfalse%
\ fF{\isacharcolon}\ {\isachardoublequoteopen}finite\ F{\isachardoublequoteclose}\ \isanewline
\ \ \ \ \ \isakeyword{and}\ xF{\isacharcolon}\ {\isachardoublequoteopen}x\ {\isasymnotin}\ F{\isachardoublequoteclose}\ \isanewline
\ \ \ \ \ \isakeyword{and}\ HI{\isacharcolon}\ {\isachardoublequoteopen}{\isasymforall}\ x{\isasymin}F{\isachardot}\ x\ {\isasymle}\ sumaConj\ F{\isachardoublequoteclose}\isanewline
\ \ \isacommand{show}\isamarkupfalse%
\ {\isachardoublequoteopen}{\isasymforall}y\ {\isasymin}\ insert\ x\ F{\isachardot}\ y\ {\isasymle}\ sumaConj\ {\isacharparenleft}insert\ x\ F{\isacharparenright}{\isachardoublequoteclose}\isanewline
\ \ \isacommand{proof}\isamarkupfalse%
\ \isanewline
\ \ \ \ \isacommand{fix}\isamarkupfalse%
\ y\ \isanewline
\ \ \ \ \isacommand{assume}\isamarkupfalse%
\ {\isachardoublequoteopen}y\ {\isasymin}\ insert\ x\ F{\isachardoublequoteclose}\isanewline
\ \ \ \ \isacommand{show}\isamarkupfalse%
\ {\isachardoublequoteopen}y\ {\isasymle}\ sumaConj\ {\isacharparenleft}insert\ x\ F{\isacharparenright}{\isachardoublequoteclose}\isanewline
\ \ \ \ \isacommand{proof}\isamarkupfalse%
\ {\isacharparenleft}cases\ {\isachardoublequoteopen}y\ {\isacharequal}\ x{\isachardoublequoteclose}{\isacharparenright}\isanewline
\ \ \ \ \ \ \isacommand{assume}\isamarkupfalse%
\ {\isachardoublequoteopen}y\ {\isacharequal}\ x{\isachardoublequoteclose}\isanewline
\ \ \ \ \ \ \isacommand{then}\isamarkupfalse%
\ \isacommand{have}\isamarkupfalse%
\ {\isachardoublequoteopen}y\ {\isasymle}\ x\ {\isacharplus}\ {\isacharparenleft}sumaConj\ F{\isacharparenright}{\isachardoublequoteclose}\ \isacommand{by}\isamarkupfalse%
\ simp\isanewline
\ \ \ \ \ \ \isacommand{also}\isamarkupfalse%
\ \isacommand{have}\isamarkupfalse%
\ {\isachardoublequoteopen}{\isasymdots}\ {\isacharequal}\ sumaConj\ {\isacharparenleft}insert\ x\ F{\isacharparenright}{\isachardoublequoteclose}\ \isanewline
\ \ \ \ \ \ \ \ \isacommand{by}\isamarkupfalse%
\ {\isacharparenleft}simp\ add{\isacharcolon}\ fF\ sumaConj{\isacharunderscore}def\ xF{\isacharparenright}\ \isanewline
\ \ \ \ \ \ \isacommand{finally}\isamarkupfalse%
\ \isacommand{show}\isamarkupfalse%
\ {\isacharquery}thesis\ \isacommand{{\isachardot}}\isamarkupfalse%
\isanewline
\ \ \ \ \isacommand{next}\isamarkupfalse%
\isanewline
\ \ \ \ \ \ \isacommand{assume}\isamarkupfalse%
\ {\isachardoublequoteopen}y\ {\isasymnoteq}\ x{\isachardoublequoteclose}\isanewline
\ \ \ \ \ \ \isacommand{then}\isamarkupfalse%
\ \isacommand{have}\isamarkupfalse%
\ {\isachardoublequoteopen}y\ {\isasymin}\ F{\isachardoublequoteclose}\ \isacommand{using}\isamarkupfalse%
\ {\isacharbackquoteopen}y\ {\isasymin}\ insert\ x\ F{\isacharbackquoteclose}\ \isacommand{by}\isamarkupfalse%
\ simp\isanewline
\ \ \ \ \ \ \isacommand{then}\isamarkupfalse%
\ \isacommand{have}\isamarkupfalse%
\ {\isachardoublequoteopen}y\ {\isasymle}\ sumaConj\ F{\isachardoublequoteclose}\ \isacommand{using}\isamarkupfalse%
\ HI\ \isacommand{by}\isamarkupfalse%
\ simp\isanewline
\ \ \ \ \ \ \isacommand{also}\isamarkupfalse%
\ \isacommand{have}\isamarkupfalse%
\ {\isachardoublequoteopen}{\isasymdots}\ {\isasymle}\ x\ {\isacharplus}\ {\isacharparenleft}sumaConj\ F{\isacharparenright}{\isachardoublequoteclose}\ \isacommand{by}\isamarkupfalse%
\ simp\isanewline
\ \ \ \ \ \ \isacommand{also}\isamarkupfalse%
\ \isacommand{have}\isamarkupfalse%
\ {\isachardoublequoteopen}{\isasymdots}\ {\isacharequal}\ sumaConj\ {\isacharparenleft}insert\ x\ F{\isacharparenright}{\isachardoublequoteclose}\ \isacommand{using}\isamarkupfalse%
\ fF\ xF\isanewline
\ \ \ \ \ \ \ \ \isacommand{by}\isamarkupfalse%
\ {\isacharparenleft}simp\ add{\isacharcolon}\ sumaConj{\isacharunderscore}def{\isacharparenright}\isanewline
\ \ \ \ \ \ \isacommand{finally}\isamarkupfalse%
\ \isacommand{show}\isamarkupfalse%
\ {\isacharquery}thesis\ \isacommand{{\isachardot}}\isamarkupfalse%
\isanewline
\ \ \ \ \isacommand{qed}\isamarkupfalse%
\isanewline
\ \ \isacommand{qed}\isamarkupfalse%
\isanewline
\isacommand{qed}\isamarkupfalse%
%
\endisatagproof
{\isafoldproof}%
%
\isadelimproof
%
\endisadelimproof
%
\begin{isamarkuptext}%
En esta última demostración hemos usado el método de prueba por
 casos,el método blast y también el simp("simplificador") añadiéndole 
\isa{sumaConj{\isacharunderscore}def}.%
\end{isamarkuptext}\isamarkuptrue%
%
\isadelimtheory
%
\endisadelimtheory
%
\isatagtheory
%
\endisatagtheory
{\isafoldtheory}%
%
\isadelimtheory
%
\endisadelimtheory
%
\end{isabellebody}%
\endinput
%:%file=~/Escritorio/TFG/ConjuntosFinitos.thy%:%
%:%6=1%:%
%:%20=9%:%
%:%24=11%:%
%:%28=13%:%
%:%37=15%:%
%:%49=18%:%
%:%50=19%:%
%:%51=20%:%
%:%52=21%:%
%:%53=22%:%
%:%54=23%:%
%:%55=24%:%
%:%56=25%:%
%:%57=26%:%
%:%58=27%:%
%:%59=28%:%
%:%60=29%:%
%:%61=30%:%
%:%62=31%:%
%:%63=32%:%
%:%64=33%:%
%:%65=34%:%
%:%66=35%:%
%:%67=36%:%
%:%68=37%:%
%:%69=38%:%
%:%70=39%:%
%:%71=40%:%
%:%72=41%:%
%:%73=42%:%
%:%74=43%:%
%:%75=44%:%
%:%76=45%:%
%:%77=46%:%
%:%78=47%:%
%:%79=48%:%
%:%80=49%:%
%:%81=50%:%
%:%82=51%:%
%:%83=52%:%
%:%84=53%:%
%:%85=54%:%
%:%86=55%:%
%:%87=56%:%
%:%88=57%:%
%:%89=58%:%
%:%90=59%:%
%:%91=60%:%
%:%92=61%:%
%:%93=62%:%
%:%94=63%:%
%:%95=64%:%
%:%96=65%:%
%:%97=66%:%
%:%98=67%:%
%:%99=68%:%
%:%108=71%:%
%:%120=73%:%
%:%121=74%:%
%:%122=75%:%
%:%123=76%:%
%:%125=79%:%
%:%126=79%:%
%:%127=80%:%
%:%129=82%:%
%:%131=85%:%
%:%132=85%:%
%:%135=86%:%
%:%136=87%:%
%:%140=87%:%
%:%150=89%:%
%:%152=91%:%
%:%153=91%:%
%:%154=92%:%
%:%157=93%:%
%:%161=93%:%
%:%162=93%:%
%:%171=96%:%
%:%172=97%:%
%:%173=98%:%
%:%174=99%:%
%:%175=100%:%
%:%176=101%:%
%:%177=102%:%
%:%178=103%:%
%:%179=104%:%
%:%180=105%:%
%:%181=106%:%
%:%182=107%:%
%:%183=108%:%
%:%192=111%:%
%:%204=113%:%
%:%206=115%:%
%:%207=115%:%
%:%210=116%:%
%:%214=116%:%
%:%215=116%:%
%:%216=117%:%
%:%217=117%:%
%:%218=118%:%
%:%219=118%:%
%:%220=119%:%
%:%230=121%:%
%:%231=122%:%
%:%232=123%:%
%:%233=124%:%
%:%234=125%:%
%:%243=128%:%
%:%255=130%:%
%:%257=132%:%
%:%258=132%:%
%:%261=133%:%
%:%265=133%:%
%:%266=133%:%
%:%267=134%:%
%:%281=136%:%
%:%293=138%:%
%:%295=140%:%
%:%296=140%:%
%:%297=141%:%
%:%304=142%:%
%:%305=142%:%
%:%306=143%:%
%:%307=143%:%
%:%308=143%:%
%:%309=144%:%
%:%310=144%:%
%:%311=145%:%
%:%312=145%:%
%:%313=146%:%
%:%314=146%:%
%:%315=147%:%
%:%316=148%:%
%:%317=149%:%
%:%318=149%:%
%:%319=150%:%
%:%320=150%:%
%:%321=151%:%
%:%322=151%:%
%:%323=152%:%
%:%324=152%:%
%:%325=153%:%
%:%326=153%:%
%:%327=154%:%
%:%328=154%:%
%:%329=155%:%
%:%330=155%:%
%:%331=156%:%
%:%332=156%:%
%:%333=156%:%
%:%334=156%:%
%:%335=157%:%
%:%336=157%:%
%:%337=157%:%
%:%338=158%:%
%:%339=158%:%
%:%340=159%:%
%:%341=159%:%
%:%342=159%:%
%:%343=159%:%
%:%344=160%:%
%:%345=160%:%
%:%346=161%:%
%:%347=161%:%
%:%348=162%:%
%:%349=162%:%
%:%350=162%:%
%:%351=162%:%
%:%352=162%:%
%:%353=163%:%
%:%354=163%:%
%:%355=163%:%
%:%356=163%:%
%:%357=163%:%
%:%358=164%:%
%:%359=164%:%
%:%360=164%:%
%:%361=164%:%
%:%362=165%:%
%:%363=165%:%
%:%364=165%:%
%:%365=165%:%
%:%366=166%:%
%:%367=166%:%
%:%368=167%:%
%:%369=167%:%
%:%370=167%:%
%:%371=167%:%
%:%372=168%:%
%:%373=168%:%
%:%374=169%:%
%:%375=169%:%
%:%376=170%:%
%:%386=173%:%
%:%387=174%:%
%:%388=175%:%
\chapter{Teoría de funciones}
%
\begin{isabellebody}%
\setisabellecontext{CancelacionInyectiva}%
%
\isadelimtheory
%
\endisadelimtheory
%
\isatagtheory
%
\endisatagtheory
{\isafoldtheory}%
%
\isadelimtheory
%
\endisadelimtheory
%
\isadelimdocument
%
\endisadelimdocument
%
\isatagdocument
%
\isamarkupsection{Cancelación de funciones inyectivas%
}
\isamarkuptrue%
%
\endisatagdocument
{\isafolddocument}%
%
\isadelimdocument
%
\endisadelimdocument
%
\begin{isamarkuptext}%
El siguiente teorema prueba una caracterización de las funciones
 inyectivas, en otras palabras, las funciones inyectivas son
 monomorfismos en la categoría de conjuntos. Un monomorfismo es un
 homomorfismo inyectivo y la categoría de conjuntos es la categoría
 cuyos objetos son los conjuntos.
  
  \begin{teorema}
    $f$ es una función inyectiva, si y solo si, para todas $g$ y $h$
    tales que \isa{f\ {\isasymcirc}\ g\ {\isacharequal}\ f\ {\isasymcirc}\ h} se tiene que $g = h$. 
  \end{teorema}

Vamos a hacer dos lemas de nuestro teorema, ya que podemos la doble 
implicación en dos implicaciones y demostrar cada una de ellas por
 separado.

\begin {lema}
$f$ es una función inyectiva si para todas $g$ y $h$ tales que $f \circ
 g = f \circ h$ se tiene que $g = h.$
\end {lema}
  \begin{demostracion}
    La demostración la haremos por doble implicación: 
\begin {enumerate}
\item Supongamos que tenemos que $f \circ g = f \circ h$, queremos
 demostrar que $g = h$, usando que f es inyectiva tenemos que: \\
$$(f \circ g)(x) = (f \circ h)(x) \Longrightarrow f(g(x)) = f(h(x)) = 
g(x) = h(x)$$
\item Supongamos ahora que $g = h$, queremos demostrar que  $f \circ g
 = f \circ h$. \\
$$(f \circ g)(x) = f(g(x)) = f(h(x)) = (f \circ h)(x)$$
\end {enumerate}
.
  \end{demostracion}

\begin {lema} 
Si para toda $g$ y $h$ tales que $f \circ g =  f \circ h$ se tiene que $g
= h$ entonces f es inyectiva.
\end {lema} 

\begin {demostracion}
Supongamos que el dominio de nuestra función $f$ es distinto del vacío.
Tenemos que demostrar que $\forall a,b$ tales que $f(a) = f(b),$ esto
 implica que $a = b.$ \\
Sean $a,b$ tales que $f(a) = f(b)$, sean ahora $g(x) = a \forall x$ y
 $h(x) = b \forall x$ entonces 
$$(f \circ g) = (f \circ h) \Longrightarrow  f(g(x)) = f(h(x)) \Longrightarrow f(a) = f(b)$$
Por hipótesis tenemos entonces que $a = b,$ como queríamos demostrar.
\end {demostracion}


  Su especificación es la siguiente, pero al igual que hemos hecho en la demostración
a mano vamos a demostrarlo a través de dos lemas:%
\end{isamarkuptext}\isamarkuptrue%
\isacommand{theorem}\isamarkupfalse%
\ \isanewline
\ \ {\isachardoublequoteopen}inj\ f\ {\isasymlongleftrightarrow}\ {\isacharparenleft}f\ {\isasymcirc}\ g\ {\isacharequal}\ f\ {\isasymcirc}\ h{\isacharparenright}\ {\isacharequal}\ {\isacharparenleft}g\ {\isacharequal}\ h{\isacharparenright}{\isachardoublequoteclose}\isanewline
%
\isadelimproof
\ \ %
\endisadelimproof
%
\isatagproof
\isacommand{oops}\isamarkupfalse%
%
\endisatagproof
{\isafoldproof}%
%
\isadelimproof
%
\endisadelimproof
%
\begin{isamarkuptext}%
Sus lemas son los siguientes:%
\end{isamarkuptext}\isamarkuptrue%
\isacommand{lemma}\isamarkupfalse%
\ \isanewline
{\isachardoublequoteopen}{\isasymforall}g\ h{\isachardot}\ {\isacharparenleft}f\ {\isasymcirc}\ g\ {\isacharequal}\ f\ {\isasymcirc}\ h\ {\isasymlongrightarrow}\ g\ {\isacharequal}\ h{\isacharparenright}\ {\isasymLongrightarrow}\ inj\ f{\isachardoublequoteclose}\isanewline
%
\isadelimproof
\ \ %
\endisadelimproof
%
\isatagproof
\isacommand{oops}\isamarkupfalse%
%
\endisatagproof
{\isafoldproof}%
%
\isadelimproof
\isanewline
%
\endisadelimproof
\isanewline
\isacommand{lemma}\isamarkupfalse%
\ \isanewline
{\isachardoublequoteopen}inj\ f\ {\isasymLongrightarrow}\ {\isacharparenleft}f\ {\isasymcirc}\ g\ {\isacharequal}\ f\ {\isasymcirc}\ h{\isacharparenright}\ {\isacharequal}\ {\isacharparenleft}g\ {\isacharequal}\ h{\isacharparenright}{\isachardoublequoteclose}\isanewline
%
\isadelimproof
\ \ %
\endisadelimproof
%
\isatagproof
\isacommand{oops}\isamarkupfalse%
%
\endisatagproof
{\isafoldproof}%
%
\isadelimproof
%
\endisadelimproof
%
\begin{isamarkuptext}%
En la especificación anterior, \isa{inj\ f} es una 
  abreviatura de \isa{inj\ f} definida en la teoría
  \href{http://bit.ly/2XuPQx5}{Fun.thy}. Además, contiene la definición
  de \isa{inj{\isacharunderscore}on}
  \begin{itemize}
    \item[] \isa{inj{\isacharunderscore}on\ f\ A\ {\isacharequal}\ {\isacharparenleft}{\isasymforall}x{\isasymin}A{\isachardot}\ {\isasymforall}y{\isasymin}A{\isachardot}\ f\ x\ {\isacharequal}\ f\ y\ {\isasymlongrightarrow}\ x\ {\isacharequal}\ y{\isacharparenright}} \hfill (\isa{inj{\isacharunderscore}on{\isacharunderscore}def})
  \end{itemize} 
  Por su parte, \isa{UNIV} es el conjunto universal definido en la 
  teoría \href{http://bit.ly/2XtHCW6}{Set.thy} como una abreviatura de 
  \isa{top} que, a su vez está definido en la teoría 
  \href{http://bit.ly/2Xyj9Pe}{Orderings.thy} mediante la siguiente
  propiedad 
  \begin{itemize}
    \item[] \isa{\mbox{}\inferrule{\mbox{ordering{\isacharunderscore}top\ less{\isacharunderscore}eq\ less\ top}}{\mbox{less{\isacharunderscore}eq\ a\ top}}} 
      \hfill (\isa{ordering{\isacharunderscore}top{\isachardot}extremum})
  \end{itemize} 
  En el caso de la teoría de conjuntos, la relación de orden es la
  inclusión de conjuntos.

  Presentaremos distintas demostraciones de los lemas. La primera
  demostración es applicativa:%
\end{isamarkuptext}\isamarkuptrue%
\isacommand{lemma}\isamarkupfalse%
\ inyectivapli{\isacharcolon}\isanewline
\ \ {\isachardoublequoteopen}inj\ f\ {\isasymLongrightarrow}\ {\isacharparenleft}f\ {\isasymcirc}\ g\ {\isacharequal}\ f\ {\isasymcirc}\ h{\isacharparenright}\ {\isacharequal}\ {\isacharparenleft}g\ {\isacharequal}\ h{\isacharparenright}{\isachardoublequoteclose}\isanewline
%
\isadelimproof
\ \ %
\endisadelimproof
%
\isatagproof
\isacommand{apply}\isamarkupfalse%
\ {\isacharparenleft}simp\ add{\isacharcolon}\ inj{\isacharunderscore}on{\isacharunderscore}def\ fun{\isacharunderscore}eq{\isacharunderscore}iff{\isacharparenright}\ \isanewline
\ \ \isacommand{apply}\isamarkupfalse%
\ auto\isanewline
\ \ \isacommand{done}\isamarkupfalse%
%
\endisatagproof
{\isafoldproof}%
%
\isadelimproof
\ \isanewline
%
\endisadelimproof
\isanewline
\isacommand{lemma}\isamarkupfalse%
\ inyectivapli{\isadigit{2}}{\isacharcolon}\isanewline
{\isachardoublequoteopen}{\isasymforall}g\ h{\isachardot}\ {\isacharparenleft}f\ {\isasymcirc}\ g\ {\isacharequal}\ f\ {\isasymcirc}\ h\ {\isasymlongrightarrow}\ g\ {\isacharequal}\ h{\isacharparenright}\ {\isasymLongrightarrow}\ inj\ f{\isachardoublequoteclose}\isanewline
%
\isadelimproof
\ \ %
\endisadelimproof
%
\isatagproof
\isacommand{apply}\isamarkupfalse%
\ {\isacharparenleft}rule\ injI{\isacharparenright}\isanewline
\ \ \isacommand{by}\isamarkupfalse%
\ {\isacharparenleft}metis\ fun{\isacharunderscore}upd{\isacharunderscore}apply\ fun{\isacharunderscore}upd{\isacharunderscore}comp{\isacharparenright}%
\endisatagproof
{\isafoldproof}%
%
\isadelimproof
%
\endisadelimproof
%
\begin{isamarkuptext}%
En las demostraciones anteriores se han usado los siguientes
 lemas:
  \begin{itemize}
    \item[] \isa{{\isacharparenleft}f\ {\isacharequal}\ g{\isacharparenright}\ {\isacharequal}\ {\isacharparenleft}{\isasymforall}x{\isachardot}\ f\ x\ {\isacharequal}\ g\ x{\isacharparenright}} 
      \hfill (\isa{fun{\isacharunderscore}eq{\isacharunderscore}iff})
  \end{itemize} 
  \begin{itemize}
    \item[] \isa{{\isacharparenleft}f{\isacharparenleft}x\ {\isacharcolon}{\isacharequal}\ y{\isacharparenright}{\isacharparenright}\ z\ {\isacharequal}\ {\isacharparenleft}\textsf{if}\ z\ {\isacharequal}\ x\ \textsf{then}\ y\ \textsf{else}\ f\ z{\isacharparenright}} 
      \hfill (\isa{fun{\isacharunderscore}upd{\isacharunderscore}apply})
  \end{itemize} 
  \begin{itemize}
    \item[] \isa{{\isacharparenleft}f\ {\isacharequal}\ g{\isacharparenright}\ {\isacharequal}\ {\isacharparenleft}{\isasymforall}x{\isachardot}\ f\ x\ {\isacharequal}\ g\ x{\isacharparenright}} 
      \hfill (\isa{fun{\isacharunderscore}upd{\isacharunderscore}comp})
  \end{itemize} 

  La demostración applicativa sin auto es%
\end{isamarkuptext}\isamarkuptrue%
\isacommand{lemma}\isamarkupfalse%
\isanewline
\ \ {\isachardoublequoteopen}inj\ f\ {\isasymLongrightarrow}\ {\isacharparenleft}f\ {\isasymcirc}\ g\ {\isacharequal}\ f\ {\isasymcirc}\ h{\isacharparenright}\ {\isacharequal}\ {\isacharparenleft}g\ {\isacharequal}\ h{\isacharparenright}{\isachardoublequoteclose}\isanewline
%
\isadelimproof
\ \ %
\endisadelimproof
%
\isatagproof
\isacommand{apply}\isamarkupfalse%
\ {\isacharparenleft}unfold\ inj{\isacharunderscore}on{\isacharunderscore}def{\isacharparenright}\ \isanewline
\ \ \isacommand{apply}\isamarkupfalse%
\ {\isacharparenleft}unfold\ fun{\isacharunderscore}eq{\isacharunderscore}iff{\isacharparenright}\ \isanewline
\ \ \isacommand{apply}\isamarkupfalse%
\ {\isacharparenleft}unfold\ o{\isacharunderscore}apply{\isacharparenright}\isanewline
\ \ \isacommand{apply}\isamarkupfalse%
\ {\isacharparenleft}rule\ iffI{\isacharparenright}\isanewline
\ \ \ \isacommand{apply}\isamarkupfalse%
\ simp{\isacharplus}\isanewline
\ \ \isacommand{done}\isamarkupfalse%
%
\endisatagproof
{\isafoldproof}%
%
\isadelimproof
%
\endisadelimproof
%
\begin{isamarkuptext}%
En la demostración anterior se ha introducido los siguientes
  hechos
  \begin{itemize}
    \item \isa{{\isacharparenleft}f\ {\isasymcirc}\ g{\isacharparenright}\ x\ {\isacharequal}\ f\ {\isacharparenleft}g\ x{\isacharparenright}} \hfill (\isa{o{\isacharunderscore}apply})
    \item \isa{{\isasymlbrakk}P\ {\isasymLongrightarrow}\ Q{\isacharsemicolon}\ Q\ {\isasymLongrightarrow}\ P{\isasymrbrakk}\ {\isasymLongrightarrow}\ P\ {\isacharequal}\ Q} \hfill (\isa{iffI})
  \end{itemize} 

  La demostración automática es%
\end{isamarkuptext}\isamarkuptrue%
\isacommand{lemma}\isamarkupfalse%
\isanewline
\ \ \isakeyword{assumes}\ {\isachardoublequoteopen}inj\ f{\isachardoublequoteclose}\isanewline
\ \ \isakeyword{shows}\ {\isachardoublequoteopen}{\isacharparenleft}f\ {\isasymcirc}\ g\ {\isacharequal}\ f\ {\isasymcirc}\ h{\isacharparenright}\ {\isacharequal}\ {\isacharparenleft}g\ {\isacharequal}\ h{\isacharparenright}{\isachardoublequoteclose}\isanewline
%
\isadelimproof
\ \ %
\endisadelimproof
%
\isatagproof
\isacommand{using}\isamarkupfalse%
\ assms\isanewline
\ \ \isacommand{by}\isamarkupfalse%
\ {\isacharparenleft}auto\ simp\ add{\isacharcolon}\ inj{\isacharunderscore}on{\isacharunderscore}def\ fun{\isacharunderscore}eq{\isacharunderscore}iff{\isacharparenright}%
\endisatagproof
{\isafoldproof}%
%
\isadelimproof
%
\endisadelimproof
%
\begin{isamarkuptext}%
La demostración declarativa%
\end{isamarkuptext}\isamarkuptrue%
\isacommand{lemma}\isamarkupfalse%
\isanewline
\ \ \isakeyword{assumes}\ {\isachardoublequoteopen}inj\ f{\isachardoublequoteclose}\isanewline
\ \ \isakeyword{shows}\ {\isachardoublequoteopen}{\isacharparenleft}f\ {\isasymcirc}\ g\ {\isacharequal}\ f\ {\isasymcirc}\ h{\isacharparenright}\ {\isacharequal}\ {\isacharparenleft}g\ {\isacharequal}\ h{\isacharparenright}{\isachardoublequoteclose}\isanewline
%
\isadelimproof
%
\endisadelimproof
%
\isatagproof
\isacommand{proof}\isamarkupfalse%
\ \isanewline
\ \ \isacommand{assume}\isamarkupfalse%
\ {\isachardoublequoteopen}f\ {\isasymcirc}\ g\ {\isacharequal}\ f\ {\isasymcirc}\ h{\isachardoublequoteclose}\isanewline
\ \ \isacommand{show}\isamarkupfalse%
\ {\isachardoublequoteopen}g\ {\isacharequal}\ h{\isachardoublequoteclose}\isanewline
\ \ \isacommand{proof}\isamarkupfalse%
\isanewline
\ \ \ \ \isacommand{fix}\isamarkupfalse%
\ x\isanewline
\ \ \ \ \isacommand{have}\isamarkupfalse%
\ {\isachardoublequoteopen}{\isacharparenleft}f\ {\isasymcirc}\ g{\isacharparenright}{\isacharparenleft}x{\isacharparenright}\ {\isacharequal}\ {\isacharparenleft}f\ {\isasymcirc}\ h{\isacharparenright}{\isacharparenleft}x{\isacharparenright}{\isachardoublequoteclose}\ \isacommand{using}\isamarkupfalse%
\ {\isacharbackquoteopen}f\ {\isasymcirc}\ g\ {\isacharequal}\ f\ {\isasymcirc}\ h{\isacharbackquoteclose}\ \isacommand{by}\isamarkupfalse%
\ simp\isanewline
\ \ \ \ \isacommand{then}\isamarkupfalse%
\ \isacommand{have}\isamarkupfalse%
\ {\isachardoublequoteopen}f{\isacharparenleft}g{\isacharparenleft}x{\isacharparenright}{\isacharparenright}\ {\isacharequal}\ f{\isacharparenleft}h{\isacharparenleft}x{\isacharparenright}{\isacharparenright}{\isachardoublequoteclose}\ \isacommand{by}\isamarkupfalse%
\ simp\isanewline
\ \ \ \ \isacommand{then}\isamarkupfalse%
\ \isacommand{show}\isamarkupfalse%
\ {\isachardoublequoteopen}g{\isacharparenleft}x{\isacharparenright}\ {\isacharequal}\ h{\isacharparenleft}x{\isacharparenright}{\isachardoublequoteclose}\ \isacommand{using}\isamarkupfalse%
\ {\isacharbackquoteopen}inj\ f{\isacharbackquoteclose}\ \isacommand{by}\isamarkupfalse%
\ {\isacharparenleft}simp\ add{\isacharcolon}inj{\isacharunderscore}on{\isacharunderscore}def{\isacharparenright}\isanewline
\ \ \isacommand{qed}\isamarkupfalse%
\isanewline
\isacommand{next}\isamarkupfalse%
\isanewline
\ \ \isacommand{assume}\isamarkupfalse%
\ {\isachardoublequoteopen}g\ {\isacharequal}\ h{\isachardoublequoteclose}\isanewline
\ \ \isacommand{show}\isamarkupfalse%
\ {\isachardoublequoteopen}f\ {\isasymcirc}\ g\ {\isacharequal}\ f\ {\isasymcirc}\ h{\isachardoublequoteclose}\isanewline
\ \ \isacommand{proof}\isamarkupfalse%
\isanewline
\ \ \ \ \isacommand{fix}\isamarkupfalse%
\ x\isanewline
\ \ \ \ \isacommand{have}\isamarkupfalse%
\ {\isachardoublequoteopen}{\isacharparenleft}f\ {\isasymcirc}\ g{\isacharparenright}\ x\ {\isacharequal}\ f{\isacharparenleft}g{\isacharparenleft}x{\isacharparenright}{\isacharparenright}{\isachardoublequoteclose}\ \isacommand{by}\isamarkupfalse%
\ simp\isanewline
\ \ \ \ \isacommand{also}\isamarkupfalse%
\ \isacommand{have}\isamarkupfalse%
\ {\isachardoublequoteopen}{\isasymdots}\ {\isacharequal}\ f{\isacharparenleft}h{\isacharparenleft}x{\isacharparenright}{\isacharparenright}{\isachardoublequoteclose}\ \isacommand{using}\isamarkupfalse%
\ {\isacharbackquoteopen}g\ {\isacharequal}\ h{\isacharbackquoteclose}\ \isacommand{by}\isamarkupfalse%
\ simp\isanewline
\ \ \ \ \isacommand{also}\isamarkupfalse%
\ \isacommand{have}\isamarkupfalse%
\ {\isachardoublequoteopen}{\isasymdots}\ {\isacharequal}\ {\isacharparenleft}f\ {\isasymcirc}\ h{\isacharparenright}\ x{\isachardoublequoteclose}\ \isacommand{by}\isamarkupfalse%
\ simp\isanewline
\ \ \ \ \isacommand{finally}\isamarkupfalse%
\ \isacommand{show}\isamarkupfalse%
\ {\isachardoublequoteopen}{\isacharparenleft}f\ {\isasymcirc}\ g{\isacharparenright}\ x\ {\isacharequal}\ {\isacharparenleft}f\ {\isasymcirc}\ h{\isacharparenright}\ x{\isachardoublequoteclose}\ \isacommand{by}\isamarkupfalse%
\ simp\isanewline
\ \ \isacommand{qed}\isamarkupfalse%
\isanewline
\isacommand{qed}\isamarkupfalse%
%
\endisatagproof
{\isafoldproof}%
%
\isadelimproof
%
\endisadelimproof
%
\begin{isamarkuptext}%
Otra demostración declarativa es%
\end{isamarkuptext}\isamarkuptrue%
\isacommand{lemma}\isamarkupfalse%
\ \isanewline
\ \ \isakeyword{assumes}\ {\isachardoublequoteopen}inj\ f{\isachardoublequoteclose}\isanewline
\ \ \isakeyword{shows}\ {\isachardoublequoteopen}{\isacharparenleft}f\ {\isasymcirc}\ g\ {\isacharequal}\ f\ {\isasymcirc}\ h{\isacharparenright}\ {\isacharequal}\ {\isacharparenleft}g\ {\isacharequal}\ h{\isacharparenright}{\isachardoublequoteclose}\isanewline
%
\isadelimproof
%
\endisadelimproof
%
\isatagproof
\isacommand{proof}\isamarkupfalse%
\ \isanewline
\ \ \isacommand{assume}\isamarkupfalse%
\ {\isachardoublequoteopen}f\ {\isasymcirc}\ g\ {\isacharequal}\ f\ {\isasymcirc}\ h{\isachardoublequoteclose}\ \isanewline
\ \ \isacommand{then}\isamarkupfalse%
\ \isacommand{show}\isamarkupfalse%
\ {\isachardoublequoteopen}g\ {\isacharequal}\ h{\isachardoublequoteclose}\ \isacommand{using}\isamarkupfalse%
\ {\isacharbackquoteopen}inj\ f{\isacharbackquoteclose}\ \isacommand{by}\isamarkupfalse%
\ {\isacharparenleft}simp\ add{\isacharcolon}\ inj{\isacharunderscore}on{\isacharunderscore}def\ fun{\isacharunderscore}eq{\isacharunderscore}iff{\isacharparenright}\ \isanewline
\isacommand{next}\isamarkupfalse%
\isanewline
\ \ \isacommand{assume}\isamarkupfalse%
\ {\isachardoublequoteopen}g\ {\isacharequal}\ h{\isachardoublequoteclose}\ \isanewline
\ \ \isacommand{then}\isamarkupfalse%
\ \isacommand{show}\isamarkupfalse%
\ {\isachardoublequoteopen}f\ {\isasymcirc}\ g\ {\isacharequal}\ f\ {\isasymcirc}\ h{\isachardoublequoteclose}\ \isacommand{by}\isamarkupfalse%
\ simp\isanewline
\isacommand{qed}\isamarkupfalse%
%
\endisatagproof
{\isafoldproof}%
%
\isadelimproof
%
\endisadelimproof
%
\begin{isamarkuptext}%
En consecuencia, la demostración de nuestro teorema:%
\end{isamarkuptext}\isamarkuptrue%
\isacommand{theorem}\isamarkupfalse%
\ \isanewline
{\isachardoublequoteopen}{\isasymforall}g\ h{\isachardot}\ {\isacharparenleft}f\ {\isasymcirc}\ g\ {\isacharequal}\ f\ {\isasymcirc}\ h\ {\isasymlongrightarrow}\ g\ {\isacharequal}\ h{\isacharparenright}\ {\isasymlongleftrightarrow}\ inj\ f{\isachardoublequoteclose}\isanewline
%
\isadelimproof
\ \ %
\endisadelimproof
%
\isatagproof
\isacommand{oops}\isamarkupfalse%
\isanewline
%
\endisatagproof
{\isafoldproof}%
%
\isadelimproof
%
\endisadelimproof
%
\isadelimtheory
%
\endisadelimtheory
%
\isatagtheory
%
\endisatagtheory
{\isafoldtheory}%
%
\isadelimtheory
%
\endisadelimtheory
%
\end{isabellebody}%
\endinput
%:%file=~/ownCloud/alonso/curso-TFG/Carlos/TFG/CancelacionInyectiva.thy%:%
%:%24=8%:%
%:%36=10%:%
%:%37=11%:%
%:%38=12%:%
%:%39=13%:%
%:%40=14%:%
%:%41=15%:%
%:%42=16%:%
%:%43=17%:%
%:%44=18%:%
%:%45=19%:%
%:%46=20%:%
%:%47=21%:%
%:%48=22%:%
%:%49=23%:%
%:%50=24%:%
%:%51=25%:%
%:%52=26%:%
%:%53=27%:%
%:%54=28%:%
%:%55=29%:%
%:%56=30%:%
%:%57=31%:%
%:%58=32%:%
%:%59=33%:%
%:%60=34%:%
%:%61=35%:%
%:%62=36%:%
%:%63=37%:%
%:%64=38%:%
%:%65=39%:%
%:%66=40%:%
%:%67=41%:%
%:%68=42%:%
%:%69=43%:%
%:%70=44%:%
%:%71=45%:%
%:%72=46%:%
%:%73=47%:%
%:%74=48%:%
%:%75=49%:%
%:%76=50%:%
%:%77=51%:%
%:%78=52%:%
%:%79=53%:%
%:%80=54%:%
%:%81=55%:%
%:%82=56%:%
%:%83=57%:%
%:%84=58%:%
%:%85=59%:%
%:%86=60%:%
%:%88=63%:%
%:%89=63%:%
%:%90=64%:%
%:%93=65%:%
%:%97=65%:%
%:%107=68%:%
%:%109=70%:%
%:%110=70%:%
%:%111=71%:%
%:%114=72%:%
%:%118=72%:%
%:%124=72%:%
%:%127=73%:%
%:%128=74%:%
%:%129=74%:%
%:%130=75%:%
%:%133=76%:%
%:%137=76%:%
%:%147=79%:%
%:%148=80%:%
%:%149=81%:%
%:%150=82%:%
%:%151=83%:%
%:%152=84%:%
%:%153=85%:%
%:%154=86%:%
%:%155=87%:%
%:%156=88%:%
%:%157=89%:%
%:%158=90%:%
%:%159=91%:%
%:%160=92%:%
%:%161=93%:%
%:%162=94%:%
%:%163=95%:%
%:%164=96%:%
%:%165=97%:%
%:%166=98%:%
%:%167=99%:%
%:%169=101%:%
%:%170=101%:%
%:%171=102%:%
%:%174=103%:%
%:%178=103%:%
%:%179=103%:%
%:%180=104%:%
%:%181=104%:%
%:%182=105%:%
%:%188=105%:%
%:%191=106%:%
%:%192=107%:%
%:%193=107%:%
%:%194=108%:%
%:%197=109%:%
%:%201=109%:%
%:%202=109%:%
%:%203=110%:%
%:%204=110%:%
%:%213=113%:%
%:%214=114%:%
%:%215=115%:%
%:%216=116%:%
%:%217=117%:%
%:%218=118%:%
%:%219=119%:%
%:%220=120%:%
%:%221=121%:%
%:%222=122%:%
%:%223=123%:%
%:%224=124%:%
%:%225=125%:%
%:%226=126%:%
%:%227=127%:%
%:%228=128%:%
%:%230=130%:%
%:%231=130%:%
%:%232=131%:%
%:%235=132%:%
%:%239=132%:%
%:%240=132%:%
%:%241=133%:%
%:%242=133%:%
%:%243=134%:%
%:%244=134%:%
%:%245=135%:%
%:%246=135%:%
%:%247=136%:%
%:%248=136%:%
%:%249=137%:%
%:%259=139%:%
%:%260=140%:%
%:%261=141%:%
%:%262=142%:%
%:%263=143%:%
%:%264=144%:%
%:%265=145%:%
%:%266=146%:%
%:%268=148%:%
%:%269=148%:%
%:%270=149%:%
%:%271=150%:%
%:%274=151%:%
%:%278=151%:%
%:%279=151%:%
%:%280=152%:%
%:%281=152%:%
%:%290=154%:%
%:%292=156%:%
%:%293=156%:%
%:%294=157%:%
%:%295=158%:%
%:%302=159%:%
%:%303=159%:%
%:%304=160%:%
%:%305=160%:%
%:%306=161%:%
%:%307=161%:%
%:%308=162%:%
%:%309=162%:%
%:%310=163%:%
%:%311=163%:%
%:%312=164%:%
%:%313=164%:%
%:%314=164%:%
%:%315=164%:%
%:%316=165%:%
%:%317=165%:%
%:%318=165%:%
%:%319=165%:%
%:%320=166%:%
%:%321=166%:%
%:%322=166%:%
%:%323=166%:%
%:%324=166%:%
%:%325=167%:%
%:%326=167%:%
%:%327=168%:%
%:%328=168%:%
%:%329=169%:%
%:%330=169%:%
%:%331=170%:%
%:%332=170%:%
%:%333=171%:%
%:%334=171%:%
%:%335=172%:%
%:%336=172%:%
%:%337=173%:%
%:%338=173%:%
%:%339=173%:%
%:%340=174%:%
%:%341=174%:%
%:%342=174%:%
%:%343=174%:%
%:%344=174%:%
%:%345=175%:%
%:%346=175%:%
%:%347=175%:%
%:%348=175%:%
%:%349=176%:%
%:%350=176%:%
%:%351=176%:%
%:%352=176%:%
%:%353=177%:%
%:%354=177%:%
%:%355=178%:%
%:%365=180%:%
%:%367=182%:%
%:%368=182%:%
%:%369=183%:%
%:%370=184%:%
%:%377=185%:%
%:%378=185%:%
%:%379=186%:%
%:%380=186%:%
%:%381=187%:%
%:%382=187%:%
%:%383=187%:%
%:%384=187%:%
%:%385=187%:%
%:%386=188%:%
%:%387=188%:%
%:%388=189%:%
%:%389=189%:%
%:%390=190%:%
%:%391=190%:%
%:%392=190%:%
%:%393=190%:%
%:%394=191%:%
%:%404=193%:%
%:%406=195%:%
%:%407=195%:%
%:%408=196%:%
%:%411=197%:%
%:%415=197%:%
%:%416=197%:%
%
\begin{isabellebody}%
\setisabellecontext{CancelacionSobreyectiva}%
%
\isadelimtheory
%
\endisadelimtheory
%
\isatagtheory
%
\endisatagtheory
{\isafoldtheory}%
%
\isadelimtheory
%
\endisadelimtheory
%
\isadelimdocument
%
\endisadelimdocument
%
\isatagdocument
%
\isamarkupsection{Cancelación de las funciones sobreyectivas%
}
\isamarkuptrue%
%
\endisatagdocument
{\isafolddocument}%
%
\isadelimdocument
%
\endisadelimdocument
%
\begin{isamarkuptext}%
El siguiente teorema prueba una propiedad de las funciones
 sobreyectivas. El enunciado es el siguiente: 
\begin {teorema}
Las funciones sobreyectivas son cancelativas por la derecha. Es decir,
 si f es sobreyectiva entonces para todas funciones g y h tal que g o f
 = h o f se tiene que g = h.
\end {teorema}
 
\begin {demostracion}
\begin {itemize}
\item Supongamos que tenemos que $g o f = h o f$, queremos probar que $g =
 h.$ Usando la definición de sobreyectividad $(\forall y \in Y,
 \exists x | y = f(x))$ y nuestra hipótesis, tenemos que:
$$g(y) = g(f(x)) = (g o f) (x) = (h o f) (x) = h(f(x)) = h(y)$$
\item Supongamos que $g = h$, hay que probar que $g o f = h o f.$ Usando
nuestra hipótesis, tenemos que:
$$ (g o f)(x) = g(f(x)) = h(f(x)) = (h o f) (x).$$
\end {itemize}
.
\end {demostracion}

Su especificación es la siguiente:%
\end{isamarkuptext}\isamarkuptrue%
\isacommand{lemma}\isamarkupfalse%
\ {\isachardoublequoteopen}surj\ f\ {\isasymLongrightarrow}\ {\isacharparenleft}\ g\ {\isasymo}\ f\ {\isacharequal}\ h\ {\isasymo}\ f\ {\isacharparenright}\ {\isacharequal}\ {\isacharparenleft}g\ {\isacharequal}\ h{\isacharparenright}{\isachardoublequoteclose}\isanewline
%
\isadelimproof
\ \ %
\endisadelimproof
%
\isatagproof
\isacommand{oops}\isamarkupfalse%
%
\endisatagproof
{\isafoldproof}%
%
\isadelimproof
%
\endisadelimproof
%
\begin{isamarkuptext}%
En la especificación anterior, \isa{surj\ f} es una abreviatura de 
  \isa{surj\ f}, donde \isa{range\ f} es el rango o imagen
de la función f.
 Por otra parte, \isa{UNIV} es el conjunto universal definido en la 
  teoría \href{http://bit.ly/2XtHCW6}{Set.thy} como una abreviatura de 
  \isa{top} que, a su vez está definido en la teoría 
  \href{http://bit.ly/2Xyj9Pe}{Orderings.thy} mediante la siguiente
  propiedad 
  \begin{itemize}
    \item[] \isa{\mbox{}\inferrule{\mbox{ordering{\isacharunderscore}top\ less{\isacharunderscore}eq\ less\ top}}{\mbox{less{\isacharunderscore}eq\ a\ top}}} 
      \hfill (\isa{ordering{\isacharunderscore}top{\isachardot}extremum})
  \end{itemize} 
Además queda añadir que la teoría donde se encuentra definido \isa{surj\ f}
 es en \href{http://bit.ly/2XuPQx5}{Fun.thy}. Esta teoría contiene la
 definicion \isa{surj{\isacharunderscore}def}.
 \begin{itemize}
    \item[] \isa{surj\ f\ {\isacharequal}\ {\isacharparenleft}{\isasymforall}y{\isachardot}\ {\isasymexists}x{\isachardot}\ y\ {\isacharequal}\ f\ x{\isacharparenright}} \hfill (\isa{inj{\isacharunderscore}on{\isacharunderscore}def})
  \end{itemize} 

Presentaremos distintas demostraciones del teorema. La primera es la
 detallada:%
\end{isamarkuptext}\isamarkuptrue%
\isacommand{lemma}\isamarkupfalse%
\ \isanewline
\ \ \isakeyword{assumes}\ {\isachardoublequoteopen}surj\ f{\isachardoublequoteclose}\ \isanewline
\ \ \isakeyword{shows}\ {\isachardoublequoteopen}{\isacharparenleft}\ g\ {\isasymcirc}\ f\ {\isacharequal}\ h\ {\isasymcirc}\ f\ {\isacharparenright}\ {\isacharequal}\ {\isacharparenleft}g\ {\isacharequal}\ h{\isacharparenright}{\isachardoublequoteclose}\isanewline
%
\isadelimproof
%
\endisadelimproof
%
\isatagproof
\isacommand{proof}\isamarkupfalse%
\ {\isacharparenleft}rule\ iffI{\isacharparenright}\isanewline
\ \ \isacommand{assume}\isamarkupfalse%
\ {\isadigit{1}}{\isacharcolon}\ {\isachardoublequoteopen}\ g\ {\isasymcirc}\ f\ {\isacharequal}\ h\ {\isasymcirc}\ f\ {\isachardoublequoteclose}\isanewline
\ \ \isacommand{show}\isamarkupfalse%
\ {\isachardoublequoteopen}g\ {\isacharequal}\ h{\isachardoublequoteclose}\ \isanewline
\ \ \isacommand{proof}\isamarkupfalse%
\ \isanewline
\ \ \ \ \isacommand{fix}\isamarkupfalse%
\ x\ \isanewline
\ \ \ \ \isacommand{have}\isamarkupfalse%
\ {\isachardoublequoteopen}\ {\isasymexists}y\ {\isachardot}\ x\ {\isacharequal}\ f{\isacharparenleft}y{\isacharparenright}{\isachardoublequoteclose}\ \isacommand{using}\isamarkupfalse%
\ assms\ \isacommand{by}\isamarkupfalse%
\ {\isacharparenleft}simp\ add{\isacharcolon}surj{\isacharunderscore}def{\isacharparenright}\isanewline
\ \ \ \ \isacommand{then}\isamarkupfalse%
\ \isacommand{obtain}\isamarkupfalse%
\ {\isachardoublequoteopen}y{\isachardoublequoteclose}\ \isakeyword{where}\ {\isadigit{2}}{\isacharcolon}{\isachardoublequoteopen}x\ {\isacharequal}\ f{\isacharparenleft}y{\isacharparenright}{\isachardoublequoteclose}\ \isacommand{by}\isamarkupfalse%
\ {\isacharparenleft}rule\ exE{\isacharparenright}\isanewline
\ \ \ \ \isacommand{then}\isamarkupfalse%
\ \isacommand{have}\isamarkupfalse%
\ {\isachardoublequoteopen}g{\isacharparenleft}x{\isacharparenright}\ {\isacharequal}\ g{\isacharparenleft}f{\isacharparenleft}y{\isacharparenright}{\isacharparenright}{\isachardoublequoteclose}\ \isacommand{by}\isamarkupfalse%
\ simp\isanewline
\ \ \ \ \isacommand{then}\isamarkupfalse%
\ \isacommand{have}\isamarkupfalse%
\ {\isachardoublequoteopen}{\isachardot}{\isachardot}{\isachardot}\ {\isacharequal}\ {\isacharparenleft}g\ {\isasymcirc}\ f{\isacharparenright}\ {\isacharparenleft}y{\isacharparenright}\ \ {\isachardoublequoteclose}\ \isacommand{by}\isamarkupfalse%
\ simp\isanewline
\ \ \ \ \isacommand{then}\isamarkupfalse%
\ \isacommand{have}\isamarkupfalse%
\ {\isachardoublequoteopen}{\isachardot}{\isachardot}{\isachardot}\ {\isacharequal}\ {\isacharparenleft}h\ o\ f{\isacharparenright}\ {\isacharparenleft}y{\isacharparenright}{\isachardoublequoteclose}\ \isacommand{using}\isamarkupfalse%
\ {\isadigit{1}}\ \isacommand{by}\isamarkupfalse%
\ simp\isanewline
\ \ \ \ \isacommand{then}\isamarkupfalse%
\ \isacommand{have}\isamarkupfalse%
\ {\isachardoublequoteopen}{\isachardot}{\isachardot}{\isachardot}\ {\isacharequal}\ h{\isacharparenleft}f{\isacharparenleft}y{\isacharparenright}{\isacharparenright}{\isachardoublequoteclose}\ \isacommand{by}\isamarkupfalse%
\ simp\isanewline
\ \ \ \ \isacommand{then}\isamarkupfalse%
\ \isacommand{have}\isamarkupfalse%
\ {\isachardoublequoteopen}{\isachardot}{\isachardot}{\isachardot}\ {\isacharequal}\ h{\isacharparenleft}x{\isacharparenright}{\isachardoublequoteclose}\ \isacommand{using}\isamarkupfalse%
\ {\isadigit{2}}\ \ \ \isacommand{by}\isamarkupfalse%
\ {\isacharparenleft}simp\ add{\isacharcolon}\ {\isacartoucheopen}x\ {\isacharequal}\ f\ y{\isacartoucheclose}{\isacharparenright}\isanewline
\ \ \ \ \isacommand{then}\isamarkupfalse%
\ \isacommand{show}\isamarkupfalse%
\ {\isachardoublequoteopen}\ g{\isacharparenleft}x{\isacharparenright}\ {\isacharequal}\ h{\isacharparenleft}x{\isacharparenright}\ {\isachardoublequoteclose}\ \isanewline
\ \ \ \ \ \ \isacommand{using}\isamarkupfalse%
\ {\isacartoucheopen}{\isacharparenleft}g\ {\isasymcirc}\ f{\isacharparenright}\ y\ {\isacharequal}\ {\isacharparenleft}h\ {\isasymcirc}\ f{\isacharparenright}\ y{\isacartoucheclose}\ {\isacartoucheopen}{\isacharparenleft}h\ {\isasymcirc}\ f{\isacharparenright}\ y\ {\isacharequal}\ h\ {\isacharparenleft}f\ y{\isacharparenright}{\isacartoucheclose}\isanewline
\ \ \ \ {\isacartoucheopen}g\ {\isacharparenleft}f\ y{\isacharparenright}\ {\isacharequal}\ {\isacharparenleft}g\ {\isasymcirc}\ f{\isacharparenright}\ y{\isacartoucheclose}\ {\isacartoucheopen}g\ x\ {\isacharequal}\ g\ {\isacharparenleft}f\ y{\isacharparenright}{\isacartoucheclose}\ {\isacartoucheopen}h\ {\isacharparenleft}f\ y{\isacharparenright}\ {\isacharequal}\ h\ x{\isacartoucheclose}\ \isacommand{by}\isamarkupfalse%
\ presburger\isanewline
\ \ \isacommand{qed}\isamarkupfalse%
\isanewline
\isacommand{next}\isamarkupfalse%
\isanewline
\ \ \isacommand{assume}\isamarkupfalse%
\ {\isachardoublequoteopen}g\ {\isacharequal}\ h{\isachardoublequoteclose}\ \isanewline
\ \ \isacommand{show}\isamarkupfalse%
\ {\isachardoublequoteopen}g\ {\isasymcirc}\ f\ {\isacharequal}\ h\ {\isasymcirc}\ f{\isachardoublequoteclose}\isanewline
\ \ \isacommand{proof}\isamarkupfalse%
\isanewline
\ \ \ \ \isacommand{fix}\isamarkupfalse%
\ x\isanewline
\ \ \ \ \isacommand{have}\isamarkupfalse%
\ {\isachardoublequoteopen}{\isacharparenleft}g\ {\isasymcirc}\ f{\isacharparenright}\ x\ {\isacharequal}\ g{\isacharparenleft}f{\isacharparenleft}x{\isacharparenright}{\isacharparenright}{\isachardoublequoteclose}\ \isacommand{by}\isamarkupfalse%
\ simp\isanewline
\ \ \ \ \isacommand{also}\isamarkupfalse%
\ \isacommand{have}\isamarkupfalse%
\ {\isachardoublequoteopen}{\isasymdots}\ {\isacharequal}\ h{\isacharparenleft}f{\isacharparenleft}x{\isacharparenright}{\isacharparenright}{\isachardoublequoteclose}\ \isacommand{using}\isamarkupfalse%
\ {\isacharbackquoteopen}g\ {\isacharequal}\ h{\isacharbackquoteclose}\ \isacommand{by}\isamarkupfalse%
\ simp\isanewline
\ \ \ \ \isacommand{also}\isamarkupfalse%
\ \isacommand{have}\isamarkupfalse%
\ {\isachardoublequoteopen}{\isasymdots}\ {\isacharequal}\ {\isacharparenleft}h\ {\isasymcirc}\ f{\isacharparenright}\ x{\isachardoublequoteclose}\ \isacommand{by}\isamarkupfalse%
\ simp\isanewline
\ \ \ \ \isacommand{finally}\isamarkupfalse%
\ \isacommand{show}\isamarkupfalse%
\ {\isachardoublequoteopen}{\isacharparenleft}g\ {\isasymcirc}\ f{\isacharparenright}\ x\ {\isacharequal}\ {\isacharparenleft}h\ {\isasymcirc}\ f{\isacharparenright}\ x{\isachardoublequoteclose}\ \isacommand{by}\isamarkupfalse%
\ simp\isanewline
\ \ \isacommand{qed}\isamarkupfalse%
\isanewline
\isacommand{qed}\isamarkupfalse%
%
\endisatagproof
{\isafoldproof}%
%
\isadelimproof
%
\endisadelimproof
%
\begin{isamarkuptext}%
En la demostración hemos introducido: 
 \begin{itemize}
    \item[] \isa{\mbox{}\inferrule{\mbox{{\isasymexists}x{\isachardot}\ P\ x}\\\ \mbox{{\isasymAnd}x{\isachardot}\ \mbox{}\inferrule{\mbox{P\ x}}{\mbox{Q}}}}{\mbox{Q}}} 
      \hfill (\isa{rule\ exE}) 
  \end{itemize} 
 \begin{itemize}
    \item[] \isa{{\isasymlbrakk}P\ {\isasymLongrightarrow}\ Q{\isacharsemicolon}\ Q\ {\isasymLongrightarrow}\ P{\isasymrbrakk}\ {\isasymLongrightarrow}\ P\ {\isacharequal}\ Q} 
      \hfill (\isa{iffI})
  \end{itemize} 

La demostración aplicativa es:%
\end{isamarkuptext}\isamarkuptrue%
\isacommand{lemma}\isamarkupfalse%
\ {\isachardoublequoteopen}surj\ f\ {\isasymLongrightarrow}\ {\isacharparenleft}{\isacharparenleft}g\ {\isasymcirc}\ f{\isacharparenright}\ {\isacharequal}\ {\isacharparenleft}h\ {\isasymcirc}\ f{\isacharparenright}\ {\isacharparenright}\ {\isacharequal}\ {\isacharparenleft}g\ {\isacharequal}\ h{\isacharparenright}{\isachardoublequoteclose}\isanewline
%
\isadelimproof
\ \ %
\endisadelimproof
%
\isatagproof
\isacommand{apply}\isamarkupfalse%
\ {\isacharparenleft}simp\ add{\isacharcolon}\ surj{\isacharunderscore}def\ fun{\isacharunderscore}eq{\isacharunderscore}iff{\isacharparenright}\isanewline
\ \ \isacommand{apply}\isamarkupfalse%
\ \ metis\isanewline
\ \ \isacommand{done}\isamarkupfalse%
%
\endisatagproof
{\isafoldproof}%
%
\isadelimproof
%
\endisadelimproof
%
\begin{isamarkuptext}%
En esta demostración hemos introducido:
 \begin{itemize}
    \item[] \isa{{\isacharparenleft}f\ {\isacharequal}\ g{\isacharparenright}\ {\isacharequal}\ {\isacharparenleft}{\isasymforall}x{\isachardot}\ f\ x\ {\isacharequal}\ g\ x{\isacharparenright}} 
      \hfill (\isa{fun{\isacharunderscore}eq{\isacharunderscore}iff})
  \end{itemize}%
\end{isamarkuptext}\isamarkuptrue%
%
\isadelimtheory
%
\endisadelimtheory
%
\isatagtheory
%
\endisatagtheory
{\isafoldtheory}%
%
\isadelimtheory
%
\endisadelimtheory
%
\end{isabellebody}%
\endinput
%:%file=/home/carlos/Escritorio/TFG-v1/EjerciciosDELMF/CancelacionSobreyectiva.thy%:%
%:%24=7%:%
%:%36=10%:%
%:%37=11%:%
%:%38=12%:%
%:%39=13%:%
%:%40=14%:%
%:%41=15%:%
%:%42=16%:%
%:%43=17%:%
%:%44=18%:%
%:%45=19%:%
%:%46=20%:%
%:%47=21%:%
%:%48=22%:%
%:%49=23%:%
%:%50=24%:%
%:%51=25%:%
%:%52=26%:%
%:%53=27%:%
%:%54=28%:%
%:%55=29%:%
%:%56=30%:%
%:%57=31%:%
%:%59=34%:%
%:%60=34%:%
%:%63=35%:%
%:%67=35%:%
%:%77=38%:%
%:%78=39%:%
%:%79=40%:%
%:%80=41%:%
%:%81=42%:%
%:%82=43%:%
%:%83=44%:%
%:%84=45%:%
%:%85=46%:%
%:%86=47%:%
%:%87=48%:%
%:%88=49%:%
%:%89=50%:%
%:%90=51%:%
%:%91=52%:%
%:%92=53%:%
%:%93=54%:%
%:%94=55%:%
%:%95=56%:%
%:%96=57%:%
%:%97=58%:%
%:%99=61%:%
%:%100=61%:%
%:%101=62%:%
%:%102=63%:%
%:%109=64%:%
%:%110=64%:%
%:%111=65%:%
%:%112=65%:%
%:%113=66%:%
%:%114=66%:%
%:%115=67%:%
%:%116=67%:%
%:%117=68%:%
%:%118=68%:%
%:%119=69%:%
%:%120=69%:%
%:%121=69%:%
%:%122=69%:%
%:%123=70%:%
%:%124=70%:%
%:%125=70%:%
%:%126=70%:%
%:%127=71%:%
%:%128=71%:%
%:%129=71%:%
%:%130=71%:%
%:%131=72%:%
%:%132=72%:%
%:%133=72%:%
%:%134=72%:%
%:%135=73%:%
%:%136=73%:%
%:%137=73%:%
%:%138=73%:%
%:%139=73%:%
%:%140=74%:%
%:%141=74%:%
%:%142=74%:%
%:%143=74%:%
%:%144=75%:%
%:%145=75%:%
%:%146=75%:%
%:%147=75%:%
%:%148=75%:%
%:%149=76%:%
%:%150=76%:%
%:%151=76%:%
%:%152=77%:%
%:%153=77%:%
%:%154=78%:%
%:%155=78%:%
%:%156=79%:%
%:%157=79%:%
%:%158=80%:%
%:%159=80%:%
%:%160=81%:%
%:%161=81%:%
%:%162=82%:%
%:%163=82%:%
%:%164=83%:%
%:%165=83%:%
%:%166=84%:%
%:%167=84%:%
%:%168=85%:%
%:%169=85%:%
%:%170=85%:%
%:%171=86%:%
%:%172=86%:%
%:%173=86%:%
%:%174=86%:%
%:%175=86%:%
%:%176=87%:%
%:%177=87%:%
%:%178=87%:%
%:%179=87%:%
%:%180=88%:%
%:%181=88%:%
%:%182=88%:%
%:%183=88%:%
%:%184=89%:%
%:%185=89%:%
%:%186=90%:%
%:%196=93%:%
%:%197=94%:%
%:%198=95%:%
%:%199=96%:%
%:%200=97%:%
%:%201=98%:%
%:%202=99%:%
%:%203=100%:%
%:%204=101%:%
%:%205=102%:%
%:%206=103%:%
%:%208=105%:%
%:%209=105%:%
%:%212=106%:%
%:%216=106%:%
%:%217=106%:%
%:%218=107%:%
%:%219=107%:%
%:%220=108%:%
%:%230=110%:%
%:%231=111%:%
%:%232=112%:%
%:%233=113%:%
%:%234=114%:%
\chapter{Teoría de conjuntos}
%
\begin{isabellebody}%
\setisabellecontext{TeoremaCantor}%
%
\isadelimtheory
\isanewline
%
\endisadelimtheory
%
\isatagtheory
%
\endisatagtheory
{\isafoldtheory}%
%
\isadelimtheory
%
\endisadelimtheory
%
\begin{isamarkuptext}%
\comentario{Estructurar en secciones.}%
\end{isamarkuptext}\isamarkuptrue%
%
\begin{isamarkuptext}%
\comentario{Hacer demostraciones detalladas.}%
\end{isamarkuptext}\isamarkuptrue%
%
\begin{isamarkuptext}%
\comentario{Añadir lemas usados al Soporte.}%
\end{isamarkuptext}\isamarkuptrue%
%
\begin{isamarkuptext}%
El siguiente, denominado  teorema de Cantor por el matemático
 Georg Cantor, es un resultado importante de la teoría
 de conjuntos. 

El matemático Georg Ferdinand Ludwig Philipp Cantor fue un matemático y
lógico nacido en Rusia en el siglo XIX. Fue inventor junto con Dedekind
 y Frege de la teoría de conjuntos, que es la base de las matemáticas
 modernas.



Para la comprensión del teorema vamos a definir una serie de conceptos:

\begin {itemize}

\item Conjunto de potencia $A$  $(\mathcal{P}(A))$: conjunto formado por
todos los subconjuntos de $A$.
\item Cardinal del conjunto $A$ (Denotado $\# A$): número de elementos del propio
 conjunto.

\end {itemize}
El enunciado original del teorema es el siguiente : 


\begin {teorema}
El cardinal del conjunto potencia de cualquier conjunto A es
 estrictamente mayor que el cardinal de A, o lo que es lo mismo,
$\# \mathcal{P}(A) > \# A.$


\end {teorema}
Pero el enunciado del teorema lo podemos reformular como: 
\begin{teorema}
Dado un conjunto A, $\nexists  f : A \longrightarrow \mathcal{P}(A)$ que
sea sobreyectiva.

\end{teorema}

El teorema lo hemos podido reescribir de la anterior forma, ya que si
 suponemos que $\exists f$ tal que $f: A \longrightarrow \mathcal{P}(A)$
es sobreyectiva, entonces tenemos que $f(A) = \mathcal{P}(A)$ y por lo
 tanto, $\# f(A) \geq \# \mathcal{P}(A)$, de lo que se deduce esta
 reformulación. Reciprocamente, es trivial ver que esta reformulación
 implica la primera.
 con el teorema. \\
El teorema de Cantor es trivial para conjuntos finitos, ya que el
 conjunto potencia, de conjuntos finitos de n elementos tiene
 $2^n$ elementos.

Por ello,  vamos a realizar la prueba para conjuntos infinitos. 


\begin{demostracion}
 
Vamos a realizar la prueba por reducción al absurdo.\\
Supongamos que $\exists f : A \longrightarrow \mathcal{P}(A)$ sobreyectiva, es
 decir, $\forall C \in \rho(A) ,  \exists x \in A$ tal que $C = f(x)$.
En particular, tomemos el conjunto $$B = \{ x \in A : x \notin f(x) \}$$
 y  supongamos que $\exists a \in A : B = f(a)$, ya que $B$ es un
 subconjunto de A, luego podemos distinguir dos casos $:$ \\
$1.$ Si $a \in B$, entonces por definición del conjunto $B$ tenemos que
$a \notin B$, luego llegamos a una contradicción. \\
$2.$ Si $a \notin B$, entonces por definición de B tenemos que $a \in 
B$, luego hemos llegado a otra contradicción. 

En las dos hipótesis hemos llegado a una contradicción,
por lo que no existe $a$ y $f$ no es sobreyectiva.
\end{demostracion}


Para la especificación del teorema en Isabelle, primero debemos notar
 que $$f :: \, 'a \Rightarrow \,'a \: set$$
 significa que es una función 
de tipos, donde $'a$ significa un tipo y para poder denotar
el conjunto potencia tenemos que poner $'a \ set$ que significa que es
 de un tipo formado por conjuntos del tipo $'a$.




El enunciado del teorema es el siguiente :%
\end{isamarkuptext}\isamarkuptrue%
\isacommand{theorem}\isamarkupfalse%
\ Cantor{\isacharcolon}\ {\isachardoublequoteopen}{\isasymnexists}f\ {\isacharcolon}{\isacharcolon}\ {\isacharprime}a\ {\isasymRightarrow}\ {\isacharprime}a\ set{\isachardot}\ {\isasymforall}A{\isachardot}\ {\isasymexists}x{\isachardot}\ A\ {\isacharequal}\ f\ x{\isachardoublequoteclose}\isanewline
\isanewline
%
\isadelimproof
\isanewline
\ \ %
\endisadelimproof
%
\isatagproof
\isacommand{oops}\isamarkupfalse%
%
\endisatagproof
{\isafoldproof}%
%
\isadelimproof
%
\endisadelimproof
%
\begin{isamarkuptext}%
La demostración la haremos por la regla la introducción a la
negación, la cual es una simplificación de la regla de 
reducción al absurdo, cuyo esquema mostramos a continuación:   
 \begin{itemize}
  \item[] \isa{{\isacharparenleft}P\ {\isasymLongrightarrow}\ False{\isacharparenright}\ {\isasymLongrightarrow}\ {\isasymnot}\ P} \hfill (\isa{notI})
  \end{itemize}


Esta es la demostración detallada del teorema:%
\end{isamarkuptext}\isamarkuptrue%
\isacommand{theorem}\isamarkupfalse%
\ CantorDetallada{\isacharcolon}\ {\isachardoublequoteopen}{\isasymnexists}f\ {\isacharcolon}{\isacharcolon}\ {\isacharprime}a\ {\isasymRightarrow}\ {\isacharprime}a\ set{\isachardot}\ {\isasymforall}B{\isachardot}\ {\isasymexists}x{\isachardot}\ B\ {\isacharequal}\ f\ x{\isachardoublequoteclose}\isanewline
%
\isadelimproof
%
\endisadelimproof
%
\isatagproof
\isacommand{proof}\isamarkupfalse%
\ {\isacharparenleft}rule\ notI{\isacharparenright}\isanewline
\ \ \isacommand{assume}\isamarkupfalse%
\ {\isachardoublequoteopen}{\isasymexists}f\ {\isacharcolon}{\isacharcolon}\ {\isacharprime}a\ {\isasymRightarrow}\ {\isacharprime}a\ set{\isachardot}\ {\isasymforall}A{\isachardot}\ {\isasymexists}x{\isachardot}\ A\ {\isacharequal}\ f\ x{\isachardoublequoteclose}\isanewline
\ \ \isacommand{then}\isamarkupfalse%
\ \isacommand{obtain}\isamarkupfalse%
\ f\ {\isacharcolon}{\isacharcolon}\ {\isachardoublequoteopen}{\isacharprime}a\ {\isasymRightarrow}\ {\isacharprime}a\ set{\isachardoublequoteclose}\ \isakeyword{where}\ {\isacharasterisk}{\isacharcolon}\ {\isachardoublequoteopen}{\isasymforall}A{\isachardot}\ {\isasymexists}x{\isachardot}\ A\ {\isacharequal}\ f\ x{\isachardoublequoteclose}\ \isacommand{by}\isamarkupfalse%
\ {\isacharparenleft}rule\isanewline
\ \ \ \ \ \ \ \ exE{\isacharparenright}\isanewline
\ \ \isacommand{let}\isamarkupfalse%
\ {\isacharquery}B\ {\isacharequal}\ {\isachardoublequoteopen}{\isacharbraceleft}x{\isachardot}\ x\ {\isasymnotin}\ f\ x{\isacharbraceright}{\isachardoublequoteclose}\isanewline
\ \ \isacommand{from}\isamarkupfalse%
\ {\isacharasterisk}\ \isacommand{obtain}\isamarkupfalse%
\ {\isachardoublequoteopen}\ {\isasymexists}x{\isachardot}\ {\isacharquery}B\ {\isacharequal}\ f\ x\ {\isachardoublequoteclose}\ \isacommand{by}\isamarkupfalse%
\ {\isacharparenleft}rule\ allE{\isacharparenright}\isanewline
\ \ \isacommand{then}\isamarkupfalse%
\ \ \isacommand{obtain}\isamarkupfalse%
\ a\ \isakeyword{where}\ {\isadigit{1}}{\isacharcolon}{\isachardoublequoteopen}{\isacharquery}B\ {\isacharequal}\ f\ a{\isachardoublequoteclose}\ \isacommand{by}\isamarkupfalse%
\ {\isacharparenleft}rule\ exE{\isacharparenright}\isanewline
\ \ \isacommand{show}\isamarkupfalse%
\ False\isanewline
\ \ \isacommand{proof}\isamarkupfalse%
\ {\isacharparenleft}cases{\isacharparenright}\isanewline
\ \ \ \ \isacommand{assume}\isamarkupfalse%
\ {\isachardoublequoteopen}a\ {\isasymin}\ {\isacharquery}B{\isachardoublequoteclose}\ \ \isanewline
\ \ \ \ \isacommand{then}\isamarkupfalse%
\ \isacommand{show}\isamarkupfalse%
\ False\ \ \isacommand{using}\isamarkupfalse%
\ {\isadigit{1}}\ \isacommand{by}\isamarkupfalse%
\ blast\isanewline
\ \ \isacommand{next}\isamarkupfalse%
\ \isanewline
\ \ \ \ \isacommand{assume}\isamarkupfalse%
\ {\isachardoublequoteopen}a\ {\isasymnotin}\ {\isacharquery}B{\isachardoublequoteclose}\isanewline
\ \ \ \ \isacommand{thus}\isamarkupfalse%
\ False\ \isacommand{using}\isamarkupfalse%
\ {\isadigit{1}}\ \isacommand{by}\isamarkupfalse%
\ blast\isanewline
\ \ \isacommand{qed}\isamarkupfalse%
\isanewline
\isacommand{qed}\isamarkupfalse%
%
\endisatagproof
{\isafoldproof}%
%
\isadelimproof
%
\endisadelimproof
%
\begin{isamarkuptext}%
Esta es la demostración aplicativa del teorema:%
\end{isamarkuptext}\isamarkuptrue%
\isacommand{theorem}\isamarkupfalse%
\ CantorAplicativa\ {\isacharcolon}\isanewline
\ {\isachardoublequoteopen}{\isasymnexists}f\ {\isacharcolon}{\isacharcolon}\ {\isacharprime}a\ {\isasymRightarrow}\ {\isacharprime}a\ set{\isachardot}\ {\isasymforall}A{\isachardot}\ {\isasymexists}x{\isachardot}\ A\ {\isacharequal}\ f\ x{\isachardoublequoteclose}\isanewline
%
\isadelimproof
\ \ %
\endisadelimproof
%
\isatagproof
\isacommand{apply}\isamarkupfalse%
\ {\isacharparenleft}rule\ notI{\isacharparenright}\isanewline
\ \ \isacommand{apply}\isamarkupfalse%
\ {\isacharparenleft}erule\ exE{\isacharparenright}\isanewline
\ \ \isacommand{apply}\isamarkupfalse%
\ {\isacharparenleft}erule{\isacharunderscore}tac\ x\ {\isacharequal}\ {\isachardoublequoteopen}{\isacharbraceleft}x{\isachardot}\ x\ {\isasymnotin}\ f\ x{\isacharbraceright}{\isachardoublequoteclose}\ \isakeyword{in}\ allE{\isacharparenright}\isanewline
\ \ \isacommand{apply}\isamarkupfalse%
\ {\isacharparenleft}erule\ exE{\isacharparenright}\isanewline
\ \ \isacommand{apply}\isamarkupfalse%
\ \ blast\ \isanewline
\ \ \isacommand{done}\isamarkupfalse%
%
\endisatagproof
{\isafoldproof}%
%
\isadelimproof
%
\endisadelimproof
%
\begin{isamarkuptext}%
Esta es la demostración automática del teorema:%
\end{isamarkuptext}\isamarkuptrue%
\isacommand{theorem}\isamarkupfalse%
\ CantorAutomatic{\isacharcolon}\ {\isachardoublequoteopen}{\isasymnexists}f\ {\isacharcolon}{\isacharcolon}\ {\isacharprime}a\ {\isasymRightarrow}\ {\isacharprime}a\ set{\isachardot}\ {\isasymforall}B{\isachardot}\ {\isasymexists}x{\isachardot}\ B\ {\isacharequal}\ f\ x{\isachardoublequoteclose}\isanewline
%
\isadelimproof
\ \ %
\endisadelimproof
%
\isatagproof
\isacommand{by}\isamarkupfalse%
\ best%
\endisatagproof
{\isafoldproof}%
%
\isadelimproof
%
\endisadelimproof
%
\begin{isamarkuptext}%
En la demostración de isabelle hemos utilizado el método de prueba
rule con las siguientes reglas, tanto en la aplicativa como en la
 detallada:
 \begin{itemize}
  \item[] \isa{\mbox{}\inferrule{\mbox{\mbox{}\inferrule{\mbox{P}}{\mbox{False}}}}{\mbox{{\isasymnot}\ P}}} \hfill (\isa{notI})
  \end{itemize}
 \begin{itemize}
  \item[] \isa{\mbox{}\inferrule{\mbox{{\isasymexists}x{\isachardot}\ P\ x}\\\ \mbox{{\isasymAnd}x{\isachardot}\ \mbox{}\inferrule{\mbox{P\ x}}{\mbox{Q}}}}{\mbox{Q}}} \hfill (\isa{exE})
  \end{itemize}
 \begin{itemize}
  \item[] \isa{\mbox{}\inferrule{\mbox{{\isasymforall}x{\isachardot}\ P\ x}\\\ \mbox{\mbox{}\inferrule{\mbox{P\ x}}{\mbox{R}}}}{\mbox{R}}} \hfill (\isa{allE})
  \end{itemize}
También hacemos uso de blast, que es un conjunto de reglas lógicas y 
 la demostración automática la hacemos por medio de "best".%
\end{isamarkuptext}\isamarkuptrue%
%
\isadelimtheory
%
\endisadelimtheory
%
\isatagtheory
%
\endisatagtheory
{\isafoldtheory}%
%
\isadelimtheory
%
\endisadelimtheory
%
\end{isabellebody}%
\endinput
%:%file=~/ownCloud/alonso/curso-TFG/Carlos/TFG_de_Carlos/TeoremaCantor.thy%:%
%:%6=1%:%
%:%20=9%:%
%:%24=11%:%
%:%28=13%:%
%:%32=15%:%
%:%33=16%:%
%:%34=17%:%
%:%35=18%:%
%:%36=19%:%
%:%37=20%:%
%:%38=21%:%
%:%39=22%:%
%:%40=23%:%
%:%41=24%:%
%:%42=25%:%
%:%43=26%:%
%:%44=27%:%
%:%45=28%:%
%:%46=29%:%
%:%47=30%:%
%:%48=31%:%
%:%49=32%:%
%:%50=33%:%
%:%51=34%:%
%:%52=35%:%
%:%53=36%:%
%:%54=37%:%
%:%55=38%:%
%:%56=39%:%
%:%57=40%:%
%:%58=41%:%
%:%59=42%:%
%:%60=43%:%
%:%61=44%:%
%:%62=45%:%
%:%63=46%:%
%:%64=47%:%
%:%65=48%:%
%:%66=49%:%
%:%67=50%:%
%:%68=51%:%
%:%69=52%:%
%:%70=53%:%
%:%71=54%:%
%:%72=55%:%
%:%73=56%:%
%:%74=57%:%
%:%75=58%:%
%:%76=59%:%
%:%77=60%:%
%:%78=61%:%
%:%79=62%:%
%:%80=63%:%
%:%81=64%:%
%:%82=65%:%
%:%83=66%:%
%:%84=67%:%
%:%85=68%:%
%:%86=69%:%
%:%87=70%:%
%:%88=71%:%
%:%89=72%:%
%:%90=73%:%
%:%91=74%:%
%:%92=75%:%
%:%93=76%:%
%:%94=77%:%
%:%95=78%:%
%:%96=79%:%
%:%97=80%:%
%:%98=81%:%
%:%99=82%:%
%:%100=83%:%
%:%101=84%:%
%:%102=85%:%
%:%103=86%:%
%:%104=87%:%
%:%105=88%:%
%:%106=89%:%
%:%107=90%:%
%:%108=91%:%
%:%109=92%:%
%:%110=93%:%
%:%111=94%:%
%:%112=95%:%
%:%114=97%:%
%:%115=97%:%
%:%116=98%:%
%:%119=99%:%
%:%120=100%:%
%:%124=100%:%
%:%134=102%:%
%:%135=103%:%
%:%136=104%:%
%:%137=105%:%
%:%138=106%:%
%:%139=107%:%
%:%140=108%:%
%:%141=109%:%
%:%142=110%:%
%:%144=112%:%
%:%145=112%:%
%:%152=113%:%
%:%153=113%:%
%:%154=114%:%
%:%155=114%:%
%:%156=115%:%
%:%157=115%:%
%:%158=115%:%
%:%159=115%:%
%:%160=116%:%
%:%161=117%:%
%:%162=117%:%
%:%163=118%:%
%:%164=118%:%
%:%165=118%:%
%:%166=118%:%
%:%167=119%:%
%:%168=119%:%
%:%169=119%:%
%:%170=119%:%
%:%171=120%:%
%:%172=120%:%
%:%173=121%:%
%:%174=121%:%
%:%175=122%:%
%:%176=122%:%
%:%177=123%:%
%:%178=123%:%
%:%179=123%:%
%:%180=123%:%
%:%181=123%:%
%:%182=124%:%
%:%183=124%:%
%:%184=125%:%
%:%185=125%:%
%:%186=126%:%
%:%187=126%:%
%:%188=126%:%
%:%189=126%:%
%:%190=127%:%
%:%191=127%:%
%:%192=128%:%
%:%202=130%:%
%:%204=133%:%
%:%205=133%:%
%:%206=134%:%
%:%209=135%:%
%:%213=135%:%
%:%214=135%:%
%:%215=136%:%
%:%216=136%:%
%:%217=137%:%
%:%218=137%:%
%:%219=138%:%
%:%220=138%:%
%:%221=139%:%
%:%222=139%:%
%:%223=140%:%
%:%233=142%:%
%:%235=143%:%
%:%236=143%:%
%:%239=144%:%
%:%243=144%:%
%:%244=144%:%
%:%253=146%:%
%:%254=147%:%
%:%255=148%:%
%:%256=149%:%
%:%257=150%:%
%:%258=151%:%
%:%259=152%:%
%:%260=153%:%
%:%261=154%:%
%:%262=155%:%
%:%263=156%:%
%:%264=157%:%
%:%265=158%:%
%:%266=159%:%
\chapter{Teoría de retículos}
\input{TeoremaKnasterTarski}


\appendix

\chapter{Métodos de pruebas y reglas}
%
\begin{isabellebody}%
\setisabellecontext{Metodosdepruebasyreglas}%
%
\isadelimtheory
\isanewline
%
\endisadelimtheory
%
\isatagtheory
%
\endisatagtheory
{\isafoldtheory}%
%
\isadelimtheory
%
\endisadelimtheory
%
\begin{isamarkuptext}%
Métodos de pruebas de demostraciones:

 \begin{itemize}
  \item[] \isa{{\isasymlbrakk}P\ {\isadigit{0}}{\isacharsemicolon}\ {\isasymAnd}nat{\isachardot}\ P\ nat\ {\isasymLongrightarrow}\ P\ {\isacharparenleft}Suc\ nat{\isacharparenright}{\isasymrbrakk}\ {\isasymLongrightarrow}\ P\ nat} \hfill (\isa{nat{\isachardot}induct})
  \end{itemize}

 \begin{itemize}
  \item[] \isa{{\isasymlbrakk}P\ {\isasymLongrightarrow}\ Q{\isacharsemicolon}\ Q\ {\isasymLongrightarrow}\ P{\isasymrbrakk}\ {\isasymLongrightarrow}\ P\ {\isacharequal}\ Q} \hfill (\isa{iffI})
  \end{itemize}

 \begin{itemize}
  \item[] \isa{{\isasymlbrakk}finite\ x{\isacharsemicolon}\ P\ {\isasymemptyset}{\isacharsemicolon}\ {\isasymAnd}A\ a{\isachardot}\ finite\ A\ {\isasymand}\ P\ A\ {\isasymLongrightarrow}\ P\ {\isacharparenleft}{\isacharbraceleft}a{\isacharbraceright}\ {\isasymunion}\ A{\isacharparenright}{\isasymrbrakk}\ {\isasymLongrightarrow}\ P\ x} \hfill (\isa{finite{\isachardot}induct})
  \end{itemize}

 \begin{itemize}
  \item[] \isa{{\isacharparenleft}P\ {\isasymLongrightarrow}\ False{\isacharparenright}\ {\isasymLongrightarrow}\ {\isasymnot}\ P} \hfill (\isa{notI})
  \end{itemize}


Reglas usadas:

 \begin{itemize}
  \item[] \isa{inj{\isacharunderscore}on\ f\ A\ {\isacharequal}\ {\isacharparenleft}{\isasymforall}x{\isasymin}A{\isachardot}\ {\isasymforall}y{\isasymin}A{\isachardot}\ f\ x\ {\isacharequal}\ f\ y\ {\isasymlongrightarrow}\ x\ {\isacharequal}\ y{\isacharparenright}} \hfill (\isa{inj{\isacharunderscore}on{\isacharunderscore}def})
  \end{itemize}
 \begin{itemize}
  \item[] \isa{\mbox{}\inferrule{\mbox{ordering{\isacharunderscore}top\ less{\isacharunderscore}eq\ less\ top}}{\mbox{less{\isacharunderscore}eq\ a\ top}}} \hfill
 (\isa{ordering{\isacharunderscore}top{\isachardot}extremum})
  \end{itemize}
 \begin{itemize}
  \item[] \isa{{\isacharparenleft}f\ {\isacharequal}\ g{\isacharparenright}\ {\isacharequal}\ {\isacharparenleft}{\isasymforall}x{\isachardot}\ f\ x\ {\isacharequal}\ g\ x{\isacharparenright}} \hfill (\isa{fun{\isacharunderscore}eq{\isacharunderscore}iff})
  \end{itemize}
 \begin{itemize}
  \item[] \isa{{\isacharparenleft}f\ {\isasymcirc}\ g{\isacharparenright}\ x\ {\isacharequal}\ f\ {\isacharparenleft}g\ x{\isacharparenright}} \hfill (\isa{o{\isacharunderscore}apply})
  \end{itemize}
 \begin{itemize}
  \item[] \isa{\mbox{}\inferrule{\mbox{\mbox{}\inferrule{\mbox{P}}{\mbox{Q}}}\\\ \mbox{\mbox{}\inferrule{\mbox{Q}}{\mbox{P}}}}{\mbox{P\ {\isacharequal}\ Q}}} \hfill (\isa{iffI})
  \end{itemize}
 \begin{itemize}
  \item[] \isa{\mbox{}\inferrule{\mbox{ListMem\ x\ xs}}{\mbox{ListMem\ x\ {\isacharparenleft}y\ {\isasymcdot}\ xs{\isacharparenright}}}} \hfill (\isa{insert})
  \end{itemize}
 \begin{itemize}
  \item[] \isa{\mbox{}\inferrule{\mbox{{\isasymexists}x{\isachardot}\ P\ x}\\\ \mbox{{\isasymAnd}x{\isachardot}\ \mbox{}\inferrule{\mbox{P\ x}}{\mbox{Q}}}}{\mbox{Q}}} \hfill (\isa{exE})
  \end{itemize}
 \begin{itemize}
  \item[] \isa{\mbox{}\inferrule{\mbox{{\isasymforall}x{\isachardot}\ P\ x}\\\ \mbox{\mbox{}\inferrule{\mbox{P\ x}}{\mbox{R}}}}{\mbox{R}}} \hfill (\isa{allE})
  \end{itemize}
 \begin{itemize}
  \item[] \isa{\mbox{}\inferrule{\mbox{\mbox{}\inferrule{\mbox{P}}{\mbox{False}}}}{\mbox{{\isasymnot}\ P}}} \hfill (\isa{notI})
  \end{itemize}
 \begin{itemize}
  \item[] \isa{{\isacharparenleft}{\isacharparenleft}P\ {\isasymlongrightarrow}\ Q{\isacharparenright}\ {\isasymand}\ {\isacharparenleft}{\isasymnot}\ P\ {\isasymlongrightarrow}\ Q{\isacharparenright}{\isacharparenright}\ {\isacharequal}\ Q} \hfill (\isa{cases})
  \end{itemize}%
\end{isamarkuptext}\isamarkuptrue%
%
\isadelimtheory
%
\endisadelimtheory
%
\isatagtheory
%
\endisatagtheory
{\isafoldtheory}%
%
\isadelimtheory
%
\endisadelimtheory
%
\end{isabellebody}%
\endinput
%:%file=~/Escritorio/TFG/Metodosdepruebasyreglas.thy%:%
%:%6=1%:%
%:%20=9%:%
%:%21=10%:%
%:%22=11%:%
%:%23=12%:%
%:%24=13%:%
%:%25=14%:%
%:%26=15%:%
%:%27=16%:%
%:%28=17%:%
%:%29=18%:%
%:%30=19%:%
%:%31=20%:%
%:%32=21%:%
%:%33=22%:%
%:%34=23%:%
%:%35=24%:%
%:%36=25%:%
%:%37=26%:%
%:%38=27%:%
%:%39=28%:%
%:%40=29%:%
%:%41=30%:%
%:%42=31%:%
%:%42=32%:%
%:%43=33%:%
%:%44=34%:%
%:%45=35%:%
%:%46=36%:%
%:%47=37%:%
%:%48=38%:%
%:%49=39%:%
%:%50=40%:%
%:%51=41%:%
%:%52=42%:%
%:%53=43%:%
%:%54=44%:%
%:%55=45%:%
%:%56=46%:%
%:%57=47%:%
%:%58=48%:%
%:%59=49%:%
%:%60=50%:%
%:%61=51%:%
%:%62=52%:%
%:%63=53%:%
%:%64=54%:%
%:%65=55%:%
%:%66=56%:%
%:%67=57%:%
%:%68=58%:%
%:%69=59%:%
%:%70=60%:%
%:%71=61%:%

\chapter{Lemas de HOL usados}
%
\begin{isabellebody}%
\setisabellecontext{Soporte}%
%
\isadelimtheory
%
\endisadelimtheory
%
\isatagtheory
%
\endisatagtheory
{\isafoldtheory}%
%
\isadelimtheory
%
\endisadelimtheory
%
\begin{isamarkuptext}%
En este apéndice se recogen la lista de los lemas usados en
  el trabajo indicando la página del
  \href{http://bit.ly/2OMbjMM}{libro de HOL} donde se encuentra.%
\end{isamarkuptext}\isamarkuptrue%
%
\begin{isamarkuptext}%
\comentario{Añadir el libro de HOL a la bibliografía.}%
\end{isamarkuptext}\isamarkuptrue%
%
\begin{isamarkuptext}%
\comentario{Completar la lista de lemas usados.}%
\end{isamarkuptext}\isamarkuptrue%
%
\isadelimdocument
%
\endisadelimdocument
%
\isatagdocument
%
\isamarkupsection{Números naturales (16)%
}
\isamarkuptrue%
%
\isamarkupsubsection{Operaciones aritméticas (16.3)%
}
\isamarkuptrue%
%
\endisatagdocument
{\isafolddocument}%
%
\isadelimdocument
%
\endisadelimdocument
%
\begin{isamarkuptext}%
\begin{itemize}
  \item (p. 348) \isa{{\isadigit{0}}\ {\isacharasterisk}\ n\ {\isacharequal}\ {\isadigit{0}}}
    \hfill (\isa{mult{\isacharunderscore}{\isadigit{0}}}) 
  \item (p. 348) \isa{Suc\ m\ {\isacharasterisk}\ n\ {\isacharequal}\ n\ {\isacharplus}\ m\ {\isacharasterisk}\ n}
    \hfill (\isa{mult{\isacharunderscore}Suc}) 
  \item (p. 348) \isa{m\ {\isacharasterisk}\ Suc\ n\ {\isacharequal}\ m\ {\isacharplus}\ m\ {\isacharasterisk}\ n}
    \hfill (\isa{mult{\isacharunderscore}Suc{\isacharunderscore}right}) 
\end{itemize}%
\end{isamarkuptext}\isamarkuptrue%
%
\isadelimtheory
%
\endisadelimtheory
%
\isatagtheory
%
\endisatagtheory
{\isafoldtheory}%
%
\isadelimtheory
%
\endisadelimtheory
%
\end{isabellebody}%
\endinput
%:%file=~/ownCloud/alonso/curso-TFG/Carlos/TFG_de_Carlos/Soporte.thy%:%
%:%19=11%:%
%:%20=12%:%
%:%21=13%:%
%:%25=15%:%
%:%29=17%:%
%:%38=19%:%
%:%42=21%:%
%:%54=24%:%
%:%55=25%:%
%:%56=26%:%
%:%57=27%:%
%:%58=28%:%
%:%59=29%:%
%:%60=30%:%
%:%61=31%:%

% optional bibliography
\nocite{LMF, tutorial}
\bibliographystyle{plain}
\bibliography{root}

% Pendientes
\todototoc
\listoftodos

\end{document}

%%% Local Variables:
%%% mode: latex
%%% TeX-master: t
%%% End:
